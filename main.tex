\documentclass[12pt]{book}
\usepackage[a5paper]{geometry}

% Input characters
\usepackage[utf8]{inputenc}
\usepackage[T1]{fontenc}
\usepackage{tipa}

% Tables
\usepackage{longtable}
\usepackage{multicol}
\usepackage{multirow}

% Remove table floating
\usepackage[section]{placeins}

\usepackage{url}
\usepackage{graphicx}
\usepackage{makeidx}
\usepackage{glossaries}

\setlength{\parindent}{1em}

% Where input should look for
\makeatletter 
\def\input@path{{../}} 
\makeatother

\makeindex
% CMD: makeindex main.idx
\makeglossaries
% CMD: makeglossaries main

\DeclareUnicodeCharacter{0166}{\barredT}
\DeclareUnicodeCharacter{0167}{\barredt}

\DeclareRobustCommand{\barredT}{\barredTt{0.5}{0.05}{1.5}{T}}
\DeclareRobustCommand{\barredt}{\barredTt{0.4}{0}{1.15}{t}}
\newcommand{\reducedhyphen}[2]{%
	\raisebox{#1ex}{\scalebox{#2}[0.5]{-}}%
}
\newcommand{\barredTt}[4]{%
	\begingroup
	\vphantom{#4}%
	\ooalign{%
		#4\cr
		\hidewidth\kern#2em\reducedhyphen{#1}{#3}\hidewidth\cr
	}%
	\endgroup
}

\begin{document}
	
\newcommand{\plogo}{\fbox{$\mathcal{KRIN}$}} % Generic dummy publisher logo

\begin{titlepage} % Suppresses headers and footers on the title page
	
	\centering % Centre everything on the title page
	
	\scshape % Use small caps for all text on the title page
	
	\vspace*{\baselineskip} % White space at the top of the page
	
	%------------------------------------------------
	%	Title
	%------------------------------------------------
	
	\rule{\textwidth}{1.6pt}\vspace*{-\baselineskip}\vspace*{2pt} % Thick horizontal rule
	\rule{\textwidth}{0.4pt} % Thin horizontal rule
	
	\vspace{0.75\baselineskip} % Whitespace above the title
	
	{\LARGE NOVOSLOVNICA} % Title
	
	\vspace{0.75\baselineskip} % Whitespace below the title
	
	\rule{\textwidth}{0.4pt}\vspace*{-\baselineskip}\vspace{3.2pt} % Thin horizontal rule
	\rule{\textwidth}{1.6pt} % Thick horizontal rule
	
	\vspace{2\baselineskip} % Whitespace after the title block
	
	%------------------------------------------------
	%	Subtitle
	%------------------------------------------------
	
	Guide through a Slavic constructed language % Subtitle or further description
	
	\vspace*{3\baselineskip} % Whitespace under the subtitle
	
	%------------------------------------------------
	%	Editor(s)
	%------------------------------------------------
	
	Edited By
	
	\vspace{0.5\baselineskip} % Whitespace before the editors
	
	{\scshape\Large George Carpow \\} % Editor list
	
	\vspace{0.5\baselineskip} % Whitespace below the editor list
	
	\textit{Krizhanich Research Institute of Novoslovnica \\ Moscow} % Editor affiliation
	
	\vfill % Whitespace between editor names and publisher logo
	
	%------------------------------------------------
	%	Publisher
	%------------------------------------------------
	
	\plogo % Publisher logo
	
	\vspace{0.3\baselineskip} % Whitespace under the publisher logo
	
	2015-2019 % Publication year
	
	{\large publisher} % Publisher
	
\end{titlepage}


% \author{Krizhanich Research Institute of Novoslovnica}
% \title{%
%	Novoslovnica \\
%	\large Guide through a Slavic constructed language
%\vfill
%}
%\date{2015-2019}

%\end{titlepage}
\frontmatter
% \maketitle
\chapter{Acknowledgements}

Starting this book I would like to appreciate some persons who had great influence on me writing this book. Such a work never can be done with efforts of a single man. I am grateful to those who believed in the project for the whole development period or just for a while.

First of all I want to thank Rafail Gasparyan. I suppose without his help and collaboration this book would have never been finished. And maybe so Novoslovnica would.

I would like to thank my Macedonian friend Kristijan Cvetanovski. His trust in Novoslovnica have been making me continue my work for all these five years. And with the help of him I discovered some important points that shook the grounds of the project a bit of times.

I want to thank my American friend Harris Mowbray and my Belarusian friend Sergey Yanchenko for editing English and Russian text of the work respectively.

Also I want to thank my Czech colleague Vojtěh Merunka for his critical view on this project. This made me rethink an amount of language details. Without that I don't believe Novoslovnica would have ever become a serious project. 

I would like to thank my brother Ivan Carpow who spent his time and efforts on readting this book and making comments and corrections.

Moreover, I would like to appreciate persons that keep the inspiration inside me for these years: Manol Terziev (Bulgaria), Alexandra Getsich (Serbia), Ljiljana Bradich (Serbia).

\chapter{Foreword}

Every language is alive - it is born, it grows up, it bares children and it dies.

The process of developing a language is unstoppable and it is cause by a lot of factors we cannot track.

What a beauty is a moment a new language appears. A lot of languages are old and appeared many years ago. However, there exist a very young ones were witnesses to be born i.e. Afrikaans (20 century), Yiddish (19 century) and extinct i.e. Prussian (18 century), Slovinian (20 century), Livonian (2013 year).

The language is naturally born doubly:

\begin{itemize}
	\item by dialect divergence
	\item by two languages pidginating 
\end{itemize}

The first way is much popular, though in pure form they both appear pretty rarely. This leads to differentiating of languages into a close language group.

Nevertheless, people sometime get into idea of reuniting these groups into a single language. This is not a natural way but we have examples of such a reunion. For example, modern German language was created from a group of different german languages and dialects to unite people into a single strong country. And they succeeded, so today we ahve a stable literary German language (though some region dialects still exist).

Moreover, people want to unite languages that have diverged much greater than German dialects did. It is rather enviable, though so many examples took place (Volapük, Novial). However, one of them, called Esperanto created by L.L. Zamenhof in 1887 year was very successful. Today more than 2 million people speak Esperanto and several thousand of them use it as a native language.

These examples show that the languages develop both ways - unification and divergence.

Novoslovnica as such a bridge for Slavic languages is working in direction of uniting the language. Collected expressiveness, beauty and purity of all Slavic languages, it tries to rework the idea of what s Slavic language should be.

It is not the only project in this sphere. Starting from Cyril and Methodius, Slavs continue to think about the idea of a common language when loosing the previous project in mind.

The need of a secular common language for Slavs is especially important since the twentieth century. The process of globalization is destroying weak languages and washing out a lot of root nests from others.

Novoslovnica in a row with other Slavic auxiliary language projects try to create something that Slavs could use to struggle the globalization process.

This book is for those who is interested in such an idea of a Slavic language reunion. Moreover, it reveals some historic features of our languages.

Hope you will enjoy reading this book and get something important for yourself when you finish it.

Novoslovnica is a standardized language project with strict rules in it that make the language able to be codified. Hope this will be useful for the interslavic community in developing the bridge between Slavic languages.

Whatever would be, anything you find useful in this book will make it worth writing, my dear reader.

You are welcome to write me all your wishes, comments and marks via email: support@novoslovnica.com

Best wishes,

George Carpow
\tableofcontents
\mainmatter
\printglossary
\chapter{Background}

I would like the book to be started with the following article \cite{intro}.

The Slavic world has a deep and rich history. The tribes of the Slavs appeared in the vastness of Europe, and during its history of development have reached an extraordinary territorial vastness of settlement. We can say that the formed tribes of the Slavs as a result of settlement lived in a huge territory by the standards of Europe: from the Elbe in the West to the Volga in the East, from the Aegean sea in the South to the Neva river in the North. Further, thanks to the development of Kievan Rus and the Russian state, the Slavs settled throughout Eastern Europe and Siberia. Despite all the contradictions in politics and strife, the cultural and linguistic component of the Slavs continued to be preserved. Of course, the tribes bordering on other Nations have experienced enormous pressure of culture and language, including Turkic, Roman, Finno-Ugric peoples.

Despite this, the Slavs retained their cultural identity and mentality. So, we can easily understand a person from the same group (Eastern, Western, southern) and with little difficulty can establish communication with a person from another group of Slavs. This gives us the right to talk about the brotherhood of the Slavic peoples.

Many of the Slavic peoples have sunk into Oblivion, some in very recent times. For example, the POLAB culture left with the language by the middle of the 20th century. However, most of them remained alive, and from our common efforts depends on what will be the future of the Slavs.

From modern Slavic peoples only East Slavic group had relative stability and self-sufficiency. Russian experienced several waves of foreign language intervention — Mongolian, German, French and English, from which he, generally speaking, came out the winner, in many cases enriching and diversifying its vocabulary. At the moment, the English intervention, the strongest since the Tatar-Mongolian yoke, which has withdrawn many native Russian words from our everyday life, has not ended. But Slavs do it, assimilating came, and sometimes returning lost.

The situation is different in other Slavic groups. Not possessing constant sovereign statehood, Western and Eastern Slavs were subject to constant influence other peoples. Thus, the intervention of the Turkish language in the South Slavic languages and culture can be correlated with the Tatar-Mongolian yoke for Russia. After liberation from Turks, southern Slavs have become to experience Western influence, and we can to see enough a large number of borrowing from English, French and camping on p. in Bulgarian and Serbian languages.

Throughout its history, Western Slavs were constantly terrorized by the influence of Romanesque languages. So, under the onslaught of the Germans, the Slavs lost their land along the Elbe and retreated to the Oder. Thus, the influence of the Romanesque tribes was also facilitated by the adoption of the Latin alphabet by the languages of the Western Slavs, instead of a possible Glagolitic or Cyrillic. Anyway, Slavs, though, and have undergone numerous impacts on their culture, managed to keep a common language to the present day.

The problem of reconsolidation of the Slavic peoples worried all many centuries ago and attempts were made to implement it. This can be considered the creation of Kievan Rus, the Grand Duchy of Lithuania, the expansion of the Russian Empire to the West, the creation of friendly relations with the Balkans, the creation of ATS, Yugoslavia, Czechoslovakia. However, the problem of the impossibility or short-lived nature of this Union arose not because of the establishment of friendly, fraternal relations, but the prevalence of the interests of one nation over all the other members of the Union, which inevitably led to the collapse and failure.

There is a fair idea that the inability to undertake such an Association is not only in the political interests of any country, Yes, undoubtedly, but also in the language of communication, state or Union language. It is impossible to make official all languages and adverbs existing in any district, and declaring one language more important than another we want this or not we turn ourselves to the further collapse of such relations.

The necessity and importance of the creation of the Slavic (pan-Slavic) language was realized in the 16th century, when the first attempts were made to create a common language for all Slavs. However, they did not bring the expected results and sank into oblivion.

Most recent developments of pan-Slavic languages are based on the principle of simplification, minimizing all vocabulary and grammar of languages. For example, these Words and Mezhduslovnyh English Wuhan. We believe this is fundamentally wrong, as the new language should not be aimed at the degradation of cultures, but rather to maintain the grandeur and beauty of Slavic culture and identity. Therefore, our principle is not to "Discard all that is foreign", but to "Unite all the best".

If we analyze why the Slavs need a new Slavic language, we can distinguish several positions:

\begin{itemize}
	\item the need for an additional consolidating factor to unite the policy of the Slavic peoples.
	\item The need to strengthen mutual understanding and improve the Slavs ' perception of each other.
	\item Preserve the original richness and beauty of all Slavic languages.
	\item The impossibility of taking the role of the Slavic language of Russian (and Polish) due to historical reasons.
	\item Inability to accept the role of the Slavic language of the previous planned projects in full due to objective reasons.
\end{itemize}

\section{Historical background}

One of the Central ideas of pan-Slavism is the creation of the Slavic language. In one form or another, this problem was solved many times and in different ways, starting with the old Church Slavonic language Svv. brothers Constantine-Cyril and Methodius. With the advent of the first common Slavic language is associated with the appearance of the first writing, which was to become common Slavic — Glagolitic.

But over time, the part of the Slavs adopted the Cyrillic alphabet, the piece took Latin, and some in course of some time was the Glagolitic alphabet, the Latin alphabet and a modified Cyrillic alphabet, known as bosancica or horvatich (there are other names). Glagolitic also changed, becoming from rounded "Bulgarian" angular "Croatian". But the main thing — changed the language itself. Once a single drevnearmyansky began to share on Harry, and Harry began to converge with the living Slavic languages. Of the harass is possible, for example, to call: slavyanorusskogo, Slav, slavyanoserbsky, slavyanobolgarskaya and other.

Attempts to solve the problem of separation of Slavic languages and scripts gave several options:

\begin{itemize}
	\item To revive the Church in one way or another, but adding to it elements of living languages or without adding.
	\item Take one of the living languages, but with certain modifications — the addition of letters, words from other Slavic languages, grammatical structures and so on.
	\item As in the previous paragraph, but do not change anything, and try to spread the living language through educational courses and other forms of education.
	\item As in the previous point, but forcibly.
\end{itemize}

Create a new artificial / semi-artificial language based on those elements of the Slavic languages that remain common or can be relatively easily reduced to common + a number of different elements, to facilitate the "entry" into the language of the Slavs who speak different Slavic languages, or on the basis of only common elements.

The second part of the problem is writing. Due to the fact that the Glagolitic alphabet and bosancica virtually disappeared and remained only in limited use (bosancica disappeared almost completely, and supported the Glagolitic alphabet in Croatia), the remaining candidates for the Slavic alphabet is Cyrillic and Latin.
In General it can be said that novoalekseevsky projects often use the Latin alphabet that cuts obsessiveness projects such as Cyrillic and Latin alphabets still are native to Slavic writing systems: in one case, a literature began almost immediately with the Latin alphabet, the other with the Cyrillic alphabet, somewhere in the course of the Cyrillic and the Latin alphabet from the Cyrillic alphabet there was a transition to the Latin alphabet.

In Yugoslavia, Cyrillic and Latin tried to lead to a common denominator through the creation of the so-called Slavica. The essence of Slavica was to leave the backbone of the Latin alphabet (all the basic letters + letters, the same for the Latin and Cyrillic), replace all the digraphs, letters with diacritics and digraphs with diacritics corresponding Cyrillic letters. In Yugoslavia, this was all the easier to do because the Yugoslav Cyrillic and Latin alphabet are completely identical in composition, being mutual transliteration of each other. However, because of the resistance of nationalists in Croatia, this project failed.

\section{Modern background}

We all know that the idea of a common language for the Slavs hovers among the Slavic peoples for more than a Millennium. The first and only successful project to date was the Church Slavonic language of Cyril and Methodius. This language has successfully found its niche in religious use and is used with some changes to the present day. However, such attempts for the secular language were resumed only in the 17th century in the works \cite{krizhanich} \cite{matija}. Moreover, the working project was neither created nor introduced into society, while some similar projects of planned languages for romance languages have successfully taken root and found a wide response in society (Esperanto, Interlingua).

The revival of the idea of the inter-Slavic language in our days occurred in 1999, when mark Gucco from Slovakia created the language of Slovio. This language has a huge number of shortcomings and is currently not recognized as capable by any of the developers of the inter-Slavic project, but it served as a catalyst for the emergence of a large number of such projects and the development of the idea of creating a planned inter-Slavic language that could meet the needs of modern society. 

After that, more than 20 similar projects, more or less developed, appeared in 10 years. In 2006 there was a project slavianski, created (Ondrej Rečnik and Gabriel Svoboda). This language was developed in parallel in different degrees of detail, he put his ideas Jan van Steenbergen and Igor Polyakov. In 2009, the project broke away from the project Slovioski (Steeven Radzikowski, Andrej Moraczewski and Michal Borovička), which tried to unite the ideas of Slovio and Slovyanski. However, in 2010, all these projects were merged into one under the name Interslavic (see below - Interslavic-2).

At the same time, the Czech programmer Vojtech Merunka published under the name of Neoslavonic. This project suggested the idea of how the Church Slavonic language could develop in free development. In 2011, these two projects began cooperative work on a common case. (see figure 1) then, this project has taken a leadership position on the issue of building medullablastoma language. Only in 2012 there was one project of “Northern Slavic language” (Venedčyna), presented by Nikolai Kuznetsov, who then joined Interslavic, and in 2014 there was a project of Novoslovnitsa, presented by George Carpow and the development team mainly from the countries of Eastern and southern Slavs, which at the moment remains the only living independent project outside the project Interslavic.

Interslavic introduced the idea of favoritely, which resulted in a valid language differentiation of dialects and spellings and the lack of codification.  In 2017, the project Neoslavonic and Interslavic merged into a new project called interslavic-2, which was a compromise between the ideas of Maranki Vojtech and Jan van Steenbergen (who headed the project interslavic-1). They presented their new ideas in the article \cite{interslavic-2}, which was published in July 2017.

\chapter{Phonology}

Phonology\index{phonology} - a branch of linguistics concerned with the systematic organization of sounds in the language. Phonology describes all the sounds that the language possesses.

Sounds can be divided into different classes. One of the characteristics for separating different sounds is the ability to pronounce them with an open vocal tract. You can notice that vowels (such as \textit{a, o, u} in English) are able to be pronounced with open vocal tract, whereas consonants are pronounced with partly or completely closed vocal tract.

Moreover the term of allophone should be concerned before going further.

\textit{Allophone\index{allophone}} - one of a set of multiple possible spoken sounds or signs used to pronounce a single phoneme in a particular language.

This leads to the definition of a phoneme:

A \textit{phoneme\index{phoneme}} is one of the units of sound that distinguish one word from another in a particular language.

Therefore, an allophone is such a sound (or a set of sounds) that does not influence the meaning of the word. 

\section{Vowels}

In the beginning of this paragraph the description of a vowel should be given.

\newglossaryentry{vow}{name=V, description={Vowel}}

A \textit{vowel\index{vowel} (\gls{vow})} - a sound produced with no constriction in the vocal tract.

With this information we can distinguish different types of vowels. The classification of vowels is based on two main factors:

\begin{itemize}
	\item{What is the row of the sound}
	\item{What is the height of the sound}
	\item{Whether the vowel is rounded or not}
\end{itemize}

The \textbf{row\index{vowel!row}} is the position of the tongue when you pronounce a vowel. There are three main rows: front, central and back. When you pronounce a vowel at the front row, you move your tongue toward the teeth. The descriptions of central and back rows are similar - you move your tongue to the center or toward the larynx to pronounce them.

The \textbf{height\index{vowel!height}} of the sound is a characteristic of tongue convexity and tension in your mouth. If it is positive, it means your tongue does not touches the palate nor bottom of the mouth, and the tip of your tongue is tense- or closed. If your tongue is flat and parallel to the bottom of your oral cavity, moreover, it lies on it - it is an open one. Between these two positions a middle position can be found.

The \textbf{roundedness\index{vowel!roundedness}} of the vowel is the amount of rounding in the lips during the articulation of a vowel. Vowels can be categorized as rounded and unrounded. Thus, to pronounce a rounded vowel you should round your lips. To pronounce an unrounded vowel, you should relax your lips during the articulation of a vowel.

Finally, we can talk about Novoslovnica phonology. Novo-slovnica consists of 20 ordinary vowel phonemes. Among them there are seven closed vowels, three open vowels, ten middle vowels as well as seven front vowels, five center vowels and eight back vowels. On the table 1.1 you can see a chart position of the vowels in Novoslovnica.

\begin{table}[h]
\caption{Vowels in Novoslovnica}
	\begin{tabular}{lllll}
		& Front & Center & Near-back & Back \\
		Close      &  \textipa{i}          & \textipa{1|\textbf{0}} &                      &  \textipa{W|\textbf{u}} \\
		Near-close &  \textipa{\textbf{I}} &                        & \textipa{\textbf{U}} &                         \\
		Close-mid  &  \textipa{e}          & \textipa{\textbf{8}}   &                      &   \textipa{\textbf{o}}         \\
		Mid        &  \textipa{\|`e}       & \textipa{@}            &                      &  \textipa{\textbf{\|`o}} \\
		Open-mid   &  \textipa{E}          & \textipa{3}            &                      &   \textipa{2|\textbf{O}}    \\
		Near-open  &  \textipa{\ae}        &                        &                      &                 \\
		Open       &  \textipa{a}          &                        &                      &   \textipa{A} 
	\end{tabular}
\end{table}

In this table you can also see different vowels are font-styled differently. The bold ones are for rounded vowels. The normal ones are for unrounded vowels.

If you know Czech or Finnish, you might be concerned by the absence long vowels in this chart. It’s time to speak about allophones in Novoslovnica.

Novoslovnica has allophones of open and long vowels. This means that it does not matter how you pronounce a “modified” vowel - as a long one or as an open one - the meaning of the word will not change. To make this more clear, look at table 1.2.

\begin{table}[h]
	\caption{Long vowel allophones in Novoslovnica}
	\begin{tabular}{lll}
  \multirow{2}{*}{Main vowel} & \multicolumn{2}{c}{Modified vowel} \\
  	&	First allophone & Second allophone \\
  \textipa{E} & \textipa{E:} & \textipa{3} \\
  \textipa{a} & \textipa{a:} & \textipa{A} \\
  \textipa{u} & \textipa{u:} & \textipa{U} \\
  \textipa{o} & \textipa{o:} & \textipa{O} \\
  \textipa{\|`o} & - & \textipa{W}
	\end{tabular}
\end{table}

\textbf{Exception:}

\textipa{\v*{o}} (only one modified vowel - \textipa{W})

\textbf{Examples:}

\textit{Buda} \textipa{[‘buda]} (Buddha)

\textit{búda} \textipa{[‘bu:da]} (building) - \textipa{[‘bUda]}

If you have some knowledge about Slavic languages and their origins, you should know that the Proto-Slavic language had nasal vowels, which we can nowadays be found in Polish and Lithuanian. Do they exist in Novoslovnica? Of course they do! There are two allophones for pronouncing nasal vowels. The first one is actually a nasal vowel, when you pronounce an ordinary vowel through your nose. The second is more common, such as in French, when you add a nasal consonant \textipa{[N]} to an ordinary vowel. Look at the sounds on the next table.

\begin{table}[h]
	\caption{Nasal vowel allophones in Novoslovnica}
	\begin{tabular}{lll}
  \multirow{2}{*}{Ordinary vowel} & \multicolumn{2}{c}{Nasal vowel} \\
  &	First allophone & Second allophone \\
  \textipa{E} & \textipa{\~E} & \textipa{EN} \\
  \textipa{i} & \textipa{\~E} & \textipa{iN} \\
  \textipa{a} & \textipa{\~a} & \textipa{aN} \\
  \textipa{u} & \textipa{\~u} & \textipa{uN} \\
  \textipa{o} & \textipa{\~o} & \textipa{oN}
	\end{tabular}
\end{table}

\textbf{Examples:}

\textit{Dųb} (oak) \textipa{[d\~{O}b]} - \textipa{[duNb]}

\textit{Męso} (meet)\textipa{[‘m\~{E}so]} - \textipa{[`meNso]}

As you can see, Novoslovnica distinguishes between nasal vowels in two categories - O-nasality (hard) and E-nasality (soft). 

The last topic that we will speak of pertaining to vowels, is the firmness and the softness of the vowels. Scientists argue about what is primary in making sound soft or hard - consonants or vowels. Novoslovnica claim vowels are softer or hard primarily, although consonants itself also can be either soft or hard.

There are vowels that tend to make their surrounding soft and there are vowels that tend to make their surrounding hard. As you already know, there are two nasal vowels - one hard (O-nasality) and one soft (E-nasality). However, there are other pairs of hard-soft vowels among ordinary vowels. Look at table 1.4 for more information.


\begin{table}
	\caption{Hard and soft vowels}
	\begin{tabular}{ll}
		Hard vowel & Soft vowel \\
		\textipa{\~o} & \textipa{\~E} \\
		\textipa{u} & \textipa{0} \\
		\textipa{1} & \textipa{i} \\
		\textipa{E} & \textipa{\|`e} \\
		\textipa{o} & \textipa{8} \\
		\textipa{a} & \textipa{\ae} \\
		\textipa{I} & \textipa{e}
	\end{tabular}
\end{table}

\textbf{Examples:}

\textit{Dodatek} (addition) [do’datek]

\textit{Smërtj} (death) \textipa{[sm\t{e}rt’]}

It should be mentioned that the sounds \textsc and e that are shown in italic are the allophones for one phoneme. However, both these sounds exist in Novoslovnica and you can use whatever you like.

There is only one vowel that has no pair in softness/hardness. It is \textipa{@}. This vowel is named the “Schwa” sound and it can be described as the most “middle” sound among vowels. To pronounce it you should relax your oral cavity and pronounce a sound with weakened muscles. This is the “Schwa” sound.




\section{Consonants}

In this paragraph about consonants, I would like to begin with the definition of a consonant.

\newglossaryentry{cons}{name=C, description={Consonant}}

\textit{Consonant\index{consonant} (\gls{cons})} - a sound that is articulated with complete or particular closure of the vocal tract. 

Likewise vowels, consonants have three characteristics that determine their position in your articulation. These three parameters are:

\begin{itemize}
	\item{Place, where the consonant is pronounced in the mouth}
	\item{Manner, how the consonant is pronounced }
	\item{Sonority, whether you use your vocal cords or not}
\end{itemize}

\textbf{Place\index{consonant!place}} of the consonant can be quite different. Here are possible types: \textit{Bilabial, labiodental, dental, alveolar, postalveolar, palato-alveolar, retroflex, alveolo-palatal, palatal, labio-velar, velar}. There are more types, but they do not exist in Novoslovnica.

\textbf{Manner\index{consonant!manner}} is the way how you pronounce the sound. There are also different manners, that are used in Novoslovnica. They are: \textit{nasal, stop, affricate (sibilant), sibilant fricative, non-sibilant fricative, approximant, trill, lateral approximant}.

\textbf{Sonority\index{consonant!sonority}} is the boolean attribute of pronunciation. You can either use your voice \textit{with} the sound you pronounce or \textit{not}. Notice that vowels cannot be pronounced without the use of your voice. 

Combining these three parameters, we get the unique consonant that we want to pronounce. We cannot draw  a 3-dimensional table, because there are three parameters on input, so we will combine information into 2-dimensional space as in paragraph about vowels. So, look at the figure and see the different consonants that are used in Novoslovnica.

\begin{figure}[h]
	\includegraphics[width=\linewidth]{./sources/consonants.png}
	\caption{Consonant sounds in Novoslovnica}
	\label{fig:consonants}
\end{figure}

Different colors of the cells show the sonority of the consonant. Yellow color shows that the sound is voiced, while green ones are for voiceless sounds.

Blue cells in the table show that sounds in it can be used both in voiced and voiceless forms as allophones.

Novoslovnica has 51 consonants, 21 of them are voiceless and 30 are voiced.

However, not all of these consonants are language phonemes. So, let’s talk about the allophones among these sounds.

\begin{table}[h]
	\caption{Consonant allophones in Novoslivnica }
	\begin{tabular}{ll}
		Main consonant & Allophones \\
		\textipa{\t{\:d\textctz}} & \textipa{\t{\:dZ}} \\
		\textipa{\t{tC}}  & \textipa{\t{tS}} \\
		 \textipa{Z} & \textipa{\textctz}  \\
		  \textipa{S}  &  \textipa{C} \\
		  \textipa{l}   &  \textipa{(\r*l} \\
		   \textipa{r} (voiced)  &  \textipa{r} (voiceless) \\
		   \textipa{n} (alveolar) & \textipa{n} (palato-alveolar), \textipa{n} (dental) \\
		    \textipa{(\r*r} (voiced) & \textipa{(\r*r} (voiceless) \\
		    \textipa{v} & \textipa{V} \\
		    \textipa{f} & \textipa{\r*V} \\
		    \textipa{H} & \textipa{G} \\
		    \textipa{x} & \textipa{h} \\
	\end{tabular}
\end{table}

Likewise vowels, consonants can be compared with each other in terms of softness/hardness. The common rule is that every consonant has its soft or hard partner.

Exception: Three sounds are exceptions to this rule.
The sound \textipa{T} has no pair, because its pair ð has never been used in Slavic languages.
Also nasal velar consonant \textipa{N} has no soft pair.
And vice versa, the sound j is soft and has no hard pair. 

Remember this exception, let’s look at the table 1.7 to get acquainted with pairs of consonants.

\begin{longtable}{ll}
%	\caption{Softness in Novoslivnica}
		Hard consonant & Soft consonant \\
		\endhead 
		b & bj \\
		v & vj \\
		g & gj \\
		\textipa{H} & \textipa{Hj} \\
		d & \textipa{J} \\
		\textipa{\t{\:d\:z}} & \textipa{\t{\:d\textctz}} \\
		\textipa{\t{dz}} & \textipa{\t{dzj}} \\
		\textipa{\:z} & \textipa{Z}  \\
		z & zj \\
		k & kj \\
		l & \textipa{L} \\
		m & mj \\
		n & \textltailn \\ 
		p & pj \\
		r & rj \\
		\textipa{\r*r} & \textipa{\r*rj} \\
		s & sj \\
		t & \textipa{C} \\
		f & fj \\
		x & xj \\
		\textipa{\t{ts}} & \textipa{\t{tsj}} \\
		\textipa{\t{t\:s}}  & \textipa{\t{tC}} \\
		\textipa{\:s} & \textipa{S} \\
		w & \textvibyy \\
\end{longtable}

As you see, every consonant from table has its own soft pair. The soft consonant is usually written as a hard consonant with the sound j attached, but some of them are provided as unique sounds by IPA, because in some languages they can be phonemes. (footnote)


% любопытным следтвием наличия особых значков в IPA для мягких были попытки построения чисто фонетических алфавитов на основе только символов IPA, которые должны были использоваться в качестве букв. (дополнение можно внести в текст книги).
\section{Vowels}

By reading this paragraph you should be aware about Novoslovnica phonemes. At the beginning we will speak about two main features of the language: reproduction and alternation.

Reproduction (extra sounds) - is a process of adding a new sound (consonant or vowel) in the word.

Alternation - is a process of changing some sound (consonant or vowel) or sounds to another one(s).

What are the purposes of alternation and reproduction? These terms both obtain the single aim - to make our speech comfortable for ourselves.

Every language has its own comfortable combinations of sounds and avoided ones. Some languages alternate these uncomfortable character rows into another ones that are comfortable for language speaker in writing, some languages keep writing etymological and alternate the pronunciation of the uncomfortable words. Take into account the fact, that the concept of comfort is relative, that means two different languages (which are spoken by different nations) have different modes of comfort.
Example: you can look at and compare German and Dutch languages. They are very close, but they have different ways for comfortable pronunciation. The word “book” in German sounds like “Buch”, but in Holland it sounds like “boek”. Thus, you see that these words are familiar, but their pronunciation differs a lot.

Moreover, even one language spoken in different areas can differ greatly in the areas it is spoken.
Example: American and British English. You know that it is still the same language, but pronunciation differs so greatly that American person can not always understand in ordinary speech (quick temp of speaking) what the British person have said.

This is the example how the same language differs in areas it is spoken. Furthermore, these areas may not be far away from each other. You can find information about different patois and even dialects in very small regions. 

Novoslovnica as a panslavic language absorbs different conceptions of the comfort term of Slavic languages.

When reproduction is used, we add a new sound before the word we want to say. It can be whether a vowel or a consonant, depending on the previous letter in the word.

When should we reproduce a sound? To deal with this problem, you should know a thesis, which is widely used in Novoslovnica.

\textbf{Rule №2}: After consonant there should be placed a vowel. And after a vowel there should be placed a consonant.

This rule will help you in speaking and writing, when you construct your words with the reproduction.

There is a limited row of sounds that are allowed to participate in reproduction. In the table 2.1 you can see all of them.





\section{Alternation}

Alternation is a very important feature of Slavic languages. All of them provide some cases, when one letter changes to another one(s) and controversially. The cause is the fact that some sound combinations are difficult to be spoken or not comfortable for that. For example, Slavs can say “napisati” (to write), but “napi???at” (they (will) write). The variant “napisat” is not comfortable to pronounce and, moreover, it is not understandable. What are you speaking - “to write” with hardened T or “they write” with depalatalized S. So you can see, that these rules are often experimental and cannot be explained in a common way.

Alternation can be found mostly in conjugation or declension of the word, because the process is in changing uncomfortable forms to better ones while changing the base and endings of the word. 

In this paragraph I will speak about different palatalizations of vowels and consonants. Let’s begin with the consonants.

\textbf{Alternation S//Š}

This alteration appears in words with the letter S before a vowel A. The basic sound is S, which changes to Š before vowels I, E and in some cases before A in conjugation. Let’s look at examples to understand how it works.

\underline{Examples:}

\textit{Pisati} (write) \textipa{[’pisat1]} - \textit{piši} \textipa{[pi’\:s1]}

\textbf{Alternation K//Č}

This alteration appears in words with the letter K before a vowel A without a following vowel. The basic sound is K, which changes to Č before vowels Ě, N. Let us look at examples to understand how it works.

\underline{Examples:}

\textit{Věk} \textipa{[vIk]} - \textit{věčen} \textipa{['vI\t{tS}en]}

\textbf{Alternation C//Č}

This alteration appears in words with the letter C before a vowel A. The basic sound is C, which changes to Č before vowels I, E. Let us look at examples.

\underline{Examples:}

Ptica (bird) \textipa{[’pti\t{ts}a]} - ptičen (birdish) \textipa{[’pti\t{tS}en]}


\textbf{Alternation D//Đ}

This one appears in words with the letter D before a vowel I. This consonant changes to Đ before vowels A, E and U. Let us look at the examples.

\underline{Examples:}

\textit{Voditi} (drive) \textipa{[‘vodit1]} - \textit{vođati} \textipa{[‘vo\texttoptiebar{\textctdctzlig}at1]}

\textbf{Alternation Ǐ//J}

This alternation is very simple. We write Ǐ before a consonant or in the end of the word and we write J before a vowel. The exception is the case, when we write Ǐ in the end of the word, but the first letter of the next word is a vowel - then we pronounce J, but write Ǐ.

\underline{Examples:}

\textit{Môǐ} (my) \textipa{[mUj]} - \textit{moja} \textipa{[m2'Ja]}

However, not only consonants can change in the word when we conjugate or decline it. There are some alterations of vowels too.

\textbf{Alternation Ò//-}

This alternation appears in some old roots (see next paragraph).

\underline{Examples:}

\textit{Tòk} (stick) \textipa{[t@k]} - \textit{tkati} \textipa{['tkatI]} (to weave)


\textbf{Alternation E//-}

This alternation is similar to the previous ones, but exist in word suffixes.

\underline{Examples:}

\textit{Krasen} (Nice) \textipa{['kras@n]} - \textit{krasna} \textipa{['krasna]}

% \textbf{Alternation O//Ö}

% This alteration is used in some old roots (see next paragraph).

% \underline{Examples:}

% Ïdiot (idiot) \textipa{[id1’ot]} - ïdijot \textipa{[id1’Jot]}

% P.S. What are the differences between these two alternations? The variant which we choose depends on the exact word and its root. These roots ascend to proto-Slavic roots with vowels “Ò” and “J” (see next paragraph), that’s why they transforms to “-” and “??” respectively.

% \textbf{Alternation Ö//J}

\textbf{Alternation Ę//En}

This alternation is rather narrow, because it is used in the case of declension nouns of type 3 (look paragraph about noun declension), when the nasal vowel ??? alternates to non-nasal pair of vowel “E” and consonant “N” (non-nasal!). To understand it look at the example

\textbf{Examples:}

\textit{Vremę} (time) \textipa{[‘vrEmj\~E]} - \textit{vremenï} \textipa{[‘vrEmE\textltailn i]}

\textit{Plemę} (tribe) \textipa{[‘plEmj\~E]} - \textit{plemenï} \textipa{[‘plEmE\textltailn i]}

In the conclusion of this paragraph it should be mentioned that alterations are very important in Slavic languages and Novoslovnica as well. You can use reproduction in your speech as a recommended feature, while alterations are complimentary in this language. As it was noted before, we cannot ignore anything that can bring a misunderstanding in our speech. 

\section{Runaway vowels}

Looking back in the Slavic language history we can find out that there were roots with two strange for an ordinary person vowels - “Ò” and “J”. First one was named “Yer” and denotes a hard mid central vowel (Shwa). The second one was named “Yerj” (with soft R) and denotes a soft mid central vowel. Now the second one is lost and we use only shwa sound in the letter “Yer”. However, the words are still and we need to pronounce them in some way. Novoslovnica uses the soft “E” sound to represent roots with old soft shwa sound.

Main feature of these sounds was to fall out from the root, when a vowel appears afterwards. That’s why there are many words with two consonants consecutively - there is an imaginary shwa sound between them that has been fallen out from the root.

Nevertheless, despite falling out of “Yer”, soft “E” in this places does not fall out. So, in the previous paragraph you could see that there are two alternations O//- and O//Ë, that are handled in the similar positions. So the answer on the question, why in the first case there is no sound and in the second there is a soft E is the fact, that words satisfying the first case comprise old hard shwa sound and remain comprise old soft shwa sound, that has transformed into soft E.

I should mention also the fact, that nowadays the letter Ò exists only in roots of the words. In suffixes the letter E is used for this sound and in the prefixes the letter O is used. Look at the examples:
pod
ek
pòk  

You should remember that speaking the words with these letters we should pronounce them just as they are written - Ò as shwa, E as E, O as O. You should not reduce all the sounds to shwa. 

\section{Accent}

Accent is a very difficult topic in most languages, because it is not permanent. There are some exceptions i.e. Czech and French, but in most cases we cannot say where accent will be put in the word without studying definite language. This causes problems for beginners.

Novoslovnica has a dynamic accent, but it has been formalized. There is a rule, that determines the place you should put the accent. 

Rule \#2: The accent should be put on the first syllable of the word root.

This rule covers about 80\% of the words in the lexicon. You know the well-known 80/20 rule, or Pareto principle. It is something alike with the accent. Remained 20 per cent of the words we should cover by introducing extra-rule cases. Accepting these cases, you will be able to cover more than 99 per cent of Novoslovnica word amount.

These cases were created in attempt to unify the accents in different Slavic languages. Surely the Slavic languages have greatly changed since then, as they were one language. Therefore, accents in different Slavic languages often differ. Nevertheless, Novoslovnica tries to obliterate differences between them, producing accent patterns that could be comfortable to pronounce and to hear for all Slavs.

Below you can see the list of all these cases, that you should remember while speaking Novoslovnica.

Accent shifting cases:
Accented endings (Nouns)
-a (Dual, Nominative)
-y (Singular, Genitive/Partitive)
-ami (Plural, Instrumentative)
-ama (Dual, Dative)
-am (Plural, Dative)
Accented endings (Verbs)
“-I” Imperative (see paragraph about verb moods)
Accented suffixes
-ova- (Verb)
-ôva- (Verb)
-ava- (Verb)
-óv- (Adjective)
verb suffixes in Present Concrete Tense (see paragraph about verb tenses)
Accent shift in the root
If the word is a borrowed one, then the accent is put on the place it is in the original word.
If the root loses its vowel, the accents moves one vowel to the left (if it is possible)
Words that have more than one root (complex words) have their accent on the first syllable of the main word’s root. (see paragraph about complex words for what root is main) 
Adverbs or other parts of speech, formed with the prepositional construction, have their accents on the first syllable of the main word (see paragraph about collocations)

These rules are enough for you to speak Novoslovnica properly with a few efforts for it. 

\section{Accent Integrity}

There is also one term, that you should to know when you use Novoslovnica or any other Slavic language in your speech. This term is called accent integrity. Firstly I will introduce a term of a dependency structure:

\textit{Dependency structure} is a prepositional construction or a collocation.

That means, that this abstract term expands the term of collocation by involving prepositional into itself. 

\textit{Accent integrity} is a property of a dependency structure to unify elements of this structure with only one accent on the main word.

What does this definition say? If we have a collocation or a prepositional construction and we want to pronounce it, we should pronounce dependent words without any accent and put an accent on the main word of the structure. Look at the examples.

Examples:


d

Surely, you should remember that this might be applied only for brief structures, most often with one dependent word (or a preposition) of one or two syllables. If we have a long dependent word or there are too many dependent words in the construction, we pronounce them with a proper accent on each of the structure elements.

Examples:


d



\chapter{Orthography}

\begin{figure}
	\includegraphics[width=\linewidth]{./sources/alphabet.jpg}
	\caption{Alphabet of Novoslovnica}
	\label{fig:alphabet}
\end{figure}

\section{Alphabet}

Let’s summarize what we have known about Novoslovnica phonology. Afterwards we will get the list of phonemes and allophones and their connections with Novoslovnica letters in the alphabet.

We have learned that Novoslovnica has 51 consonant sounds and 22 vowels. 13 consonants and 4 vowels are allophones among them. Hence, the amount of phonemes is (51-13) + (22 - 4) = 56 phonemes.

You should know that in Novoslovnica, soft and hard consonants do not differ in writing. That is because of the fact that by the combination of “consonant + vowel” we can always determinedly get what the consonant is like - hard or soft. With this information, the amount of letters needed is reduced to 47.

Nevertheless, let’s now look at table 1.8 with the alphabet list and see how Novoslovnica is written.

% Table
% Note
\section{Pronunciation}

Novoslovnica is a phonetic language, that’s why Novoslovnica has an important rule, which you have to apply to speaking in Novoslovnica.

Rule \#1: All words are pronounced as they are written.

This rule means that you cannot reduce sounds when speak in Novoslovnica. It is a very important thing because you can make mistakes if you speak improperly. There are some exceptions but they all will be mentioned in this guidebook.

When you pronounce a word you are not restricted to use only main sounds - if it’s more comfortable, you can pronounce allophones with the same level of softness and sonority with the main sound of the letter. Let’s look at the examples below to understand what we can choose in speaking and what we cannot.
Examples:
jsdasd
asdasd

However, you are restricted in what consonant sounds to use from the allophone list. You can see the next rule which will help you to speak.

Rule \#2: You cannot mess soft and hard consonants when you pronounce a word.

This prohibit you to make hard consonants when you need a soft one, or to use a soft one when you need a hard one. Here you can see a table, where it is shown which sound you must pronounce in different combinations of letters.

\begin{longtable}{lllll}
%	\begin{tabular}{lllll}
		Letter & Next letter & IPA & Examples & Translations \\
		\endhead
		b & a o e i u y ų  & [b] && \\
		b & ä ö ë ï ü & [bj] && \\
		v & a o e i u y ų & [v] && \\
		v & ä ö ë ï ü & [vj] && \\
		g & a o e i u y ų  & && \\
		g & ä ö ë ï ü &&& \\
		ĝ & a o e i u y ų  & [g] && \\
		ĝ & ä ö ë ï ü & [gj] && \\
		d & a o e i u y ų & [d] && \\
		d & ä ö ë ï ü &&& \\	
		đ & a o e i u y ų  &&& \\
		đ & ä ö ë ï ü &&& \\
		ŝ & a o e i u y ų  &&& \\
		ŝ & ä ö ë ï ü &&& \\
		k & a o e i u y ų &&& \\  
		k & ä ö ë ï ü  &&& \\ 
		l & a o e i u y ų  &&& \\  
		l & ä ö ë ï ü  &&& \\ 
		m & a o e i u y ų  &&& \\  
		m & ä ö ë ï ü  &&& \\
		n & a o e i u y ų  &&& \\  
		n & ä ö ë ï ü  &&& \\
		p & a o e i u y ų &&& \\  
		p & ä ö ë ï ü  &&& \\ 
		r & a o e i u y ų  &&& \\  
		r & ä ö ë ï ü  &&& \\
		ř & a o e i u y ų  &&& \\  
		ř & ä ö ë ï ü  &&& \\ 
		s & a o e i u y ų &&& \\  
		s & ä ö ë ï ü  &&& \\ 
		t & a o e i u y ų  &&& \\ 
		t & ä ö ë ï ü  &&& \\ 
		ŧ & a o e i u y ų  &&& \\  
		ŧ & ä ö ë ï ü  &&& \\ 
		f & a o e i u y ų  &&& \\  
		f & ä ö ë ï ü  &&& \\
		h & a o e i u y ų   &&& \\
		h & ä ö ë ï ü  &&& \\
		c & a o e i u y ų   &&& \\
		c & ä ö ë ï ü  &&& \\
		č & a o e i u y ų   &&& \\
		č & ä ö ë ï ü  &&& \\
		š & a o e i u y ų   &&& \\
		š & ä ö ë ï ü  &&& \\		
%	\end{tabular}
\end{longtable}

% table
%  Ь and Ј
% надо добавить, что в югославянских Ь, Й и Ј слились в Ј (в полесском только Й замещена Ј), что, впрочем имело параллекли и в глаголице (как болгарской, так и хорватской), где кроме отдельной буквы для Ь был значок "штапик/штапић" который соответствовал и Ь и Ј.
NOTE: Cyrillic has two different letters ... that have different functions - the first one defines that the previous consonant is soft (we need this in case vowel is absent) and the second defines a [\textctj] sound. Latin version that you see in the table has no such difference, so you should remember, that J means a soft symbol when you see a C-”J”-C row (where C is for “Consonant”) and means a [\textctj] sound when you see a C-”J”-V (where V is for “Vowel”), or use Cyrillic to prevent such a collision. Only in the first case consonant before J is soft while in the second one it is hard.

\section{Some features}

I and Ï

Many people will surely be confused by these two letters. They will ask whether there is the rule when we need to write the first ot the second one. However, you should look up and remember that these both letters produce different sounds. And that is the point. 

The first letter produces the soft sound “I” while the second one stands for the hard sound. But the question is, where are used such sounds? We can't list the rules for usage these letters in the root, because it is etymological issue. However, we can list some prefixes and suffixes that contain the soft or the hard letter.

Marks for writing “I”:

Prefixes 

- Iz

Suffixes

- Nic

- Nik

- Itelj

- I

Conjunction

- I

- Ili

- Či

- Li

Marks for writing “Ï”:

Prefixes

- Nïz

Suffixes

- Ïc (female animals)

\section{Latin and Cyrillic}

You can see in table 1.8 that Novoslovnica utilizes two alphabets- the Latin one and the Cyrillic one. They both are practically equal, but is there is a preferred alphabet for the language? The answer is “yes”, Cyrillic is preferable.

The reasons of choosing such a script goes back in history. There were two scripts in the beginning of Slavic writing: the Glagolitic and Cyrillic scripts. They were created to cover all the sounds that existed in that era of Slavic languages. Glagolitic script practically has no borrowed letters from other writing systems- all letters are unique.

By this case in Cyrillic and Glagolitic scripts we can find the bijective mapping between sounds (phonemes) and letters. Latin script does not provide such orthography in any Slavic language which uses the Latin script. For example:
% “ch” for [x], while “c” is for [t͡s] and h is for [ɦ]
% “sz” for [ʂ], while “s” is for [s] and “z” is for [z]

Novoslovnica provides an artificial Latin script system, where the bijective mapping has almost been achieved. The Latin alphabet can seem strange or uncomfortable to native Slavs (though it can be used rather conveniently by non-Slavs). That’s why the Glagolitic or Cyrillic script should be used primarily.

Why hasn’t the Glagolitic script been mentioned yet? The same reason that the Latin script should not be used primarily: to prevent misunderstanding. Nowadays only one in a hundred Slavs can understand the Glagolitic script because all of its letters are original. That’s why this language, which has the goal of being used on the international level, cannot use Glagolitic script as its primary script.

The only script, that satisfies all the requirements to be the primary script of this Slavic constructed language, is the Cyrillic alphabet. In this book you will find many examples in different paragraphs. First you will see a primary (Cyrillic) variant of the example in normal font and then a Latin one in grey italic font. This will help you to learn primary script of Novoslovnica quickly. Nevertheless, if we speak about exact letters or letter combinations, I will write them only in Latin for not to mess the text of the book.

Now you know the sounds and the letters which are used in Novoslovnica and you are ready to go deeper!

\chapter{Grammar}

Grammar\index{grammar} is the core of every language. Grammar comprises different themes that are united in a system of changing words enough to build sensible syntax constructions. We will look at Novoslovnica grammar from the point of the parts of speech.

\newglossaryentry{pos}{name=POS, description={Part of speech}}

\textit{Part of speech\index{part of speech} (\gls{pos})} - is a category of words which have similar grammatical properties.

Thus, we unite a group of words into parts of speech when they have a lot of similar grammar properties such as “number”, “person”, “case”, “tense” etc. In this chapter we will go through different parts of speech and study what differences they have and how we should identify each of them and combine them to be able of creating phrases and sentences. 

Parts of speech comprise two main categories - independent\index{independent POS} POS and auxiliary\index{auxiliary POS} POS. The fact the POS is independent or not refers to its semantic value. An independent POS has a full semantic value that can be used separately.  That means when you say a word of an independent POS you reproduce some semantic meaning that the interlocutor can recognize. An auxiliary POS a semantic value that is partially defined and can be distinguished only in pair with a word of an independent POS.

\textbf{Example:}

\textit{River} - this word can be recognized by interlocutor as some concept of water flowing in a restricted area.

\textit{Beautiful} - this word is recognized as an attribute of any concept that is nice, pleasant to the person (i.e. speaker or interlocutor)

\textit{Came} - this word can be recognized as any concept’s action of going to the destination point and having reached it. 

And so on. Nouns, adjectives, verbs, adverbs, participles, numerals, pronouns refer to independent POS. You see that every word has its own meaning full enough to imagine yourself some information you have received. This is not true for a word of an auxiliary POS. 

\textit{On} - this word can be recognized as a placement of an object against some horizontal surface. It can refer to some time moment (be on time). It can also be used with a fully different value (come on - to increase the activity of doing something).

\textit{For} - it can be recognized as an aim for something or an object that participates in some action. Also this word can refer to a duration of a process.

Moreover, words “and” or “to” cannot give you any map in your mind into any sensible meanings. Though these words have no determined semantic value, they are extremely important in the whole phrase connecting two words of the same POS with logical value (AND, OR, NOT). English language is an analytical one, that is why words are mostly connected with each other in the phrase by an auxiliary POS. Without them you are not able to understand what the person is speaking about. Slavic languages are fusional, however, there are enough analytic features in them, hence auxiliary parts of speech are also important. Articles, prepositions and conjunctions are referred to auxiliary parts of speech.

There are also two additional categories - particles and interjections. Some allocate them into separate categories, some claim they belong to an auxiliary category. Nevertheless, these are both separated parts of speech because they have different grammar properties.

These are all categories of POS. If we speak about an independent POS, we should take into account that there are different semantic, morphological and syntax functions can be described by it. There are several types of semantic functions: the concept, the attribute, the predicate and the demonstrative.

The \textit{concept} is something that correlates with the object or subject in the real world. It could be either abstract or imaginary, but we can ask a questions “Who? What?” about it.

The \textit{predicate} is something that determines the action, corresponding to a concept. We often ask questions “What to do?” to reveal a predicate.

The \textit{attribute} is something that is correspond to a concept or an action. We ask questions “What concept is like?” or “What action is like?” to find out the determine value of the attribute.

The \textit{demonstrative} points out the concept. It has the same question with the attribute yet has no semantic value but demonstrating the concept it corresponds with. 

Parts of speech that have properties of a concept are: \textit{nouns, cardinal numerals, verbal noun and some kinds of pronouns}.

Parts of speech that have properties of an attribute are: \textit{adjectives, participles, ordinal numerals and adverbs}.

Parts of speech that have properties of a predicate are: \textit{verbs, transgressive, and gerund}.

Parts of speech that have demonstrative properties are most kinds of \textit{pronouns}.

Thus, noun, adjective, verb, adverb, numeral, pronoun, gerund and participle are independent POSes, while preposition, particle, interjection (with Onomatopoetic), article and conjunction are auxiliary ones.

Independent parts of speech are also divided into nominal and verbal ones. It is extremely important because this division shows differences in grammar forms of nominal and verbal POS. Verb, participle, transgressive, gerund are verbal parts of speech, while noun, adjective, numeral are nominal. Adverb and pronoun stay separately - the first one because of its immutability and the second one because of its heterogenity. 

In the chapter it will be spoken about the very POS exists in Novoslovnica. The table with grammar and semantic properties of a POS will be given in the corresponding section.

First of all you should know some facts about different grammatical properties in Novoslovnica.

\section{Case}

Case is a grammatical property of a nominal POS, that shows what references this nominal POS has with other words in a sentence (phrase). This property is widely known in fusional languages, while analytical languages do not often possess this property. Thus English has only two active cases - Nominative and Oblique case. Moreover, oblique case is used practically only within pronouns while nouns have no such case. That means case is not the only way to show references between nominal POS and other words in a sentence. Case is one of the ways to show and Slavic languages as being fusional widely use this grammar category.

Different Slavic languages have different amounts of cases. For example, Russian language has six cases when Serbian language has seven. We can find exceptions in Bulgarian and Macedonian languages, which are analytical, that’s why they have only one case for a noun and adjective and three cases for a pronoun.

Different cases are referred to different semantic links between words. That’s why we see ambiguity of cases in different languages (that have different amount of cases and different usages of cases). Novoslovnica provides most common and wide means to use cases with most determination. When you speak Novoslovnica, you have to use the case of exact semantic value and not of the longstanding phraseology of your own language.

With this principle Novoslovnica establishes nine cases. Nine changing patterns that determine alterations of all words of nominal POS. Here I want to introduce them to you:

Nominative
Genitive
Partitive
Dative
Accusative
Instrumental
Prepositional
Locative
Vocative

Nominative case is used when we are talking to a concept as an actor. If the sentence is full, the subject is in Nominative. You can ask questions like “Who? What?” to it.

Genitive case is used when we are talking to an object being related to another one. Thus this case show what generation the object is of and what is it made from or to whom does it belong. The questions that determine the case are “Whose? Which? What?”

Partitive case is used when we are talking about some amount of an object having or being supposed to have uncountable properties. The questions that determine the case are “Of what? With what?”. 

Dative case is used when we are talking about a subject of perception. We can ask a question for a word in this case as “Whom? For whom?”  …. 

Accusative case is used to describe a direct object of the action. The questions determining the case are “What? Whom?”

Instrumental case is used to describe an instrument of an action that affect the object of the action. The questions related with this case are: “With whom? With what?”.

Prepositional case 

Locative case

Vocative case

These cases cover 99,99% of possible nominal POS declension.  

\section{Number}

How do people understand what is the numeric characteristic of the object? Of course, the simpliest way is to use numerals. We can call to a numeral and link it with some noun, thus, people will understand that there is an amount shown by a numeral of the concept shown by a noun. But it is rather uncomfortable to use near every noun an additional numeral. Mostly due to the fact we seldom know the exact amount of something. That is why there exists a term of number.

Number\index{number} shows what is the amount of some concept without using numeral before the noun.

Number is a grammar property of the word, its alteration. That means when we change number of the word, we change the word itself and not add some additional words or particles around the word we are speaking about.

There are three numbers in Novoslovnica: \textit{singular}, \textit{dual} and \textit{plural}. Singular\index{number!singular} and plural\index{number!plural} are familiar to an English speaker. They show whether the object is single or not. Dual is rather peculiar \footnote{Using non-boolean logic in a number property is rather specific. Triple logic (Single-Dual-Plural) can be found in some derived from Indo-European languages: Sanscrit, Slavic, though other branches lost the dual number in ancient time (Greek, Latin, German, Baltic); Arabic, Hebrew, Khoe languages also have triple number logic. Quadruple logic (Single-Dual-Triple-Plural) can be found i.e. in Tok-Pisin. Sursurunga is famous for having a five-way grammatical number distinction.}, so we should take an additional account on it.

A dual\index{number!dual} number \footnote{In modern Slavic languages Slovenian, Upper- and Lower-Sorbian languages still have a dual number.} is used when we speak about a pair of something - hands, legs, eyes etc. of one person. Two-doors gates, two boolean values, two antipodes etc. In these cases we use a dual number. Having a pair is a rather frequent fact, so this number appeared in Proto-Indo-European. Dual number so as plural one depends on the word which is spoken so we cannot determine a static rule about choosing a form for a dual number. We can get a dual form of the word by using a declension function with a type of declension corresponding with the word given as a parameter.

There also exists an additional number form we should speak about that is so-called as a counting form. A counting form is used with the nouns. It only occurs when we use the \underline{noun} \textbf{with} the numerals “\underline{two}”, “\underline{three}” or “\underline{four}” (cardinal numbers). The counting form is equal to a dual number in writing so we can speak of it as an extension of a dual number usage.

\textbf{Examples:}

\textit{Dva doma} (two houses) - counting form

\textit{Doma} (two ones) - dual number

\textit{Tri doma} (three houses) - counting form

\textit{Domy} (three ones) - plural number 

\section{Person}

This grammar category determines the person who is spoken about. There are three points of view:

\begin{itemize}
	\item the point of the speaker (First person)
	\item the point of the interlocutor (Second person)
	\item the point of other persons, that are not involved into discussion (Third person)
\end{itemize}

That is how a Slavic discussion could be seen. Practically, this concept is similar to all European languages, particularly English. There is a total equivalency with English in Novoslovnica, so it is not necessary to describe the usage all of these person types. Just look at the following examples to get sure of it:

\textbf{Examples:}

\textit{Ja glědaju v prozorec cěl věčôr.} - I am looking outside the window for the whole evening. (The first person)

\textit{Vy kažete že sámo ïsto byše včera? }- Do you mean that the very same thing was yesterday? (The second person)

\textit{Ony hlåpcy niĝda ne mogut pjiti tïho.} - Those guys never can drink quitely. (The third person)

\section{Tense}

Grammatical tense is a category that expresses time reference with reference to the moment of speaking. In many languages there are three main categories of present, past and future, that refer to the moment of speaking, the period before and the period after it respectively. English is of the same group of languages with tense-rich grammar. It has 16 tenses, divided by categories of time and perfection. Novoslovnica has 12 tenses, based on the two principles like English. They are:

\begin{itemize}
	\item Present Indefinite Tense
	\item Present Definite Tense
	\item Future Indefinite Tense
	\item Future Definite Tense
	\item Pre-future Tense
	\item Future-in-the-Past Tense
	\item Pre-future-in-the-Past Tense
	\item Aorist
	\item Perfect
	\item Imperfect
	\item Plusquamperfect
	\item Past Indefinite Tense
\end{itemize}

First two tenses describe the present moment, the next three ones refers to the future period of time and the last six ones describe the period of time that has already passed.

We can divide past tenses in two groups - past tenses itself and future-in-the-past tenses, that describe actions that refer to the future moment with reference to the moment in the past.
Novoslovnica tense system can be described better with the help of the following diagram:

\begin{figure}
	\includegraphics[width=\linewidth]{./sources/tenses.jpg}
	\caption{Tenses in Novoslovnica}
	\label{fig:tenses}
\end{figure}

Present group of tenses consists of Present Indefinite Tense (\textit{Pritomen čas}) and Present Definite Tense (\textit{Sëdyšen čas}).

Present Indefinite Tense (\textit{Pritomen čas}) is used when the action does not depend on time. For example, in the first example we know that the Earth always revolves and nothing but the apocalypsis can change it. This tense is very similar to the English one and has similar use cases. We can find this tense in indicative (examples 1, 2) and declarative (example 3, 4) moods mostly (in some cases it can be found in different moods too).

Note that in declarative there are two variants of present indefinite, because of its resultive semantics. In example 3 you can see the verb in declarative mood, in present indefinite tense, and in example 4 you can see the same tense and mood but within a predicateless sentence.

Also take care of the translation in example 3. You can mess it with English Present Perfect Tense (He has bought two cars), but this sentence should be translated with the verb «to have» in Present Indefinite with the participle III of the verb «to buy» (bought).

\textbf{Examples:}

\textit{Zemlä sę vreta okolo slånca.} - The Earth revolves around the Sun.

\textit{Vsękyǐ denj ja hođam do učilišta prez lěpyǐ park.} — Every day I walk to school through the beautiful park.

\textit{On imá kupeno dva vozidla.} — He has two cars bought.

\textit{Lěs i tïhota...} — There are forest and silence around me...

Present Definite Tense (\textit{Sëdyšen čas}) is used if the action depends on time. In the first example we can see a sentense with the following semantics: now I think about the rest, though in an hour I can forget about this thought.

Generally, this tense is close to Present Continuous Tense in English, though has a different interpretation. If you want to get in differences, we can give you such an example: «The sun is revolving around the Sun at the moment». Read this hypothetic sentence that is syntaxicly valid in English.

What would be the differences of English and Novoslovnica here? It is in a term of differentiation. English operates with the time and Novoslovnica operates with the mutability. That is why here Novoslovnica will still use Present Definite Tense with the word «sëdy» (at the moment).

\textbf{Examples:}

\textit{Ja myslü o počïvkě.} — I am thinking about the rest.

\textit{On idaje do råboty, zato ne može da odgovori vam.} — He is going to the work, that is why he cannot answer you.

The group of future tenses comprises two ones: Future Tense and Pre-future Tense.

Future Definite Tense (\textit{Bųdešt čas}) defines that the action will appear in the future with reference with the moment of speaking. There are three variants of how you can use Future Tense in Novoslovnica.

Firstly, you can use the verb «hteti» (to will) in 3-person form with the main verb in Present Indefinite Tense (example 1). This varitant is most similar to English Future Simple Tense.

Secondly, you can use the verb «byti» (to be) with inifinitive form of the main verb (example 2). This variant is close to English Future Continuous Tense. It is often used with verbs of A-type (read the chapter about verbal types in Novoslovnica).

Finally, you can use a synthetic verb form with the future tense conjugation (example 3). In English, it should be translated in Future Simple so as the first one.

\textbf{Examples:}

\textit{Ja hte kazam ti něčto.} — I will tell you something.

\textit{Ja bųdu sluhati gudbu.} — I will listen to music. (I will be listening to music).

\textit{Nakonec ja tę viđahtem}. – Finally, I will meet you.

Pre-future Tense (\textit{Predbųdešt čas}) is a grammatical tense that is used when the speaker should explicitly show that the action will take place before another future action. So, this tense deals with the comparison of future actions rather than with the moment of speaking.

In Novoslovnica Pre-future Tense is formed by using the verb «hteti» in 3-person form with the main verb in perfect form (examples 1, 2). This tense should be translated in English with Future Perfect Tense.

\textbf{Examples:}

\textit{Ja hte sòm kupil květy, dy ty hte priǐdaš do města.} — I will have bought flowers, when you will come to the place.

\textit{On hte je zakončil učilišto, dy otec mu hte sę vreta dodomu}. — He will have graduated from school, when his father will come back home.

Future Indefinite Tense (\textit{Něĝdašen čas}) is a rather rare tense to be used in Novoslovnica. It has no direct equivalents in English and should be translated in Future Simple with some keywords (such as «somewhen», «ever», «once»). The sense of this tense is to show that the action takes place in a future moment or period of time, but we do not care of when it will occur or how long it will last.
This tense is formed by Pre-future form of the verb «byti» with the past passive participle of the main verb. Look at the examples to get acquainted.

\textbf{Examples:}

\textit{On hte je byl nagrådil medalom mę.} — Once he will grant me with a medal.

\textit{Ja hte sòm byl kupil vašu věčj.} — Once I will but your item.

All other tenses are related to the past period of time. Firstly, we will consider real past tenses and then future-in-the-past ones.

Aorist (\textit{Prost minul čas}) determines the fact of the committed action without semantic details. It is similar in usage with English Past Simple Tense. Using Aorist we consider only the time the action occurs, but do not think about its duration.

In Novoslovnica it is formed by verb-base vowels with past definite endings: «-h», «-ša», «-še», «-hma», «-hta», «-ha», «-hme», «-hte», «-hu».

\textbf{Examples:}

\textit{Ja dělah råbotu-ta včera.} — I did the work yesterday.

\textit{Pred dvě ročiny ja jěŝih v gråd-òt.} — Two years ago I travelled to the city.

Imperfect (\textit{Neporęden minul čas}) determines the imperfect aspect of the past action. That means it is used with habitual, durable repeated actions etc, that took place in the past. Using imperfect we add the information, that our action took some exact time in the past. This tense is similar in usage with English Past Continuous Tense or Past Perfect.

It is formed so as Aorist with the one difference. There is a «-ě-» vowel before endings, not the verb-base vowel. So, for any verb type (see a chapter about verbal types) there is only a single imperfect form.

\textbf{Examples:}

\textit{Ja pišěh nadomnu råbotu dvě godiny včera.} — I was writing my homework for two hours yesterday.

\textit{Prijatelj mi ráděše na zavodu dvě ročiny.} — My friend had worked on a plant for two years.

Perfect (\textit{Svòršen minul čas}) determines the action has been committed before the present moment. That means, that we take account on the result of the action, on the fact it is finished, while Aorist determines the fact of the action itself and the time when it occured. So, it is a full equivalent of English Present Perfect Tense.

In Novoslovnica Perfect is formed by the auxiliary verb «byti» in Present Indefinite and an L-participle following it.

\textbf{Examples:}

\textit{On je izmyslïl novu ŧeoriju o problemě-ta.} — He has devised a new theory on the problem.

\textit{Môǐ brat je zakončil vysšojučilišto.} — My brother has graduated from University.

Plusquamperfect (\textit{Predminul čas}) determines the fact our action is further from us than another action in the past. It is akin Pre-Future tense, just with past actions. Simply speaking, it is an equivalent of English Past Perfect Tense. So as Pre-Future tense, actions in Plusquamperfect are usually used in pair with another past action that stands in Aorist.

Plusquamperfect in Novoslovnica is formed with aorist form of the auxiliary verb «byti» with an L-participle of the main verb.

\textbf{Examples:}

\textit{Ja byh doǐdal do ulicy-ta, dy on mi odŝvoniše, če ne može da priǐde.} — I had arrived to the street by the time he called be to say he cannot come.

\textit{On byše zdělal model, ĝda ja kazaše mu, če to ne máše potrěbnostï.}

Now we should consider two last tenses: Future-in-the-Past tense and Pre-Future-in-the-Past tense.
Future-in-the-Past tense (Bųdešt v minulom čas) is used when we speak about past actions, that occured after some other past actions. To emphasize this fact we use a past variant of the future tense — Future-in-the-Past. English has an equivalent, so it is easy to build a parallel between these two ones. In Novoslovnica this time is build with imperfect of the auxiliary verb «hteti» and DA-construction of the main verb (example 1).

This tense has also another meaning, that can be expressed with «would like to» phrase in English. It shows a polite intention to do something (example 2).

\textbf{Examples:}

\textit{My zapytahme dali vozidlo htěše da priǐde včas.} — We wondered if the bus would arrive on time.

\textit{Ja htěh da kazam ti něčto.} — I would like to say you something.

Pre-Future-in-the-Past tense (\textit{Predbųdešt v minulom čas}) is used, when we speak about some actions, that happened earlier than some future actions in the past (analogue of pre-future tense in the past). It has a similar meaning with Future-in-the-Past Perfect tense in English and it is rather rarely used.

It is formed with imperfect of the auxiliary verb «hteti» with perfect form of the main verb via DA-construction (look at the example).

\textbf{Examples:}

\textit{Vy kazahte, če htěhte da jeste podpisali pismo pred tym, kak htěhte da započïnate råbotati.} — You said you would have signed the paper before you would start working.

Past Indefinite Tense (\textit{Davnominul čas}) is the last official tense in Novoslovnica. It describes an action that occured in some moment in the past we do not know. In English we should translate it with Past Simple or with the «used to»-construction. It can be considered as a pair to Future Indefinite Tense.

\textbf{Examples:}

\textit{On je byl pisal knigu.} — He used to write a book.

Let us summarize all the equivalents of Novoslovnica's Tenses in English in the next table.

\begin{table}
	\caption{English equivalents of tenses in Novoslovnica}
	\begin{tabular}{ p{11em} p{11em} }
		\textbf{Tense in Novoslovnica}       & \textbf{English equivalent}                  \\
		Present Indefinite Tense             & Present Simple                               \\
		Present Definite Tense               & Present Continuous                           \\
		Future Indefinite Tense              & “once” + Future Simple                       \\
		Future Definite Tense (with “hteti”) & Future Simple                                \\
		Future Definite Tense (with “byti”)  & Future Continuous                            \\
		Future Definite Tense (single form)  & Future Simple                                \\
		Pre-future Tense                     & Future Perfect (Cont.)                       \\
		Future-in-the-Past Tense             & Future-in-the-Past Simple or “would like to” \\
		Pre-future-in-the-Past Tense         & Future-in-the-Past Perfect (Cont.)           \\
		Aorist                               & Past Simple                                  \\
		Perfect                              & Present Perfect (Cont.)                      \\
		Imperfect                            & Past Continuous                              \\
		Plusquamperfect                      & Past Perfect (Cont.)                         \\
		Past Indefinite Tense                & “used to”                                    
	\end{tabular}
\end{table}

The third verbal category can be found in Novoslovnica which is shown only in differences between Aorist and Imperfect: perfection, while other tenses differ only by determinacy and time. So, you can use complex tenses as Perfect, Plusquamperfect, Past Indefinite etc. with aorist and imperfect participles and receive two shades of perfection meaning.

\section{Mood}

Mood\index{mood} is a grammatical feature of verbs, used for signaling modality \cite{mood}. Mood is distinct from grammatical tense or grammatical aspect, although the same word patterns are used for expressing more than one of these meanings at the same time. There are several moods in Novoslovnica \cite{nsl-naklony}:

\begin{itemize}
	\item Indicative
	\item Declarative
	\item Subjunctive
	\item Conjunctive I
	\item Conjunctive II
	\item Imperative
	\item Optative
	\item Jussive
	\item Hortative
	\item Supine
	\item Inferential
\end{itemize}

Moods are divided into realis and irrealis moods. Realis\index{mood!realis} mood is a grammatical mood which is used principally to indicate that something is a statement of fact. More precisely, it is used to express what the speaker considers to be a known state of affairs. Novoslovnica has two realis moods - indicative and declarative.

Indicative mood\index{mood!indicative} (\textit{Oznamitelen})  is used when we need simply to indicate that something is a statement of fact. Indicative has a variety of forms, including positive, negative and interrogative. Due to this fact it is chosen to be a basic mood, that is compared to the others. Indicative mood will be extensively considered in the next chapters. Just look at few examples.

\textbf{Examples:}

\textit{On glěděše iz prozorca doma si.} - He was looking out of the window of his house.

\textit{Ja bųdu hoditi vò vysšojučilišto črez dvě godiny.} - I will attend the university in two years.

Declarative mood\index{mood!declarative} (\textit{Objavitelen}) is used when we describe a state of something, considering the action it was caused by. Declarative mood is of the same degree of variety as indicative mood, though is not used as often as indicative.

The first difference with indicative mood is in using auxiliary verb “imáti” (to have) instead of “byti” (to be) while forming sentences. This makes some semantic shifts in Novoslovnica. Look at the second example: the auxiliary verb is used in present tense, though it has a resultive semantic value. For present semantics the verbless sentences are used (first example).

The second difference is in the particle being used with the auxiliary verb in two moods. In indicative we use L-particles, while in declarative mood we use passive ones.

\textbf{Examples:}

\textit{Rěka krasna i slånco.} - (There are) Beautiful river and the sun (around me).

\textit{Ja mám rođeno dvôh synoŭ.} - I have two sons born.

Irrealis\index{mood!irrealis} mood indicates that a certain situation or action is not known to have happened at the moment the speaker is talking. 

Subjunctive mood\index{mood!subjunctive} (\textit{Predpokladen}) is used when we want to express various states of unreality: wish, emotion, possibility, obligation etc. Subjunctives occur most often, although not exclusively, in subordinate clauses, using particles “aby”, “žeby”, “čtoby”, “išby” (example 1).

Take a note about absence of auxiliary verb while using L-particles. We just use a subjunctive particle with them. Though, we can also use infinitive instead of L-particle with the object in dative (example 2). Moreover, DA-forms can be used to express subjunctive (example 3). Nevertheless, there are cases you can use subjunctive in the main clause (example 4). 

\textbf{Examples:}

\textit{Az upëram aby ty odidal.} - I would like you to leave.

\textit{Az upëram aby ti odidati.} - It’s better for you to leave.

\textit{Ja htem da ty odidaš.} - I want you to leave.

\textit{Ty by odidal odde.} - You better leave here.

Conjunctive mood I\index{mood!conjunctive I} (\textit{Domyslen I}) is used to express real wishes relatively present moment. It means, using Conjunctive I we show that our wish is still able to be implemented. It is a classic variant of conjunctive and is formed by conjunctive form of the verb “byti” (to be) with an L-participle.

In fact, this mood is rather similar to Conditional II in English with the same tense analogues used (examples 2, 3). If we use just conjunctive clause, it is similar with “wish-construction” usage area in English (example 1). However, we can express conjunctive with the single clause using past transgressive (example 4).

Note, that having two clauses we need to use conditional conjunctions (“ako”, “jestjli”, “koli”, “dali”, “či” etc.).

\textbf{Examples:}

\textit{Az bih htěl da viđu slånco v ovyǐ obvlåčnyǐ denj.} - I wish I see the sun in this cloudy day.

\textit{On biše doǐdal do nas, ako my zazovahme ĝo.} - He would come if we called him.

\textit{Ako znaše on novoslovnicu, mogal biše da govori so vsïmi slověnami.} - If he knew Novoslovnica, he would speak with all Slavs.

\textit{Znavšy novoslovnicu, on mogal biše da govori so vsïmi slověnami}. - If he knew Novoslovnica, he would speak with all Slavs.

Conjunctive mood II\index{mood!conjunctive II} (\textit{Domyslen II}) is the second conjunctive mood and it is used to express unreal wishes that have not been realized relatively to a moment in the past. It can be formed by the verb “byti” in conjunctive form either  with supine (example 1) or with plusquamperfekt form of the main verb (example 2). It can be related to Conditional III in English. 

Note, that using supine you do not have to use conditional conjunctions between clauses. Moreover, we can use past active participle in the conditional clause (example 3). Attention! Compare this with the past transgressive in the single clause in Conjunctive I.

\textbf{Examples:}

\textit{Htětj da sę vidim s nim včera, bih byl mogal to da sdělam.} - If  I had wanted to see him yesterday, I would have meet him.

\textit{Ako byh htel da sę vidim s nim včera, bih byl mogal to da sdělam.} - If  I had wanted to see him yesterday, I would have meet him.

\textit{Ako byh htel da sę vidim s nim včera, bih byl mogavšym to sdělati.} - the same translation

Imperative mood\index{mood!imperative} (\textit{Zapověden})  is used when we want to tell somebody a command or a request. In Novoslovnica it has only I-person (in dual and plural) and II-person (in singular, dual and plural) forms. It is indicated by special endings (look forward for details). In the examples you can see how imperative is used.

Please note that imperative is indicated only by single clause. Complex or compound sentences are not of this mood.

\textbf{Examples:}

\textit{Piši ovu knigu.} - [Please] write this book. (you, sg)

\textit{Kažite mu poslanije-to.} - [Please] tell him the message. (you, pl)

\textit{Pročitaǐte ovu knigu.} - [Please] read this book. (you, pl)

\textit{Predgotuǐmo juž sëdy pokôǐ-ot.} - [Please] prepare the room now! (we, pl)

Optative mood\index{mood!optative} (\textit{Žadatelen}) is a grammatical mood that expresses wish or hope. It is used when we want something or somebody to succeed in any action. It is formed by DA-construction in the main clause (look at the examples). 

In English we can find similarity in LET-forms (examples 1, 2) or some general expressions that reveal our wish (example 3).

\textbf{Examples:}

\textit{Da daǐ mi pomognuti ti.} - Let me help you.

\textit{Da bųde tako.} - Let it be so.

\textit{Da žive Bòlĝarija.} - Long live Bulgaria.

Jussive mood\index{mood!jussive} (\textit{Umožnitelen}) is a grammatical mood of verbs for issuing orders and commanding. In Novoslovnica it is used to make orders for third-person expressions. There are no direct equivalents in English for this mood, but we can translate it with impersonal sentences (example 1) or with MUST-modal expressions. It is formed by indicative with “haǐ” modal word.

\textbf{Examples:}

\textit{Haǐ toǐ ne tòpta trevy.} - Do not walk on grass.

\textit{Haǐ on podide.} - He must come closer.

Hortative mood\index{mood!hortative} (\textit{Predložen})  is a grammatical mood that let verb express encourage or discourage of doing something. It can be translated with “let us” (encourage) or “might not” (discourage) constructions in English. In Novoslovnica it is formed with “něhaǐ” modal word with indicative and is able to have negative (example 4) and positive (examples 1-3) form. It is generally used just with I-person expressions.

\textbf{Examples:}

\textit{Něhaǐ grame v ovu gru.} - Let us play this game.

\textit{Něhaǐ pějama pěsnü.} - Let us (both) sing a song.

\textit{Něhaǐ govorim to otcu.} - I might say this to father.

\textit{Něhaǐ ne idame v dom-òt.} - We might not go into the house.


Supine\index{supine} (\textit{Dostęgatelen}) is rather a grammar form than a mood. Nevertheless, it is often used to express aiming at something and the action of approaching to the goal that has been defined. It is similar to infinitive but the ending (infinitive has “-ti” ending, while supine has “-tj” endind). It is usually used with modal verbs and verbs of moving.

Supine can be translated in English through complex predicate. That is because in Novoslovnica supine is practically never used as a single verb form. It is generally used with main verb that determine the background action of the circumstance. Look at the examples to get acquainted with it.

\textbf{Examples:}

\textit{Moǐca je priǐdala povědatj dobru novinu.} - Mojca has come to message a good news.

\textit{Běgi skoro sę preoblekatj.} - Run fast to change clothes.

\textit{Poǐdame kupovatj.} - Let’s go to buy something.

Inferential mood\index{mood!inferential} (\textit{Prekazatelen}) is used to report an unwitnessed event without confirming it. It is often used in stories or fiction books and is very similar with indicative. The only difference is in 3-person in past tenses, where the auxiliary verb “byti” disappears and we use just L-participle. In English it should be translated with ordinary indicative (examples 1-4), sometimes “there”-forms also can be used (example 5). Note that we use L-participle in Novoslovnica in the past tense (you can mess it with Perfect tense), while it should be translated in Past Simple.
Look at the examples.

\textbf{Examples:}

\textit{Jesòm mu ĝo kupil.} - I bought it for him.

\textit{Byl sòm kupil naǐ-prosty martenicy, ale toǐ izbral naǐ-råzkošny.} - I had bought simpliest martenitsy, but he bought the most luxurious.

\textit{On byl bogatym.} - He was rich.

\textit{Kųde byl master?} - Where was the master?

\textit{Žila žena i mųž v malomu domu.} - There lived a woman and a man in a small house.

\section{Noun}

%Table

Nouns can be differentiated by three parameters: gender, animacy and the type of declension.

Animacy determines whether the object is animate (we are able to ask “Who is it?” to the object) and inanimate (we are able to ask “What is it?” to the object).

Gender determines whether the object is masculine, feminine or undefined (we cannot say it is one of the previous genders). Hence, there are three genders: masculine (with masculine properties), feminine (with feminine properties) and neutral (with undefined properties).

Despite English, Novoslovnica make us always show word gender explicitly of both animacies. We can say “it” to the object if we aren’t coupled with it in English. In Novoslovnica (as in every Slavic language) we should use the predefined gender when we speak about some concept (noun). Using wrong genders shows your ignorance and language nescience.

Type of declension is a parameter of declension function. Declension is a function of word alteration. It has two input parameters - the word itself and the type of declension that includes the terms of animacy, gender and some morphological features (such as word endings) in it. The output is a list of forms that the noun can be changed into. Novoslovnica supports 27-cell output list with (3 numbers) * (9 cases) elements in it. Further you can see tables of different declension types. These tables cover all use cases of declension function.

P.S. In tables abbrevs “A” and “I” are for “animate” and “inanimate” respectively.

% Tables
\section{Adjective}

\begin{table}[h]
	\caption{Noun characteristics}
	\begin{tabular}{lllll}
		\textbf{Title}              & \textbf{Value}               \\
		Semantic value              & Attribute                    \\
		Category                    & Independent                  \\
		Subcategory                 & Nominal                      \\
		Alteration                  & Declension                   \\
		Alteration parameters       & Case, Number, Gender, Degree \\
		Differentiation parameters  & Gender, Type, Form
	\end{tabular}
\end{table}

Adjective is one of POS that determines an attribute of the concept. There are two types of adjectives - relative and qualitative. 

Relative adjectives are called so, because they show relations between two concepts or a concept and an action.

\textbf{Examples:}

\textit{South (adj) pole} = South (noun) <= relation <= Pole

\textit{Južnyǐ pôl} = Jug <= relation <= Pôl

Qualitative adjectives are called so, because they show the quality of a concept’s property. This quality could be relative or quantitative or purely qualitative (showing concept condition, position, measure etc).

\textbf{Examples:}

There is the only type of adjective declension. However, we will divide declension tables by gender and base softness.

\section{Pronoun}

\begin{table}[h]
	\caption{Pronoun characteristics}
	\begin{tabular}{lllll}
		\textbf{Title}              & \textbf{Value}               \\
		Semantic value              & Attribute, Concept           \\
		Category                    & Independent                  \\
		Subcategory                 & Nominal                      \\
		Alteration                  & Declension                   \\
		Alteration parameters       & Case, Numbers, Gender, Person\\
		Differentiation parameters  & Gender, Type, Group
	\end{tabular}
\end{table}

Pronoun\index{pronoun} is a POS that has different meanings. It can play the role of an attribute or a concept, depending on what is replaced with the pronoun. However, pronoun has a verbal property of person.

Pronouns are divided into three types: nominal\index{pronoun!nominal} (noun-like declension), substantive\index{pronoun!substantive} (substantive declension), attributive\index{pronoun!attributive} (adjective-like declension).

There are also several groups of pronouns, depending on their semantic value. They are: personal, possessive, interrogative, relative, indefinite, definitive, reflexive, demonstrative, negative, reciprocal.

\subsection{Personal pronouns}

One of the most important groups are personal pronouns\index{pronoun!personal}. They replace nouns in sentences, where we do not want to use an unreasonable repeat. Personal pronouns have different forms for every person-number cell. In the table you can see them.

\begin{table}
	\begin{tabular}{llll}
		& Singular & Dual & Plural \\
		1 person & Ja (Az) & Ma & My \\
		2 person & Ty & Va & Vy \\
		3 person, ms & On & Ona & Oni \\
		3 person, fm & Ona & One & Oně \\
		3 person, neu & Ono & Oná & Onji
	\end{tabular}
\end{table}

I should mention the form “Vy” (You). As in English or polite Russian, we can use this pronoun for a single person when we want to emphasize our respect to the interlocutor. Moreover, we also have to use a plural form of the verb while speaking “Vy” in polite form.

Also you can see two forms of the 1 person - singular pronoun (I). Etymologically the form “Az” is full while “Ja” is only a short one. However, different Slavic languages have retained different forms and now we should support both ones. In Novoslovnica the difference between using “Az” and “Ja” lies in phonetics. As you see, “Az” begins with the vowel and ends with a consonant, “Ja” - controversially. If we remember rule 1 we will get such number of rules:

\begin{itemize}
	\item If the word before the pronoun ends with a vowel and the word after begins with a consonant - you should use “Ja”
	\item If the word before the pronoun ends with a consonant and the word after begins with a vowel - you should use “Az”
	\item In other cases you can use either one or another variant (“Az” is more formal than “Ja”)
\end{itemize}

Now let us speak about declension of personal pronouns. Personal pronouns relate to nominal pronouns (with noun-like declension). This is the most difficult type to learn. But everything has its order and beauty.

\begin{table}[!htb]
	\begin{tabular}{llll}
		I & Singular & Dual & Plural \\
		Nominative & Ja (Az) & Ma & My \\
		Genitive & Menę & Naju & Nas \\
		Partitive & Menä & Naju & Nas \\
		Accusative & Mene & Naju & Nas \\
		Dative & Meni & Nama & Nam \\
		Instrumental & Mnom (Mnoǐ) & Nama & Nami \\
		Prepositional & O mně & O naju & O nas \\
		Locative & Vo mnu & V naju & V nas \\
		Vocative & - & - & -
	\end{tabular}
\end{table}

\begin{table}[!htb]
	\begin{tabular}{llll}
		You (sg) & Singular & Dual & Plural \\
		Nominative & Ty & Va & Vy \\
		Genitive & Tebę & Vaju & Vas \\
		Partitive & Tebä & Vaju & Vas \\
		Accusative & Tebe & Vaju & Vas \\
		Dative & Tebi & Vama & Vam \\
		Instrumental & Tobom (Toboǐ) & Vama & Vami \\
		Prepositional & O mně & O vaju & O vas \\
		Locative & Vo mnu & V vaju & V vas \\
		Vocative & - & - & -
	\end{tabular}
\end{table}

\begin{table}[!htb]
	\begin{tabular}{llll}
		He & Singular & Dual & Plural \\
		Nominative & On & Ona & Oni \\
		Genitive & Jeĝa & Onaju & Ih \\
		Partitive & Jeĝu & Onaju & Ih \\
		Accusative & Jeĝo & Onaju & Ih \\
		Dative & Jemu & Onama & Im \\
		Instrumental & Nim & Onama & Nimi \\
		Prepositional & O nëm & Ob onaju & O nih \\
		Locative & V nëmu & V onaju & V nih \\
		Vocative & - & - & -
	\end{tabular}
\end{table}

\begin{table}[!htb]
	\begin{tabular}{llll}
		She & Singular & Dual & Plural \\
		Nominative & Ona & Oně & Oni \\
		Genitive & Ji & Oněju & Ih \\
		Partitive & Ji & Oněju & Ih \\
		Accusative & Ju & Oněju & Ih \\
		Dative & Ji & Oněma & Im \\
		Instrumental & Neju & Oněju & Nimi \\
		Prepositional & O neǐ & Ob oněju & O nih \\
		Locative & V neji & V oněju & V nih \\
		Vocative & - & - & -
	\end{tabular}
\end{table}

\begin{table}[!htb]
	\begin{tabular}{llll}
		It & Singular & Dual & Plural \\
		Nominative & Ono & Oná & Oni \\
		Genitive & Jeĝa & Náju & Ih \\
		Partitive & Jeĝu & Oněju & Ih \\
		Accusative & Jeĝo & Oněju & Ih \\
		Dative & Jemu & Onáma & Ih \\
		Instrumental & Nim & Onáma & Nimi \\
		Prepositional & O nem & O náju & O nih \\
		Locative & V nemu & V náju & V nih \\
		Vocative & - & - & -
	\end{tabular}
\end{table}


\subsection{Reflexive pronoun}

This\index{pronoun!reflexive} is a separate group of pronouns. There is the only reflexive pronoun in Novoslovnica - “sebę”. Its feature is the absence of nominative form. It has only 7 cases to be alternated. Vocative and Nominative are absent.

\begin{table}
	\begin{tabular}{lll}
		Case & Full & Short \\
		Genitive & Sebę & Sę \\
		Partitive & Sebä & Sä \\
		Accusative & Sebe & Se \\
		Dative & Sebi & Si \\ 
		Insrumental & Sobom (Soboǐ) & - \\ 
		Prepositional & O sobě & - \\
		Locative & V sobu & -
	\end{tabular}
\end{table}


“Sę” is used very often as a reflexive suffix in verbs. It determines a way of creating medial voice sentences (look the paragraph about it).

However, there is a term of a complex reflexive form (CRF) also. It is formed by the sequence of a personal pronoun and the definitive pronoun “sám”. This form shows a partly-developed reflection of a subject.  

There is practically no difference in using a reflexive pronoun or a complex reflexive form. However, it is recommended to us a CRF for Nominative and to use a reflexive pronoun in other cases.

\subsection{Possessive pronouns}

Possessive\index{pronoun!possessive} pronouns show whom any object belongs to. For example, we can say “It is a thing of Bob (which Bob possesses)”. In Novoslovnica you can use a similar expression: “To je věčj Boba” (remember rules of case using). Also Novoslovnica has another expression with a possessive adjective: “To je Bobóva věčj”. This adjective shows that this thing belongs to Bob. However, if we have already used the name of Bob in our sentence, we should not repeat it again. English also uses possessive pronouns in such cases: “Bob is my friend and that’s his thing”. We do not repeat the word Bob, we just say - his (which he possesses). That is what the possessive pronouns look like.

These pronouns are attributive, so we have no need to rewrite their declension, just to name nominative forms. Further, you take a nominative form of a possessive pronoun, look through declension tables for adjectives and transform your pronoun so as you did it with an adjective. Now let us look at the nominative forms of possessive pronouns.

\begin{table}
	\begin{tabular}{llll}
		& Singular & Dual & Plural \\
		1 person & Môǐ & Naš & Naš \\
		2 person & Tvôǐ & Vaš & Vaš \\
		3 person, ms & Jegôǐ & Onyš & Ih \\
		3 person, fm & Jejôǐ & Oneš & Ih \\
		3 person, neu & Jegôǐ & Onyš & Ih
	\end{tabular}
\end{table}

As usual, bold letters are under an accent.

The only feature of possessive pronouns is that you should add a “virtual” ending for 1-person and 2-person pronouns to start declining of it.

\textbf{Examples:}

- \textit{Môǐ - Mojyǐ} (virtual full form) - \textit{Mojoga, mojomu, mojym} (ordinary attributive declension) 

To avoid this difficulty you may use an “adjectived” form

\begin{table}
	\begin{tabular}{llll}
		& Singular & Dual & Plural \\
		1 person & Môǐnyǐ & Našnyǐ & Našnyǐ \\
		2 person & Tvôǐnyǐ & Vašnyǐ & Vašnyǐ \\
		3 person, ms & Jegôǐnyǐ & Onyšnyǐ & Ihnyǐ \\
		3 person, fm & Jejôǐnyǐ & Onešnyǐ & Ěhnyǐ \\
		3 person, neu & Jegôǐnyǐ & Onyšnyǐ & Ihnyǐ
	\end{tabular}
\end{table}

\subsection{Interrogative and relative pronouns}

Interrogative\index{pronoun!interrogative} pronouns are used when we create a question. They are never in plural or dual. Also they have no declension. The only role of interrogative pronouns is to reveal an aim of the question (How much? Who? What?) - to guide the interlocutor to the right answer (you need).

Relative\index{pronoun!relative} pronouns aim is to introduce a relative clause.

\textit{Relative clause}\index{clause!relative} is a special kind of subordinate clause whose primary function is as modifier to a noun or nominal. We examine the case of relative clause modifiers in NPs first, and then extend the description to cover less prototypical relative constructions.\cite{english-grammar}

Interrogative and relative pronouns are practically the same, only usage differs, that is why I talk about them in the same paragraph. Let us look at them in the table.

\begin{table}
	\begin{tabular}{lll}
		English equivalent & Interrogative pronoun & Relative pronoun \\
		Who & Kto? & Ke \\
		What & Čto & Če \\
		What & Kakyǐ? & Kak \\
		Which & Ktoryǐ? & Ktor \\
		Whose & Čyǐ? & Čyǐ \\
		What & Jakyǐ? & Jak \\
		What & Kakvyǐ ? & Kakòv \\
		What & Kolïkyǐ ? & Kolïk 
	\end{tabular}
\end{table}

Pronouns “Kolïko”, “” do not decline. Pronouns “Kakyǐ”, “Ktoryǐ” decline so as adjectives do. Other pronouns have a similar declension with possessive pronouns.
  
\subsection{Indefinite and negative pronouns}

Another group of pronouns that are similar to each other are indefinite\index{pronoun!indefinite} and negative\index{pronoun!negative} pronouns. They are derived from interrogative pronouns by adding the negative “ni-” and the indefinite “ne-” prefixes. Their declension equals to their ancestor’s one.

\begin{table}
	\begin{tabular}{lll}
		Interrogative pronoun & Negative pronoun & Indefinite pronoun \\
		Kto? & Nikto & Někto \\
		Čto & Ničto & Něčto \\
		Kakyǐ? & Nikakyǐ & Někakyǐ \\
		Ktoryǐ? & Niktoryǐ & Něktoryǐ \\
		Čyǐ? & Ničyǐ & Něčyǐ 
	\end{tabular}
\end{table}

\subsection{Demonstrative pronouns}

Demonstrative\index{pronoun!demonstrative} pronouns are usually used to make the interlocutor pay attention to something.
There are only four demonstrative pronouns. And this topic is closely connected with the articles. In the table … you can see these pronouns, divided by terms of visibility and distance of the object to name with respect to the speaker.


\begin{table}
	\begin{tabular}{lll}
		& Visible & Invisible \\
		Far & Onyǐ & Tyǐ \\
		\multirow{2}{*}{Close} & Ovyǐ & - \\ & \multicolumn{2}{c}{Sïǐ}  
	\end{tabular}
\end{table}

These pronouns decline so as adjectives do, so that is very easy. However, you see that there is no pronoun for a close object that is not seen to you. Maybe it is logical, maybe not, nevertheless, Slavic languages do not support this semantic value.

Examples:

\subsection{Definitive pronouns}

Attributive\index{pronoun!definitive} pronouns indicate a generalized feature of an object. 

\begin{table}
	\begin{tabular}{ll}
		English equivalent & Definitive pronoun \\
		Whole & Vesj \\
		All & Vsë, vsä, vsï \\
		Every & Vsękyǐ, lübyǐ \\
		Each & Káždyǐ \\
		Another, other & Ïnyǐ, drugyǐ \\
		Reflexive pronouns & Personal pronouns + sám
	\end{tabular}
\end{table}

All these pronouns decline as adjectives. I should say just about the last one - the pronoun “sám”. This pronoun is used with personal pronouns to create a complex reflexive form (look paragraph about reflexive pronouns). However, it keeps adjective-like declension.

\subsection{Reciprocal pronouns}


The last type of pronouns is reciprocal\index{pronoun!reciprocal}. It is used to refer to a noun phrase mentioned earlier in a sentence. English has only two such pronouns - each other and another one.

Slavic languages have much more variants:

\textit{Drug so drugom}

\textit{Raz za razom}


\section{Numeral}

\begin{table}[h]
	\caption{Adverb characteristics}
	\begin{tabular}{lllll}
		\textbf{Title}              & \textbf{Value}      \\
		Semantic value              & Attribute           \\
		Category                    & Independent         \\
		Subcategory                 & Nominal             \\
		Alteration                  & Declension          \\
		Alteration parameters       &               \\
		Differentiation parameters  & Type
	\end{tabular}
\end{table}

Numerals can be attributive (with a nount) or pronominal (without a noun).
	
% 	Им. п.	Р. п. 	К. п.	В. п.	Д. п.	Тв. п.	П. п.	М. п.
% М. р.	Једен	Једнога	Једногу	Једного	Једному	Једным	Једном	Једному
% Ж. р.	Једна	Једної	Једної	Једну	Једной	Једноју	Једной	Једної
% Ср. р.	Једно	Једнога	Једногу	Једного	Једному	Једным	Једном	Једному


%	Им. п.	Р. п. 	К. п.	В. п.	Д. п.	Тв. п.	П. п.	М. п.
% М. р.	Два	Двôх	Двôх	Два	Двôм	Двомя	Двôх	Двому
% Ж. р.	Двѣ	Двôх	Двôх	Двѣ	Двôм	Двомя	Двôх	Двому

% 	Им. п.	Р. п. 	К. п.	В. п.	Д. п.	Тв. п.	П. п.	М. п.
% М. р.	Три	Трёх	Трёх	Три	Трём	Трємя	Трёх	Трєму
% Ж. р.	Трѣ	Трёх	Трёх	Трѣ	Трём	Трємя	Трёх	Трєму

% Дальше наблюдается система – числительные от 5 до 20 имеют следующую парадигму склонения (о которой будет изложено чуть ниже), далее 21,31,41 и т.п. имеют первую парадигму, 22,32,42 и т.п. – вторую, 23,24,33,34 и т.п. – третью, и 25-30, 35-40, 45-50 и т.п. – четвёртую.
% Приводим четвёртую парадигму склонения на примере числительного 5 (Пѣт)


% 	Им. п.	Р. п. 	К. п.	В. п.	Д. п.	Тв. п.	П. п.	М. п.
% М. р.	Пѧт	Пѧті	Пѧті	Пѧть	Пѧті	Пѧтію	Пѧті	Пѧтї
% Ж. р.	Пѧць	Пѧті	Пѧті	Пѧць	Пѧті	Пѧтію	Пѧті	Пѧтї

\subsection{Cardinal numbers}

\begin{itemize}
	\item One (1) - Jeden, Jedna, Jedno
	\item Two (2) - Dva, Dvě
	\item Three (3) - Tri, Trě
	\item Four (4) - Četyri, Četyrě
	\item Five (5) - Pęt
	\item Six (6) - Šest
	\item Seven (7) - Sedem
	\item Eight (8) - Osem
	\item Nine (9) - Devęt
	\item Ten (10) - Desęt
	\item Eleven (11) - Jedennadesęt
	\item Twelve (12) - Dvanadesęt
	\item Thirteen (13) - Trinadesęt
	\item Fourteen (14) - Četyrinadesęt
	\item Fifteen (15) - Pętnadesęt
	\item Sixteen (16) - Šestnadesęt
	\item Seventeen (17) - Sedemnadesęt
	\item Eighteen (18) - Osemnadesęt
	\item Nineteen (19) - Devętnadesęt
	\item Twenty (20) - Dvadesęt
	\item Twenty-one (21) - Dvadesęt jeden
	\item Thirty (30) - Tridesęt
	\item Fourty (40) - Četyridesęt
	\item Fifty (50) - Pętdesęt
	\item Sixty (60) - Šestdesęt
	\item Seventy (70) - Sedemdesęt
	\item Eighty (80) Osemdesęt
	\item Ninety (90) - Devętdesęt
	\item Hundred (100) - Sto
	\item One hundred one (101) - Sto jeden
	\item One hundred ten (110) - Sto desęt
	\item Two hundred (200) - Dvěstě (Dvasta)
	\item Three hundred (300) - Trista
	\item Four hundred (400) - Četyrista
	\item Five hundred (500) - Pętsòt
	\item Six hundred (600) - Šestsòt
	\item Seven hundred (700) - Sedemsòt
	\item Eight hundred (800) - Osemsòt
	\item Nine hundred (900) - Devętsòt
	\item Thousand (1000) - Tysęča
	\item One thousand one (1001) - Tysęča jeden
	\item One thousand ten (1010) - Tysęča desęt
	\item One thousand one hundred (1100) - Tysęča sto
	\item Two thousand (2000)- Dvě tysęčy
	\item Five thousand (5000) - Pęt tysęč
	\item Million (1000000) - Milïon
	\item One million one thousand one (1001001) - Milïon sto jeden
	\item Billion (10\^9) - Bilïon (Milïard)
	\item Trillion (10\^12) - Trilïon
	\item Quadrillion (10\^15) - Kŭadrilïon
	\item Quintillion (10\^18) - Kŭintilïon
	\item Sextillion (10\^21) - Sekstilïon
	\item Septillion (10\^24) - Septilïon
	\item Octillion (10\^27) - Oktilïon
	\item Nonillion (10\^30) - Nontilïon
	\item Decillion (10\^33) - Decilïon
	\item Googol (10\^100) - Ĝuĝòl
	\item Googolplex (10\^(10\^100)) - Ĝuĝlopleks
\end{itemize}

Declension



\subsection{Ordinal numerals}

\begin{itemize}
	\item First (1) - Pòrvyǐ
	\item Second (2) - Vtoryǐ
	\item Third (3) - Tretjyǐ
	\item Fourth (4) - Četvòrtyǐ
	\item Fifth (5) - Pętyǐ
	\item Sixth (6) - Šestyǐ
	\item Seventh (7) - Sedmyǐ
	\item Eighth (8) - Osmyǐ
	\item Ninth (9) - Devętyǐ
	\item Tenth (10) - Desętyǐ
	\item Eleventh (11) - Jedennadesętyǐ
	\item Twelfth (12) - Dvanadesętyǐ
	\item Thirteenth (13) - Trinadesętyǐ
	\item Fourteenth (14) - Četyrinadesętyǐ
	\item Fifteenth (15) - Pętnadesętyǐ
	\item Sixteenth (16) - Šestnadesętyǐ
	\item Seventeenth (17) - Sedemnadesętyǐ
	\item Eighteenth (18) - Osemnadesętyǐ
	\item Nineteenth (19) - Devętnadesętyǐ
	\item Twentieth (20) - Dvadesętyǐ
	\item Twenty first (21) - Dvadesęt pòrvyǐ
	\item Thirtieth (30) - Tridesętyǐ
	\item Fourtieth (40) - Četyridesętyǐ
	\item Fiftieth (50) - Pętdesętyǐ
	\item Sixtieth (60) - Šestdesętyǐ
	\item Seventieth (70) - Sedemdesętyǐ
	\item Eightieth (80) Osemdesętyǐ
	\item Ninetieth (90) - Devętdesętyǐ
	\item Hundredth (100) - Sòtyǐ
	\item One hundred and first (101) - Sto pòrvyǐ
	\item One hundred and tenth (110) - Sto desętyǐ
	\item Two hundredth (200) - Dvôhsòtyǐ
	\item Three hundredth (300) - Tröhsòtyǐ
	\item Four hundredth (400) - Četyröhsòtyǐ
	\item Five hundredth (500) - Pętsòtyǐ
	\item Six hundredth (600) - Šestsòtyǐ
	\item Seven hundredth (700) - Sedemsòtyǐ
	\item Eight hundredth (800) - Osemsòtyǐ
	\item Nine hundredth (900) - Devętsòtyǐ
	\item Thousandth (1000) - Tysęčnyǐ
	\item One thousand and first (1001) - Tysęča pòrvyǐ
	\item One thousand and tenth (1010) - Tysęča desętyǐ
	\item One thousand and one hundredth (1100) - Tysęča sòtyǐ
	\item Two thousandth (2000)- Dvě tysęčnyǐ
	\item Five thousandth (5000) - Pęt tysęčnyǐ
	\item Millionth (1000000) - Milïonnyǐ
	\item One million one thousand and first (1001001) - Milïon sto pòrvyǐ
	\item Billionth (10\^9) - Bilïonnyǐ (Milïardnyǐ)
	\item Trillionth (10\^12) - Trilïonnyǐ
	\item Quadrillionth (10\^15) - Kŭadrilïonnyǐ
	\item Quintillionth (10\^18) - Kŭintilïonnyǐ
	\item Sextillionth (10\^21) - Sekstilïonnyǐ
	\item Septillionth (10\^24) - Septilïonnyǐ
	\item Octillionth (10\^27) - Oktilïonnyǐ
	\item Nonillionth (10\^30) - Nontilïonnyǐ
	\item Decillionth (10\^33) - Decilïonnyǐ
	\item Googolth (10\^100) - Ĝuĝòlnyǐ
	\item Googolplexth (10\^(10\^100)) - Ĝuĝlopleksnyǐ
\end{itemize}

\subsection{Pronominal numerals}

% СОБИРАТЕЛЬНОЕ ЧИСЛИТЕЛЬНОЕ
% Собирательное числительное обозначает совокупность объектов названного количество. Как и существительные данного типа, они склоняются только в единственном числе. Характерной особенностью является наличие суффикса «–ЕР–» или «–ОЈ–». 


% Числительное	Им. п.	Р. п. 	К. п.	В. п.	Д. п.	Тв. п.	П. п.	М. п.
% Два	Двојє	Двојіх		=Р.п.	Двојім	Двојіми	Двојіх	
% Три	Тројє	Тројіх		=Р.п.	Тројім	Тројіми	Тројіх	
% Четыри	Четверо	Четверых		=Р.п.	Четверым	Четверыми	Четверых	
% Пѧт	Пѧтеро	Пѧтерых		=Р.п.	Пѧтерым	Пѧтерыми	Пѧтерых	
% Шест	Шестеро	Шестерых		=Р.п.	Шестерым	Шестерыми	Шестерых	
% Седем	Седмеро	Седмерых		=Р.п.	Седмерым	Седмерыми	Седмерых	
% Осем	Осмеро	Осмерых		=Р.п.	Осмерым	Осмерыми	Осмерых	
% Девѧт	Девѧтеро	Девѧтерых		=Р.п.	Девѧтерым	Девѧтерыми	Девѧтерых	
% Десѧт	Десѧтеро	Десѧтерых		=Р.п.	Десѧтерым	Десѧтерыми	Десѧтерых	
% Једеннадесѧт	-десѧтеро	=||=	=||=	=||=	=||=	=||=	=||=	=||=
% Двадесѧт	-десѧтеро	=||=	=||=	=||=	=||=	=||=	=||=	=||=
% ИТД								


\section{Adverb}

\begin{table}[h]
	\caption{Adverb characteristics}
	\begin{tabular}{lllll}
		\textbf{Title}              & \textbf{Value}      \\
		Semantic value              & Attribute           \\
		Category                    & Independent         \\
		Subcategory                 & Nominal             \\
		Alteration                  & Comparison          \\
		Alteration parameters       & Degree              \\
		Differentiation parameters  & Type, Group
	\end{tabular}
\end{table}

\section{Predicative}

\begin{table}[h]
	\caption{Noun characteristics}
	\begin{tabular}{lllll}
		\textbf{Title}              & \textbf{Value}                            \\
		Semantic value              & Predicate                                 \\
		Category                    & Independent                               \\
		Subcategory                 & Nominal                                   \\
		Alteration                  & None                                      \\
		Alteration parameters       & None                                      \\
		Differentiation parameters  & Tense                                  
	\end{tabular}
\end{table}

Predicative\index{predicative} is a POS that is closely deriving to the predicate. Formally, it is an adverb that plays the role of the predicate (concluded in the predicate as a part of it). 

You should divide a syntax and morphological definitions of predicative. Now we are talking about the second one. In russian philology you can find the term of “condition category” - the equivalent for predicative (morphological definition). In this paragraph we will speak just about this term.

Predicative determines the class of words indicating an attribute or a condition of a person, environment etc., especially mental. Predicatives cannot be alternated, though they can be differentiated by tenses using an auxiliary verb “to be”. 

Predicatives have homonyms with adverbs, and nouns.  

\section{Verb}

Verb\index{verb} is a POS that describes the predicate and its properties. 


\subsection{Verbal types}

Learning Slavic languages you could mention that there is a set of verb suffixes that is very eliminated. Novoslovnica provides a theory that allows you to construct a right verb form.
Verbs with different verb suffixes represent different verbal types. There are four types of verbs in Novoslovnica.

\begin{itemize}
	\item A-type\index{verb!a-type}: verbs of this type define that the action in common sense.  
	\item E-type\index{verb!e-type}: verbs of this type define that the action is long-termed.
	\item I-type\index{verb!i-type}: verbs of this type define that the action is short-termed. This type comprises verbs with suffixes I and O. The suffix O is used when the consonant before the constructed suffix is involved in alterations depending on soft vowels that the vowel I is.
	\item U-type\index{verb!u-type}: verbs of this type define that the action is dotty.
	\item Extra type\index{verb!extra-type}. Is formed with the suffix “-OVA-” and defines the repeated action.  
\end{itemize}

When you speak about the action, you find what characteristic is suitable for the action and then use one of the predefined verbal types.

Somebody can ask what are the differences in tenses and types, because it might be confusing. However, verbal type determines the durability of the action (or its repeating property) while tense determines tense characteristic of the action such as completeness, stability in time, result, order etc.

All tenses provide difference between conjugation of different verbal types except imperfect. This tense has the common conjugation table for all verbal types. 

Further we will speak about conjugation itself. We will look at verb conjugation in indicative mood first of all. Then we will speak about other moods.

\subsection{Active voice}

Active\index{voice!active} voice shows that the person makes the action by himself. So, the subject of the sentence and the actor are the same.

\subsubsection{Indicative mood}

Verbs in indicative\index{mood!indicative} mood can be found in every tense that Novoslovnica possesses. Let us look at some tables with verb of different verbal types conjugation.

\subsubsection{Present Tenses}

\begin{table}[!htb]
	\caption{A-type conjugation in Present Indefinite}
	\begin{tabular}{llll}
		Present Indefinite & Singular & Dual & Plural \\
		1 person & -am & -ama & -ame \\
		2 person & -aš & -ata & -ate \\
		3 person & -a & -at & -ut
	\end{tabular}
\end{table}


\begin{table}[!htb]
	\caption{A-type conjugation in Present Definite}
	\begin{tabular}{llll}
		Present Definite & Singular & Dual & Plural \\
		1 person & -aju & -ajema & -ajeme \\
		2 person & -aješ & -ajeta & -ajete \\
		3 person & -aje & -ajat & -ajut
	\end{tabular}
\end{table}

\begin{table}[!htb]
	\caption{E-type conjugation in Present Indefinite}
	\begin{tabular}{llll}
		Present Indefinite & Singular & Dual & Plural \\
		1 person & -em & -ema & -eme \\
		2 person & -eš & -eta & -ete \\
		3 person & -e & -at & -ut
	\end{tabular}
\end{table}


\begin{table}[!htb]
	\caption{E-type conjugation in Present Definite}
	\begin{tabular}{llll}
		Present Definite & Singular & Dual & Plural \\
		1 person & -u & -ujema & -ujeme \\
		2 person & -uješ & -ujeta & -ujete \\
		3 person & -uje & -ujat & -ujut
	\end{tabular}
\end{table}


\begin{table}[!htb]
	\caption{I-type conjugation in Present Indefinite}
	\begin{tabular}{llll}
		Present Indefinite & Singular & Dual & Plural \\
		1 person & -im & -ima & -ime \\
		2 person & -iš & -ita & -ite \\
		3 person & -i & -at & -ut
	\end{tabular}
\end{table}


\begin{table}[!htb]
	\caption{I-type conjugation in Present Definite}
	\begin{tabular}{llll}
		Present Definite & Singular & Dual & Plural \\
		1 person & -ujim & -ujima & -ujime \\
		2 person & -ujiš & -ujita & -ujite \\
		3 person & -uji & -ujat & -ujut
	\end{tabular}
\end{table}


\begin{table}[!htb]
	\caption{U-type conjugation in Present Indefinite}
	\begin{tabular}{llll}
		Present Indefinite & Singular & Dual & Plural \\
		1 person & -nam & -nama & -name \\
		2 person & -naš & -nata & -nate \\
		3 person & -na & -nat & -nut
	\end{tabular}
\end{table}


\begin{table}[!htb]
	\caption{U-type conjugation in Present Definite}
	\begin{tabular}{llll}
		Present Definite & Singular & Dual & Plural \\
		1 person & - & - & - \\
		2 person & - & - & - \\
		3 person & - & - & -
	\end{tabular}
\end{table}

\begin{table}[!htb]
	\caption{Extra-type conjugation in Present Indefinite}
	\begin{tabular}{llll}
		Present Indefinite & Singular & Dual & Plural \\
		1 person & -ovam & -ovama & -ovame \\
		2 person & -ovaš & -ovata & -ovate \\
		3 person & -ova & -ovat & -ovut
	\end{tabular}
\end{table}


\begin{table}[!htb]
	\caption{Extra-type conjugation in Present Definite}
	\begin{tabular}{llll}
		Present Definite & Singular & Dual & Plural \\
		1 person & -uju & -ujema & -ujeme \\
		2 person & -uješ & -ujeta & -ujete \\
		3 person & -uje & -ujat & -ujut
	\end{tabular}
\end{table}

\begin{table}[!htb]
	\caption{The verb "Byti" (Exception)}
	\begin{tabular}{lllllll}
		Pr. Indef.
			& \multicolumn{2}{c}{Singular}
			 & \multicolumn{2}{c}{Dual}
			 & \multicolumn{2}{c}{Plural} \\
		1 person & Jesòm & Sòm & Jesma & Sma & Jesme & Sme \\
		2 person & Jesi & Si & Jesta & Sta & Jeste & Ste \\
		3 person & Jestj & Je & Jesų & Sų & Jesu & Su
	\end{tabular}
\end{table}

\subsubsection{Future Tenses}

\textbf{Future Definite Tense (Bųdešt čas)}

The Future Definite Tense is formed with the following:

\begin{itemize}
	\item Using HTE (3p. sg. of the verb "hteti" (to will)) + the verb in Present Definite or Indefinite Tense
	\item Using the future form of the verb "byti" + infinitive
	\item Using single-form conjugation 
\end{itemize}

\begin{table}[!htb]
	\caption{Future single-form conjugation}
	\begin{tabular}{llll}
		Future
		& Singular
		& Dual
		& Plural \\
		1 person & -ahtem & -ahtema & ahteme \\
		2 person & -ahteš & -ahteta & -ahtete \\
		3 person & -ahte & -ahtat & -ahtut
	\end{tabular}
\end{table}

\begin{table}[!htb]
	\caption{The verb "Byti" future conjugation (Exception)}
	\begin{tabular}{llll}
		Future
		& Singular
		& Dual
		& Plural \\
		1 person & Bųdu & Bųdema & Bųdeme \\
		2 person & Bųdeš & Bųdeta & Bųdete \\
		3 person & Bųde & Bųdat & Bųdut
	\end{tabular}
\end{table}

\textbf{Pre-Future Tense}

\begin{table}[!htb]
	\caption{The verb "Byti" pre-future conjugation (Exception)}
	\begin{tabular}{llll}
		Future
		& Singular
		& Dual
		& Plural \\
		1 person & Bųdeh & Bųdehma & Bųdehme \\
		2 person & Bųdeša & Bųdehta & Bųdehte \\
		3 person & Bųdeše & Bųdeha & Bųdehu
	\end{tabular}
\end{table}

\textbf{Future Indefinite Tense}

\subsubsection{Past Tenses}

\begin{table}[!htb]
	\caption{A-type conjugation in Aorist}
	\begin{tabular}{llll}
		Present Definite & Singular & Dual & Plural \\
		1 person & -ah & -ahma & -ahme \\
		2 person & -aša & -ahta & -ahte \\
		3 person & -aše & -aha & -ahu
	\end{tabular}
\end{table}

\begin{table}[!htb]
	\caption{E-type conjugation in Aorist}
	\begin{tabular}{llll}
		Present Definite & Singular & Dual & Plural \\
		1 person & -eh & -ehma & -ehme \\
		2 person & -eša & -ehta & -ehte \\
		3 person & -eše & -eha & -ehu
	\end{tabular}
\end{table}

\begin{table}[!htb]
	\caption{I-type conjugation in Aorist}
	\begin{tabular}{llll}
		Present Definite & Singular & Dual & Plural \\
		1 person & -ih & -ihma & -ihme \\
		2 person & -iša & -ihta & -ihte \\
		3 person & -iše & -iha & -ihu
	\end{tabular}
\end{table}

\begin{table}[!htb]
	\caption{U-type conjugation in Aorist}
	\begin{tabular}{llll}
		Present Definite & Singular & Dual & Plural \\
		1 person & -uh & -uhma & -uhme \\
		2 person & -uša & -uhta & -uhte \\
		3 person & -uše & -uha & -uhu
	\end{tabular}
\end{table}

\begin{table}[!htb]
	\caption{Extra-type conjugation in Aorist}
	\begin{tabular}{llll}
		Present Definite & Singular & Dual & Plural \\
		1 person & -ovah & -ovahma & -ovahme \\
		2 person & -ovaša & -ovahta & -ovahte \\
		3 person & -ovaše & -ovaha & -ovahu
	\end{tabular}
\end{table}

\begin{table}[!htb]
	\caption{The verb "byti" conjugation in Aorist}
	\begin{tabular}{llll}
		Present Definite & Singular & Dual & Plural \\
		1 person & byh & byhma & byhme \\
		2 person & byša & byhta & byhte \\
		3 person & byše & byha & byhu 
	\end{tabular}
\end{table}

\begin{table}[!htb]
	\caption{Сonjugation in Imperfect}
	\begin{tabular}{llll}
		Present Definite & Singular & Dual & Plural \\
		1 person & -ěh & -ěhma & -ěhme \\
		2 person & -ěša & -ěhta & -ěhte \\
		3 person & -ěše & -ěha & -ěhu
	\end{tabular}
\end{table}

\begin{table}[!htb]
	\caption{Сonjugation in Perfect}
	\begin{tabular}{llll}
		Present Definite & Singular & Dual & Plural \\
		1 person & sòm + PPP & jesma + PPP & jesme + PPP \\
		2 person & si + PPP & jesta + PPP & jeste  + PPP \\
		3 person & je + PPP & jesų + PPP & jesu + PPP 
	\end{tabular}
\end{table}

\begin{table}[!htb]
	\caption{Conjugation in Plusquamperfect}
	\begin{tabular}{llll}
		Present Definite & Singular & Dual & Plural \\
		1 person & byh + PPP & byhma + PPP & byhme + PPP \\
		2 person & byša + PPP & byhta + PPP & byhte  + PPP \\
		3 person & byše + PPP & byha + PPP & byhu + PPP 
	\end{tabular}
\end{table}

\subsubsection{Future-in-the-Past Tenses}


\textbf{Future-in-the-Past}

\textbf{Pre-Future-in-the-Past}

\subsubsection{Subjunctive mood}

Subjunctive\index{mood!subjunctive} mood shows two states of an actions. In the first state it can show a desirable action. In the second one it can show an action that has not become a real one. 

Subjunctive mood has only one-tense form. It is constructed with the verb “byti” in a subjunctive form with an aorist or imperfect participle. Special forms of the verb “byti” you can see in the next table.

\begin{table}[!htb]
	\begin{tabular}{llll}
		Subjunctive mood & Singular & Dual & Plural \\
		1 person & Bih & Bihma & Bihme \\
		2 person & Biša & Bihta & Bihte \\
		3 person & Biše & Biha & Bihu
	\end{tabular}
\end{table}

\subsubsection{Imperative mood}

Imperative\index{mood!imperative} mood shows that the actor make somebody do some action (imperate another object). Imperative mood also has only one tense to be used in. 

\begin{table}[!htb]
	\caption{A-type}
	\begin{tabular}{llll}
		Imperative mood & Singular & Dual & Plural \\
		1 person &  & -aǐma & -aǐmo \\
		2 person & -aǐ & -aǐta & -aǐte \\
		3 person &  &  & 
	\end{tabular}
\end{table}



\begin{table}[!htb]
	\caption{E-type}
	\begin{tabular}{llll}
		Imperative mood & Singular & Dual & Plural \\
		1 person &  & -eǐma & -eǐmo \\
		2 person & -i & -eǐta & -eǐte \\
		3 person &  &  & 
	\end{tabular}
\end{table}



\begin{table}[!htb]
	\caption{I-Type}
	\begin{tabular}{llll}
		Imperative mood & Singular & Dual & Plural \\
		1 person &  & -iǐma & -iǐmo \\
		2 person & -i & -iǐta & -iǐte \\
		3 person &  &  & 
	\end{tabular}
\end{table}

\begin{table}[!htb]
	\caption{U-Type}
	\begin{tabular}{llll}
		Imperative mood & Singular & Dual & Plural \\
		1 person &  & -niǐma & -niǐmo \\
		2 person & -ni & -niǐta & -niǐte \\
		3 person &  &  & 
	\end{tabular}
\end{table}


\begin{table}[!htb]
	\caption{Extra-Type}
	\begin{tabular}{llll}
		Imperative mood & Singular & Dual & Plural \\
		1 person &  & -uǐǐma & -uǐǐmo \\
		2 person & -uǐ & -uǐǐta & -uǐte \\
		3 person &  &  & 
	\end{tabular}
\end{table}

\subsubsection{Inferential mood}

Inferential\index{mood!inferential} mood is used when we speak about actions that we did not observe by ourselves. This mood mostly is used in tales, histories two show that we were not witnesses of what we are speaking about.

Inferential mood is created by using the verb “ïmáti’ with the past passive participle (PPP). This mood can be used in all tenses of the indicative mood (we should only place the verb “ïmáti” into this tense and add a PPP of the main verb to it). 

\textbf{Examples:}

\textit{Ja sòm rođen svojoǐ mamoǐ} - I am born by my own mother. (This is a real fact, but the actor (Mother) is not at the subject place in the sentence - Passive Voice).

\textit{Ona má rođeno ove rebętko.} - Some say she bore this child (This is a rumor, and the actor is in the place of a subject - Inferential mood of AV)

\subsection{Passive voice}

Passive\index{voice!passive} voice shows that the person is an object for the action. That means, that the subject of the sentence semantically is not an actor, but the object. 

Passive voice has only two moods - Indicative and Subjunctive. Passive voice is created by using the verb “byti” with the PPP. Here is the narrow border between Inferential mood of Active Voice and Indicative mood of Passive Voice.

We can describe the difference between these two terms in such a way. If we speak about real actions that we have imagined or we heard about them (however, the actor of these sentences is placed in the subject) - we use Inferential mood. And if we speak about any action that has no actor in the subject syntax role - we use Passive Voice. Let’s look at the examples.


\subsection{Infinitive and Supine}

Infinitive is the main form of the verb. In fact all languages that have an infinitive form of the verb use it in that way. So when you look through the dictionary looking for a verb, you should remember that it will stand in infinitive. Infinitive form is created by adding the ending “-ti” to the end of the verb. (Compare with English, you put the particle “to” before the verb to create the same form - there is some similarity in both cases). Infinitive has no parameters for declination.

The second unchangeable form of the verb is supine. It has no equivalent in English. Semantically, it is similar to the construction “to be going to do something”, determining the aims of the subject. Supine in Novoslovnica is built by adding the “-tj” ending to the end of the verb.

Supine had a great usage area in the past, now it is still used in Uppersorbian, Lowersorbian and Slovenian languages. It is mostly used with the verbs of motion, such as “to go”, “to swim”, “to move”, “to fly” etc.

Examples:

Ty hteš kazati mi něčto, da li? - You want to say me something, don’t you? (Infinitive)

Mamo, ja idaju spatj dnesj po-pano. - Mom, I’m going to sleep now earlier. (Supine)




\section{Participle}

\begin{table}[h]
	\caption{Adjective characteristics}
	\begin{tabular}{lllll}
		\textbf{Title}              & \textbf{Value}               \\
		Semantic value              & Attribute                    \\
		Category                    & Independent                  \\
		Subcategory                 & Verbal                       \\
		Alteration                  & Declension                   \\
		Alteration parameters       & Case, Number                 \\
		Differentiation parameters  & Type, Voice, Tense, Gender
	\end{tabular}
\end{table}

Participle is an independent verbal POS, that defines the action of another actor as its attribute. Participle is a POS between the nominal and verbal type, so as a pronoun, because it has properties of tense, voice, gender, case and number.

There are essential participles and auxiliary participles. The first ones can be used as attributes of nouns, they are far distanciated from the verb. They are Active and Passive Participles.

Auxiliary participles are quite closer to verbs, they participate in grammar constructions. They are Aorist and Imperfect participles.

Let us look at the tables to define how to use different participles.

\section{Gerund}

Gerund is a form of a verb that determines the process of an action. That means it is not an independent POS, but has some features that do not allow us to include the description about gerund in the paragraph about verb.

Gerund is formed from the infinitive of a verb by reducing a “-ti” ending and adding a suffix “-n-” and an ending “-e-”. However, gerunds are often created only from verbs of A-type. However, you can construct gerunds from verbs of other types. Look at the examples:

\textbf{Examples:}

\textit{Pisati} (verb) - pisane (gerund)

\textit{Nositi} (verb) - nosine (gerund, bad form) - nošane (gerund, recommended form)

\textit{Pěti} (verb) - Pěne (gerund, bad form) - pějane (gerund, recommended form)

Unlike infinitive or supine, gerund can be declined, but only in the singular. This fact unites it with the nominal POS. Simply speaking, gerund declines so as soft neutral nouns do despite the nominal/accusative form of “e”.

\begin{table}
	\begin{tabular}{ll}
	Case & Ending \\
	Nominal & -e \\
	Genitive & -ä \\
	Partitive & -ü \\
	Dative & -ü \\
	Accusative & -e \\
	Instrumental & -ëm \\
	Prepositional & -ě \\
	Locative & -ü \\
	Vocative & no form
	\end{tabular}
\end{table}

Let us look at some examples:

Examples:



\section{Article}

If you tried to learn a Slavic language, you would notice that there are no articles\index{article} in it. The term of definiteness\index{definiteness} is ruled mostly by demonstrative pronouns. However, there are two Slavic languages that provide this possibility - Bulgarian\footnote{With Pomak dialects that are considered to be a separate language by some scientists.} and Macedonian: these are Slavic analytic marvel (a pair of languages, that has worked out a completely different model of grammar). Novoslovnica let you use this achievement. Going further, we will compare an English and a Slavic systems of articles.

Firstly, you should remember that there is no indefinite article. The word itself shows that you are speaking about a designatum and there is no concrete information about it, no details. However, if you want to accentuate the term of indefiniteness you can use an indefinite pronoun before the word (“Někyǐ”, “Něktoryǐ”).

Secondly, there are huge differences between definite articles\index{article!definite} in English and in Novoslovnica. English has only one article - “the”. It shows only the term of definiteness. Novoslovnica provides in articles additional meanings, which you have already seen in the paragraph about demonstrative pronouns - distance and visibility of the object. So, there are three definite articles in Novoslovnica: “-òt”, “-òn”, “-òv”. Also you should know that these articles have the differentiation by gender and number. 

òt – ta – to – te

òv – va – vo – ve

òn – na – no – ne

Despite of the other POS articles have only two numbers (singular and plural) and are differentiated by gender only in singular. Words in dual manage the article in plural. To remind you, we will repeat that the article “òv” is used, when the object is in the field of view and it is rather close to you. The article “òn” is used, when you still see the object, but it is rather far from you. The article “òt” is used, when you cannot see the object or it is abstract, however you are talking about the definite instance of the designatum.

English article "the" is derived from a demonstrative\index{pronoun!demonstrative} pronoun "that". Novoslovnica articles are derived as it follows: \textit{òn} from \textit{onyǐ}, \textit{òv} - from \textit{òvyǐ} and \textit{òt} - from \textit{tyǐ}. Nonetheless, you can see that the pronoun “sïǐ” has not developed into the article and there is no article equivalent.

Finally, there are differences in the way to use the article. In English you place the article just before the word, dividing the word and the article by the space (different words). In Novoslovnica you put the article just after the word, dividing the word and the article by the hyphen (one word).

\textbf{Examples:} 

\textit{Kniga-na je na stolu} - The book is on table.

\textit{Dom-òv je mnogo starym} - This house is very old.

\subsection{Adjective article}

Articles in Novoslovnica are used with nouns only. However, there are tools for expressing definiteness of adjectives.

Adjective article is a form of using short personal pronouns with adjectives.\footnote{A similar phenomenon is used nowadays in Macedonian to express the definiteness of an object of speaking. However, the "double object" (which stands for a personal pronoun form here) is placed right before the predicate.}

\textbf{Examples:}

Dobòr - Dobòr \textbf{i} 

Dobra - Dobra \textbf{ja} 

Dobro - Dobro \textbf{je} 

You can use such an old form. Though, better you a full adjective form which has been derived from adjective articles in East Slavic languages.

\textbf{Examples:}

Dobòr \textbf{i} - Dobryǐ

Dobra \textbf{ja} - Dobraja

Dobro \textbf{je} - Dobroje

Follow the next rules of using articles:

\begin{itemize}
	\item Use only one definite form in a \gls{np}.
	\item If there are several adjectives in a definite \gls{np}, use a full adjective only with the first one.
	\item If the full form of an adjective is equal to the short one (Due to cases), you should use an article on the noun in \gls{np} or an adjective article itself (in a split form).
\end{itemize}
\section{Particle}

Particle\index{particle} is a dependent part of word. Particles add an auxiliary meaning to the main word. There are some groups of particles.

Particles added to words are close to some additional functions. If you delete them from your phrase, there will be no change in the whole meaning (that is why one cannot say they are of an auxiliary POS), but with presence of particles you will get more additional semantic or sometimes emotional information named \textit{color}. English language has very few particles. The most known is “to” as an infinitive indicator particle of a verb. Controversially, Slavic languages have a bit more particles that are rather popular in the spoken language. They form several groups by their semantic color.

\textbf{Examples:}

\textit{Ja sòm govoril tobě.} - I have told you.

\textit{Ja sòm govoril že tobě!} - I have told you!

Additional semantic meaning. The speaker shifts emphasis from undefined (neutral phrase) to the word “govoril” (told). So the interlocutor now has a determined emphasis of the phrase. The speaker wants to say that he has already told the same fact to the interlocutor and he was right because something happened confirming them.

Positive

* Aga - Yeap

* Ugu - Yeah

* Da - Yes

Negative

* Ne - Not
* Ni...ni - Neither...nor

Interrogative

* Či - Whether

* Li - Whether

Estimative

* Kakto - Like

* Mòl - Supposedly

Comparative

Incentive

Exclamative

Amplificative


Specifying

Restrictive

Demonstrative



\section{Preposition}

Prepositions\index{preposition} also are not an independent POS. They are closely related to the main word (often it is a noun). Most prepositions show the direction or the location of the action. 

We can divide prepositions into two main groups, so as we did with adverbs. We can distinguish \textit{primary} and \textit{secondary} prepositions. Primary prepositions are very ancient and we cannot refer to any word form they have formed from. Secondary prepositions are longer, and they appeared by semantic shift of an adverb, transgressive or a cased-noun. 

Here I will list primary prepositions with English translations and controlled cases. Complex cases are commented in notes.

\textit{Bez} (Gen.) - Without

\textit{V} (Acc.) - In, into

\textit{Dlä} (Gen) - For

\textit{Do} (Gen.) - To

\textit{Za} (Instr.) - For

\textit{Iz} (Gen.) - From (inside the object)

\textit{K} (Dat.) - To

\textit{Krôz} (Skrôz) (Gen.) - Through

\textit{Na} (Acc.) - On

\textit{Nad} (Instr.) - Above

\textit{O} (Prep.) - About

\textit{Od} (Gen.) - From (the object)

\textit{Po} (Dat.) - Along

\textit{Pod} (Instr.) - Under

\textit{Pri} (Loc.) - At

\textit{Pro} (Acc.) - About (the difference between “O” and “Pro” is in the detail view on the object. When we say the second variant we just mention the object in our speech, while using the first one we talk about it in details).

\textit{S} (Instr.) - With

\textit{U} (Gen.) - At (the difference between “Pri” and “U” is in the object of speaking. When we use “U” we mention real object in space and place the object of speaking near it. “Pri” is used when we speak about proximity in time, i.e. some events are close to each other.)

\textit{Črez} (Acc.) - After,  in (time)

The following figure shows the semantics of most primary prepositions.

\begin{figure}
	\includegraphics[width=\linewidth]{./sources/prepositions.jpeg}
	\caption{Prepositions in Novoslovnica}
	\label{fig:prepositions}
\end{figure}

Secondary prepositions are derived from nouns or adverbs with the shift of semantic from independent to an auxiliary one. For examples, preposition "Pred" is derived from the noun "Pred" (Front). Using separately in refers to a frontal part of something. Using with an additional independent word it becomes a preposition defining the frontal part of the word that follows it.

\textbf{Examples:}

\textit{Svòrh} - Over

\textit{Među} - Between
\section{Conjunction}

You saw that there is a rather big amount of prepositions on Novoslovnica. However, the amount of conjunctions\index{conjunction} is much smaller. 

Conjunctions are divided into two classes: coordinating and subordinating conjunctions.

Coordinating\index{conjunction!coordinating} conjunctions usually connect sentence elements of the same grammatical class (N + N, V + V etc.). There are four kinds of coordinating conjunctions: copulative, adversative, disjunctive and illative.

\textbf{Conjunctive}:

I - and

Da - and

Ta - and

Ili - or

Či - or

Abo -or

\textbf{Adversative}:

Ale - but

Ama - but

No - but

Subordinating\index{conjunction!subordinating} conjunctions are used to connect clauses in subordinate sentence. They complement the functionality of corresponding adverbs.

\textbf{Subordinative}:

Dabi - for

Aby - for

Da - for

This is the whole list of existing conjunctions in Novoslovnica today.

\section{Interjection}

This interesting POS is used to describe emotions within the sentence. Interjections\index{interjection} are neither independent nor dependent POS. They even are not involved into sentence structure. They just show the color of the sentence to let the interlocutor show your feelings. Interjections are divided into three groups depending on their aim: emotional, imperative and etiquette. 

\textbf{Emotional interjections}

Emotional interjections can be divided into negative and positive interjections.

Positive:

Ah - Ah

Ŭaŭ - Wow

Ŭah - Wow

Ura - Hurray

Ogo - 

Uf - 

\textbf{Negative}

Oh - 

O-o - Oh

Be - 

Hehe -

Heh -  

Éh - 

Jo - 

Fu - 

\textbf{Ambiguous} (depend on the context):

Uh - Uh

Oǐ - Oh

Aǐ - 

Išty - 

Hm - Hmm

\textbf{Imperative interjections}

Let us look at them:

Éǐ - Hey

Na - Take it

Stop - Stop

Bre - Man

A-u - 

Allo - Hello

Brysj - Go out

Von - Out

\textbf{Etiquette interjections}

These interjections are often whole words, that we use without the sentence context in some situations that need our etiquette.

Hvála - Thanks

Dobrodošli - Welcome

Dękujem - Thanks

Zdråveǐ - Hi

Zdråveǐte - Hello

\section{Onomatopoetic words}

This group of words determines the reflection of animal, bird, babe, technic sounds that person reproduces in his speech.



\chapter{Phrase}

\section{Collocation}

A collocation is an expression consisting of two or more words, that the one is main and the other are subordinative. Usually, collocation consists from 2 to 5 words. Subornative words are connected to the main one by several types of links: coherence, management and adjunction.

\textit{Coherence} between the main and the subordinative words lies in corellation of grammar forms. That means the subordinative word should correspond with the main one while the latter changes (i.e. case or gender). Thus, coherent link introduces a strong connection between the two words, so they change both in primary link establishment and in word change (declension or conjugation).

Examples:
- 

\textit{Management} link has a weaker connection between the words. The word with a management link has a determined grammar form. So the subordinative word should be changed once when is connected to the main word and then is not changed while the main word is declining or conjugating. The primary word form is managed by the main word within a rule-case. It can be of a determined case or a preposition to be used with.

Examples:
- 

\textit{Adjunction} means we simply add a subordinative word to the main one to form a collocation. We usually speak about this type while linking when we have immutable words.

Examples:
- 

Using proper cases in a collocation has a great mean for constructing an euphonious and understandable sentence. However, we cannot define all the cases of such links, so you can feel-in the language to achieve this. If you are Slavic, simply try to use those cases that you use in your native language. If you are not, try to use English logic that is close to in a number of cases.
\section{Emphasis}

Emphasis is a logical accent in a phrase. Often we allocate the accented words with the voice while speaking, but sometimes we need to show the accent in writing or strongify it. So here you can take a look at tools of showing the accent in your phrase.

The most suitable way of it is to use particles “že”, “ž”, “pretož”, “vědj”. The accent depends on place where the particle stands in your sentence. 

\begin{itemize}
	\item In the beginning - this makes the accent on the first word in a phrase (sentence)
	\item After the first word - this makes the accent on the whole phrase (sentence)
	\item After the second … last word - this makes the accent on the second … last word in your phrase (sentence)
\end{itemize}

\textbf{Examples:}

- \textit{Že ja ti kazah to včera!} (That was me who said that to you yesterday)

- \textit{Ja že ti kazah to včera!} (I told you that yesterday [But you did not listen to me]).

- \textit{ Ja ti kazah to včera že!} (I told you that yesterday [But you has already forgotten about it])
\section{Phraseology}

\textbf{Ja mám něčto - Něčto je u menę}

These are two constructions we can use to describe an English phrase “I have something”.

As you see, the exact translation of the English phrase is the first variant. Here we use the verb “máti” (to have). However, there is the second variant in Novoslovnica too. In this variant we can watch the interchange between the actor and the object of an action by syntax roles. In “\textit{Ja mám něčto}” we see that the actor is the subject of this short sentence, and the action object has a syntax role of a direct object. In “\textit{Něčto je u menę}” we see, that the actor has a role of indirect object and the object has a subject syntax role.    

There are some connections between the second variant of the phrase and the English “There is/are” construction. In English we use that when the object is located in some place so as in the second variant (if we change the actor (indirect object) with the adverbial (or indirect object of location)). In Novoslovnica we still can use this construction with the soulful objects such as personal pronouns or animated nouns. 

\textbf{Jesòm někto - Jesòm někym}

You can notice that there are different construction of translation a phrase “to be somebody” in Novoslovnica. These two variants, with normal form \footnote{The term of "normal form" is rather virtual though it refers to a Nominative case. However, we know Nominative can be used only with subjects. In \textit{Jesòm někto} "někto" plays the role of an object, so it cannot be Nominative in the term of cases. So-called \textit{Normal form} indicates an analytic affect on Slavic languages. On one hand is a direct object so no prepositions are used. On the other, no case is used. We can correlate this one with infinitive as a "normal form" of the verb.} and Instrumentative, are semantically equal. 

The accent of using these two forms lays in the linguistic formalism that you use in your speech. Using a normal form is more analytic while using Instrumentative is more synthetic.

Nevertheless, they both can be used in speech.


\textbf{...ili... - ...ali...}

\textbf{<nothing> - kakto}


\section{Preposition managing}

Many phrases have not one determined preposition to be managed. It is a part of phraseology to detect which one suits better in which case. Here we list some examples that show those differences.

\textbf{Jeden z - Jeden od}

Both these constructions are used to describe an element of some sort. However, two different prepositions make different semantic sense here.

\begin{itemize}
	\item Preposition "z" describes that the following word is a category or an infinite set
	\item Preposition "od" describes that the following word is a finite set.
\end{itemize}

Look at the examples:

\textit{Jeden od prijatelëǐ mi kazaše mi užasnu novinu} - One of (I have just several friends) told me a terrifying news.

\textit{Jeden z vozidlov sę zlomaše včera na stanicji} - One of autobuses (We do not know how many there are autobuses) cracked yesterday.

\textbf{Blïzko do - Blïzko k}

Both these constructions describe something that is close to the following word. However, there are also semantic differences between each other.

\begin{itemize}
	\item Preposition "do" is used to show the approaching destination (direction)
	\item Preposition "k" is used to show the closity of an object to another one (placement)
\end{itemize}

Examples:

\textit{My jěhajeme juž dvě godiny, ale sme juž blïzko do doma.} - We have been riding for two hours, though we are close to home. (Our house is close to our due to our moving)

\textit{Naš dom stoji blizko kò grådu.} - Our house situates near the city. (The placement of the house is close to city)


\section{Common Slavic proverbs and adages}


\chapter{Syntax}

\section{Clause}

Clause is the smallest grammatical unit that represents a complete proposition. An ordinary clause consists of a subject and a predicate, sometimes with some additional auxiliary members. Predicate usually is composed of a verb with adverbial modifiers.

The subject determines the concept which is the actor in the sentence and the predicate determines the action which the actor is connected with.



% http://www2.ivcc.edu/rambo/eng1001/sentences.htm
% https://www.lexico.com/en/grammar/clauses

\section{Simple sentense}

Originally, a simple sentence consists of a single independent clause with a finite verb in it. Complex and compound sentences (look further) may be of multiple clauses, though a simple sentence has the only one.

\subsection{Subject}
The actor, that subject is indeed, can be described by different means. However, the most frequently used one is a noun. Every POS that is a subject in the sentence should be put into initial form. For a noun, its initial form is Nominative case. Noun in other cases cannot play the role of a subject.
Frankly speaking, a lot of POS in proper cases can play role of a subject. Now I want to list the most popular ones to let you know what is the point.

% Table

The most popular ones are first three rows.

At the end I will list the examples that could help you in understanding of how to put the word into its initial form to let it be the subject of a sentence. 

\textbf{Examples:}


\subsection{Predicate}
The action, which is provided by the actor, can be also described in different ways. The most popular is the verb, of course. The verb, determining the tense of the action, the person that does this action, the mood of the action, can be formed into every available for a verb form… Novoslovnica does not provide a noun being as a predicate (as it is available in Russian). Such sentences are transformed into the ones with the auxiliary verb “to be” as connection between the term and the determining word. 

\subsection{Auxiliary members}
The subject and the predicate are the main members of a simple sentence. However, you can notice that they do not provide  enough information to the interlocutor. That is why there are auxiliary members of a sentence, that give us necessary information.

\subsubsection{Object}
This auxiliary member gives the information about objects that are influenced by the action actor does. 
Here we should speak about one more classification of verb: by the mean of transitivity.

The transitive verb can have an object in accusative case without any prepositions.

The intransitive verb can have no objects or only with a preposition. 

\subsubsection{Adverbial}
Adverbial helps us to get some more information about predicate. Sometimes, they are involved in semantic connection with the verb and become actants. 

\subsubsection{Modifier}
Modifier brings some additional information about attributes of the subject, objects etc. They help us to build the exact picture of the situation that is described by a predicate.

\subsection{Word order}
There are languages with free word order and with strict one. Novoslovnica is a language with a declared word order, when changing the order of the words shifts a semantic value of the whole sentence. However, you can change some words in the sentences as you like. Let us see what the order is.

The common structure of simple sentence with neutral semantic value is SVO: \textit{Subject + Verb + Object}. From this thesis we can write down two rules:

Predicate is placed after the Subject

Object is placed after the Predicate it is connected to.

These rules are main if you want to create a grammatically proper sentence with neutral semantic value. 

There are also three additional rules:

\begin{itemize}
	\item Consistent grammar modifiers should be placed before the modified word.
	\item Inconsistent grammar modifiers should be placed after the modified word.
	\item Adverbials should be placed after the Predicate.
	\item Interrogatives should be placed at the beginning of the sentence.
\end{itemize}

These six rules can help you to understand what order is suitable for a Slavic sentence. There are also some constraints of what should not appear in your sentence:

\begin{itemize}
	\item Participial should be placed after or before the modified word.
	\item The particle “b/by” should not be placed in the beginning of the sentence.
\end{itemize}

Thus, now you are able to create a normal simple sentence with neutral semantic value. Nevertheless, you can gain a certain accent on the word you want by replacing words within the sentence.

Using SOV idiom you put the accent on the Object. 

Using VSO idiom you put the accent on the Verb (the Predicate)

Using OVS idiom you put the accent on the pair of Object and Predicate.

Using OSV idiom you put the accent on the pair of Object and Subject.

Using VOS idiom is equal to using a VSO idiom.

The same thing is about auxiliary members of the sentence - if you want to put an accent on the word, you should place it at the beginning of the sentence. Subject is an exception - its normal place is first, so you can put an accent on it only in your pronunciation.

\subsection{Incomplete sentence}

However, sometimes we do not need to say everything in our sentence, because it will cause tautology. Then we reduce some words calling for ellipsis. A sentence is named incomplete, when the subject or the predicate is not used in it.

\subsection{Substantive sentence}

Sometimes it is enough to say only a collocation or even the only word to show your idea to the interlocutor. This is what the substantive sentence is like.

We would not speak about them, because we suppose everything is quite clear. Just use phrases or the only word to express your idea.

\subsection{Relative clause}

Relative clause is one of grammar modifiers. It is a separate definition or adverbial, which has turned into a sentence that depends on the main one.

\section{Compound Sentence}

A compound\index{sentence!compound} sentence refers to a sentence that consists of two (or more) independent clauses in it. They are connected to one another with \textit{coordinating} link.

Coordinating sentences describe the relationship between the clauses with no dependency in it. Coordinating sentences are divided into three categories: \textit{copulative}, \textit{adversative} and \textit{consecutive}. These types of sentences are connected with the so-called types of conjunctions or adverbs. The only exception is consecutive sentences where no auxiliary word is used.
 
Following examples show the usage of coordinate sentences:

\textbf{Examples:}

\textit{Otec mi stroji búdu, i materj mi govori o tom dobro.} - My father is building a house, and my mother says great about it.

\textit{On uči anĝliǐskyǐ jazyk četyrě ročiny, ale ja učaju samo dva dnä.} - He has been studying English for four years, but I have been studying it only for two days.

\section{Complex Sentence}

A complex\index{sentence!complex} sentence refers to a sentence that consists of one (or more) dependent clauses connected to the single main one.

A dependent\index{clause!dependent} clause alone cannot stand for a complete sentence. The only way to form a complete one is to connect the dependent clause to an independent one and to form a complex sentence. Independent clauses are connected to the independent one with subordinate, conditional or relative conjunctions.

Thus, subordinating sentences are used to describe a relationship between sentence parts with a dependency from one main clause (superordinate) to other subordinate.

\textbf{Examples:}

\textit{On råzumi, že to ne je naǐ-lěpym vyrěšenëm} - He understands that it is not the best decision.

\textit{Mysläm, če ty ne hte máš uspěha v tom} - I think you will not succeed in it.

A conditional\index{clause!conditional} sentence describes something that is possible or probable.

\textbf{Examples:}

\textit{Ja bųdu lěkarom, dy hte zakončim vysšojučilišto} - I will be a teacher when I graduate from university

\textit{Ako ne sę hybam, to bųde trudno} - If I am right, this will be difficult.

A relative\index{clause!relative} clause is used for description of a part of the main clause.

\textbf{Examples:}

\textit{Ja uvidih děvušku, ktoru byh sretil včera v parku} - I saw a girl I meet yesterday in the park.

\textit{Môǐ gråd, v ktorom jesòm sę rodil, je vëlïk} - My city, I was born in, is large.

The dependent clauses can go first followed by the independent one, and conversely.


\section{Reported speech}

There are some ways of describing person’s speech in the sentence. The first way is to use direct speech. Look at the examples:

\textit{I say: “We need to go to the theatre”.} - (in quotes is situated my own phrase about going to the theatre)

\textit{- Come on, we need to go!, - exclaimed his brother.} - (after the hyphen we see a phrase of somebody’s brother).

Another way is to use a reported speech - a way of describing what somebody has said without using of quotes. Look at the examples of reported speech in English.

\textit{He says (that) we’re going to stay here today.} (We just mention what HE says to us).

\textit{I said that we had finished this work the day before.} (I mention my words that I said previously without exact reproducing them).

Reported speech is used when we refer to person’s speech in our sentence implicitly. It is constructed with the ordinary sentence describing real situation and then a relative cause of a reported speech with reference to another expression, which is connected to the main clause with the relative pronoun or a subordinative conjunction.

However, like English, Novoslovnica has a tense-shift in using reported speech. You can see what tenses shift to which one in the next table:

\begin{table}
	\begin{tabular}{ll}
		Tense in Direct Speech & Tense in Reported Speech \\
		Present Common & Present Common/Aorist \\
		Present Concrete & Aorist/Imperfect \\
		Future I & Future-in-the-Past I \\
		Future II & Future-in-the-Past II \\
		Future-in-the-Past I & Future-in-the-Past I \\
		Future-in-the-Past II & Future-in-the-Past II \\
		Aorist & Plusquamperfect (aorist part.) \\
		Imperfect & Plusquamperfect (imperfect part.) \\
		Perfect & Plusquamperfect \\
		Plusquamperfect & Plusquamperfect 
	\end{tabular}
\end{table}


Subjunctive mood is reproduced in reported speech so as it is written in the direct one. 

Examples:


Imperative mood is reproduced in reported speech with using modal verbs and subjunctive mood of the main verb.

Examples:

Inferential mood can be reproduced in reported speech doubly: with using modal analogs of the English verb “might” with a tense-shifted verb in Indicative mood or the verb in Inferential mood itself can be tense-shifted (we shift the tense of the auxiliary verb “ïmáti”).

Examples: 

\chapter{Semantics}

There are three main parts of the sentence analysis that lead us to understanding: grammar, syntax and semantics. First two chapters were reviewed on the previous pages of the book. Here we are going to talk about the third main part of the linguistic analysis - Semantics.

We are not talking here about how to produce something like machine-assisted translator of Novoslovnica. However, we will consider some facts, that could help to deal with it. 

This chapter comprises three parts: the semantics of prefixes, the semantics of suffixes and common semantics that will describe the sense of roots and dependent \gls{pos}.

\section{Prefix}

\newglossaryentry{pow}{name=POW, description={Part of word}}
\newglossaryentry{prep}{name=PREP, description={Preposition}}

Prefix\index{prefix} is a part of a word (\gls{pow}) that stands before the word stem. One word may have more than one prefix. In other words, prefixes can be added one by one right to the beginning of the stem. Each prefix has its own semantic value and word changes its meaning due to the presence of exact prefixes.

We can divide prefixes into several groups. 

\begin{itemize}
	\item Borrowed - prefixes that were borrowed from other languages (i.e. Latin, English etc.)
	\item Native - Slavic prefixes, that have a clear semantic value
\end{itemize}

Novoslovnica possesses 35 native\index{prefix!native} and 9 borrowed\index{prefix!borrowed} prefixes.

Native: bez, v, vně, vųtrě, voz, vy, do, za, zad, iz, k, kų, među, na, nad, naǐ, ne, ni, o, od, pa, po, pod, poslě, pra, pre, pred, prez, pri, pro, råz, s, sô, u, črez.

Borrowed: a, antï, aŭto, kŭazï, mono, multï, pan, para, super

The chapter is to discuss the main prefixes, giving each of them a description and examples. Also one should consider that major part of prefixes are deeply connected with the prepositions. That means many prefixes can be used separately from the stem in a prepositional role. Such prefixes are marked with the PREP label. 

\textbf{Bez} \gls{prep}

Meaning: “without”

This prefix is an equivalent of the word “without” in English. It means the absence of something that we mention. As a prefix, it has no equivalence in English. Nevertheless, the suffix “-less” has a very similar meaning. Look at the example to get it.

\underline{Examples:}

\textit{Bez doma} - Without home

\textit{Bezdušen} - Without soul/Soulless 


\textbf{V} \gls{prep}

Meaning: “into” (direction or placement)

This prefix has a semantic value of “leading into something”. The preposition equals to “in” or “into” in English.

\underline{Examples:}

\textit{V búdji }- In the building

\textit{Vhoditi} - Enter

\textbf{Vně} \gls{prep}

Meaning: “outside” (placement)

\underline{Examples:}

\textit{Vně trudnostëǐ} - Out of problems

\textit{Vněrečen} - Outspoken

\textbf{Vųtrě}

Meaning: “inside” (placement)

\underline{Examples:}

\textit{Vųtrě dušy} - Inside a soul

\textit{Vųtrěseben} - Thoughtful

\textbf{Voz}

Meaning: “upside” (direction)

\underline{Examples:}

\textit{Vozmožen} - Possible

\textit{Vozběg} - Take-off

\textbf{Vy}

Meaning: “outside” (direction)

This one means “leading outside something”. The prefix has no prepositional equivalent, remember word “Vy” means “You”.

\underline{Examples:}

\textit{Vyhod} - Exit

\textit{Vyglěd} - Outlook

\textbf{Do} \gls{prep}

Meaning: “to” (destination)

This prefix shows the destination of a process to some point. In English we can find an equivalent “to”.

\underline{Examples:}

\textit{Idati do stěny }- go to the wall

\textit{Dodati} - add (something we give to a set of something to increase its amount or capacity)

\textbf{Za} \gls{prep}

Meaning: “for” (aim)

The semantic value of this prefix is the aim of an action. We use it when we want to reach something, some object. English “for” can be found as revealing one of its semantic values in equal way. 

\underline{Examples:}

\textit{Idati za cělïü} - go for the goal

\textit{Zavariti čaǐ }- To brew tea (in order to make it hot)


\textbf{Zad} \gls{prep}

Meaning: “behind, after” (follow)

This prefix has an artificial origin (This semantic value was divided from the previous prefix). It means placing an object behind another one. The equivalent is “behind”.

\underline{Examples:}

Bez doma - Without home
Bezdušen - Without soul/Soulless 

\textbf{Iz} \gls{prep}

Meaning: “from”

This prefix equals the English word “from”. 

\underline{Examples:}

\textit{Jesòm iz Moskvy} - I am from Moscow

\textit{Izhodnyǐ} - basic (from that we can develop something new)

\textbf{K} \gls{prep}

Meaning: “to” (direction)

\underline{Examples:}

Bez doma - Without home
Bezdušen - Without soul/Soulless 

\textbf{Kų}

Meaning: “what”

\underline{Examples:}

\textit{Kųda} - Where (What direction)

\textit{Kųdy} - When (What time) 

\textbf{Na} \gls{prep}

Meaning: “on” (placement)

Examples:
Bez doma - Without home
Bezdušen - Without soul/Soulless 

\textbf{Nad} \gls{prep}

Meaning: “above” (placement)”

Examples:
Bez doma - Without home
Bezdušen - Without soul/Soulless 

\textbf{Naǐ}

Meaning: “the most”

Examples:
Bez doma - Without home
Bezdušen - Without soul/Soulless 


\textbf{Ne} \gls{prep}

Meaning: “not”

Examples:
Bez doma - Without home
Bezdušen - Without soul/Soulless 

\textbf{Ni} \gls{prep}

Meaning: “even this/so”

Examples:
Bez doma - Without home
Bezdušen - Without soul/Soulless 

\textbf{O}  \gls{prep}

Meaning: “about”

Examples:
Bez doma - Without home
Bezdušen - Without soul/Soulless 

\textbf{Od} \gls{prep}

Meaning: “from” (direction)

Examples:
Bez doma - Without home
Bezdušen - Without soul/Soulless 

\textbf{Pa}

Meaning: “not real”

Examples:
Bez doma - Without home
Bezdušen - Without soul/Soulless 


\textbf{Po} \gls{prep}

Meaning: “along” (direction)

Examples:
Bez doma - Without home
Bezdušen - Without soul/Soulless 

\textbf{Pod} \gls{prep}

Meaning: “under” (placement)

Examples:
Bez doma - Without home
Bezdušen - Without soul/Soulless 

\textbf{Pra}

Meaning: “grand”

Examples:
Bez doma - Without home
Bezdušen - Without soul/Soulless 

\textbf{Pre}

Meaning: “more than”

Examples:
Bez doma - Without home
Bezdušen - Without soul/Soulless 

\textbf{Pred} \gls{prep}

Meaning: “before, in front of” (placement)

Examples:
Bez doma - Without home
Bezdušen - Without soul/Soulless 

\textbf{Prez} \gls{prep}

Meaning: “rapidly through”

Examples:
Bez doma - Without home
Bezdušen - Without soul/Soulless 


\textbf{Pri} \gls{prep}

Meaning: “close to, approach” (direction)

Examples:
Bez doma - Without home
Bezdušen - Without soul/Soulless 

\textbf{Pro} \gls{prep}

Meaning: “rapidly through”

Examples:
Bez doma - Without home
Bezdušen - Without soul/Soulless 

\textbf{Råz}

Meaning: “different”

Examples:
Bez doma - Without home
Bezdušen - Without soul/Soulless 

\textbf{S} \gls{prep}

Meaning: “with”

Examples:
Bez doma - Without home
Bezdušen - Without soul/Soulless 

\textbf{Sô}

Meaning: “alongside”

Examples:
Bez doma - Without home
Bezdušen - Without soul/Soulless 

\textbf{U} \gls{prep}

Meaning: “near”

Examples:
Bez doma - Without home
Bezdušen - Without soul/Soulless 

\textbf{Črez} \gls{prep}

Meaning: “slowly through”:

Examples:
Bez doma - Without home
Bezdušen - Without soul/Soulless 


\section{Suffix}


\textbf{Ak}

This suffix has a semantic value of “person of some quality”.

\textbf{Examples:}

\textit{Bědnäk - Běden (adj)} - Poor man - Poor (adj)

\textit{Glupak - Glup (adj)} - Fool - fool (adj)

\textbf{An}

This suffix has a semantic value of “having some property”.

Examples:
Bědnäk - Běden (adj) - Poor man - Poor (adj)
Glupak - Glup (adj) - Fool - fool (adj)

\textbf{An}

This suffix has a semantic value of “person of some profession”. The suffix has a similar meaning with English "-er".

\textbf{Examples:}

\textit{Rybar} - Fisher

Glupak - Glup (adj) - Fool - fool (adj)


Ar

Ač

Dl

En

Ot

Ostj

Ec

Izn

Ik

In

Ih

Ic

K

Ni

Nik

Sl

Stv

Telj

Un

Yš

\section{Stem}

A word \textit{stem} is a semantic core of the word. It includes the main semantic value presented by a word form.

Each word of an independent \gls{pos} has at least one stem.

A \textit{stem nest} is a group of words existing in the language that have the same word stem. For example, look through the next words:

\textit{Sųd - sųdba - sųditi - sųden}

\textit{Hod - hodka - hoditi - poholdka - zahoden}

You see the words are of different meanings and different \gls{pos}. Thought, every word in each group has an equal semantic core - something connected with the judgement (1) or with walking (2).

Stems can be divided by several categories. For examples, there are borrowed and native stems. Native stems are derived from some proto-language forms. Thus, words which stems are derived from proto-Slavic, proto-Baltic, proto-German are supposed to be native in Novoslovnica. Also, some neologisms created by Slavic languages are also native.

Words of all other stems are borrowed. On of Novoslovnica's goals is to reduce the borrowings. However, we can list the languages from which Novoslovnica borrows the most. We use the term "factor" to describe the affect on Novoslovnica lexicon of different non-Slavic languages instead of absolute or percentage value. The more the factor the more influence on the language the borrowing from that language have.

\begin{table}[!htb]
	\begin{tabular}{ll}
		Language & Factor \\
		Roman & 43 \\
		i.e. French & 2 \\
		i.e. Latin & 40 \\
		German & 26 \\
		i.e. English & 25 \\
		Greek & 12 \\ 
		Turkish & 6 \\
		Baltic & 0.01 \\
	\end{tabular}
\end{table}

The word can have multiple stems. Novoslovnica usually use words with up to three stems in a word. The stems are connected with stem-forming vowels. They are: \textit{o, ë, ï} and \textit{ô}.

Words with multiple stems usually describes a \gls{np} that has a strong forces between the words. The following examples show when you should use each of stem-forming vowels.

\textbf{Examples:}

- \textit{Zemlëtręs} - \textit{Tręs zemlï} - Earthquake (auxiliary word (soft) + main word)

- \textit{Sámovoz} - \textit{Vozi sámo} - Automobile (auxiliary word (hard) + main word)

- \textit{Blågodariti - Blågo dariti} - To thank (auxiliary word (hard) + main word)

- \textit{Pętïkrátno} - \textit{Pętï krátno} - Five times (Simple concatination of numerals.\footnote{Pętokrátno is also valid according to the second case (Kpátno pętï - Pętokrátno)})

- \textit{Dvôkrátno - Dva kráta} - Twice (Particular case of the previous example.\footnote{Dvokrátno from Krátno dvôm is also available.})

\section{New word creating}

The process of creating the language’s vocabulary is one of the most important processes in language constructing. We need to build an understandable lexeme, moreover, it can be named as a Slavic one. Here I want to tell you about this process we follow every time when the new word of Novoslovnica appears in this world. By following this algorithm you will be able to create new words that are suitable for Novoslovnica by yourself. Now look here:
We are looking through the vocabularies of Slavic languages and find out the translations from our native language for the exact word.
Then we look at the frequency of roots within this list.
If there is an absolute leader there, and it is Slavic, we take this root.
Then we look at the frequency of suffixes and prefixes within this list.
If there is an absolute leader, and it stays for the primary semantic value, we take this form.
If not, we create a new form by following described above semantic values of different prefixes and suffixes.
If there is some forms that are prevail others, we can add them all into Novoslovnica vocabulary.
If not, we check whether there are some non-absolute leaders within the words. If they are, we go to 3.a to each of them.
If not, and there is a non-Slavic absolute or some relative leaders, we should take into account the next points:
There is no Slavic root among them
We cannot form an artificial Slavic form for a word.
If the previous forms are true, we add a non-Slavic word into vocabulary. However, if we can form a Slavic word - we add it into vocabulary with additional non-Slavic “crutch”. 

Frankly, this is the whole algorithm. You can apply it in your everyday speaking. However, it takes time to build a new word for the only semantic value, so it is difficult to create all your words by these rules by yourself. It is better for you to use a dictionary, that have already been created by our team and contains more than 5 thousand forms of Novoslovnica’s vocabulary.. 
\chapter{Appendices}

Language Core 207 (Swadesh list)

Comparison between Novoslovnica and Interslavic with Neoslavonic

Example texts in Novoslovnica

\backmatter
\bibliographystyle{ieeetr}
\bibliography{biblio}
\addcontentsline{toc}{chapter}{Bibliography}
\printindex

% SOURCES

% https://cals.nu/language/novoslovnica/
% https://novoslovnica.com

% Rodney D. Huddleston, Geoffrey K. Pullum, A Student's Introduction to English Grammar, CUP 2005, p. 183.
% А.А. Кривицкий, А.Е. Михневич, А.И. Подлужный Белорусский язык. Для говорящих по-русски - Минск, Выш. шк., 2008. - 383 с.
% Palmer, F. R., Mood and Modality, Cambridge Univ. Press, 1986 (second edition 2001).

% bibliography, glossary and index would go here.

\end{document}