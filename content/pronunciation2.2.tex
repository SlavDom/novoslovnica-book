\section{Pronunciation}

Novoslovnica is a phonetic language, that’s why Novoslovnica has an important rule, which you have to apply to speaking in Novoslovnica.

Rule \#1: All words are pronounced as they are written.

This rule means that you cannot reduce sounds when speak in Novoslovnica. It is a very important thing because you can make mistakes if you speak improperly. There are some exceptions but they all will be mentioned in this guidebook.

When you pronounce a word you are not restricted to use only main sounds - if it’s more comfortable, you can pronounce allophones with the same level of softness and sonority with the main sound of the letter. Let’s look at the examples below to understand what we can choose in speaking and what we cannot.
Examples:
jsdasd
asdasd

However, you are restricted in what consonant sounds to use from the allophone list. You can see the next rule which will help you to speak.

Rule \#2: You cannot mess soft and hard consonants when you pronounce a word.

This prohibit you to make hard consonants when you need a soft one, or to use a soft one when you need a hard one. Here you can see a table, where it is shown which sound you must pronounce in different combinations of letters.

\begin{table}
	\begin{tabular}{lllll}
		Letter & Next letter & IPA & Examples & Translations \\
		b & a o e i u y ų  & [b] && \\
		b & ä ö ë ï ü & [bj] && \\
		v & a o e i u y ų & [v] && \\
		v & ä ö ë ï ü & [vj] && \\
		g & a o e i u y ų  & && \\
		g & ä ö ë ï ü &&& \\
		ĝ & a o e i u y ų  & [g] && \\
		ĝ & ä ö ë ï ü & [gj] && \\
		d & a o e i u y ų & [d] && \\
		d & ä ö ë ï ü &&& \\	
		đ & a o e i u y ų  &&& \\
		đ & ä ö ë ï ü &&& \\
		ŝ & a o e i u y ų  &&& \\
		ŝ & ä ö ë ï ü &&& \\
		k & a o e i u y ų &&& \\  
		k & ä ö ë ï ü  &&& \\ 
		l & a o e i u y ų  &&& \\  
		l & ä ö ë ï ü  &&& \\ 
		m & a o e i u y ų  &&& \\  
		m & ä ö ë ï ü  &&& \\
		n & a o e i u y ų  &&& \\  
		n & ä ö ë ï ü  &&& \\
		p & a o e i u y ų &&& \\  
		p & ä ö ë ï ü  &&& \\ 
		r & a o e i u y ų  &&& \\  
		r & ä ö ë ï ü  &&& \\
		ř & a o e i u y ų  &&& \\  
		ř & ä ö ë ï ü  &&& \\ 
		s & a o e i u y ų &&& \\  
		s & ä ö ë ï ü  &&& \\ 
		t & a o e i u y ų  &&& \\ 
		t & ä ö ë ï ü  &&& \\ 
		ŧ & a o e i u y ų  &&& \\  
		ŧ & ä ö ë ï ü  &&& \\ 
		f & a o e i u y ų  &&& \\  
		f & ä ö ë ï ü  &&& \\
		h a o e i u y ų   &&& \\
		h ä ö ë ï ü  &&& \\
		c a o e i u y ų   &&& \\
		c ä ö ë ï ü  &&& \\
		č a o e i u y ų   &&& \\
		č ä ö ë ï ü  &&& \\
		š a o e i u y ų   &&& \\
		š ä ö ë ï ü  &&& \\		
	\end{tabular}
\end{table}

% table
%  Ь and Ј
% надо добавить, что в югославянских Ь, Й и Ј слились в Ј (в полесском только Й замещена Ј), что, впрочем имело параллекли и в глаголице (как болгарской, так и хорватской), где кроме отдельной буквы для Ь был значок "штапик/штапић" который соответствовал и Ь и Ј.
NOTE: Cyrillic has two different letters ... that have different functions - the first one defines that the previous consonant is soft (we need this in case vowel is absent) and the second defines a [\textctj] sound. Latin version that you see in the table has no such difference, so you should remember, that J means a soft symbol when you see a C-”J”-C row (where C is for “Consonant”) and means a [\textctj] sound when you see a C-”J”-V (where V is for “Vowel”), or use Cyrillic to prevent such a collision. Only in the first case consonant before J is soft while in the second one it is hard.
