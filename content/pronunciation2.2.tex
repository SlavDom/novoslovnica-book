\section{Pronunciation}

Novoslovnica is a phonetic language, that’s why Novoslovnica has an important rule, which you have to apply to speaking in Novoslovnica.

\textbf{Rule n. 3}: All words are pronounced as they are written.

This rule means that you cannot reduce sounds when speak in Novoslovnica. It is a very important thing because you can make mistakes if you speak improperly. There are some exceptions but they all will be mentioned in this guidebook.

When you pronounce a word you are not restricted to use only main sounds - if it’s more comfortable, you can pronounce allophones with the same level of softness and sonority with the main sound of the letter. Let’s look at the examples below to understand what we can choose in speaking and what we cannot.

\textbf{Examples:}

jsdasd

asdasd

However, you are restricted in what consonant sounds to use from the allophone list. You can see the next rule which will help you to speak.

\textbf{Rule n. 4}: You cannot mess soft and hard consonants when you pronounce a word.

This prohibit you to make hard consonants when you need a soft one, or to use a soft one when you need a hard one. Here you can see a table, where it is shown which sound you must pronounce in different combinations of letters.

\begin{longtable}{lllll}
%	\begin{tabular}{lllll}
		Letter & Next letter & IPA & Examples & Translations \\
		\endhead
		b & a o e i u y ų ě  & [b] & Banä & \\
		b & ä ö ë ï ü ę & [bj] & Büro & Bureau\\
		v & a o e i u y ų ě & [v] & Voda & Water \\
		v & ä ö ë ï ü ę & [vj] & Vëdro & \\
		g & a o e i u y ų ě & \textipa{[H]} & Godina & Hour \\
		g & ä ö ë ï ü ę & \textipa{[Hj]} & G & \\
		ĝ & a o e i u y ų ě & [g] & Ĝadslo & Password \\
		ĝ & ä ö ë ï ü ę & [gj] & Ĝöŧe & Goethe \\
		d & a o e i u y ų ě & [d] & Dom & House \\
		d & ä ö ë ï ü ę & \textipa{[\textbardotlessj]} & Düna & \\	
		đ & a o e i u y ų ě & \textipa{[\t{\:d\:z}]} & & \\
		đ & ä ö ë ï ü ę & \textipa{[\t{dZ}]} && \\
		ŝ & a o e i u y ų ě & \textipa{[\t{dz}]} && \\
		ŝ & ä ö ë ï ü ę & \textipa{[\t{dzj}]} && \\
		k & a o e i u y ų ě & [k] && \\  
		k & ä ö ë ï ü ę & [kj] && \\ 
		l & a o e i u y ų ě & [l] && \\  
		l & ä ö ë ï ü ę & \textipa{[L]} && \\ 
		m & a o e i u y ų ě & [m] && \\  
		m & ä ö ë ï ü ę & [mj] && \\
		n & a o e i u y ų ě & [n] && \\  
		n & ä ö ë ï ü ę & \textipa{[\textltailn]} && \\
		p & a o e i u y ų ě & [p] && \\  
		p & ä ö ë ï ü ę & [pj] && \\ 
		r & a o e i u y ų ě & [r] && \\  
		r & ä ö ë ï ü ę & [rj] && \\
		ř & a o e i u y ų ě & \textipa{[\r*r]} && \\  
		ř & ä ö ë ï ü ę & \textipa{[\r*rj]} && \\ 
		s & a o e i u y ų ě & [s] && \\  
		s & ä ö ë ï ü ę & [sj] && \\ 
		t & a o e i u y ų ě & [t] && \\ 
		t & ä ö ë ï ü ę & [c] && \\ 
		ŧ & a o e i u y ų ě & \textipa{[T]} && \\  
		ŧ & ä ö ë ï ü ę & \textipa{[Tj]} && \\ 
		f & a o e i u y ų ě & [f] && \\  
		f & ä ö ë ï ü ę & [fj] && \\
		h & a o e i u y ų ě & [h] && \\
		h & ä ö ë ï ü ę & [hj] && \\
		c & a o e i u y ų ě  & \textipa{[\t{ts}]} & & \\
		c & ä ö ë ï ü ę & \textipa{[\t{tsj}]} && \\
		č & a o e i u y ų ě  & \textipa{[\t{tS}]} && \\
		č & ä ö ë ï ü ę & \textipa{[\t{tSj}]} && \\
		š & a o e i u y ų ě  & \textipa{[\v{s}]} & & Neck \\
		š & ä ö ë ï ü ę & \textipa{[\v{s}j]} & & \\		
%	\end{tabular}
\end{longtable}
