\section{Person}

This grammar category determines the person who is spoken about. There are three points of view:

\begin{itemize}
	\item the point of the speaker (First person)
	\item the point of the interlocutor (Second person)
	\item the point of other persons, that are not involved into discussion (Third person)
\end{itemize}

That is how a Slavic discussion could be seen. Practically, this concept is similar to all European languages, particularly English. There is a total equivalency with English in Novoslovnica, so it is not necessary to describe the usage all of these person types. Just look at the following examples to get sure of it:

\textbf{Examples:}

\textit{Ja glědaju v prozorec cěl věčôr.} - I am looking outside the window for the whole evening. (The first person)

\textit{Vy kažete že sámo ïsto byše včera? }- Do you mean that the very same thing was yesterday? (The second person)

\textit{Ony hlåpcy niĝda ne mogut pjiti tïho.} - Those guys never can drink quitely. (The third person)
