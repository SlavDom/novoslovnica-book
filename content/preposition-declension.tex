\subsection{Analytical Declension}

Bulgarian and Macedonian languages involved a non-synthetic declension to our Slavic languages. Thus, we can use prepositions instead of cases to express the connections between words in a \gls{np}. In the following table you can see the correspondence between cases in Novoslovnica and the prepositions that are able to express the same connection.

\begin{table}[!htb]
	\begin{tabular}{ll}
		Case & Preposition \\
		Nominative & word itself \\
		Genitive & Na + Nominative \\
		Partitive & Od \\
		Accusative & Accusative \\
		Dative & Na + Accusative \\
		Instrumental & S \\
		Prepositional & Za \\
		Locative & V \\
		Vocative & word vocative
	\end{tabular}
\end{table}

\textbf{Examples:}

\textit{Čaška od čaǐ - Čaška čaju }- A cup of tea

\textit{Dati kniga na Natašu - Dati Natašě knigu} - Give the book to Natasha

\textit{Govoriti za lěs - Govoriti o lěsě} - Talking about forest

\textit{Znajomec na brat - Znajomec brata} - My brother's friend.

\textit{Jesòm v Měnsk - Jesòm v Měnsku} - I am in Minsk