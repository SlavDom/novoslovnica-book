\section{Phraseology}

\textbf{Ja mám něčto - Něčto je u menę}

These are two constructions we can use to describe an English phrase “I have something”.

As you see, the exact translation of the English phrase is the first variant. Here we use the verb “máti” (to have). However, there is the second variant in Novoslovnica too. In this variant we can watch the interchange between the actor and the object of an action by syntax roles. In “\textit{Ja mám něčto}” we see that the actor is the subject of this short sentence, and the action object has a syntax role of a direct object. In “\textit{Něčto je u menę}” we see, that the actor has a role of indirect object and the object has a subject syntax role.    

There are some connections between the second variant of the phrase and the English “There is/are” construction. In English we use that when the object is located in some place so as in the second variant (if we change the actor (indirect object) with the adverbial (or indirect object of location)). In Novoslovnica we still can use this construction with the soulful objects such as personal pronouns or animated nouns. 

\textbf{Jesòm někto - Jesòm někym}

You can notice that there are different construction of translation a phrase “to be somebody” in Novoslovnica. These two variants, with normal form \footnote{The term of "normal form" is rather virtual though it refers to a Nominative case. However, we know Nominative can be used only with subjects. In \textit{Jesòm někto} "někto" plays the role of an object, so it cannot be Nominative in the term of cases. So-called \textit{Normal form} indicates an analytic affect on Slavic languages. On one hand is a direct object so no prepositions are used. On the other, no case is used. We can correlate this one with infinitive as a "normal form" of the verb.} and Instrumentative, are semantically equal. 

The accent of using these two forms lays in the linguistic formalism that you use in your speech. Using a normal form is more analytic while using Instrumentative is more synthetic.

Nevertheless, they both can be used in speech.


\textbf{...ili... - ...ali...}

\textbf{<nothing> - kakto}

