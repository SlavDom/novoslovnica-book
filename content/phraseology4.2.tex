\section{Phraseology}

Ja mám něčto - Něčto je u menę

There are two constructions that we can use to describe an English variant “I have something”.
As you see, the exact translation of the English phrase is the first variant. Here we use the verb “máti” (to have). However, there is the second variant in Novoslovnica too. In this variant we can watch the interchange between the actor and the object of an action by syntax roles. In “Ja mám něčto” we see that the actor is the subject of this short sentence, and the action object is has a syntax role as a direct object. In “Něčto je u menę” we see, that the actor has a role of indirect object and the object has a subject syntax role.    

There are some connections between the second variant of the phrase and the English “There is/are” construction. In English we use this when the object is situated in some place, so as in the second variant (if we change the actor (indirect object) with the adverbial (or indirect object of location). In Novoslovnica we still can use this construction with the soulful objects such as personal pronouns or animated nouns. 

Jesòm někto - Jesòm někym

You can notice that there are different construction of translation a phrase “to be somebody” in Novoslovnica. These two variants, with normal form and Instrumentative, are equal. 
The accent of using these two forms lays in the linguistic formalism that you use in your speech. Using a normal form instance of cases (Remember, that you cannot use Nominative itself in the non-Subject roles in your sentence) is more analytic while using Instrumentative is more synthetic.
Nevertheless, they both can be used in speech.
