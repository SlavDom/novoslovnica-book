\section{Compound Sentence}

A compound\index{sentence!compound} sentence refers to a sentence that consists of two (or more) independent clauses in it. There are connected to one another with \textit{coordinating} link.

Coordinating sentences describe the relationship between the clauses with no dependency in it. Coordinating sentences are divided into three categories: \textit{copulative}, \textit{adversative} and \textit{consecutive}. These types of sentences are connected with the so-called types of conjunctions or adverbs. The exception is consecutive sentences where no auxiliary word is used.
 
Following examples shows the usage of coordinate sentences:

Examples:



\section{Complex Sentence}

A complex\index{sentence!complex} sentence refers to a sentence that consists of one (or more) dependent clauses connected to the single main one.

A dependent\index{clause!dependent} clause alone cannot stand for a complete sentence. The only way to form a complete one is to be connected to an independent clause and to form a complex sentence. Independent clauses are connected to the independent one with subordinate, conditional or relative conjunctions.

Thus, subordinating sentences are used to describe a relationship between parts with a dependency from one main clause (superordinate) to other subordinate.

Examples:

A conditional\index{clause!conditional} sentence describes something that is possible or probable.

Examples:

A relative\index{clause!relative} clause is used for description of a part of the main clause.

The dependent clauses can go first followed by the independent one, and conversely.

