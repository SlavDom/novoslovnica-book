\section{Compound Sentence}

A compound\index{sentence!compound} sentence refers to a sentence that consists of two (or more) independent clauses in it. They are connected to one another with \textit{coordinating} link.

Coordinating sentences describe the relationship between the clauses with no dependency in it. Coordinating sentences are divided into three categories: \textit{copulative}, \textit{adversative} and \textit{consecutive}. These types of sentences are connected with the so-called types of conjunctions or adverbs. The only exception is consecutive sentences where no auxiliary word is used.
 
Following examples show the usage of coordinate sentences:

\textbf{Examples:}

\textit{Otec mi stroji búdu, i materj mi govori o tom dobro.} - My father is building a house, and my mother says great about it.

\textit{On uči anĝliǐskyǐ jazyk četyrě ročiny, ale ja učaju samo dva dnä.} - He has been studying English for four years, but I have been studying it only for two days.

\section{Complex Sentence}

A complex\index{sentence!complex} sentence refers to a sentence that consists of one (or more) dependent clauses connected to the single main one.

A dependent\index{clause!dependent} clause alone cannot stand for a complete sentence. The only way to form a complete one is to connect the dependent clause to an independent one and to form a complex sentence. Independent clauses are connected to the independent one with subordinate, conditional or relative conjunctions.

Thus, subordinating sentences are used to describe a relationship between sentence parts with a dependency from one main clause (superordinate) to other subordinate.

\textbf{Examples:}

\textit{On råzumi, že to ne je naǐ-lěpym vyrěšenëm} - He understands that it is not the best decision.

\textit{Mysläm, če ty ne hte máš uspěha v tom} - I think you will not succeed in it.

A conditional\index{clause!conditional} sentence describes something that is possible or probable.

\textbf{Examples:}

\textit{Ja bųdu lěkarom, dy hte zakončim vysšojučilišto} - I will be a teacher when I graduate from university

\textit{Ako ne sę hybam, to bųde trudno} - If I am right, this will be difficult.

A relative\index{clause!relative} clause is used for description of a part of the main clause.

\textbf{Examples:}

\textit{Ja uvidih děvušku, ktoru byh sretil včera v parku} - I saw a girl I meet yesterday in the park.

\textit{Môǐ gråd, v ktorom jesòm sę rodil, je vëlïk} - My city, I was born in, is large.

The dependent clauses can go first followed by the independent one, and conversely.

