\chapter{Foreword}

Every language is alive - it is born, it grows up, it bares children and it dies.

The process of developing a language is unstoppable and it is cause by a lot of factors we cannot track.

What a beauty is a moment a new language appears. A lot of languages are old and appeared many years ago. However, there exist a very young ones were witnesses to be born i.e. Afrikaans (20 century), Yiddish (19 century) and extinct i.e. Prussian (18 century), Slovinian (20 century), Livonian (2013 year).

The language is naturally born doubly:

\begin{itemize}
	\item by dialect divergence
	\item by two languages pidginating 
\end{itemize}

The first way is much popular, though in pure form they both appear pretty rarely. This leads to differentiating of languages into a close language group.

Nevertheless, people sometime get into idea of reuniting these groups into a single language. This is not a natural way but we have examples of such a reunion. For example, modern German language was created from a group of different german languages and dialects to unite people into a single strong country. And they succeeded, so today we ahve a stable literary German language (though some region dialects still exist).

Moreover, people want to unite languages that have diverged much greater than German dialects did. It is rather enviable, though so many examples took place (Volapük, Novial). However, one of them, called Esperanto created by L.L. Zamenhof in 1887 year was very successful. Today more than 2 million people speak Esperanto and several thousand of them use it as a native language.

These examples show that the languages develop both ways - unification and divergence.

Novoslovnica as such a bridge for Slavic languages is working in direction of uniting the language. Collected expressiveness, beauty and purity of all Slavic languages, it tries to rework the idea of what s Slavic language should be.

It is not the only project in this sphere. Starting from Cyril and Methodius, Slavs continue to think about the idea of a common language when loosing the previous project in mind.

The need of a secular common language for Slavs is especially important since the twentieth century. The process of globalization is destroying weak languages and washing out a lot of root nests from others.

Novoslovnica in a row with other Slavic auxiliary language projects try to create something that Slavs could use to struggle the globalization process.

This book is for those who is interested in such an idea of a Slavic language reunion. Moreover, it reveals some historic features of our languages.

Hope you will enjoy reading this book and get something important for yourself when you finish it.

Novoslovnica is a standardized language project with strict rules in it that make the language able to be codified. Hope this will be useful for the interslavic community in developing the bridge between Slavic languages.

Whatever would be, anything you find useful in this book will make it worth writing, my dear reader.

You are welcome to write me all your wishes, comments and marks via email: support@novoslovnica.com

Best wishes,

George Carpow