\section{Number}

How do people understand what is the numeric characteristic of the object? Of course, the simpliest way is to use numerals. We can call to a numeral and link it with some noun, thus, people will understand that there is an amount shown by a numeral of the concept shown by a noun. But it is rather uncomfortable to use near every noun an additional numeral. Mostly due to the fact we seldom know the exact amount of something. That is why there exists a term of number.

Number\index{number} shows what is the amount of some concept without using numeral before the noun.

Number is a grammar property of the word, its alteration. That means when we change number of the word, we change the word itself and not add some additional words or particles around the word we are speaking about.

There are three numbers in Novoslovnica: \textit{singular}, \textit{dual} and \textit{plural}. Singular\index{number!singular} and plural\index{number!plural} are familiar to an English speaker. They show whether the object is single or not. Dual is rather peculiar \footnote{Using non-boolean logic in a number property is rather specific. Triple logic (Single-Dual-Plural) can be found in some derived from Indo-European languages: Sanscrit, Slavic, though other branches lost the dual number in ancient time (Greek, Latin, German, Baltic); Arabic, Hebrew, Khoe languages also have triple number logic. Quadruple logic (Single-Dual-Triple-Plural) can be found i.e. in Tok-Pisin. Sursurunga is famous for having a five-way grammatical number distinction.}, so we should take an additional account on it.

A dual\index{number!dual} number \footnote{In modern Slavic languages Slovenian, Upper- and Lower-Sorbian languages still have a dual number.} is used when we speak about a pair of something - hands, legs, eyes etc. of one person. Two-doors gates, two boolean values, two antipodes etc. In these cases we use a dual number. Having a pair is a rather frequent fact, so this number appeared in Proto-Indo-European. Dual number so as plural one depends on the word which is spoken so we cannot determine a static rule about choosing a form for a dual number. We can get a dual form of the word by using a declension function with a type of declension corresponding with the word given as a parameter.

There also exists an additional number form we should speak about that is so-called as a counting form. A counting form is used with the nouns. It only occurs when we use the \underline{noun} \textbf{with} the numerals “\underline{two}”, “\underline{three}” or “\underline{four}” (cardinal numbers). The counting form is equal to a dual number in writing so we can speak of it as an extension of a dual number usage.

\textbf{Examples:}

\textit{Dva doma} (two houses) - counting form

\textit{Doma} (two ones) - dual number

\textit{Tri doma} (three houses) - counting form

\textit{Domy} (three ones) - plural number 
