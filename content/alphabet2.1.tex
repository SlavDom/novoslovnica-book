\section{Alphabet}

Let’s summarize what we have known about Novoslovnica phonology. Afterwards we will get the list of phonemes and allophones and their connections with Novoslovnica letters in the alphabet.

We have learned that Novoslovnica has 51 consonant sounds and 22 vowels. 13 consonants and 4 vowels are allophones among them. Hence, the amount of phonemes is (51-13) + (22 - 4) = 56 phonemes.

You should know that in Novoslovnica, soft and hard consonants do not differ in writing. That is because of the fact that by the combination of “consonant + vowel” we can always determinedly get what the consonant is like - hard or soft. With this information, the amount of letters needed is reduced to 49 \cite{nsl-alphabet}.

Nevertheless, let’s now look at the table with the alphabet list and see how Novoslovnica is written.

% \begin{table}
	\begin{longtable}{llllp{4em}p{6em}}
		No. & Char & Cyr & Phoneme & Allophone & Examples in English \\
		\endhead
		1 & A a & А а & \textipa{[a]} & & Ex\textbf{a}mple, p\textbf{a}st (British forms) \\
		2 & Ä ä & Я я & \textipa{[\ae]} &  & C\textbf{a}t, r\textbf{a}t (American English) \\
		3 & Á á & Ā ā & \textipa{[a:]} & \textipa{[A]} & M\textbf{a}rk, p\textbf{a}rk \\
		4 & Å å & Å å & \textipa{[2]} & & C\textbf{u}t, w\textbf{o}nder \\
		5 & B b & Б б & \textipa{[b]} & \textipa{[bj]} & \textbf{B}etter, \textbf{b}ear \\
		6 & C c & Ц ц & \textipa{[\t{ts}]} & \textipa{[\t{ts}j]} & \textbf{T}ea, \textbf{t}eam (In some American or British dialects) \\
		7 & Č č & Ч ч & \textipa{[\t{tS}]} & \textipa{[\t{tC}], [\t{t\:s}]} & \textbf{Ch}eese, \textbf{ch}eck \\
		8 & D d & Д д & \textipa{[d]} & \textipa{[\textbardotlessj]} & \textbf{D}o, \textbf{d}inner \\
		9 & Đ đ & Џ, џ & \textipa{[\t{\:d\:z}]} & \textipa{[\t{d\textctz}], [\t{dZ}]} & \textbf{J}ohn, \textbf{J}une \\
		10 & E e & Е е & \textipa{[E]} & & P\textbf{e}t, s\textbf{e}t \\
		11 & Ë ë & Є є & \textipa{[\|`e]} & & M\textbf{e}n\textbf{e}n (Finnish), sul\textbf{e} (Estonian) \\
		12 & É é & Ē ē & \textipa{[E:]} & \textipa{[3]} & \textbf{Э}му (Russian) \\
		13 & Ě ě & Ѣ ѣ & \textipa{[e]} & \textipa{[I]} & Medv\textbf{e}d (Slovenian), V\textbf{e}k (Slovenian) \\
		14 & Ę ę & \cyrsyus & \textipa{[\~E]} & \textipa{[eN]} & Si\textbf{ę} (Polish), V\textbf{inght} (French) \\
		15 & F f & Ф ф & \textipa{[f]} & \textipa{[fj], [\r*U], [\r*Uj]} & \textbf{F}eather, \textbf{ph}one \\
		16 & G g & Г г & \textipa{[H]} & \textipa{[G], [Gj], [Hj]} & \textbf{H}orn, be\textbf{h}ind (Australian English) \\
		17 & Ĝ đ & \CYRGUP \cyrgup & \textipa{[g]} & \textipa{[gj]} & \textbf{G}o, \textbf{g}row \\
		18 & H h & Х х & \textipa{[x]} & \textipa{[xj], [h], [hj]} & \textbf{H}air, \textbf{h}orror \\
		19 & I i & И и & \textipa{[I]} & & K\textbf{i}tten, r\textbf{i}d \\
		20 & Ï ï & I i & \textipa{[i]} & & M\textbf{ee}t, s\textbf{ea}t, pol\textbf{i}ce \\
		21 & Ǐ ǐ & Й й & \textipa{[j]}  & & M\textbf{y}, t\textbf{ie} (the last part of the diphthong) \\
		22 & Į į & Į į (\cyryn) & \textipa{[\~E]} & \textipa{[iN]} & Even\textbf{ing}, morn\textbf{ing} \\
		23 & J j & J j (ь)& \textipa{[J]} & & \textbf{Y}ogurt, \textbf{y}ard \\
		24 & K k & К к & \textipa{[k]} & & \textbf{C}alm, po\textbf{ck}et \\
		25 & L l & Л л & \textipa{[l]} & \textipa{[\r*l]} & \textbf{L}emon, c\textbf{l}imate \\
		26 & M m & М м & \textipa{[m]} & & \textbf{M}onth, \textbf{m}e\textbf{m}ory \\
		27 & N n & Н н & \textipa{[n]} & & \textbf{N}ou\textbf{n}, coi\textbf{n} \\
		28 & O o & О о & \textipa{[o]} & & C\textbf{o}tton \\
		29 & Ö ö & Ё ё & \textipa{[8]} & & B\textbf{i}rd, s\textbf{i}r \\
		30 & Ó ó & Ō ō & \textipa{[o:]} & & P\textbf{o}le \\
		31 & Ò ò & Ъ ъ & \textipa{[@]} & & Kingd\textbf{o}m, ketch\textbf{u}p  \\
		32 & Ô ô & Ô ô & \textipa{[\|`o]} & & P\textbf{oo}l, g\textbf{oo}d \\
		33 & P p & П п & \textipa{[p]} & & \textbf{P}ear, swee\textbf{p} \\
		34 & R r & Р р & \textipa{[r]} & & \textbf{R}ace, pa\textbf{r}ent (Scottish) \\
		35 & Ř ř & \cyrrz & \textipa{[\r*r]} & & \textbf{Ř}eka (Czech) \\
		36 & S s & С с & \textipa{[s]} & & Pre\textbf{ss}, co\textbf{s}tume \\
		37 & Š š & Ш ш & \textipa{[\v{s}]} && \textbf{Sh}ine, mu\textbf{sh}room \\
		38 & Ŝ ŝ & Ѕ ѕ & \textipa{[\t{dz}]} & & \textbf{D}ay, dime (In some American or British dialects)  \\
		39 & T t & Т т & \textipa{[t]} & & \textbf{T}odd, \textbf{t}orture \\
		40 & Ŧ ŧ & \CYROTLD   \cyrotld & \textipa{[T]} & & \textbf{Th}rone, heal\textbf{th} \\
		41 & U u & У у & \textipa{[u]} & & P\textbf{u}t, \\
		42 & Ü ü & Ю ю & \textipa{[0]} & & P\textbf{u}re, c\textbf{u}te \\
		43 & Ú ú & Ӯ ӯ & \textipa{[u:]} & & P\textbf{oo}r \\
		44 & Ŭ ŭ & Ў ў & \textipa{[w]} & & \textbf{W}onder, \textbf{w}ay \\
		45 & Ų ų & Ѫ ѫ & \textipa{\~o} & \textipa{[uN], [oN], [aN], [\~a], [\~u]} & W\textbf{ą}ż (Polish), S\textbf{ont} (French) \\
		46 & V v & В в & \textipa{[v]} &\textipa{[vj], [V], [Vj]} & \textbf{V}al\textbf{v}e, \textbf{v}el\textbf{v}et \\
		47 & Y y & Ы ы & \textipa{[1]} & & С\textbf{ы}р (Russian) \\
		48 & Z z & З з & \textipa{[z]} & & \textbf{Z}one, \textbf{z}oo \\
		49 & Ž ž & Ж ж & \textipa{[\:z]} &  \textipa{[Z]} & Clo\textbf{s}ure, mea\textbf{s}ure \\
	\end{longtable}
% \end{table}

Note, that the Cyrillic letters Щ, $\Psi$, Ќ, Ї in earlier versions of Novoslovnica have been replaced by Шт (Št), Пс (Ps), Кс (Ks) and Ји (Jі).\footnote{Cyrillic has two different letters \textit{Ь} and \textit{J} that have different functions - the first one defines that the previous consonant is soft (we need this in case vowel is absent) and the second defines a [\textctj] sound. Latin version that you see in the table has no such difference, so you should remember, that J means a soft symbol when you see a C-”J”-C row (where C is for “Consonant”) and means a [\textctj] sound when you see a C-”J”-V (where V is for “Vowel”), or use Cyrillic to prevent such a collision. Only in the first case consonant before J is soft while in the second one it is hard.}
% надо добавить, что в югославянских Ь, Й и Ј слились в Ј (в полесском только Й замещена Ј), что, впрочем имело параллекли и в глаголице (как болгарской, так и хорватской), где кроме отдельной буквы для Ь был значок "штапик/штапић" который соответствовал и Ь и Ј.
