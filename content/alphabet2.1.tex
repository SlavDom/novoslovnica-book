\section{Alphabet}

Let’s summarize what we have known about Novoslovnica phonology. Afterwards we will get the list of phonemes and allophones and their connections with Novoslovnica letters in the alphabet.

We have learned that Novoslovnica has 51 consonant sounds and 22 vowels. 13 consonants and 4 vowels are allophones among them. Hence, the amount of phonemes is (51-13) + (22 - 4) = 56 phonemes.

You should know that in Novoslovnica, soft and hard consonants do not differ in writing. That is because of the fact that by the combination of “consonant + vowel” we can always determinedly get what the consonant is like - hard or soft. With this information, the amount of letters needed is reduced to 49 \cite{nsl-alphabet}.

Nevertheless, let’s now look at the table with the alphabet list and see how Novoslovnica is written.

% \begin{table}
	\begin{longtable}{lllp{4em}p{9em}}
		No. & Char & Phoneme & Allophone & Examples in English \\
		\endhead
		1 & A a & \textipa{[a]} & & Ex\textbf{a}mple, p\textbf{a}st (British forms) \\
		2 & Ä ä & \textipa{[\ae]} &  & Cat, rat (American English) \\
		3 & Á á & \textipa{[a:]} & \textipa{[A]} & M\textbf{a}rk, p\textbf{a}rk \\
		4 & Å å & \textipa{[2]} & & C\textbf{u}t, w\textbf{o}nder \\
		5 & B b & \textipa{[b]} & \textipa{[bj]} & \textbf{B}etter, \textbf{b}ear \\
		6 & C c & \textipa{[\t{ts}]} & \textipa{[\t{ts}j]} & Tea, team (In some American or British dialects) \\
		7 & Č č & \textipa{[\t{tS}]} & \textipa{[\t{tC}], [\t{t\:s}]} & Cheese, check \\
		8 & D d & \textipa{[d]} & \textipa{[]} & Do, dinner \\
		9 & Đ đ & \textipa{[\t{\:d\:z}]} & & John, June \\
		10 & E e & \textipa{[E]} & & Pet, set \\
		11 & Ë ë & \textipa{[\|`e]} & & \\
		12 & É é & \textipa{[E:]} & & \\
		13 & Ě ě & \textipa{[e]} & & \\
		14 & Ę ę & \textipa{[eN]} & & \\
		15 & F f & \textipa{[f]} & & Feather, phone \\
		16 & G g & \textipa{[H]} & \textipa{[G], [Gj], [Hj]} & Horn, behind (Australian English) \\
		17 & Ĝ đ & \textipa{[g]} & \textipa{[gj]} & Go, grow \\
		18 & H h & \textipa{[x]} & & Hair, horror \\
		19 & I i & \textipa{[I]} & & Kitten, rid \\
		20 & Ï ï & \textipa{[1]} & & Meet, seat \\
		21 & Ǐ ǐ & \textipa{[j]}  & & My, tie (the last part of the diphthong) \\
		22 & Į į & \textipa{[iN]} & & Evening, morning \\
		23 & J j & \textipa{[J]} & & Yogurt, yard \\
		24 & K k & \textipa{[k]} & & Calm. pocket \\
		25 & L l & \textipa{[l]} & & Lemon, climate \\
		26 & M m & \textipa{[m]} & & Month, memory \\
		27 & N n & \textipa{[n]} & & Noun, coin \\
		28 & O o & \textipa{[o]} & & Cotton \\
		29 & Ö ö & \textipa{[8]} & & Bird, sir \\
		30 & Ó ó & \textipa{[o:]} & & Pole \\
		31 & Ò ò & \textipa{[@]} & & \\
		32 & Ô ô & \textipa{[\|`o]} & & Pool, good \\
		33 & P p & \textipa{[p]} & & Pear, sweep \\
		34 & R r & \textipa{[r]} & & Race, parent \\
		35 & Ř ř & \textipa{[\r*r]} & & \\
		36 & S s & \textipa{[s]} & & Press, costume \\
		37 & Š š & \textipa{[\v{s}]} && Shine, mushroom \\
		38 & Ŝ ŝ & \textipa{[\t{dz}]} & & Day, dime (In some American or British dialects)  \\
		39 & T t & \textipa{[t]} & & Todd, torture \\
		40 & Ŧ ŧ & \textipa{[T]} & & Throne, health \\
		41 & U u & \textipa{[u]} & & Put, \\
		42 & Ü ü & \textipa{[0]} & & Pure, cute \\
		43 & Ú ú & \textipa{[u:]} & & Poor \\
		44 & Ŭ ŭ & \textipa{[w]} & & Wonder, way \\
		45 & Ų ų & \textipa{[uN]} & & \\
		46 & V v & \textipa{[v]} &\textipa{[vj], [V], [Vj]} & \textbf{V}al\textbf{v}e, \textbf{v}el\textbf{v}et \\
		47 & Y y & \textipa{[I]} & & \\
		48 & Z z & \textipa{[z]} & & Zone, zoo \\
		49 & Ž ž & \textipa{[\:z]} & & Closure, measure \\
	\end{longtable}
% \end{table}