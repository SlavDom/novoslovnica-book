\section{Article}

If you tried to learn a Slavic language, you would notice that there are no articles\index{article} in it. The term of definiteness\index{definiteness} is ruled mostly by demonstrative pronouns. However, there are two Slavic languages that provide this possibility - Bulgarian\footnote{With Pomak dialects that are considered to be a separate language by some scientists.} and Macedonian: these are Slavic analytic marvel (a pair of languages, that has worked out a completely different model of grammar). Novoslovnica let you use this achievement. Going further, we will compare an English and a Slavic systems of articles.

Firstly, you should remember that there is no indefinite article. The word itself shows that you are speaking about a designatum and there is no concrete information about it, no details. However, if you want to accentuate the term of indefiniteness you can use an indefinite pronoun before the word (“Někyǐ”, “Něktoryǐ”).

Secondly, there are huge differences between definite articles\index{article!definite} in English and in Novoslovnica. English has only one article - “the”. It shows only the term of definiteness. Novoslovnica provides in articles additional meanings, which you have already seen in the paragraph about demonstrative pronouns - distance and visibility of the object. So, there are three definite articles in Novoslovnica: “-òt”, “-òn”, “-òv”. Also you should know that these articles have the differentiation by gender and number. 

òt – ta – to – te

òv – va – vo – ve

òn – na – no – ne

Despite of the other POS articles have only two numbers (singular and plural) and are differentiated by gender only in singular. Words in dual manage the article in plural. To remind you, we will repeat that the article “òv” is used, when the object is in the field of view and it is rather close to you. The article “òn” is used, when you still see the object, but it is rather far from you. The article “òt” is used, when you cannot see the object or it is abstract, however you are talking about the definite instance of the designatum.

English article "the" is derived from a demonstrative\index{pronoun!demonstrative} pronoun "that". Novoslovnica articles are derived as it follows: \textit{òn} from \textit{onyǐ}, \textit{òv} - from \textit{òvyǐ} and \textit{òt} - from \textit{tyǐ}. Nonetheless, you can see that the pronoun “sïǐ” has not developed into the article and there is no article equivalent.

Finally, there are differences in the way to use the article. In English you place the article just before the word, dividing the word and the article by the space (different words). In Novoslovnica you put the article just after the word, dividing the word and the article by the hyphen (one word).

\textbf{Examples:} 

\textit{Kniga-na je na stolu} - The book is on table.

\textit{Dom-òv je mnogo starym} - This house is very old.

\subsection{Adjective article}

Articles in Novoslovnica are used with nouns only. However, there are tools for expressing definiteness of adjectives.

Adjective article is a form of using short personal pronouns with adjectives.\footnote{A similar phenomenon is used nowadays in Macedonian to express the definiteness of an object of speaking. However, the "double object" (which stands for a personal pronoun form here) is placed right before the predicate.}

\textbf{Examples:}

Dobòr - Dobòr \textbf{i} 

Dobra - Dobra \textbf{ja} 

Dobro - Dobro \textbf{je} 

You can use such an old form. Though, better you a full adjective form which has been derived from adjective articles in East Slavic languages.

\textbf{Examples:}

Dobòr \textbf{i} - Dobryǐ

Dobra \textbf{ja} - Dobraja

Dobro \textbf{je} - Dobroje

Follow the next rules of using articles:

\begin{itemize}
	\item Use only one definite form in a \gls{np}.
	\item If there are several adjectives in a definite \gls{np}, use a full adjective only with the first one.
	\item If the full form of an adjective is equal to the short one (Due to cases), you should use an article on the noun in \gls{np} or an adjective article itself (in a split form).
\end{itemize}