\section{Article}

If you tried to learn a Slavic language, you would notice that there is no articles. The term of definiteness is ruled mostly by demonstrative pronouns. However, there are two Slavic languages that provide this possibility - Bulgarian and Macedonian: these are Slavic analytic marvel (a pair of languages, that has worked out an completely different model of grammar). Novoslovnica let you use this achievement. So, now there is a question: what do articles in Novoslovnica look like and what are the differences between English articles?

Firstly, you should remember that there is no indefinite article. The word itself shows that you are speaking about a designatum and there is no concrete information about it, no details. However, if you want to accentuate the term of indefiniteness you can use an indefinite pronoun before the word (“Někyǐ”, “Něktoryǐ”).

Secondly, there are huge differences between definite articles in English and in Novoslovnica. English has only one article - “the”. It shows only the term of definiteness. Novoslovnica provides in articles additional meanings, which you have already seen in the paragraph about demonstrative pronouns - distance and visibility of the object. So, there are three definite articles in Novoslovnica: “-òt”, “-òn”, “-òv”. Also you should know that these articles have the differentiation by gender and number. 

òt – ta – to – te

òv – va – vo – ve

òn – na – no – ne

Despite other POS articles have only two number and alternate by gender only in singular. Words in dual rule the article in plural. I will remind you, that the article “òv” is used, when the object is in the field of view and it is rather close to you. The article “òn” is used, when you still see the object, but it is rather far from you. The article “òt” is used, when you cannot see the object or it is abstract, however you are talking about the definite instance of the designatum.

Finally, there are differences in the way to use the article. In English you place the article just before the word, dividing the word and the article by the space (different words). In Novoslovnica you put the article just after the word, dividing the word and the article by the hyphen (one word).

Examples: 


Articles in Novoslovnica, so as in English, have been derived from demonstrative pronouns (“tyǐ”, “ovyǐ”, “onyǐ”). Nonetheless, you can see that the pronoun “sïǐ” has not developed into the article.
