\section{Interslavic background}

We all know that the idea of a common language for the Slavs hovers among the Slavic peoples for more than a Millennium. The first and only successful project to date was the Church Slavonic language of Cyril and Methodius. This language has successfully found its niche in religious use and is used with some changes to the present day. However, such attempts for the secular language were resumed only in the 17th century in the works \cite{krizhanich} \cite{matija}. Moreover, the working project was neither created nor introduced into society, while some similar projects of planned languages for romance languages have successfully taken root and found a wide response in society (Esperanto, Interlingua).

The revival of the idea of the inter-Slavic language in our days occurred in 1999, when mark Gucco from Slovakia created the language of Slovio. This language has a huge number of shortcomings and is currently not recognized as capable by any of the developers of the inter-Slavic project, but it served as a catalyst for the emergence of a large number of such projects and the development of the idea of creating a planned inter-Slavic language that could meet the needs of modern society. 

After that, more than 20 similar projects, more or less developed, appeared in 10 years. In 2006 there was a project slavianski, created (Ondrej Rečnik and Gabriel Svoboda). This language was developed in parallel in different degrees of detail, he put his ideas Jan van Steenbergen and Igor Polyakov. In 2009, the project broke away from the project Slovioski (Steeven Radzikowski, Andrej Moraczewski and Michal Borovička), which tried to unite the ideas of Slovio and Slovyanski. However, in 2010, all these projects were merged into one under the name Interslavic (see below - Interslavic-2).

At the same time, the Czech programmer Vojtěch Merunka published under the name of Neoslavonic. This project suggested the idea of how the Church Slavonic language could develop in free development. In 2011, these two projects began cooperative work on a common case. (see figure 1) then, this project has taken a leadership position on the issue of building medullablastoma language. Only in 2012 there was one project of “Northern Slavic language” (Venedčyna), presented by Nikolai Kuznetsov, who then joined Interslavic, and in 2014 there was a project of Novoslovnitsa, presented by George Carpow and the development team mainly from the countries of Eastern and southern Slavs, which at the moment remains the only living independent project outside the project Interslavic.

Interslavic introduced the idea of flavorisation, which resulted in a valid language differentiation of dialects and spellings and the lack of codification. In 2017, the project Neoslavonic and Interslavic merged into a new project called interslavic-2, which was a compromise between the ideas of Vojtěch Merunka and Jan van Steenbergen (who headed the project interslavic-1). They presented their new ideas in the article \cite{interslavic-2}, which was published in July 2017.