\section{Preposition}

Prepositions\index{preposition} also are not an independent POS. They are closely related to the main word (often it is a noun). Most prepositions show on the direction or on the location of the action. 

We can divide prepositions on two main groups, so as we did with adverbs. We can distinguish primary and secondary prepositions. Primary prepositions are very ancient and we cannot refer to any word form they have formed from. Secondary prepositions are longer, and they appeared by semantic shift of an adverb, transgressive or a cased-noun. 

Here I will list primary prepositions with English translations and controlled cases. Complex cases I will comment in brackets.

\textit{Bez} (Gen.) - Without

\textit{V} (Acc.) - In, into

\textit{Dlä} (Gen) - For

\textit{Do} (Gen.) - To

\textit{Za} (Instr.) - For

\textit{Iz} (Gen.) - From (inside the object)

\textit{K} (Dat.) - To

\textit{Krôz} (Skrôz) (Gen.) - Through

\textit{Na} (Acc.) - On

\textit{Nad} (Instr.) - Above

\textit{O} (Prep.) - About

\textit{Od} (Gen.) - From (the object)

\textit{Po} (Dat.) - Along

\textit{Pod} (Instr.) - Under

\textit{Pri} (Loc.) - At

\textit{Pro} (Acc.) - About (the difference between “O” and “Pro” is in the detail view on the object. When we say the second variant we just mention the object in our speech, while using the first one we talk about it in details).

\textit{S} (Instr.) - With

\textit{U} (Gen.) - At (the difference between “Pri” and “U” is in the object of speaking. When we use “U” we mention real object in space and place the object of speaking near it. “Pri” is used when we speak about proximity in time, i.e. some events are close to each other.)

\textit{Črez} (Acc.) - After,  in (time)
