\section{Clause}

A clause is the smallest grammatical unit that represents a complete proposition. An ordinary clause consists of a NP and VP.

NP, or Noun (Nominal) Phrase, is a construction into which nouns most commonly enter, and of which they are the head word. \cite{lang-dict}

VP, or Verb Phrase, has two senses \cite{lang-dict}. 

The term verb phrase is used in two senses. Traditionally, it refers to a group
of verbs which together have the same syntactic function as a single verb,
e.g. is coming, may be coming, get up to. In such phrases (verbal groups, verbal
clusters), one verb is the main verb (a lexical verb) and the others are subordinate to it (auxiliary verbs, catenative verbs). A verb followed by a non-verbal
particle (similar in form to a preposition or adverb) is generally referred to as
a phrasal verb.
In generative grammar, the verb phrase (VP) has a much broader definition,
being equivalent to the whole of the predicate of a sentence, as is clear from the
expansion of S as NP+VP in phrase-structure grammar. In the minimalist
programme, the head of the upper vp shell is referred to as little v.

a subject and a predicate, sometimes with additional auxiliary members. Predicate usually is composed of a verb with adverbial modifiers.

The subject determines the concept which is the actor in the sentence and the predicate determines the action which the actor is connected with.



