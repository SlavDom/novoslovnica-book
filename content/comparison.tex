\section{Comparison with other projects}

We can provide a little concept comparison of other interslavic projects: Neoslavonic \cite{neoslavonic} and Interslavic \cite{interslavic}.

These projects had been developed before Novoslovnica appeared. Each of the projects has its own flavor or aim:

\begin{itemize}
	\item Neoslavonic represents the idea of the possible development of Old Church Slavonic up to nowadays.
	\item Interslavic-1 develops the idea of finding a common core of modern Slavic languages and take features and lexicon that is common to all modern Slavic languages
	\item Novoslovnica from this point of view develops the idea of creating the language that unites most grammar features assimilated by different Slavic languages along with etymologically developed lexicon with regard to modern days.
\end{itemize}

\begin{table}[!htb]
	\caption{Comparison Novoslovnica with other projects}
	\begin{tabular}{lp{5em}p{5em}p{5em}}
		Parameter & Novoslovnica & Interslavic & Neoslavonic \\
		Numbers & 3 & 2 & 2+ (Dual - optional) \\
		Genders & 3 & 3 & 3 \\
		Cases & 9 & 6+ (Vocative optional) & 7 \\
		Irr. verbs & 1 & + & + \\
		Verb conj. & 5 & 2 & 2 \\
		Past tenses & 4 & 3 & 2 (Perfect as a present tense)\\
		Inferential & + & - & - \\
		Voices & 3 & 3 & 3 \\
		Moods & 11 & 3 & 3 \\
		Supine & + & - & - \\
		Gerund & + & + & + \\
		Verbal noun & + & - & =gerund \\
		Transgressive & + & - & =adv.participle \\
		Vocabulary & Etymological & Common & Common + a few Etymological 
	\end{tabular}
\end{table}