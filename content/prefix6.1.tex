\section{Prefix}

Prefix is a part of a word (POW) that stands before the word root. However, one word can have more than one prefix. That means prefixes can be placed one by one right to the beginning of the root. Why should not we unite all of them into one Prefix?

The answer is in two points:
Firstly, it is more comfortable to determine different prefixes in words to make easier their determining in the other word.
Secondly, each prefix has its own semantic value and word changes its semantic color due to presence of prefixes.

We do not to consider the first point- it is obvious. This paragraph will consider only the second point. We should be able to distinguish different semantic values of the word. The paragraph will built on going through the list of morphemes and giving a description about each of them.

Also please consider that many prefixes are deeply connected with the prepositions. That is why their semantic values are equal and in the paragraph about preposition I will not mention those, that have equivalents within prefixes.

Bez

This prefix and its prepositional equivalent have an equal meaning of the preposition “without” in English. It means the absence of something that we have mentioned before. As a prefix, it has no equivalence in English within prefixes. Nevertheless, the suffix “-less” has a very similar meaning. Look at the example to get it.
Examples:
Bez doma - Without home
Bezdušnyǐ - Without soul/Soulless 

Vo

This prefix and its prepositional equivalent have a semantic value of “leading into something”. The preposition equals to “in” in English language.

Voz


Vy

This is an antonym of the previous character. It means “leading outside something. Speaking about prepositions, it has an equivalent word “out”.

Do

This prefix shows the destination of a process to some point. In English we can find an equivalent “to”. Look: “go to the wall” - “idati do stěny”

Za

The semantic value of this prefix is the aim of some action. We use it when we want to reach something, some object. English “for” can be found as revealing one of its semantic values in equal way. “go for the goal” - “idati za cělïü”.

Zad
This prefix has an artificial origin (This semantic value was divided from the previous prefix). It means placing an object behind another one. The equivalent is “behind”.

Iz

This prefix equals the English word “from”. “I am from Moscow” - “Ja sòm iz Moskvy”

Ka
Ko
Kų
Na
Naǐ
Nad
Ne
Ni
O
Ob
Od
Pa
Po
Pod
Pra
Pre
Pri
Pro
Råz
So
Sų
U
Črez
