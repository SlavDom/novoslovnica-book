\section{Prefix}

\newglossaryentry{pow}{name=POW, description={Part of word}}
\newglossaryentry{prep}{name=PREP, description={Preposition}}

Prefix\index{prefix} is a part of a word (\gls{pow}) that stands before the word stem. One word may have more than one prefix. In other words, prefixes can be added one by one right to the beginning of the stem. Each prefix has its own semantic value and word changes its meaning due to the presence of exact prefixes.

We can divide prefixes into several groups. 

\begin{itemize}
	\item Borrowed - prefixes that were borrowed from other languages (i.e. Latin, English etc.)
	\item Native - Slavic prefixes, that have a clear semantic value
\end{itemize}

Novoslovnica possesses 34 native\index{prefix!native} and 9 borrowed\index{prefix!borrowed} prefixes.

Native: bez, v, vně, vųtrě, voz, vy, do, za, zad, iz, kų, među, na, nad, naǐ, ne, ni, o, od, pa, po, pod, poslě, pra, pre, pred, prez, pri, pro, råz, s, sô, u, črez.

Borrowed: a, antï, aŭto, kŭazï, mono, multï, pan, para, super

The chapter is to discuss the main prefixes, giving each of them a description and examples. Also one should consider that major part of prefixes are deeply connected with the prepositions. That means many prefixes can be used separately from the stem in a prepositional role. Such prefixes are marked with the PREP label. 

\textbf{Bez} \gls{prep}

Meaning: “without”

This prefix is an equivalent of the word “without” in English. It means the absence of something that we mention. As a prefix, it has no equivalence in English. Nevertheless, the suffix “-less” has a very similar meaning. Look at the example to get it.

\underline{Examples:}

\textit{Bez doma} - Without home

\textit{Bezdušen} - Without soul/Soulless 


\textbf{V} \gls{prep}

Meaning: “into” (direction or placement)

This prefix has a semantic value of “leading into something”. The preposition equals to “in” or “into” in English.

\underline{Examples:}

\textit{V búdji }- In the building

\textit{Vhoditi} - Enter

\textbf{Vně} \gls{prep}

Meaning: “outside” (placement)

\underline{Examples:}

\textit{Vně trudnostëǐ} - Out of problems

\textit{Vněrečen} - Outspoken

\textbf{Vųtrě}

Meaning: “inside” (placement)

\underline{Examples:}

\textit{Vųtrě dušy} - Inside a soul

\textit{Vųtrěseben} - Thoughtful

\textbf{Voz}

Meaning: “upside” (direction)

\underline{Examples:}

\textit{Vozmožen} - Possible

\textit{Vozběg} - Take-off

\textbf{Vy}

Meaning: “outside” (direction)

This one means “leading outside something”. The prefix has no prepositional equivalent, remember word “Vy” means “You”.

\underline{Examples:}

\textit{Vyhod} - Exit

\textit{Vyglěd} - Outlook

\textbf{Do} \gls{prep}

Meaning: “to” (destination)

This prefix shows the destination of a process to some point. In English we can find an equivalent “to”.

\underline{Examples:}

\textit{Idati do stěny }- go to the wall

\textit{Dodati} - add (something we give to a set of something to increase its amount or capacity)

\textbf{Za} \gls{prep}

Meaning: “for” (aim)

The semantic value of this prefix is the aim of an action. We use it when we want to reach something, some object. English “for” can be found as revealing one of its semantic values in equal way. 

\underline{Examples:}

\textit{Idati za cělïü} - go for the goal

\textit{Zavariti čaǐ }- To brew tea (in order to make it hot)

\textbf{Zad} \gls{prep}

Meaning: “behind, after” (follow)

This prefix has an artificial origin (This semantic value was divided from the previous prefix). It means placing an object behind another one. The equivalent is “behind”.

\underline{Examples:}

\textit{Zad učilišta} - Behind the institute

\textit{Zadnik} - Backdrop

\textbf{Iz} \gls{prep}

Meaning: “from”

This prefix equals the English word “from”. 

\underline{Examples:}

\textit{Jesòm iz Moskvy} - I am from Moscow

\textit{Izhodnyǐ} - basic (from that we can develop something new)

\textbf{Kų}

Meaning: “what”

\underline{Examples:}

\textit{Kųda} - Where (What direction)

\textit{Kųdy} - When (What time) 

\textbf{Na} \gls{prep}

Meaning: “on” (placement)

\underline{Examples:}

\textit{Na ulicji} - On the street

\textit{Naděja} - Hope

\textit{Naglås} - Aloud

\textbf{Nad} \gls{prep}

Meaning: “above, over” (placement)”

\underline{Examples:}

\textit{Nad lěsom} - Above the forest

\textit{Nadzemnyǐ} - Overground

\textbf{Naǐ}

Meaning: “the most”

\underline{Examples:}

\textit{Naǐdobòr} - The kindest

\textit{Naǐskoryǐ} - The 

\textbf{Ne} \gls{prep}

Meaning: “not”

It has a similar value with English "un-" prefix.

\underline{Examples:}

\textit{Nevëlïkyǐ} - Not big

\textit{Nesměšnyǐ} - Unfunny

\textbf{Ni} \gls{prep}

Meaning: “even this/so”

\underline{Examples:}

\textit{Ni doma, ni råboty} - Neither home nor work 

\textit{Niĝda} - Never

\textbf{O}  \gls{prep}

Meaning: “about”

\underline{Examples:}

Bez doma - Without home

Bezdušen - Without soul/Soulless 

\textbf{Od} \gls{prep}

Meaning: “from, since” (direction)

\underline{Examples:}

\textit{Odlično} - Outstanding, OK

\textit{Od časa} - From the time (Since ...)

\textit{Odidati} - Go away

\textbf{Pa}

Meaning: “not real”

\underline{Examples:}

\textit{Pasyn} - Son-in-law

\textit{Padčerj} - Daughter-in-law

\textbf{Po} \gls{prep}

Meaning: “along” (direction)

\textit{Examples:}

\textit{Po ulicě} - Along the street

\textit{Poznane} - Studying

\textbf{Pod} \gls{prep}

Meaning: “under, beneath” (placement)

\underline{Examples:}

\textit{Pod vodoju} - Under the water

\textit{Podvoden} - Underwater

\textbf{Pra}

Meaning: “grand”

\textit{Examples:}

\textit{Prababa} - Great-grandmother

\textit{Praotec} - Ancestor

\textbf{Pre}

Meaning: “more than”

\underline{Examples:}

\textit{Prehod} - Transition

\textit{Premnogo} - A lot

\textbf{Pred} \gls{prep}

Meaning: “before, in front of” (placement)

\underline{Examples:}

\textit{Pred stěnoju} - In front of the wall

\textit{Predloga} - Offer

\textbf{Prez} \gls{prep}

Meaning: “rapidly through”

\underline{Examples:}

\textit{Prez trudnostj} - Through the challenge

\textit{Prezumen} - Witty

\textbf{Pri} \gls{prep}

Meaning: “close to, approach” (direction)

\underline{Examples:}

\textit{Pri lěsu} - Close to the forest

\textit{Prijěhati} - To arrive

\textbf{Pro} \gls{prep}

Meaning: “rapidly through”

\underline{Examples:}

\textit{Pro novinu} - About news (Brief speaking)

\textit{Proglås} - Declaration

\textbf{Råz}

Meaning: “ex, extra”

\underline{Examples:}

\textit{Råzličije} - Difference

\textit{Råzměn} - Exchange

\textbf{S} \gls{prep}

Meaning: “with”

\underline{Examples:}

\textit{S prijatelëm} - With friend

\textit{Stvoriti} - To create

\textbf{Sô}

Meaning: “alongside”

\underline{Examples:}

\textit{Sôrádnik} - Collaborator

\textit{Sôglåsnik} - Consonant

\textbf{U} \gls{prep}

Meaning: “near, ready to start an action”

\underline{Examples:}

\textit{U gråda} - Near the city

\textit{Uhoditi} - Go away

\textbf{Črez} \gls{prep}

Meaning: “slowly through”:

\underline{Examples:}

\textit{Črez lěs} - Through the forest

\textit{Črezměren} - Excessive
