\section{Mood}

Mood is a grammatical feature of verbs, used for signaling modality []. Mood is distinct from grammatical tense or grammatical aspect, although the same word patterns are used for expressing more than one of these meanings at the same time. There are several moods in Novoslovnica:
Indicative
Declarative
Subjunctive
Conjunctive I
Conjunctive II
Imperative
Optative
Jussive
Hortative
Supine
Inferential

Moods are divided into realis and irrealis moods. Realis mood is a grammatical mood which is used principally to indicate that something is a statement of fact. More precisely, it is used to express what the speaker considers to be a known state of affairs. Novoslovnica has two realis moods - indicative and declarative.

Indicative mood (Oznamitelnyǐ)  is used when we need simply to indicate that something is a statement of fact. Indicative has a variety of forms, including positive, negative and interrogative. Due to this fact it is chosen to be a basic mood, that is compared to the others. Indicative mood will be extensively considered in the next chapters. Just look at few examples.

Examples:
On glěděše iz prozorca doma si. - He was looking out of the window of his house.
Ja bųdu hoditi vò vysšojučilišto črez dvě godiny. - I will attend the university in two years.

Declarative mood (Objavitelnyǐ) is used when we describe a state of something, considering the action it was caused by. Declarative mood is of the same degree of variety as indicative mood, though is not used as often as indicative.
The first difference with indicative mood is in using auxiliary verb “imáti” (to have) instead of “byti” (to be) while forming sentences. This makes some semantic shifts in Novoslovnica. Look at the second example: the auxiliary verb is used in present tense, though it has a resultive semantic value. For present semantics the verbless sentences are used (first example).
The second difference is in the particle being used with the auxiliary verb in two moods. In indicative we use L-particles, while in declarative mood we use passive ones.

Examples:
Rěka krasna i slånco. - (There are) Beautiful river and the sun (around me).
Ja mám rođeno dvôh synoŭ. - I have two sons born.

Subjunctive mood (Predpokladnyǐ) is used when we want to express various states of unreality: wish, emotion, possibility, obligation etc. Subjunctives occur most often, although not exclusively, in subordinate clauses, using particles “aby”, “žeby”, “čtoby”, “išby” (example 1).
Take a note about absence of auxiliary verb while using L-particles. We just use a subjunctive particle with them. Though, we can also use infinitive instead of L-particle with the object in dative (example 2). Moreover, DA-forms can be used to express subjunctive (example 3). Nevertheless, there are cases you can use subjunctive in the main clause (example 4). 

Examples:
Az upëram aby ty odidal. - I would like you to leave.
Az upëram aby ti odidati. - It’s better for you to leave.
Ja htem da ty odidaš. - I want you to leave.
Ty by odidal odde. - You better leave here.

Conjunctive mood I (Domyselnyǐ) is used to express real wishes relatively present moment. It means, using conjunctive I we show that our wish is still able to be implemented. It is a classic variant of conjunctive and is formed by conjunctive form of the verb “byti” (to be) with an L-participle.
In fact, this mood is rather similar to Conditional II in English with the same tense analogues used (examples 2, 3). If we use just conjunctive clause, it is similar with “wish-construction” usage area in English (example 1). However, we can express conjunctive with the single clause using past transgressive (example 4).
Note, that having two clauses we need to use conditional conjunctions (“ako”, “jestjli”, “koli”, “dali”, “či” etc.).

Examples:
Az bih htěl da viđu slånco v ovyǐ obvlåčnyǐ denj. - I wish I see the sun in this cloudy day.
On biše doǐdal do nas, ako my zazovahme ĝo. - He would come if we called him.
Ako znaše on novoslovnicu, mogal biše da govori so vsïmi slověnami. - If he knew Novoslovnica, he would speak with all Slavs.
Znavšy novoslovnicu, on mogal biše da govori so vsïmi slověnami. - If he knew Novoslovnica, he would speak with all Slavs.

Conjunctive mood II (Domyselnyǐ) is the second conjunctive mood and it is used to express unreal wishes that have not been realized relatively a moment in the past. It can be formed by the verb “byti” in conjunctive form either  with supine (example 1) or with plusquamperfekt form of the main verb (example 2). It can be related to Conditional III in English. 
Note, that using supine you do not have to use conditional conjunctions between clauses. Moreover, we can use past active participle in the conditional clause (example 3). Attention! Compare this with the past transgressive in the single clause in Conjunctive I.

Examples:
Htětj da sę vidim s nim včera, bih byl mogal to da sdělam. - If  I had wanted to see him yesterday, I would have meet him.
Ako byh htel da sę vidim s nim včera, bih byl mogal to da sdělam. - If  I had wanted to see him yesterday, I would have meet him.
Ako byh htel da sę vidim s nim včera, bih byl mogavšym to sdělati. - the same translation
Imperative mood (Zapovědnyǐ)  is used when we want to tell somebody a command or a request. In Novoslovnica it has only I-person (in dual and plural) and II-person (in singular, dual and plural) forms. It is indicated by special endings (look forward for details). In the examples you can see how imperative is used.
Please note that imperative is indicated only by single form. Complex forms are not of this mood.

Examples:
Piši ovu knigu. - [Please] write this book. (you, sg)
Kažite mu poslanije-to. - [Please] tell him the message. (you, pl)
Pročitaǐte ovu knigu. - [Please] read this book. (you, pl)
Predgotuǐmo juž sëdy pokôǐ-ot. - [Please] prepare the room now! (we, pl)

Optative mood (Žadatelnyǐ) is a grammatical mood that expresses wish or hope. It is used when we want something or somebody to let us succeed in any action we are going to do. It is formed by DA-construction in the main clause (look at the examples). 
In English we can find similarity in LET-forms (examples 1, 2) or some general expressions that reveal our wish (example 3).

Examples:
Da daǐ mi pomognuti ti. - Let me help you.
Da bųde tako. - Let it be so.
Da žive Bòlĝarija. - Long live Bulgaria.

Jussive mood (Umožnitelnyǐ) is a grammatical mood of verbs for issuing orders and commanding. In Novoslovnica it is used to make orders for third-person expressions. There are no direct equivalents in English for this mood, but we can translate it with impersonal sentences (example 1) or with MUST-modal expressions. It is formed by indicative with “haǐ” modal word.

Examples:
Haǐ toǐ ne tòpta trevy. - Do not walk on grass.
Haǐ on podide. - He must come closer.

Hortative mood (Predložnyǐ)  is a grammatical mood that let verb express encourage or discourage of doing something. It can be translated with “let us” (encourage) or “might not” (discourage) constructions in English. In Novoslovnica it is formed with “něhaǐ” modal word with indicative and is able to have negative (example 4) and positive (examples 1-3) form. It is generally used just with I-person expressions.

Examples:
Něhaǐ grame v ovu gru. - Let us play this game.
Něhaǐ pějama pěsnü. - Let us (both) sing a song.
Něhaǐ govorim to otcu. - I might say this to father.
Něhaǐ ne idame v dom-òt. - We might not go into the house.


Supine (Dostęgatelnyǐ) is rather a grammar form than a mood. Nevertheless, it is often used to express aiming something and the action of approaching to the goal that has been defined. It is similar to infinitive but the ending (infinitive has “-ti” ending, while supine has “-tj” endind). It is usually used with modal verbs and verbs of moving.
Supine can be translated in English through complex predicate. That is because in Novoslovnica supine is practically never used as a single verb form. It is generally used with main verb that determine the background action of the circumstance. Look at the examples to get acquainted with it.

Examples:
Moǐca je priǐdala povědatj dobru novinu. - Mojca has come to message a good news.
Běgi skoro sę preoblekatj. - Run fast to change clothes.
Poǐdame kupovatj. - Let’s go to buy something.

Inferential mood (Prekazatelnyǐ) is used to report an unwitnessed event without confirming it. It is often used in stories or fiction books and is very similar with indicative. The only difference is in 3-person in past tenses, where the auxiliary verb “byti” disappears and we use just L-participle. In English it should be translated with ordinary indicative (examples 1-4), sometimes “there”-forms also can be used (example 5). Not, that we use L-participle in Novoslovnica in the past tense (you can mess it with Perfect tense), while it should be translated in Past Simple.
Look at the examples.

Examples:
Jesòm mu ĝo kupil.
Byl sòm kupil naǐ-prosty martenicy, ale toǐ izbral naǐ-råzkošny. - I had bought simpliest martenitsy, but he bought the most luxurious.
On byl bogatym. - He was rich.
Kųde byl master? - Where was the master?
Žila žena i mųž v malomu domu. - There lived a woman and a man in a small house.
