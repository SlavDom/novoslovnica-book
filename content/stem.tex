\section{Stem}

A word \textit{stem} is a semantic core of the word. It includes the main semantic value presented by a word form.

Each word of an independent \gls{pos} has at least one stem.

A \textit{stem nest} is a group of words existing in the language that have the same word stem. For example, look through the next words:

\textit{Sųd - sųdba - sųditi - sųden}

\textit{Hod - hodka - hoditi - poholdka - zahoden}

You see the words are of different meanings and different \gls{pos}. Thought, every word in each group has an equal semantic core - something connected with the judgement (1) or with walking (2).

Stems can be divided by several categories. For examples, there are borrowed and native stems. Native stems are derived from some proto-language forms. Thus, words which stems are derived from proto-Slavic, proto-Baltic, proto-German are supposed to be native in Novoslovnica. Also, some neologisms created by Slavic languages are also native.

Words of all other stems are borrowed. On of Novoslovnica's goals is to reduce the borrowings. However, we can list the languages from which Novoslovnica borrows the most.

\begin{table}
	\begin{tabular}{ll}
		Language & Percent \\
		Roman & \\
		i.e. French & \\
		i.e. Latin & \\
		German & \\
		i.e. English & \\
		Greek & \\ 
		Turkish &  \\
		Baltic &  \\
	\end{tabular}
\end{table}

The word can have multiple stems. Novoslovnica usually use words with up to three stems in a word. The stems are connected with stem-forming vowels. They are: \textit{o, ë, ï} and \textit{ô}.

\textbf{Examples:}

