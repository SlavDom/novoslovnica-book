\section{Collocation}

A \textit{collocation\index{collocation}} is an expression consisting of two or more words, that the one is main and the other are subordinative. \cite{colloc} Usually, collocation consists from 2 to 5 words. Subornative words are connected to the main one by several types of links: coherence, management and adjunction.

\textbf{Coherence\index{collocation!coherence}} between the main and the subordinative words lies in corellation of grammar forms. That means the subordinative word should correspond with the main one while the latter changes (i.e. case or gender). Thus, coherent link introduces a strong connection between the two words, so they change both when a primary link is established and when the word changes (declension or conjugation).

\textbf{Examples:}

- \textit{Lěpa žena - Lěpu ženu - Lěpoǐ ženě} (Beautiful woman in Nominative, Accusative and Dative)

- \textit{Sïlen věter - Sïlnoga větra - Sïlnomu větru} (Strong wind in Nominative, Genitive and Dative)

\textbf{Management\index{collocation!management}} link has a weaker connection between the words. The word with a management link has a determined grammar form. So the subordinative word should be changed once when is connected to the main word and then it is not changed while the main word is declining or conjugating. The primary word form is managed by the main word within a rule-case. It can be of a determined case or a preposition to be used with.

\textbf{Examples:}

- \textit{Věnec života - Věnca života - Věncom života} (The culmination of life in Nominative, Genitive and Dative)

- \textit{Kniga o važlivom - Knigy o važlivom - V knigji o važlivom} (A book of something important in Nominative, Genitive and Locative)

\textbf{Adjunction\index{collocation!adjunction}} means we simply add a subordinative word to the main one to form a collocation. We usually speak about this type while linking when we have immutable words.

Examples:

- \textit{Glųboko spati} (To have a deep sleep)

- \textit{Hoditi dalïko} (To walk far away)

Using proper cases in a collocation has a great importance for constructing an euphonious and understandable sentence. However, we cannot define all the cases of such links now, so you can feel-in the language to achieve this. If you are Slavic, simply try to use those cases or constructions that you use in your native language. If you are not, try to use English logic that is sometimes close to Novoslovnica, in a number of cases.