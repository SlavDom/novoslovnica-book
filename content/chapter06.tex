\chapter{Semantics}

There are three main parts of the sentence analysis that lead us to understanding: morphology, syntax and semantics. First two chapters were reviewed on the previous pages of the book. Here we are going to talk about the third main part of the linguistic analysis - Semantics.

We are not talking here about how to produce something like machine-assisted translator of Novoslovnica. However, we will deeply consider some facts, that could help with it. 
This chapter comprises three parts: the semantics of prefixes, the semantic of suffixes and common semantics that will describe the sense of roots and dependent POS.

\section{Prefix}

\newglossaryentry{pow}{name=POW, description={Part of word}}
\newglossaryentry{prep}{name=PREP, description={Preposition}}

Prefix\index{prefix} is a part of a word (\gls{pow}) that stands before the word stem. One word may have more than one prefix. In other words, prefixes can be added one by one right to the beginning of the stem. Each prefix has its own semantic value and word changes its meaning due to the presence of exact prefixes.

We can divide prefixes into several groups. 

\begin{itemize}
	\item Borrowed - prefixes that were borrowed from other languages (i.e. Latin, English etc.)
	\item Native - Slavic prefixes, that have a clear semantic value
\end{itemize}

Novoslovnica possesses 35 native\index{prefix!native} and 9 borrowed\index{prefix!borrowed} prefixes.

Native: bez, v, vně, vųtrě, voz, vy, do, za, zad, iz, k, kų, među, na, nad, naǐ, ne, ni, o, od, pa, po, pod, poslě, pra, pre, pred, prez, pri, pro, råz, s, sô, u, črez.

Borrowed: a, antï, aŭto, kŭazï, mono, multï, pan, para, super

The chapter is to discuss the main prefixes, giving each of them a description and examples. Also one should consider that major part of prefixes are deeply connected with the prepositions. That means many prefixes can be used separately from the stem in a prepositional role. Such prefixes are marked with the PREP label. 

\textbf{Bez} \gls{prep}

Meaning: “without”

This prefix is an equivalent of the word “without” in English. It means the absence of something that we mention. As a prefix, it has no equivalence in English. Nevertheless, the suffix “-less” has a very similar meaning. Look at the example to get it.

\underline{Examples:}

\textit{Bez doma} - Without home

\textit{Bezdušen} - Without soul/Soulless 


\textbf{V} \gls{prep}

Meaning: “into” (direction or placement)

This prefix has a semantic value of “leading into something”. The preposition equals to “in” or “into” in English.

\underline{Examples:}

\textit{V búdji }- In the building

\textit{Vhoditi} - Enter

\textbf{Vně} \gls{prep}

Meaning: “outside” (placement)

\underline{Examples:}

\textit{Vně trudnostëǐ} - Out of problems

\textit{Vněrečen} - Outspoken

\textbf{Vųtrě}

Meaning: “inside” (placement)

\underline{Examples:}

\textit{Vųtrě dušy} - Inside a soul

\textit{Vųtrěseben} - Thoughtful

\textbf{Voz}

Meaning: “upside” (direction)

\underline{Examples:}

\textit{Vozmožen} - Possible

\textit{Vozběg} - Take-off

\textbf{Vy}

Meaning: “outside” (direction)

This one means “leading outside something”. The prefix has no prepositional equivalent, remember word “Vy” means “You”.

\underline{Examples:}

\textit{Vyhod} - Exit

\textit{Vyglěd} - Outlook

\textbf{Do} \gls{prep}

Meaning: “to” (destination)

This prefix shows the destination of a process to some point. In English we can find an equivalent “to”.

\underline{Examples:}

\textit{Idati do stěny }- go to the wall

\textit{Dodati} - add (something we give to a set of something to increase its amount or capacity)

\textbf{Za} \gls{prep}

Meaning: “for” (aim)

The semantic value of this prefix is the aim of an action. We use it when we want to reach something, some object. English “for” can be found as revealing one of its semantic values in equal way. 

\underline{Examples:}

\textit{Idati za cělïü} - go for the goal

\textit{Zavariti čaǐ }- To brew tea (in order to make it hot)


\textbf{Zad} \gls{prep}

Meaning: “behind, after” (follow)

This prefix has an artificial origin (This semantic value was divided from the previous prefix). It means placing an object behind another one. The equivalent is “behind”.

\underline{Examples:}

Bez doma - Without home
Bezdušen - Without soul/Soulless 

\textbf{Iz} \gls{prep}

Meaning: “from”

This prefix equals the English word “from”. 

\underline{Examples:}

\textit{Jesòm iz Moskvy} - I am from Moscow

\textit{Izhodnyǐ} - basic (from that we can develop something new)

\textbf{K} \gls{prep}

Meaning: “to” (direction)

\underline{Examples:}

Bez doma - Without home
Bezdušen - Without soul/Soulless 

\textbf{Kų}

Meaning: “what”

\underline{Examples:}

\textit{Kųda} - Where (What direction)

\textit{Kųdy} - When (What time) 

\textbf{Na} \gls{prep}

Meaning: “on” (placement)

Examples:
Bez doma - Without home
Bezdušen - Without soul/Soulless 

\textbf{Nad} \gls{prep}

Meaning: “above” (placement)”

Examples:
Bez doma - Without home
Bezdušen - Without soul/Soulless 

\textbf{Naǐ}

Meaning: “the most”

Examples:
Bez doma - Without home
Bezdušen - Without soul/Soulless 


\textbf{Ne} \gls{prep}

Meaning: “not”

Examples:
Bez doma - Without home
Bezdušen - Without soul/Soulless 

\textbf{Ni} \gls{prep}

Meaning: “even this/so”

Examples:
Bez doma - Without home
Bezdušen - Without soul/Soulless 

\textbf{O}  \gls{prep}

Meaning: “about”

Examples:
Bez doma - Without home
Bezdušen - Without soul/Soulless 

\textbf{Od} \gls{prep}

Meaning: “from” (direction)

Examples:
Bez doma - Without home
Bezdušen - Without soul/Soulless 

\textbf{Pa}

Meaning: “not real”

Examples:
Bez doma - Without home
Bezdušen - Without soul/Soulless 


\textbf{Po} \gls{prep}

Meaning: “along” (direction)

Examples:
Bez doma - Without home
Bezdušen - Without soul/Soulless 

\textbf{Pod} \gls{prep}

Meaning: “under” (placement)

Examples:
Bez doma - Without home
Bezdušen - Without soul/Soulless 

\textbf{Pra}

Meaning: “grand”

Examples:
Bez doma - Without home
Bezdušen - Without soul/Soulless 

\textbf{Pre}

Meaning: “more than”

Examples:
Bez doma - Without home
Bezdušen - Without soul/Soulless 

\textbf{Pred} \gls{prep}

Meaning: “before, in front of” (placement)

Examples:
Bez doma - Without home
Bezdušen - Without soul/Soulless 

\textbf{Prez} \gls{prep}

Meaning: “rapidly through”

Examples:
Bez doma - Without home
Bezdušen - Without soul/Soulless 


\textbf{Pri} \gls{prep}

Meaning: “close to, approach” (direction)

Examples:
Bez doma - Without home
Bezdušen - Without soul/Soulless 

\textbf{Pro} \gls{prep}

Meaning: “rapidly through”

Examples:
Bez doma - Without home
Bezdušen - Without soul/Soulless 

\textbf{Råz}

Meaning: “different”

Examples:
Bez doma - Without home
Bezdušen - Without soul/Soulless 

\textbf{S} \gls{prep}

Meaning: “with”

Examples:
Bez doma - Without home
Bezdušen - Without soul/Soulless 

\textbf{Sô}

Meaning: “alongside”

Examples:
Bez doma - Without home
Bezdušen - Without soul/Soulless 

\textbf{U} \gls{prep}

Meaning: “near”

Examples:
Bez doma - Without home
Bezdušen - Without soul/Soulless 

\textbf{Črez} \gls{prep}

Meaning: “slowly through”:

Examples:
Bez doma - Without home
Bezdušen - Without soul/Soulless 


\section{Suffix}

Suffix - is a \gls{pow}, that is placed after the word stem. It is a postfix that is not an ending of the word. That means that added a suffix to the word makes a new one with its own semantic value. Here is the list of main suffixes in Novoslovnica with their semantic value and examples.

\textbf{Ak}

This suffix has a semantic value of “person of some quality”.

\underline{Examples:}

\textit{Bědnäk - Běden (adj)} - Poor man - Poor (adj)

\textit{Glupak - Glup (adj)} - Fool - fool (adj)

\textbf{An}

This suffix has a semantic value of “having some property”.

\underline{Examples:}

\textit{Ušan - Uho (noun)} - Plecotus - Ear (noun)

\textit{Leganka - Legati (verb)} - Bed - to lie (verb)

\textbf{Ar}

This suffix has a semantic value of “person of some profession”. The suffix has a similar meaning with English "-er".

\underline{Examples:}

\textit{Rybar} - Fisher

\textit{Ovčar} - Shepherd

\textit{Ač}

This suffix has a value of "person that likes to perform an action"

\underline{Examples:}

\textit{Grač - Grati} - Player - To play

\textit{Běgač - Běgati} - Runner - To run

\textbf{B}

The suffix of a noun describing a process of continuously performing an action

\underline{Examples:}

\textit{Borba - Bořati} - Struggle - To struggle

\textit{Sųdba - Sųđati} - Fate - To judge

\textbf{Dl}

This suffix describes an object (tool) for performing an action.

\underline{Examples:}

\textit{Mydlo} - Myti

\textit{Vozidlo} - Voziti

\textbf{Ot}

The suffix represents a state of some attribute.

\underline{Examples:}

\textit{Lěpota} - Beauty

\textit{Hlådnota} - Cold

\textbf{Ostj}

This suffix is an equivalent of English "ness" suffix. In make a noun from some attribute value expressed by an adjective.

\underline{Examples:}

\textit{Vëlïkostj} - Greatness

\textit{Samostj} - Loneliness

\textbf{Ec}

The suffix is for a person with some attribute.

\underline{Examples:}

\textit{Hlåpec} - Boy

\textit{Běglec} - Escaper

\textbf{Izn}

The suffix represents a state of doing some action.

\underline{Examples:}

\textit{Žiznj - Žiti} - Life - To live

\textit{Bělizna - Běliti} - White - To white

\textbf{In}

This suffix is used to express that the subject is made of something

\underline{Examples:}

\textit{Pěrina - Pěro} - Featherbed - Feather

\textit{Sųdbina - Sųdba} - Life in fortune - Fate

\textbf{Ih}

This suffix represents a female person/animal.

\underline{Examples:}

\textit{Ježiha - Jež} - Hedgehog

\textit{Rybariha - Rybar} - Fisher

\textbf{Ïc}

This suffix represents a female animal.

\underline{Examples:}

\textit{Lisïca} - Fox (female)

\textit{Kotïca} - Cat (female)

\textbf{K}

This suffix has two meanings. The first one describes a process of the verb. 

\underline{Examples:}

\textit{Myǐka - Myti} - Washing - To wash

\textit{Budka - Buđati} - Waking - To wake

The second one is deminutive.

\underline{Examples:}

\textit{Kot - Kotek} - (small) cat

\textit{Rot - Rotek} - (small) mouth

\textbf{N}

The attributive suffix that describes an attribute of being something defined by a noun.

\underline{Examples:}

\textit{Běda - Běden} - Poor - Poor

\textbf{Nik}

The suffix is for a person characterized by a NP.

\underline{Examples:}

Bezdělnik - Bez děla

Učobnik - Učoba

\textbf{Sl}

This suffix describes an object (tool) for performing an action.

\textbf{Examples:}

\textit{Vëdslo - Vëdati} - Paddle (the verb "to lead")

\textit{Ĝadslo - Ĝadati} - Password (the verb "to guess)

\textbf{Stv}

It is a suffix for collective noun of a subject.

\textbf{Examples:}

\textit{Lïstva - Lïst} - Leafage - Leaf

\textit{Čelověkstvo - Čelověk} - Mankind - Man

\textbf{Telj}

This suffix represents a person perfoming the action of the main word.

\textbf{Examples:}

\textit{Daritelj - Dariti} - Presenter - To present

\textit{Učitelj - Učiti} - Teacher - To teach

This is not the whole list. Though, you can understand a principle of common semantic shift while adding the same suffix to the stem.

\section{New word creating}

The process of creating the language’s vocabulary is one of the most important processes in language constructing. We need to build an understandable lexeme, moreover, it can be named as a Slavic one. 

Here I want to tell you about this process we follow every time when the new word of Novoslovnica appears in this world. By following this algorithm you will be able to create new words that are suitable for Novoslovnica by yourself. Now look here:

We are looking through the vocabularies of Slavic languages and find out the translations from our native language for the exact word.

Then we look at the frequency of roots within this list.
\begin{enumerate}
	\item If there is an absolute leader there, and it is Slavic, we take this root.
	\item Then we look at the frequency of suffixes and prefixes within this list.
	\item If there is an absolute leader, and it stays for the primary semantic value, we take this form.
	\item If not, we create a new form by following described above semantic values of different prefixes and suffixes.
	\item If there is some forms that are prevail others, we can add them all into Novoslovnica vocabulary.
	\item If not, we check whether there are some non-absolute leaders within the words. If they are, we go to 3.a to each of them.
	\item If not, and there is a non-Slavic absolute or some relative leaders, we should take into account the next points:
	\subitem There is no Slavic root among them
	\subitem We cannot form an artificial Slavic form for a word.
	\item If the previous forms are true, we add a non-Slavic word into vocabulary. However, if we can form a Slavic word - we add it into vocabulary with additional non-Slavic “crutch”. 
\end{enumerate}

Frankly, this is the whole algorithm. You can apply it in your everyday speaking. However, it takes time to build a new word for the only semantic value, so it is difficult to create all your words by these rules by yourself. It is better for you to use a dictionary, that have already been created by our team and contains more than 5 thousand forms of Novoslovnica’s vocabulary.. 