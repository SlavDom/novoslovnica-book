\chapter{Semantics}

There are three main parts of the sentence analysis that lead us to understanding: grammar, syntax and semantics. First two chapters were reviewed on the previous pages of the book. Here we are going to talk about the third main part of the linguistic analysis - Semantics.

We are not talking here about how to produce something like machine-assisted translator of Novoslovnica. However, we will consider some facts, that could help to deal with it. 

This chapter comprises three parts: the semantics of prefixes, the semantics of suffixes and common semantics that will describe the sense of roots and dependent \gls{pos}.

\section{Prefix}

\newglossaryentry{pow}{name=POW, description={Part of word}}
\newglossaryentry{prep}{name=PREP, description={Preposition}}

Prefix\index{prefix} is a part of a word (\gls{pow}) that stands before the word stem. One word may have more than one prefix. In other words, prefixes can be added one by one right to the beginning of the stem. Each prefix has its own semantic value and word changes its meaning due to the presence of exact prefixes.

We can divide prefixes into several groups. 

\begin{itemize}
	\item Borrowed - prefixes that were borrowed from other languages (i.e. Latin, English etc.)
	\item Native - Slavic prefixes, that have a clear semantic value
\end{itemize}

Novoslovnica possesses 35 native\index{prefix!native} and 9 borrowed\index{prefix!borrowed} prefixes.

Native: bez, v, vně, vųtrě, voz, vy, do, za, zad, iz, k, kų, među, na, nad, naǐ, ne, ni, o, od, pa, po, pod, poslě, pra, pre, pred, prez, pri, pro, råz, s, sô, u, črez.

Borrowed: a, antï, aŭto, kŭazï, mono, multï, pan, para, super

The chapter is to discuss the main prefixes, giving each of them a description and examples. Also one should consider that major part of prefixes are deeply connected with the prepositions. That means many prefixes can be used separately from the stem in a prepositional role. Such prefixes are marked with the PREP label. 

\textbf{Bez} \gls{prep}

Meaning: “without”

This prefix is an equivalent of the word “without” in English. It means the absence of something that we mention. As a prefix, it has no equivalence in English. Nevertheless, the suffix “-less” has a very similar meaning. Look at the example to get it.

\underline{Examples:}

\textit{Bez doma} - Without home

\textit{Bezdušen} - Without soul/Soulless 


\textbf{V} \gls{prep}

Meaning: “into” (direction or placement)

This prefix has a semantic value of “leading into something”. The preposition equals to “in” or “into” in English.

\underline{Examples:}

\textit{V búdji }- In the building

\textit{Vhoditi} - Enter

\textbf{Vně} \gls{prep}

Meaning: “outside” (placement)

\underline{Examples:}

\textit{Vně trudnostëǐ} - Out of problems

\textit{Vněrečen} - Outspoken

\textbf{Vųtrě}

Meaning: “inside” (placement)

\underline{Examples:}

\textit{Vųtrě dušy} - Inside a soul

\textit{Vųtrěseben} - Thoughtful

\textbf{Voz}

Meaning: “upside” (direction)

\underline{Examples:}

\textit{Vozmožen} - Possible

\textit{Vozběg} - Take-off

\textbf{Vy}

Meaning: “outside” (direction)

This one means “leading outside something”. The prefix has no prepositional equivalent, remember word “Vy” means “You”.

\underline{Examples:}

\textit{Vyhod} - Exit

\textit{Vyglěd} - Outlook

\textbf{Do} \gls{prep}

Meaning: “to” (destination)

This prefix shows the destination of a process to some point. In English we can find an equivalent “to”.

\underline{Examples:}

\textit{Idati do stěny }- go to the wall

\textit{Dodati} - add (something we give to a set of something to increase its amount or capacity)

\textbf{Za} \gls{prep}

Meaning: “for” (aim)

The semantic value of this prefix is the aim of an action. We use it when we want to reach something, some object. English “for” can be found as revealing one of its semantic values in equal way. 

\underline{Examples:}

\textit{Idati za cělïü} - go for the goal

\textit{Zavariti čaǐ }- To brew tea (in order to make it hot)


\textbf{Zad} \gls{prep}

Meaning: “behind, after” (follow)

This prefix has an artificial origin (This semantic value was divided from the previous prefix). It means placing an object behind another one. The equivalent is “behind”.

\underline{Examples:}

Bez doma - Without home
Bezdušen - Without soul/Soulless 

\textbf{Iz} \gls{prep}

Meaning: “from”

This prefix equals the English word “from”. 

\underline{Examples:}

\textit{Jesòm iz Moskvy} - I am from Moscow

\textit{Izhodnyǐ} - basic (from that we can develop something new)

\textbf{K} \gls{prep}

Meaning: “to” (direction)

\underline{Examples:}

Bez doma - Without home
Bezdušen - Without soul/Soulless 

\textbf{Kų}

Meaning: “what”

\underline{Examples:}

\textit{Kųda} - Where (What direction)

\textit{Kųdy} - When (What time) 

\textbf{Na} \gls{prep}

Meaning: “on” (placement)

Examples:
Bez doma - Without home
Bezdušen - Without soul/Soulless 

\textbf{Nad} \gls{prep}

Meaning: “above” (placement)”

Examples:
Bez doma - Without home
Bezdušen - Without soul/Soulless 

\textbf{Naǐ}

Meaning: “the most”

Examples:
Bez doma - Without home
Bezdušen - Without soul/Soulless 


\textbf{Ne} \gls{prep}

Meaning: “not”

Examples:
Bez doma - Without home
Bezdušen - Without soul/Soulless 

\textbf{Ni} \gls{prep}

Meaning: “even this/so”

Examples:
Bez doma - Without home
Bezdušen - Without soul/Soulless 

\textbf{O}  \gls{prep}

Meaning: “about”

Examples:
Bez doma - Without home
Bezdušen - Without soul/Soulless 

\textbf{Od} \gls{prep}

Meaning: “from” (direction)

Examples:
Bez doma - Without home
Bezdušen - Without soul/Soulless 

\textbf{Pa}

Meaning: “not real”

Examples:
Bez doma - Without home
Bezdušen - Without soul/Soulless 


\textbf{Po} \gls{prep}

Meaning: “along” (direction)

Examples:
Bez doma - Without home
Bezdušen - Without soul/Soulless 

\textbf{Pod} \gls{prep}

Meaning: “under” (placement)

Examples:
Bez doma - Without home
Bezdušen - Without soul/Soulless 

\textbf{Pra}

Meaning: “grand”

Examples:
Bez doma - Without home
Bezdušen - Without soul/Soulless 

\textbf{Pre}

Meaning: “more than”

Examples:
Bez doma - Without home
Bezdušen - Without soul/Soulless 

\textbf{Pred} \gls{prep}

Meaning: “before, in front of” (placement)

Examples:
Bez doma - Without home
Bezdušen - Without soul/Soulless 

\textbf{Prez} \gls{prep}

Meaning: “rapidly through”

Examples:
Bez doma - Without home
Bezdušen - Without soul/Soulless 


\textbf{Pri} \gls{prep}

Meaning: “close to, approach” (direction)

Examples:
Bez doma - Without home
Bezdušen - Without soul/Soulless 

\textbf{Pro} \gls{prep}

Meaning: “rapidly through”

Examples:
Bez doma - Without home
Bezdušen - Without soul/Soulless 

\textbf{Råz}

Meaning: “different”

Examples:
Bez doma - Without home
Bezdušen - Without soul/Soulless 

\textbf{S} \gls{prep}

Meaning: “with”

Examples:
Bez doma - Without home
Bezdušen - Without soul/Soulless 

\textbf{Sô}

Meaning: “alongside”

Examples:
Bez doma - Without home
Bezdušen - Without soul/Soulless 

\textbf{U} \gls{prep}

Meaning: “near”

Examples:
Bez doma - Without home
Bezdušen - Without soul/Soulless 

\textbf{Črez} \gls{prep}

Meaning: “slowly through”:

Examples:
Bez doma - Without home
Bezdušen - Without soul/Soulless 


\section{Suffix}


\textbf{Ak}

This suffix has a semantic value of “person of some quality”.

\textbf{Examples:}

\textit{Bědnäk - Běden (adj)} - Poor man - Poor (adj)

\textit{Glupak - Glup (adj)} - Fool - fool (adj)

\textbf{An}

This suffix has a semantic value of “having some property”.

Examples:
Bědnäk - Běden (adj) - Poor man - Poor (adj)
Glupak - Glup (adj) - Fool - fool (adj)

\textbf{An}

This suffix has a semantic value of “person of some profession”. The suffix has a similar meaning with English "-er".

\textbf{Examples:}

\textit{Rybar} - Fisher

Glupak - Glup (adj) - Fool - fool (adj)


Ar

Ač

Dl

En

Ot

Ostj

Ec

Izn

Ik

In

Ih

Ic

K

Ni

Nik

Sl

Stv

Telj

Un

Yš

\section{Stem}

A word \textit{stem} is a semantic core of the word. It includes the main semantic value presented by a word form.

Each word of an independent \gls{pos} has at least one stem.

A \textit{stem nest} is a group of words existing in the language that have the same word stem. For example, look through the next words:

\textit{Sųd - sųdba - sųditi - sųden}

\textit{Hod - hodka - hoditi - poholdka - zahoden}

You see the words are of different meanings and different \gls{pos}. Thought, every word in each group has an equal semantic core - something connected with the judgement (1) or with walking (2).

Stems can be divided by several categories. For examples, there are borrowed and native stems. Native stems are derived from some proto-language forms. Thus, words which stems are derived from proto-Slavic, proto-Baltic, proto-German are supposed to be native in Novoslovnica. Also, some neologisms created by Slavic languages are also native.

Words of all other stems are borrowed. On of Novoslovnica's goals is to reduce the borrowings. However, we can list the languages from which Novoslovnica borrows the most. We use the term "factor" to describe the affect on Novoslovnica lexicon of different non-Slavic languages instead of absolute or percentage value. The more the factor the more influence on the language the borrowing from that language have.

\begin{table}[!htb]
	\begin{tabular}{ll}
		Language & Factor \\
		Roman & 43 \\
		i.e. French & 2 \\
		i.e. Latin & 40 \\
		German & 26 \\
		i.e. English & 25 \\
		Greek & 12 \\ 
		Turkish & 6 \\
		Baltic & 0.01 \\
	\end{tabular}
\end{table}

The word can have multiple stems. Novoslovnica usually use words with up to three stems in a word. The stems are connected with stem-forming vowels. They are: \textit{o, ë, ï} and \textit{ô}.

Words with multiple stems usually describes a \gls{np} that has a strong forces between the words. The following examples show when you should use each of stem-forming vowels.

\textbf{Examples:}

- \textit{Zemlëtręs} - \textit{Tręs zemlï} - Earthquake (auxiliary word (soft) + main word)

- \textit{Sámovoz} - \textit{Vozi sámo} - Automobile (auxiliary word (hard) + main word)

- \textit{Blågodariti - Blågo dariti} - To thank (auxiliary word (hard) + main word)

- \textit{Pętïkrátno} - \textit{Pętï krátno} - Five times (Simple concatination of numerals.\footnote{Pętokrátno is also valid according to the second case (Kpátno pętï - Pętokrátno)})

- \textit{Dvôkrátno - Dva kráta} - Twice (Particular case of the previous example.\footnote{Dvokrátno from Krátno dvôm is also available.})

\section{New word creating}

The process of creating the language’s vocabulary is one of the most important processes in language constructing. We need to build an understandable lexeme, moreover, it can be named as a Slavic one. Here I want to tell you about this process we follow every time when the new word of Novoslovnica appears in this world. By following this algorithm you will be able to create new words that are suitable for Novoslovnica by yourself. Now look here:
We are looking through the vocabularies of Slavic languages and find out the translations from our native language for the exact word.
Then we look at the frequency of roots within this list.
If there is an absolute leader there, and it is Slavic, we take this root.
Then we look at the frequency of suffixes and prefixes within this list.
If there is an absolute leader, and it stays for the primary semantic value, we take this form.
If not, we create a new form by following described above semantic values of different prefixes and suffixes.
If there is some forms that are prevail others, we can add them all into Novoslovnica vocabulary.
If not, we check whether there are some non-absolute leaders within the words. If they are, we go to 3.a to each of them.
If not, and there is a non-Slavic absolute or some relative leaders, we should take into account the next points:
There is no Slavic root among them
We cannot form an artificial Slavic form for a word.
If the previous forms are true, we add a non-Slavic word into vocabulary. However, if we can form a Slavic word - we add it into vocabulary with additional non-Slavic “crutch”. 

Frankly, this is the whole algorithm. You can apply it in your everyday speaking. However, it takes time to build a new word for the only semantic value, so it is difficult to create all your words by these rules by yourself. It is better for you to use a dictionary, that have already been created by our team and contains more than 5 thousand forms of Novoslovnica’s vocabulary.. 