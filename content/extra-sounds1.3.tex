\section{Vowels}

By reading this paragraph you should be aware about Novoslovnica phonemes. At the beginning we will speak about two main features of the language: reproduction and alternation.

Reproduction (extra sounds) - is a process of adding a new sound (consonant or vowel) in the word.

Alternation - is a process of changing some sound (consonant or vowel) or sounds to another one(s).

What are the purposes of alternation and reproduction? These terms both obtain the single aim - to make our speech comfortable for ourselves.

Every language has its own comfortable combinations of sounds and avoided ones. Some languages alternate these uncomfortable character rows into another ones that are comfortable for language speaker in writing, some languages keep writing etymological and alternate the pronunciation of the uncomfortable words. Take into account the fact, that the concept of comfort is relative, that means two different languages (which are spoken by different nations) have different modes of comfort.
Example: you can look at and compare German and Dutch languages. They are very close, but they have different ways for comfortable pronunciation. The word “book” in German sounds like “Buch”, but in Holland it sounds like “boek”. Thus, you see that these words are familiar, but their pronunciation differs a lot.

Moreover, even one language spoken in different areas can differ greatly in the areas it is spoken.
Example: American and British English. You know that it is still the same language, but pronunciation differs so greatly that American person can not always understand in ordinary speech (quick temp of speaking) what the British person have said.

This is the example how the same language differs in areas it is spoken. Furthermore, these areas may not be far away from each other. You can find information about different patois and even dialects in very small regions. 

Novoslovnica as a panslavic language absorbs different conceptions of the comfort term of Slavic languages.

When reproduction is used, we add a new sound before the word we want to say. It can be whether a vowel or a consonant, depending on the previous letter in the word.

When should we reproduce a sound? To deal with this problem, you should know a thesis, which is widely used in Novoslovnica.

\textbf{Rule №2}: After consonant there should be placed a vowel. And after a vowel there should be placed a consonant.

This rule will help you in speaking and writing, when you construct your words with the reproduction.

There is a limited row of sounds that are allowed to participate in reproduction. In the table 2.1 you can see all of them.




