\section{Extra sounds}

By reading this paragraph you should be aware about Novoslovnica phonemes. At the beginning we will speak about two main features of the language: reproduction and alternation.

\textit{Reproduction} (extra sounds) - is a process of adding a new sound (consonant or vowel) in the word.

\textit{Alternation} - is a process of changing some sound (consonant or vowel) or sounds to another one(s).

What are the purposes of alternation and reproduction? These terms both obtain the single aim - to make our speech comfortable for ourselves.

Every language has its own comfortable combinations of sounds and avoided ones. Some languages alternate these uncomfortable character rows into another ones that are comfortable for language speaker in writing, some languages keep writing etymological and alternate the pronunciation of the uncomfortable words. Take into account the fact, that the concept of comfort is relative, that means two different languages (which are spoken by different nations) have different modes of comfort.

\textbf{Example:} you can look at and compare German and Dutch languages. They are very close, but they have different ways for comfortable pronunciation. The word “book” in German sounds like “Buch”, but in Holland it sounds like “boek”. Thus, you see that these words are familiar, but their pronunciation differs a lot.

Moreover, even one language spoken in different areas can differ greatly in the areas it is spoken.
Example: American and British English. You know that it is still the same language, but pronunciation differs so greatly that American person can not always understand in ordinary speech (quick temp of speaking) what the British person have said.

This is the example how the same language differs in areas it is spoken. Furthermore, these areas may not be far away from each other. You can find information about different patois and even dialects in very small regions (footnote). 

% в порядке примечания - очень хорошо это иллюстрирует китайский язык, который раздробился не только на три основных варианта диалектов: мандаринский, тайваньский и кантонский, но и продолжает дробиться на более мелкие диалекты, которые часто не являются взаимно-понимаемыми. А ведь есть ещё классический литературный китайский (для чтения и понимания которого нужно отдельно учиться - примерно как с древнерусским или старославянским). И только консервативность иероглифической письменности ещё удерживает китайский от окончательной дезинтеграции.

Novoslovnica as a panslavic language absorbs different conceptions of the comfort term of Slavic languages.

When reproduction is used, we add a new sound before the word we want to say. It can be whether a vowel or a consonant, depending on the previous letter in the word.

When should we reproduce a sound? To deal with this problem, you should know a thesis, which is widely used in Novoslovnica.

\textbf{Rule n. 1}: After consonant there should be placed a vowel. And after a vowel there should be placed a consonant.

This rule will help you in speaking and writing, when you construct your words with the reproduction.

There is a limited row of sounds that are allowed to participate in reproduction. In the table 2.1 you can see all of them.

\textbf{Examples:}

\textit{osem} (eight) \textipa{[‘osEm]} - \textit{vosem} \textipa{[‘vosEm]}

\textit{gra} (game) \textipa{[Hra]} - \textit{ïgra} \textipa{[i’Hra]}

So now the only problem is to know about how should we reproduce these sounds.

Reproduction of consonants can be found in different parts of the word. There are three places, where reproduction can appear - the beginning of the word, middle part of the word and the ending of the word.

When we speak about the beginning of the word, we should take into account what sound is at the end of the previous word. Due to rule 2 we should try to keep the “consonant-vowel” sequence inside our sentence. That’s why, we should do actions from the list below:
If the previous word ends with the vowel and temporal word begins also with the vowel we should add in the beginning of the word a consonant V or J. The choice of what letter to add depends on the vowel which is in the temporal word.

\begin{itemize}
	\item We should choose J for soft vowels
	\item We should choose V for hard ones.
	\item If there is one consonant and one vowel in the set of the first letter from the temporal word and the last letter from the previous word, we should not add nothing.
\end{itemize}

Speaking about word endings, thesis 1 is also very important. We should add a letter B to the end of the word, if the next word begins with the vowel and temporal word ends with the vowel too.

As you see these two principles are practically equal, because wherever we add a consonant, the result will be the same. So the question is, when we should add to the end of the word a consonant and when we should add it to the beginning of the word. In practice, almost always we use the first case, when we add a consonant to the beginning of the word. The second case is used in prepositional constructions with such a word as “O” (about) (Look: “O” - “Ob”). In other cases try to use letters V and J for reproduction.

The third case of consonant reproduction is to reproduce it in the middle of the word. It is used, when there is a row of vowels in the word and it is difficult to pronounce them all together. Then you can divide them by the consonant J, which is put between the neighbor vowels. Don’t confuse it with the case, when with the addition of a vowel letter Ǐ transforms to letter J (see the next paragraph for it) and you receive to vowels divided by this consonant two. Look at the examples of consonant reproduction in the middle of the word.

\textbf{Examples:}

\textit{ïdiot} (idiot) \textipa{[idI’ot]} - \textit{ïdijot} \textipa{[idI’Jot]}

The same situation is about vowel reconstruction. You can find it in the beginning, in the middle and in the end of the word, but it is a bit simpler than a previous one. Vowel reproduction appears when there is a rather large amount of consonants in one place of the sentence. This means that you can find a row of consonants in the word or a consonant conjunction in the end of one word and in the beginning of the next one. In any case, you should concern here about whether it is comfortable for you to pronounce these combinations of sounds.

In the beginning of the word we add a vowel Ï and there are no other cases. Very simple.

\textbf{Examples:}

\textit{gra} (game) \textipa{[Hra]} - \textit{ïgra} \textipa{[i’Hra]}

In the middle or in the end of the word we add a vowel O. This case is used with prepositions and prefixes (“K” - “Ko”, “S” - “So” etc). 

\textbf{Examples:}

\textit{k domu} (to home) [k ‘domu] - \textit{ko dvoru} [ko ‘dvoru]




