\section{Slavic proverbs and adages}

In this chapter we give some proverbs that exist in different Slavic languages with the corresponding proverb in Novoslovnica.

\textbf{Vsë je dobrym ako sę konči tako.}

- \textit{Wszystko dobre, co się dobrze kończy.} (Polish)

- \textit{Всё хорошо, что хорошо кончается.} (Russian)

\textbf{Ne je zlåtom vsë, čto sija.}

- \textit{Није злато све што сија.} (Serbian)

- \textit{Не всё то золото, что блестит.} (Russian)

\textbf{V bezryben čas i rak je ryba.}

- \textit{Na bezrybiu i rak ryba.} (Polish)

- \textit{На безрыбье и рак рыба.} (Russian)

\textbf{Kakto dvě kaplï vody.}

- \textit{Като две капки вода.} (Bulgarian)

- \textit{Как две капли воды.} (Russian)

\textbf{Bez truda jěda bųde huda.}

- \textit{Без труд не се ора со плуг.} (Macedonian)

- \textit{Без труда не выловишь и рыбку из пруда.} (Russian)

- \textit{Bez práce nejsou koláče.} (Czech)

\textbf{Ako mlådostj znaše, ako starostj možeše!}

- \textit{Если б молодость знала, если б старость могла!} (Russian)

- \textit{Ако младостта знаеше, ако старостта можеше!} (Bulgarian)

\textbf{Pospěh - vsïm na směh.}

- \textit{Паспех - людзям на смех.} (Belorussian)

- \textit{Поспешишь - людей насмешишь.} (Russian)

\textbf{Čto máš v sredcu, to máš na języku.}

- \textit{Co na srdci, to na jazyki.} (Czech)

- \textit{Что на уме, то и на языку.} (Russian)

\textbf{Dobro slovo železna vråta odkryva.}

- \textit{Dobré slovo i železná vrata otvírá.} (Czech)

- \textit{Добра дума железни врата отваря.} (Bulgarian)

\textbf{Trikrátno měraǐ, jednokrátno rězaǐ.}

- \textit{Три пъти мери, един път режи.} (Bulgarian)

- \textit{Dvakrát měř, jednou řež.} (Czech)

- \textit{Семь раз отмерь, один отрежь.} (Russian)

\textbf{Ĝostj v dom, Bôg v dom.}

- \textit{Gość w dom, Bóg w dom.} (Polish)

- \textit{Гость в дом, Бог в дом.} (Russian)

\textbf{Poläk je zlobnym dokoli glådnym.}

- \textit{Polak zły, póki głodny.} (Polish)

\textbf{Čelověk sę uči svojymi hybami.}

- \textit{Chybami se člověk učí.} (Czech)

- \textit{Человек учится на своих ошибках.} (Russian)
