\chapter{Phonology}

Every language has its own phonology. But do you know what phonology is?

Phonology - a branch of linguistics concerned with the systematic organization of sounds in the language.

Thus, phonology describes all the sounds that the language possesses. Sounds can be divided into different classes. One of the characteristics separating different sounds is the ability to pronounce them with an open vocal tract. You can notice that vowels (such as a, o, u in English) are able to be pronounced with open vocal tract, whereas consonants are pronounced with partly or completely closed vocal tract. So, let's discuss the sounds from the point of this classification.

However, firstly we should speak about the term allophone.

Allophone - one of a set of multiple possible spoken sounds or signs used to pronounce a single phoneme in a particular language.

Well, what is a phoneme? A phoneme is one of the units of sound that distinguish one word from another in a particular language. Simply put, it is such a sound (or a set of sounds) that do not influence the meaning of the word. 


\section{Vowels}

In the beginning of this paragraph I would like to write a description of a vowel.

Vowel - a sound produced with no constriction in the vocal tract.

With this information we can distinguish different types of vowels. The classification of vowels is based on two main factors:
What is the row of the sound
What is the height of the sound
Whether the vowel is rounded or not

The row is the position of the tongue when you pronounce a vowel. There are three main rows: front, central and back. When you pronounce a vowel at the front row, you move your tongue toward the teeth. The descriptions of central and back rows are similar - you move your tongue to the center or toward the larynx to pronounce them.

The height of the sound is a characteristic of tongue convexity and tension in your mouth. If it is positive, it means your tongue does not touches the palate nor bottom of the mouth, and the tip of your tongue is tense- or closed. If your tongue is flat and parallel to the bottom of your oral cavity, moreover, it lies on it - it is an open one. Between these two positions a middle position can be found.

The roundedness of the vowel is the amount of rounding in the lips during the articulation of a vowel. Vowels can be categorized as rounded and unrounded. Thus, to pronounce a rounded vowel you should round your lips. To pronounce an unrounded vowel, you should relax your lips during the articulation of a vowel.

Finally, we can talk about Novoslovnica phonology. Novoslovnica consists of 20 ordinary vowel phonemes. Among them there are seven closed vowels, three open vowels, ten middle vowels as well as seven front vowels, five center vowels and eight back vowels. On the table 1.1 you can see a chart position of the vowels in Novoslovnica.

\begin{table}[]
	\begin{tabular}{lllll}
		& Front & Center & Near-back & Back \\
		Close      &       &        &           &      \\
		Near-close &       &        &           &      \\
		Close-mid  &       &        &           &      \\
		Mid        &       &        &           &      \\
		Open-mid   &       &        &           &      \\
		Open       &       &        &           &     
	\end{tabular}
\end{table}

In this table you can also see different colors of the vowels. The green ones are for rounded vowels. Yellow ones are for unrounded vowels.

If you know Czech or Finnish, you might be concerned by the absence long vowels in this chart. It’s time to speak about allophones in Novoslovnica.

Novoslovnica has allophones of open and long vowels. This means that it does not matter how you pronounce a “modified” vowel - as a long one or as an open one - the meaning of the word will not change. To make this more clear, look at table 1.2.

