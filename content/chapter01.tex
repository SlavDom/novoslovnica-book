\chapter{Phonology}

Phonology\index{phonology} - a branch of linguistics concerned with the systematic organization of sounds in the language. Phonology describes all the sounds that the language possesses.

Sounds can be divided into different classes. One of the characteristics for separating different sounds is the ability to pronounce them with an open vocal tract. You can notice that vowels (such as \textit{a, o, u} in English) are able to be pronounced with open vocal tract, whereas consonants are pronounced with partly or completely closed vocal tract.

Moreover the term of allophone should be concerned before going further.

\textit{Allophone\index{allophone}} - one of a set of multiple possible spoken sounds or signs used to pronounce a single phoneme in a particular language.

This leads to the definition of a phoneme:

A \textit{phoneme\index{phoneme}} is one of the units of sound that distinguish one word from another in a particular language.

Therefore, an allophone is such a sound (or a set of sounds) that does not influence the meaning of the word. 

\section{Vowels}

In the beginning of this paragraph the description of a vowel should be given.

\newglossaryentry{vow}{name=V, description={Vowel}}

A \textit{vowel\index{vowel} (\gls{vow})} - a sound produced with no constriction in the vocal tract.

With this information we can distinguish different types of vowels. The classification of vowels is based on two main factors:

\begin{itemize}
	\item{What is the row of the sound}
	\item{What is the height of the sound}
	\item{Whether the vowel is rounded or not}
\end{itemize}

The \textbf{row\index{vowel!row}} is the position of the tongue when you pronounce a vowel. There are three main rows: front, central and back. When you pronounce a vowel at the front row, you move your tongue toward the teeth. The descriptions of central and back rows are similar - you move your tongue to the center or toward the larynx to pronounce them.

The \textbf{height\index{vowel!height}} of the sound is a characteristic of tongue convexity and tension in your mouth. If it is positive, it means your tongue does not touches the palate nor bottom of the mouth, and the tip of your tongue is tense- or closed. If your tongue is flat and parallel to the bottom of your oral cavity, moreover, it lies on it - it is an open one. Between these two positions a middle position can be found.

The \textbf{roundedness\index{vowel!roundedness}} of the vowel is the amount of rounding in the lips during the articulation of a vowel. Vowels can be categorized as rounded and unrounded. Thus, to pronounce a rounded vowel you should round your lips. To pronounce an unrounded vowel, you should relax your lips during the articulation of a vowel.

Finally, we can talk about Novoslovnica phonology. Novo-slovnica consists of 20 ordinary vowel phonemes. Among them there are seven closed vowels, three open vowels, ten middle vowels as well as seven front vowels, five center vowels and eight back vowels. On the table 1.1 you can see a chart position of the vowels in Novoslovnica.

\begin{table}[h]
\caption{Vowels in Novoslovnica}
	\begin{tabular}{lllll}
		& Front & Center & Near-back & Back \\
		Close      &  \textipa{i}          & \textipa{1|\textbf{0}} &                      &  \textipa{W|\textbf{u}} \\
		Near-close &  \textipa{\textbf{I}} &                        & \textipa{\textbf{U}} &                         \\
		Close-mid  &  \textipa{e}          & \textipa{\textbf{8}}   &                      &   \textipa{\textbf{o}}         \\
		Mid        &  \textipa{\|`e}       & \textipa{@}            &                      &  \textipa{\textbf{\|`o}} \\
		Open-mid   &  \textipa{E}          & \textipa{3}            &                      &   \textipa{2|\textbf{O}}    \\
		Near-open  &  \textipa{\ae}        &                        &                      &                 \\
		Open       &  \textipa{a}          &                        &                      &   \textipa{A} 
	\end{tabular}
\end{table}

In this table you can also see different vowels are font-styled differently. The bold ones are for rounded vowels. The normal ones are for unrounded vowels.

If you know Czech or Finnish, you might be concerned by the absence long vowels in this chart. It’s time to speak about allophones in Novoslovnica.

Novoslovnica has allophones of open and long vowels. This means that it does not matter how you pronounce a “modified” vowel - as a long one or as an open one - the meaning of the word will not change. To make this more clear, look at table 1.2.

\begin{table}[h]
	\caption{Long vowel allophones in Novoslovnica}
	\begin{tabular}{lll}
  \multirow{2}{*}{Main vowel} & \multicolumn{2}{c}{Modified vowel} \\
  	&	First allophone & Second allophone \\
  \textipa{E} & \textipa{E:} & \textipa{3} \\
  \textipa{a} & \textipa{a:} & \textipa{A} \\
  \textipa{u} & \textipa{u:} & \textipa{U} \\
  \textipa{o} & \textipa{o:} & \textipa{O} \\
  \textipa{\|`o} & - & \textipa{W}
	\end{tabular}
\end{table}

\textbf{Exception:}

\textipa{\v*{o}} (only one modified vowel - \textipa{W})

\textbf{Examples:}

\textit{Buda} \textipa{[‘buda]} (Buddha)

\textit{búda} \textipa{[‘bu:da]} (building) - \textipa{[‘bUda]}

If you have some knowledge about Slavic languages and their origins, you should know that the Proto-Slavic language had nasal vowels, which we can nowadays be found in Polish and Lithuanian. Do they exist in Novoslovnica? Of course they do! There are two allophones for pronouncing nasal vowels. The first one is actually a nasal vowel, when you pronounce an ordinary vowel through your nose. The second is more common, such as in French, when you add a nasal consonant \textipa{[N]} to an ordinary vowel. Look at the sounds on the next table.

\begin{table}[h]
	\caption{Nasal vowel allophones in Novoslovnica}
	\begin{tabular}{lll}
  \multirow{2}{*}{Ordinary vowel} & \multicolumn{2}{c}{Nasal vowel} \\
  &	First allophone & Second allophone \\
  \textipa{E} & \textipa{\~E} & \textipa{EN} \\
  \textipa{i} & \textipa{\~E} & \textipa{iN} \\
  \textipa{a} & \textipa{\~a} & \textipa{aN} \\
  \textipa{u} & \textipa{\~u} & \textipa{uN} \\
  \textipa{o} & \textipa{\~o} & \textipa{oN}
	\end{tabular}
\end{table}

\textbf{Examples:}

\textit{Dųb} (oak) \textipa{[d\~{O}b]} - \textipa{[duNb]}

\textit{Męso} (meet)\textipa{[‘m\~{E}so]} - \textipa{[`meNso]}

As you can see, Novoslovnica distinguishes between nasal vowels in two categories - O-nasality (hard) and E-nasality (soft). 

The last topic that we will speak of pertaining to vowels, is the firmness and the softness of the vowels. Scientists argue about what is primary in making sound soft or hard - consonants or vowels. Novoslovnica claim vowels are softer or hard primarily, although consonants itself also can be either soft or hard.

There are vowels that tend to make their surrounding soft and there are vowels that tend to make their surrounding hard. As you already know, there are two nasal vowels - one hard (O-nasality) and one soft (E-nasality). However, there are other pairs of hard-soft vowels among ordinary vowels. Look at table 1.4 for more information.


\begin{table}
	\caption{Hard and soft vowels}
	\begin{tabular}{ll}
		Hard vowel & Soft vowel \\
		\textipa{\~o} & \textipa{\~E} \\
		\textipa{u} & \textipa{0} \\
		\textipa{1} & \textipa{i} \\
		\textipa{E} & \textipa{\|`e} \\
		\textipa{o} & \textipa{8} \\
		\textipa{a} & \textipa{\ae} \\
		\textipa{I} & \textipa{e}
	\end{tabular}
\end{table}

\textbf{Examples:}

\textit{Dodatek} (addition) [do’datek]

\textit{Smërtj} (death) \textipa{[sm\t{e}rt’]}

It should be mentioned that the sounds \textsc and e that are shown in italic are the allophones for one phoneme. However, both these sounds exist in Novoslovnica and you can use whatever you like.

There is only one vowel that has no pair in softness/hardness. It is \textipa{@}. This vowel is named the “Schwa” sound and it can be described as the most “middle” sound among vowels. To pronounce it you should relax your oral cavity and pronounce a sound with weakened muscles. This is the “Schwa” sound.




\section{Consonants}

In this paragraph about consonants, I would like to begin with the definition of a consonant.

\newglossaryentry{cons}{name=C, description={Consonant}}

\textit{Consonant\index{consonant} (\gls{cons})} - a sound that is articulated with complete or particular closure of the vocal tract. 

Likewise vowels, consonants have three characteristics that determine their position in your articulation. These three parameters are:

\begin{itemize}
	\item{Place, where the consonant is pronounced in the mouth}
	\item{Manner, how the consonant is pronounced }
	\item{Sonority, whether you use your vocal cords or not}
\end{itemize}

\textbf{Place\index{consonant!place}} of the consonant can be quite different. Here are possible types: \textit{Bilabial, labiodental, dental, alveolar, postalveolar, palato-alveolar, retroflex, alveolo-palatal, palatal, labio-velar, velar}. There are more types, but they do not exist in Novoslovnica.

\textbf{Manner\index{consonant!manner}} is the way how you pronounce the sound. There are also different manners, that are used in Novoslovnica. They are: \textit{nasal, stop, affricate (sibilant), sibilant fricative, non-sibilant fricative, approximant, trill, lateral approximant}.

\textbf{Sonority\index{consonant!sonority}} is the boolean attribute of pronunciation. You can either use your voice \textit{with} the sound you pronounce or \textit{not}. Notice that vowels cannot be pronounced without the use of your voice. 

Combining these three parameters, we get the unique consonant that we want to pronounce. We cannot draw  a 3-dimensional table, because there are three parameters on input, so we will combine information into 2-dimensional space as in paragraph about vowels. So, look at the figure and see the different consonants that are used in Novoslovnica.

\begin{figure}[h]
	\includegraphics[width=\linewidth]{./sources/consonants.png}
	\caption{Consonant sounds in Novoslovnica}
	\label{fig:consonants}
\end{figure}

Different colors of the cells show the sonority of the consonant. Yellow color shows that the sound is voiced, while green ones are for voiceless sounds.

Blue cells in the table show that sounds in it can be used both in voiced and voiceless forms as allophones.

Novoslovnica has 51 consonants, 21 of them are voiceless and 30 are voiced.

However, not all of these consonants are language phonemes. So, let’s talk about the allophones among these sounds.

\begin{table}[h]
	\caption{Consonant allophones in Novoslivnica }
	\begin{tabular}{ll}
		Main consonant & Allophones \\
		\textipa{\t{\:d\textctz}} & \textipa{\t{\:dZ}} \\
		\textipa{\t{tC}}  & \textipa{\t{tS}} \\
		 \textipa{Z} & \textipa{\textctz}  \\
		  \textipa{S}  &  \textipa{C} \\
		  \textipa{l}   &  \textipa{(\r*l} \\
		   \textipa{r} (voiced)  &  \textipa{r} (voiceless) \\
		   \textipa{n} (alveolar) & \textipa{n} (palato-alveolar), \textipa{n} (dental) \\
		    \textipa{(\r*r} (voiced) & \textipa{(\r*r} (voiceless) \\
		    \textipa{v} & \textipa{V} \\
		    \textipa{f} & \textipa{\r*V} \\
		    \textipa{H} & \textipa{G} \\
		    \textipa{x} & \textipa{h} \\
	\end{tabular}
\end{table}

Likewise vowels, consonants can be compared with each other in terms of softness/hardness. The common rule is that every consonant has its soft or hard partner.

Exception: Three sounds are exceptions to this rule.
The sound \textipa{T} has no pair, because its pair ð has never been used in Slavic languages.
Also nasal velar consonant \textipa{N} has no soft pair.
And vice versa, the sound j is soft and has no hard pair. 

Remember this exception, let’s look at the table 1.7 to get acquainted with pairs of consonants.

\begin{longtable}{ll}
%	\caption{Softness in Novoslivnica}
		Hard consonant & Soft consonant \\
		\endhead 
		b & bj \\
		v & vj \\
		g & gj \\
		\textipa{H} & \textipa{Hj} \\
		d & \textipa{J} \\
		\textipa{\t{\:d\:z}} & \textipa{\t{\:d\textctz}} \\
		\textipa{\t{dz}} & \textipa{\t{dzj}} \\
		\textipa{\:z} & \textipa{Z}  \\
		z & zj \\
		k & kj \\
		l & \textipa{L} \\
		m & mj \\
		n & \textltailn \\ 
		p & pj \\
		r & rj \\
		\textipa{\r*r} & \textipa{\r*rj} \\
		s & sj \\
		t & \textipa{C} \\
		f & fj \\
		x & xj \\
		\textipa{\t{ts}} & \textipa{\t{tsj}} \\
		\textipa{\t{t\:s}}  & \textipa{\t{tC}} \\
		\textipa{\:s} & \textipa{S} \\
		w & \textvibyy \\
\end{longtable}

As you see, every consonant from table has its own soft pair. The soft consonant is usually written as a hard consonant with the sound j attached, but some of them are provided as unique sounds by IPA, because in some languages they can be phonemes. (footnote)


% любопытным следтвием наличия особых значков в IPA для мягких были попытки построения чисто фонетических алфавитов на основе только символов IPA, которые должны были использоваться в качестве букв. (дополнение можно внести в текст книги).
\section{Vowels}

By reading this paragraph you should be aware about Novoslovnica phonemes. At the beginning we will speak about two main features of the language: reproduction and alternation.

Reproduction (extra sounds) - is a process of adding a new sound (consonant or vowel) in the word.

Alternation - is a process of changing some sound (consonant or vowel) or sounds to another one(s).

What are the purposes of alternation and reproduction? These terms both obtain the single aim - to make our speech comfortable for ourselves.

Every language has its own comfortable combinations of sounds and avoided ones. Some languages alternate these uncomfortable character rows into another ones that are comfortable for language speaker in writing, some languages keep writing etymological and alternate the pronunciation of the uncomfortable words. Take into account the fact, that the concept of comfort is relative, that means two different languages (which are spoken by different nations) have different modes of comfort.
Example: you can look at and compare German and Dutch languages. They are very close, but they have different ways for comfortable pronunciation. The word “book” in German sounds like “Buch”, but in Holland it sounds like “boek”. Thus, you see that these words are familiar, but their pronunciation differs a lot.

Moreover, even one language spoken in different areas can differ greatly in the areas it is spoken.
Example: American and British English. You know that it is still the same language, but pronunciation differs so greatly that American person can not always understand in ordinary speech (quick temp of speaking) what the British person have said.

This is the example how the same language differs in areas it is spoken. Furthermore, these areas may not be far away from each other. You can find information about different patois and even dialects in very small regions. 

Novoslovnica as a panslavic language absorbs different conceptions of the comfort term of Slavic languages.

When reproduction is used, we add a new sound before the word we want to say. It can be whether a vowel or a consonant, depending on the previous letter in the word.

When should we reproduce a sound? To deal with this problem, you should know a thesis, which is widely used in Novoslovnica.

\textbf{Rule №2}: After consonant there should be placed a vowel. And after a vowel there should be placed a consonant.

This rule will help you in speaking and writing, when you construct your words with the reproduction.

There is a limited row of sounds that are allowed to participate in reproduction. In the table 2.1 you can see all of them.





\section{Alternation}

Alternation is a very important feature of Slavic languages. All of them provide some cases, when one letter changes to another one(s) and controversially. The cause is the fact that some sound combinations are difficult to be spoken or not comfortable for that. For example, Slavs can say “napisati” (to write), but “napi???at” (they (will) write). The variant “napisat” is not comfortable to pronounce and, moreover, it is not understandable. What are you speaking - “to write” with hardened T or “they write” with depalatalized S. So you can see, that these rules are often experimental and cannot be explained in a common way.

Alternation can be found mostly in conjugation or declension of the word, because the process is in changing uncomfortable forms to better ones while changing the base and endings of the word. 

In this paragraph I will speak about different palatalizations of vowels and consonants. Let’s begin with the consonants.

\textbf{Alternation S//Š}

This alteration appears in words with the letter S before a vowel A. The basic sound is S, which changes to Š before vowels I, E and in some cases before A in conjugation. Let’s look at examples to understand how it works.

\underline{Examples:}

\textit{Pisati} (write) \textipa{[’pisat1]} - \textit{piši} \textipa{[pi’\:s1]}

\textbf{Alternation K//Č}

This alteration appears in words with the letter K before a vowel A without a following vowel. The basic sound is K, which changes to Č before vowels Ě, N. Let us look at examples to understand how it works.

\underline{Examples:}

\textit{Věk} \textipa{[vIk]} - \textit{věčen} \textipa{['vI\t{tS}en]}

\textbf{Alternation C//Č}

This alteration appears in words with the letter C before a vowel A. The basic sound is C, which changes to Č before vowels I, E. Let us look at examples.

\underline{Examples:}

Ptica (bird) \textipa{[’pti\t{ts}a]} - ptičen (birdish) \textipa{[’pti\t{tS}en]}


\textbf{Alternation D//Đ}

This one appears in words with the letter D before a vowel I. This consonant changes to Đ before vowels A, E and U. Let us look at the examples.

\underline{Examples:}

\textit{Voditi} (drive) \textipa{[‘vodit1]} - \textit{vođati} \textipa{[‘vo\texttoptiebar{\textctdctzlig}at1]}

\textbf{Alternation Ǐ//J}

This alternation is very simple. We write Ǐ before a consonant or in the end of the word and we write J before a vowel. The exception is the case, when we write Ǐ in the end of the word, but the first letter of the next word is a vowel - then we pronounce J, but write Ǐ.

\underline{Examples:}

\textit{Môǐ} (my) \textipa{[mUj]} - \textit{moja} \textipa{[m2'Ja]}

However, not only consonants can change in the word when we conjugate or decline it. There are some alterations of vowels too.

\textbf{Alternation Ò//-}

This alternation appears in some old roots (see next paragraph).

\underline{Examples:}

\textit{Tòk} (stick) \textipa{[t@k]} - \textit{tkati} \textipa{['tkatI]} (to weave)


\textbf{Alternation E//-}

This alternation is similar to the previous ones, but exist in word suffixes.

\underline{Examples:}

\textit{Krasen} (Nice) \textipa{['kras@n]} - \textit{krasna} \textipa{['krasna]}

% \textbf{Alternation O//Ö}

% This alteration is used in some old roots (see next paragraph).

% \underline{Examples:}

% Ïdiot (idiot) \textipa{[id1’ot]} - ïdijot \textipa{[id1’Jot]}

% P.S. What are the differences between these two alternations? The variant which we choose depends on the exact word and its root. These roots ascend to proto-Slavic roots with vowels “Ò” and “J” (see next paragraph), that’s why they transforms to “-” and “??” respectively.

% \textbf{Alternation Ö//J}

\textbf{Alternation Ę//En}

This alternation is rather narrow, because it is used in the case of declension nouns of type 3 (look paragraph about noun declension), when the nasal vowel ??? alternates to non-nasal pair of vowel “E” and consonant “N” (non-nasal!). To understand it look at the example

\textbf{Examples:}

\textit{Vremę} (time) \textipa{[‘vrEmj\~E]} - \textit{vremenï} \textipa{[‘vrEmE\textltailn i]}

\textit{Plemę} (tribe) \textipa{[‘plEmj\~E]} - \textit{plemenï} \textipa{[‘plEmE\textltailn i]}

In the conclusion of this paragraph it should be mentioned that alterations are very important in Slavic languages and Novoslovnica as well. You can use reproduction in your speech as a recommended feature, while alterations are complimentary in this language. As it was noted before, we cannot ignore anything that can bring a misunderstanding in our speech. 

\section{Runaway vowels}

Looking back in the Slavic language history we can find out that there were roots with two strange for an ordinary person vowels - “Ò” and “J”. First one was named “Yer” and denotes a hard mid central vowel (Shwa). The second one was named “Yerj” (with soft R) and denotes a soft mid central vowel. Now the second one is lost and we use only shwa sound in the letter “Yer”. However, the words are still and we need to pronounce them in some way. Novoslovnica uses the soft “E” sound to represent roots with old soft shwa sound.

Main feature of these sounds was to fall out from the root, when a vowel appears afterwards. That’s why there are many words with two consonants consecutively - there is an imaginary shwa sound between them that has been fallen out from the root.

Nevertheless, despite falling out of “Yer”, soft “E” in this places does not fall out. So, in the previous paragraph you could see that there are two alternations O//- and O//Ë, that are handled in the similar positions. So the answer on the question, why in the first case there is no sound and in the second there is a soft E is the fact, that words satisfying the first case comprise old hard shwa sound and remain comprise old soft shwa sound, that has transformed into soft E.

I should mention also the fact, that nowadays the letter Ò exists only in roots of the words. In suffixes the letter E is used for this sound and in the prefixes the letter O is used. Look at the examples:
pod
ek
pòk  

You should remember that speaking the words with these letters we should pronounce them just as they are written - Ò as shwa, E as E, O as O. You should not reduce all the sounds to shwa. 

\section{Accent}

Accent is a very difficult topic in most languages, because it is not permanent. There are some exceptions i.e. Czech and French, but in most cases we cannot say where accent will be put in the word without studying definite language. This causes problems for beginners.

Novoslovnica has a dynamic accent, but it has been formalized. There is a rule, that determines the place you should put the accent. 

Rule \#2: The accent should be put on the first syllable of the word root.

This rule covers about 80\% of the words in the lexicon. You know the well-known 80/20 rule, or Pareto principle. It is something alike with the accent. Remained 20 per cent of the words we should cover by introducing extra-rule cases. Accepting these cases, you will be able to cover more than 99 per cent of Novoslovnica word amount.

These cases were created in attempt to unify the accents in different Slavic languages. Surely the Slavic languages have greatly changed since then, as they were one language. Therefore, accents in different Slavic languages often differ. Nevertheless, Novoslovnica tries to obliterate differences between them, producing accent patterns that could be comfortable to pronounce and to hear for all Slavs.

Below you can see the list of all these cases, that you should remember while speaking Novoslovnica.

Accent shifting cases:
Accented endings (Nouns)
-a (Dual, Nominative)
-y (Singular, Genitive/Partitive)
-ami (Plural, Instrumentative)
-ama (Dual, Dative)
-am (Plural, Dative)
Accented endings (Verbs)
“-I” Imperative (see paragraph about verb moods)
Accented suffixes
-ova- (Verb)
-ôva- (Verb)
-ava- (Verb)
-óv- (Adjective)
verb suffixes in Present Concrete Tense (see paragraph about verb tenses)
Accent shift in the root
If the word is a borrowed one, then the accent is put on the place it is in the original word.
If the root loses its vowel, the accents moves one vowel to the left (if it is possible)
Words that have more than one root (complex words) have their accent on the first syllable of the main word’s root. (see paragraph about complex words for what root is main) 
Adverbs or other parts of speech, formed with the prepositional construction, have their accents on the first syllable of the main word (see paragraph about collocations)

These rules are enough for you to speak Novoslovnica properly with a few efforts for it. 

\section{Accent Integrity}

There is also one term, that you should to know when you use Novoslovnica or any other Slavic language in your speech. This term is called accent integrity. Firstly I will introduce a term of a dependency structure:

\textit{Dependency structure} is a prepositional construction or a collocation.

That means, that this abstract term expands the term of collocation by involving prepositional into itself. 

\textit{Accent integrity} is a property of a dependency structure to unify elements of this structure with only one accent on the main word.

What does this definition say? If we have a collocation or a prepositional construction and we want to pronounce it, we should pronounce dependent words without any accent and put an accent on the main word of the structure. Look at the examples.

Examples:


d

Surely, you should remember that this might be applied only for brief structures, most often with one dependent word (or a preposition) of one or two syllables. If we have a long dependent word or there are too many dependent words in the construction, we pronounce them with a proper accent on each of the structure elements.

Examples:


d


