\section{Example texts in Novoslovnica}


\textbf{The agnostic’s prayer}

English original text (R. Zelazny):

Insofar as I may be heard by anything, which may or may not care what I say, I ask, if it matters, that you be forgiven for anything you may have done or failed to do which requires forgiveness. Conversely, if not forgiveness but something else may be required to insure any possible benefit for which you may be eligible after the destruction of your body, I ask that this, whatever it may be, be granted or withheld, as the case may be, in such a manner as to insure your receiving said benefit. I ask this in my capacity as your elected intermediary between yourself and that which may not be yourself, but which may have an interest in the matter of your receiving as much as it is possible for you to receive of this thing, and which may in some way be influenced by this ceremony. Amen.


Novoslovnica translation (tr. by R. Gasparyan):

\textbf{Litany against fear}

English original text (F. Herbert):

I must not fear.
Fear is the mind-killer.
Fear is the little-death that brings total obliteration.
I will face my fear.
I will permit to pass over me and through me and when it has gone past I will turn the inner eye to see its path.
Where the fear is gone there will be nothing.
Only I will remain. 

Novoslovnica translation (tr. by R. Gasparyan):

The hymn of Slavs (Hej, Slovania)

Slovak original text (S. Tomášik):

\begin{verse}
	Hej, Slovania, ešte naša \\
	slovenská reč žije, \\
	Dokiaľ naše verné srdce \\
	za náš národ bije.
	
	Žije, žije, duch slovenský, \\
	bude žiť naveky, \\
	Hrom a peklo, márne vaše \\
	proti nám sú vzteky!
	
	Jazyka dar zveril nám Boh, \\
	Boh náš hromovládny, \\
	Nesmie nám ho teda vyrvať \\
	na tom svete žiadny;
	
	I nechže je koľko ľudí, \\
	toľko čertov v svete; \\
	Boh je s nami: kto proti nám, \\
	toho Parom zmetie.
	
	A nechže sa i nad nami \\
	hrozná búrka vznesie, \\
	Skala puká, dub sa láme \\
	a zem nech sa trasie;
	
	My stojíme stále pevne, \\
	ako múry hradné. \\
	Čierna zem pohltí toho, \\
	kto odstúpi zradne!
\end{verse}

Novoslovnica translation (tr. by. G. Carpow):

