\section{Example texts in Novoslovnica}


\textbf{The agnostic’s prayer}

English original text (R. Zelazny):

Insofar as I may be heard by anything, which may or may not care what I say, I ask, if it matters, that you be forgiven for anything you may have done or failed to do which requires forgiveness. Conversely, if not forgiveness but something else may be required to insure any possible benefit for which you may be eligible after the destruction of your body, I ask that this, whatever it may be, be granted or withheld, as the case may be, in such a manner as to insure your receiving said benefit. I ask this in my capacity as your elected intermediary between yourself and that which may not be yourself, but which may have an interest in the matter of your receiving as much as it is possible for you to receive of this thing, and which may in some way be influenced by this ceremony. Amen.


Novoslovnica translation (tr. by R. Gasparyan):

Ako někto abo něčto može učuti mę i može abo ne može uslyšati mę, če ja bųdu mòlvati, ja prosim, ako prosjba ïmáje značene, dabi byl ty odpušten za če ty byša dělal abo ne dělal i če trěba odpuštati. I ako ne odpuštene, no něčto ïnoje ješto bųde mogati služiti k tvojoǐ poljzě poslě uničtoženä tvojoga těla, ja prosim dabi to něčto ïnoje bilo dano abo ne dano, v odnošenü k slučajam, aby dati ti taku poljzu. Ja prosim to ako tvôǐ posrědnik među tobom i onym, ke može byti ne jesi tobom, no ke ïmáje korystj da ty hte prijęš jak je možno bolëǐ toga ïnoga, če ty možeš prijęti od neĝa, i čto sïǐ obręd može několïko izměniti. Aminj.

\textbf{Litany against fear}

English original text (F. Herbert):

I must not fear.
Fear is the mind-killer.
Fear is the little-death that brings total obliteration.
I will face my fear.
I will permit to pass over me and through me and when it has gone past I will turn the inner eye to see its path.
Where the fear is gone there will be nothing.
Only I will remain. 

Novoslovnica translation (tr. by R. Gasparyan):

Ja ne musim strašiti sę.
Strah je ubiǐcy råzuma.
Strah je malaja smërtj, prinošajučä pòlnoje uničtožene.
Ja bųdu sę větati s mojym strahom oblïka k oblïkě.
Ja hte umožnam sę jemu proǐdati ponad mnoju i skrôz mę, i koĝda on htěše da sę proǐdal, ja bųdu sę smotrati vųtrešnym okom i bųdu sę viděti pųtj ĝo.
Ĝde byše sę proǐdal strah, tamo bųde ničto.
Zôstam sę samo ja.

The hymn of Slavs (Hej, Slovania)

Slovak original text (S. Tomášik):

\begin{verse}
	Hej, Slovania, ešte naša \\
	slovenská reč žije, \\
	Dokiaľ naše verné srdce \\
	za náš národ bije.
	
	Žije, žije, duch slovenský, \\
	bude žiť naveky, \\
	Hrom a peklo, márne vaše \\
	proti nám sú vzteky!
	
	Jazyka dar zveril nám Boh, \\
	Boh náš hromovládny, \\
	Nesmie nám ho teda vyrvať \\
	na tom svete žiadny;
	
	I nechže je koľko ľudí, \\
	toľko čertov v svete; \\
	Boh je s nami: kto proti nám, \\
	toho Parom zmetie.
	
	A nechže sa i nad nami \\
	hrozná búrka vznesie, \\
	Skala puká, dub sa láme \\
	a zem nech sa trasie;
	
	My stojíme stále pevne, \\
	ako múry hradné. \\
	Čierna zem pohltí toho, \\
	kto odstúpi zradne!
\end{verse}

Novoslovnica translation (tr. by. G. Carpow):

\begin{verse}
	Geǐ, slověnïe, ješto živo je \\
	Slovo našyh dědov, \\
	Dokųdy odvažno sredco bjije \\
	Našyh slavnyh synov.
	
	Žije, žije duh slověnskyǐ \\
	Živ je toǐ navěkï. \\
	Pěklo je naprasno stalo \\
	Proti nas sëdy.
	
	I podariše jazyk Bôg nam \\
	Našyǐ Bôg-Vlådatelj. \\
	Toǐ je s nami, nikoǐ mogne \\
	Pěsnü našu-ta prestati.
	
	I něhaǐ nad nami búrä \\
	Vznese do něba ŭvysj, \\
	Skala pęka, dųb sę lôma \\
	Zemlötręs tęĝna ŭniz.
	
	Stojeme my tvòrdy, sïlny \\
	Jako stěny gråda \\
	I bųdi proklętym toǐ, ke \\
	Je změnivcom naroda.
\end{verse}

Dvôgovor 1

— Don’t understand anything: everything is already decided, yet you have not booked the tickets… Do you hope for good luck?
— I’m going only next week. I have eight days left to buy train tickets. If I don’t but them, I’ll go by a bike!
— You’re joking! You can’t go far such a way. I have a brilliant idea: let’s take a taxi!
— I suppose I should not notice that I have no money?
— Come on! If you want to, we’ll go by bus, but we will there in six hours. Look, why don’t you want to go by car?
— I can’t find keys
— Yeah… There’s always something going on with you! I think, we have to look for your car keys.


— Råzumim ničto: vsë je porěšano, ale ty ješto ne si zarezerviral [zakupil] bilěty… Znovu jesi sę zanadějal na uspěh?
— Ja idam samo na slědnoji sedmicji. Zostahu vsï 8 (osem) dnï aby kupiti bilěty na vlåk. Ako ne kupim, hte idam na dvojokolu!
— Ješto žartuješ! Tako ty ne hte idaš dalïko. Imám ĝenijalnu videju: haǐ bëraǐmo taksi!
— Mysläm, že ne trěba pojasnäti, če ne mám pěnęŝëǐ?
— Prestani! Ako tako hteš, hte idama na lüdovozu, ale bųdema jěhati 6 (šest) godin [časov]. Sluhaǐ, a začto ne hteš jěhati na sámovozu?
— Ne mogam najidti klüčëǐ.
— Da… Vsëĝda něčto byvaje s tobom! Mysläm, trěba ješto pošukati klüčï od tvojoga sámovoza.


Dvôgovor 2

— The country has a crisis, inflation, but the number of billionaires has doubled, as well as horror-and salaries and pensions are ridiculous. And you decided to marry…
— Times change, but there’s still no perfect moment. When you and your mother got married 30 years ago, the climate in the country was even more alarming, the economy was shaken, corruption was terrible. And I got a good job, I have a reliable job, and a lot of good amassed: and I have a house, and a wheelbarrow, and a dacha. Let’s live happily ever after!
— You’re right. And will you go only to the registry office or to the Church too?
— We are sure to go to Church! Not what you were expecting? Despite the fact that she is Orthodox and I am Catholic, we have already agreed on everything. 
Last Sunday we went to the morning service, talked to the priest, received a blessing. We were allowed to get married in the temple, all as it should be.
— Well and good, it’s already time for you to marry, so to speak, to start a new life. And where will the festive feast be organized?
— That’s sick. I want to go to some awesome restaurant, service and live music, and she wants the holiday to be full of old farm, in the village, with traditional music, accordion, balalaika, etc, as well as environmentally friendly products, a real Russian feast, she says. I mean, honestly, it’s kind of freaking me out, but I can’t lose face in front of my future wife. On Saturday we go to watch some old farm with future mother-in-law and father-in-law, whether they were born there, or grew up, but in General it is very dear to them. It is true there are repairs to do, Yes, we still have time. So in the coming months I will work hard at work, and in the evenings on the farm, so that everything is ready for the wedding, as soon as possible. If only forces sufficed, because there is so much work, and I am one…
— Well, I see what you’re getting at.…
— Father, what is it worth, you’re a foreman at the biggest construction site in the city!
— Okay, got it, I’ll send you their builders, one you clearly not cope!


— Ïmáme krizu v dòržava-ta, ïmáme inflaciju, pa samo broǐ milïardarov je sę povysil, a tako hudo — i råbotna plata, i penza sų směšno malě. A vy nahtěla sų da zaženate.
— Doba sę měni, ale idejalnoga momenta ne hte naměrima. Ĝda va s mamoǐ sę ženiha 30 ročin nazad, položenije v dòržavji byše po-nepokôǐnym, ékonomika byše råzložena, korupcija byše strašna. A ja sę ziskah dobro, råbotu mám tvòrdu, ta ǐ vękostj blåg nažil sòm: dom ïmám, vozidlu, vilu [hatku]. Bųdeme žiti zadovôlno!
— Da, jesi pravym. Vy samo do SRCS [Služba registracijy civïlnyh sprav] hte idate ili takođe do còrkvy?
— Obovęzno hte idame do còrkvy! Čto, ne čakal jesi? Navzdor če ona je pravoslávnoju, a jesòm kaŧolikom, my vsë sma dogovorila. V mïnulu nedělü hodihma do jutróvoǐ služby, govorihma s svęštenikom [duhóvnikom], dostiĝala sma blågoslovnostj. Dostala sma zvolene za da sę věnčati [korônovati] v hramu kakto trěba.
— Ta ǐ dobro, vremę juž je da sę ženiš, kako kažut, započïnati nov život. A de bųde svatëbne veselije?
— To je vëlïkyǐ zapyt. Htem v nějakomu dužo dobromu restoranu dabi posluga byla dobra i živa gudba, a ona hte dabi prázdnik byl na nějakomu staromu hutoru, v selištu, s tradiciǐnoju gudboǐ, bajanom, balalaǐkoǐ i td, a takođe ékologično čïstymi produktami pa da s russkym stolovanëm, kako ona kaže.To mi pravdě izvodi iz sebę, ale ne mogam da sę zastydim pred bųdučëǐ ženoǐ. V sųbotu idujema glědati nějak star hutor s bųdučïma tëstëǐ i tëstëm. Da li rodila sų tamo, da li izrostnula, ale ona je mnogo dråga dlä nih. Ïstinskě, tamo trěba dělati remont, ama ïmáme vremę. Tomu, v predny měsęcy hte bųdu råbotati, a věčorom pråcovati na hutoru dabi vsë bylo gotovym k svatjbě. Samo dostati sïl, bo tako mnogo råboty tamo je.
— Pa, viđu, dlä čoga ty govoriš…
— Otče, pa čto ti to stoji, jesi nadzornikom na naǐ-vëlïkoji búdovji vo grådu!
— Dobro, ugovoril jesi mę, hte poslam ti búdóvnikov si, sám ty věrno ne sę spraviš!






