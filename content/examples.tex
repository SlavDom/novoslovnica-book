\section{Example texts in Novoslovnica}


\subsection{The agnostic’s prayer}

\textbf{English original text (R. Zelazny):}

Insofar as I may be heard by anything, which may or may not care what I say, I ask, if it matters, that you be forgiven for anything you may have done or failed to do which requires forgiveness. Conversely, if not forgiveness but something else may be required to insure any possible benefit for which you may be eligible after the destruction of your body, I ask that this, whatever it may be, be granted or withheld, as the case may be, in such a manner as to insure your receiving said benefit. I ask this in my capacity as your elected intermediary between yourself and that which may not be yourself, but which may have an interest in the matter of your receiving as much as it is possible for you to receive of this thing, and which may in some way be influenced by this ceremony. Amen.


\textbf{Novoslovnica translation (tr. by R. Gasparyan):}

Ako někto abo něčto može učuti mę i može abo ne može uslyšati mę, če ja bųdu mòlvati, ja prosim, ako prosjba ïmáje značene, dabi byl ty odpušten za če ty byša dělal abo ne dělal i če trěba odpuštati. I ako ne odpuštene, no něčto ïnoje ješto bųde mogati služiti k tvojoǐ poljzě poslě uničtoženä tvojoga těla, ja prosim dabi to něčto ïnoje bilo dano abo ne dano, v odnošenü k slučajam, aby dati ti taku poljzu. Ja prosim to ako tvôǐ posrědnik među tobom i onym, ke može byti ne jesi tobom, no ke ïmáje korystj da ty hte prijęš jak je možno bolëǐ toga ïnoga, če ty možeš prijęti od neĝa, i čto sïǐ obręd može několïko izměniti. Aminj.

\subsection{Litany against fear}

\textbf{English original text (F. Herbert):}

I must not fear.

Fear is the mind-killer.

Fear is the little-death that brings total obliteration.

I will face my fear.

I will permit to pass over me and through me and when it has gone past I will turn the inner eye to see its path.

Where the fear is gone there will be nothing.

Only I will remain. 

\textbf{Novoslovnica translation (tr. by R. Gasparyan):}

Ja ne musim strašiti sę.

Strah je ubiǐcy råzuma.

Strah je malaja smërtj, prinošajučä pòlnoje uničtožene.

Ja bųdu sę větati s mojym strahom oblïka k oblïkě.

Ja hte umožnam sę jemu proǐdati ponad mnoju i skrôz mę, i koĝda on htěše da sę proǐdal, ja bųdu sę smotrati vųtrešnym okom i bųdu sę viděti pųtj ĝo.

Ĝde byše sę proǐdal strah, tamo bųde ničto.

Zôstam sę samo ja.

\subsection{The hymn of Slavs (Hej, Slovania)}

\textbf{Slovak original text (S. Tomášik):}

\begin{verse}
	Hej, Slovania, ešte naša \\
	slovenská reč žije, \\
	Dokiaľ naše verné srdce \\
	za náš národ bije.
	
	Žije, žije, duch slovenský, \\
	bude žiť naveky, \\
	Hrom a peklo, márne vaše \\
	proti nám sú vzteky!
	
	Jazyka dar zveril nám Boh, \\
	Boh náš hromovládny, \\
	Nesmie nám ho teda vyrvať \\
	na tom svete žiadny;
	
	I nechže je koľko ľudí, \\
	toľko čertov v svete; \\
	Boh je s nami: kto proti nám, \\
	toho Parom zmetie.
	
	A nechže sa i nad nami \\
	hrozná búrka vznesie, \\
	Skala puká, dub sa láme \\
	a zem nech sa trasie;
	
	My stojíme stále pevne, \\
	ako múry hradné. \\
	Čierna zem pohltí toho, \\
	kto odstúpi zradne!
\end{verse}

\textbf{Novoslovnica translation (tr. by G. Carpow):}

\begin{verse}
	Geǐ, slověnïe, ješto živo je \\
	Slovo našyh dědov, \\
	Dokųdy odvažno sredco bjije \\
	Našyh slavnyh synov.
	
	Žije, žije duh slověnskyǐ \\
	Živ je toǐ navěkï. \\
	Pěklo je naprasno stalo \\
	Proti nas sëdy.
	
	I podariše jazyk Bôg nam \\
	Našyǐ Bôg-Vlådatelj. \\
	Toǐ je s nami, nikoǐ mogne \\
	Pěsnü našu-ta prestati.
	
	I něhaǐ nad nami búrä \\
	Vznese do něba ŭvysj, \\
	Skala pęka, dųb sę lôma \\
	Zemlötręs tęĝna ŭniz.
	
	Stojeme my tvòrdy, sïlny \\
	Jako stěny gråda \\
	I bųdi proklętym toǐ, ke \\
	Je změnivcom naroda.
\end{verse}

\subsection{Pamętnik sobě sòm zbúdal az netvornyǐ… (A.S. Pushkin)}

\textbf{English translation (A.Z. Foreman)}

\begin{verse}
	I've reared a monument not built by human hands. \\
	The public path to it cannot be overgrown. \\
	With insubmissive head far loftier it stands \\
	Than Alexander's columned stone.
	
	No, I shall not all die. My soul in hallowed berth \\
	Of art shall brave decay and from my dust take wing, \\
	And I shall be renowned whilst on this mortal earth \\
	Even one poet lives to sing.
	
	Tidings of me shall spread through all the realm of Rus \\
	And every tribe in Her shall name me as they speak: \\
	The haughty western Pole, the east's untamed Tungus, \\
	North Finns and the south steppe's Kalmyk.
	
	And long shall I a man dear to the people be \\
	For how my lyre once quickened kindly sentiment, \\
	I in a tyrant age who sang of liberty, \\
	And mercy toward fallen men.
	
	To God and his commands pay Thou good heed, O Muse. \\
	To praise and slander both be nonchalant and cool. \\
	Demand no laureate's wreath, think nothing of abuse, \\
	And never argue with a fool.
\end{verse}

\textbf{Novoslovnica translation (tr. by G. Carpow)}

\begin{verse}
	Pamętnik sobě sòm zbúdal az netvornyǐ, \\
	Ne hte pogyna pųtj dozdě z narodnyh trop. \\	
	Vozstaše vysšeǐ glåvoǐ si on nepokornyǐ \\
	Něž Aleksandriǐskyǐ stòlp.
	
	Ne, cěl ne hte umjram — u žadanomu gudku \\	
	Môǐ tlěn prežive duša, sę sŭobodivša z plït, \\	
	Posláven bųdu az doĝda pod něbom lunnym \\
	Živ bųde jeden hotj pijit.
		
	Posluh o mně prez Rusj hte preǐde vmïg \\	
	I hte nazva mi vsęk jestvučïǐ jazyk v nëǐ \\	
	Slověnov gòrdyǐ ŭnuk, i fïnnov, nyně dïk \\	
	Tunĝus, kalmyk, prijatelj do polëǐ.
	
	I dôlgo bųdu ja prijętnym tym narodu, \\	
	Če gudkom az vozbųđěh dobryh čuǐstv, \\	
	Če v môǐ krut věk vozslávih az sŭobodu, \\	
	Dozovah mïlostj az do mërtvyh duš.
	
	Bože slovo, muzo, čuǐ, bųdi poslušna, \\	
	Ne sę straši ty krivdy, o věncě ty ne dbaǐ, \\	
	Pohvalu ǐ klěvetu priǐmaǐ ty råvnodušno \\	
	I so glupakom ne spëraǐ.
\end{verse}

\subsection{Dvôgovor 1}

— Don’t understand anything: everything is already decided, yet you have not booked the tickets… Do you hope for good luck?

— I’m going only next week. I have eight days left to buy train tickets. If I don’t but them, I’ll go by a bike!

— You’re joking! You can’t go far such a way. I have a brilliant idea: let’s take a taxi!

— I suppose I should not notice that I have no money?

— Come on! If you want to, we’ll go by bus, but we will there in six hours. Look, why don’t you want to go by car?

— I can’t find keys

— Yeah… There’s always something going on with you! I think, we have to look for your car keys.

\begin{center}
	\textbf{***}
\end{center}

— Råzumim ničto: vsë je porěšano, ale ty ješto ne si zarezerviral [zakupil] bilěty… Znovu jesi sę zanadějal na uspěh?

— Ja idam samo na slědnoji sedmicji. Zostahu vsï 8 (osem) dnï aby kupiti bilěty na vlåk. Ako ne kupim, hte idam na dvojokolu!

— Ješto žartuješ! Tako ty ne hte idaš dalïko. Imám ĝenijalnu videju: haǐ bëraǐmo taksi!

— Mysläm, že ne trěba pojasnäti, če ne mám pěnęŝëǐ?

— Prestani! Ako tako hteš, hte idama na lüdovozu, ale bųdema jěhati 6 (šest) godin [časov]. Sluhaǐ, a začto ne hteš jěhati na sámovozu?

— Ne mogam najidti klüčëǐ.

— Da… Vsëĝda něčto byvaje s tobom! Mysläm, trěba ješto pošukati klüčï od tvojoga sámovoza.


\subsection{Dvôgovor 2}

— I love holidays! We do nothing, relax, go to a sauna, and our husbands go fishing.

— Yes, it is true, you come home late at night and then still have a dinner to be cooked… And I would like instead of cleaning the nursery and standing by the cooker to read a magazine, to watch TV quietly, or to sit at the bar.

— You work too much: as for me, even out of vacation I go to the cinema, to bars…

— Of course, you don’t have kids yet, and I have a five-year-old son and a grown-up daughter. And you need to work less: your husband can borrow his brand new car, and my husband does not have a car at all.

— That’s also true, but my husband is always saying how I should drive the car: «it is necessary to drive at 60 km / h speed! The car has been parked unsuccessfully!» And nobody tells you how to live.»

— Um… Don’t know what’s better. I have received a drive card, I bought a car on my own, but my husband still laughs at the way I drive.

— Well, I suppose he can express his opinion only in jokes…

— Wow! You are for him!

— Come on, I’m kidding, and you are always getting offended. Let’s just go to the river. At a shallow depth there you can see fish with amazing scales, huge tails and small-small fins.

\begin{center}
	\textbf{***}
\end{center}

— Obožam odpusk! Ničto dělame, počivame, hodime do saŭny, a našy mųžy hođut rybovati.

— Da, ïstina, vretaš dodomu z råboty pozdno, poslě trěba ješto večerü gotovati… A htěla bih zaměsto něž čïstiti dětsku komnatu [světlicu] i stojati na kuhnji, to lïstati žurnal, glědati TV ili spokôǐno sidati v báru [kòrčmji].

— Ty dužo mnogo råbotaš: ja hođam v kinemu [kretadlo], v bár [kòrčmu] aǐ ne v odpusku…

— Råzbëra sę, ne máš dětëǐ, az ïmám pętïročinnoga syna i doroslu dčerj. I råbotati trěbaš po-malo: tvôǐ mųž može požičiti ti svôǐ novyǐ sámovoz, a u mojoga ne má.

— To je takođe ïstina, ale mi mųž poslě neustanno govori, čto i kako musim dělati: «Jěhati trěba s bystrostïü 60 km/g! Sámovoz je hudo zaparkovan.» Ami ti nikto govori, kako trěba žiti.

— Hm… Aǐ ne znam, čto je po-dobro. Ja davno ziskah licenzu [vozne pravo], sámovoz kupih sámostatno, ale mųž vsë pako sę směje, kako vodim.

— Z mojoga poglědu, on samo v žartah može izkazati svoje mněne…

— Prošu! Ty si za ĝo!

— Prestani, žartuju, a ty, kako vsëdy, sę obidaš. Něhaǐ idama do rěky. Na malkoji glųbinji je možno uzreti ryb s čuděsnoju šupinoǐ, vëlïkymi hvostami i malymi-malymi plytvami.


\subsection{Dvôgovor 3}


— I heard you got so much makeup for your birthday that you can open a beauty salon!

— What nonsense! Husband made a present: gifted a bit make-up.

— Didn’t you choose it yourself? I’m sure he followed your instructions.

— No, he bought everything himself: I was sick, so I did not even go to the store.

— Damn it! What was wrong with you?

— I don’t know, my stomach hurt, I had diarrhea… maybe it was because of the berries I had eaten before.

— How are you feeling now?

— The pill my doctor had prescribed me started acting very quickly and I felt better. And just in case I’ve drunk paracetamol this morning. I’d rather retire now! I think I’m just working too much.

— You look like mom. She thought about work all the time, too.

— Well, tell me about yourself. You’re going on vacation soon, right?

— Yes, Baikal. I’ve been told so much about him, and as one says, «better to see once than hear a hundred times.»

\begin{center}
	\textbf{***}
\end{center}

— Slyšala sòm, že na tvôǐ rođen denj podarihu ti tolïko mnogo kosmetiky až možno odkryti kosmetičnyǐ salon!
— Jaka glupostj! Mųž ztvoriše podarek: nemnogo makijaža.

— Da li ne ty izbëraša vsë? Postoǐna sòm, če on slědovaše tvojim ukazam.

— Ne, on kupiše vsë sámostatno: běh hŭora, tomu daže do tòrgovišta ne pojěhah.

— Preklętno! Čto sę staše s toboǐ?

— Navět ne znam: bolěše brüho, běše diareja [drisk] … Može byti, to z jaĝod, ktoryh ja byh zjědla po-rano.

— A kako jesi nyně?

— Pilulka, ktoru mi byše dal lěkar, započïnaše dějati vëlïmi bystro, i mi staše po-dobro. Za vsęko, dnešnym utrom izpjila sòm paracetamol. Skoro by preǐdti na penzu [dohodku]! Sę kaže mi, če prosto premnogo råbotim.

— Jabelko od jablonï nedalïko pada: jesi kakto svoja materj! Ona takođe vesj čas myslěše o råbotě.

— Dobro, haǐ kaži o sobě. Že skoro jěhaš odpuskovati?

— Da, do Baǐkala. Tolïko mnogo sòm råzkazovahu mi o nem. I kakto sę govori, «Je dobrëǐ jednađy uviditi, něž sto kråt uslyšati’.


\subsection{Dvôgovor 4}

— Thanks to you, I got access to an unlimited amount of information.

— Thanks to the world wide web, not me: I found this site on some forum. As soon as I was connected to the Internet, I created my blog, where I now advise such interesting resources. By the way, my blog has become quite famous and attracts a huge number of users of all ages.

— And I’m just taking the first steps on the Internet, just opened for myself e-mail. It turned out that in addition to the usual text by e-mail, you can send photos and sound messages.

— Of course, and any other files.

— But despite all the benefits that the computer gives us, there are problems. I burned down the system unit, had to drag it to the workshop, and it is heavy…

— I have a laptop. Of course, it is easier to be carried to workshop. But here are its disadvantages: sometimes I can not find my laptop, which means that my son is playing online! You know, the web has many games for adolescents and you can't drag them off the Internet. In fairness I must say that he is not only playing online, but also a lot of work.

— All mom! Deep down, I never doubted your children’s talents.

— Daughter does not care computers and programming, she’s a DJ! For almost two years she has been working on the radio, in a musical program with requests of radio listeners.

— Lucky people! I myself am a music lover, always dreamed of becoming a DJ.

— Maybe I underestimated this profession, but first I was swearing to myself, I wanted her to become someone else, and now I realize that she is a real specialist, and I look at her with my mouth open.

— It’s great that you understand that, because parents always doubt the abilities of their children.

\begin{center}
	\textbf{***}
\end{center}

— Dęka ti polųčil sòm dostųp k neograničnomu objimu informacijy.

— Dękuǐ vsëmirnoǐ mrežě, a ne mi: ja sáma znaǐdala sòm ovyǐ saǐt na nějakomu forumu. Samo poslě dovęzanä Mïromrežy, ja stvorih svôǐ bloĝ, de nyně sovětam podobny korystny resursy [zdrojy]. Među tym, môǐ bloĝ je stal dostatočno znanym i pritęga vëlïke množestvo korystnikov vsękoga vozrôsta.

— Ami ja samo dělam pòrvy krocy v Mïromrežji, samo poznal sòm élektronnu poštu. Ujavilo sę, že kromě zvyčaǐnoga teksta prez élektronnu poštu je možno dosylati fotosnïmcy i zvųkovy poslanijy.

— Råzbëra sę, ta ǐ vsęky ïnšy faǐly možno je.

— Ale naprek vygodam od komputera, jestvujut i problemy. Zgorila je moja sistemóva jednota, mál sòm da tęgam ĝo do dělnicy, a on je tęžek.

— Ïmám noŭtbuk. To je po-prosto, v dělnicu ne tako tęžko je da nosiš. Ale jestvujut i mïnusy: něĝda ne mogam naǐdati svôǐ noŭtbuk, a to znači, če môǐ syn gra onlaǐn! Znaš li, če vo mrežji su tako mnogo ïger dlä podrôstkov i ne má sïl ih iztęgati iz Mïromrežy. Ale spravidlóvo trěba da kazam, če vo mrežji on ne samo sę zabavi, no i mnogo rádi.

— Jest sám cěl kakto matj! Vnųtrě dušy niĝda ne sę kolěbah v talentah [darnostäh] tvojyh čęd [dětëǐ].

— Dčeri mi komputery i proĝramovane ne su korystny, ona je DJ. Juž dvě ročiny råboti na radivu, v muzykalnomu predavanü po zakazkam radivosluhatelëǐ.

— Udava že sę lüdäm! Jesòm melomanom [gudbolübom], vsëĝda sòm měčtal stati DJ.

— Može byti, nedocěnovah profesiju-ta, ale z počętku klęh sę, htěh aby ji stati někto drugym, ale nyně råzumim, če ona je pravdivym specijalistom i glědam na ju s odkrytymi okami.

— Dobro je, že ty to si zråzumila, žbo roditelï vsëĝda sę kolěbut v sposobnostäh dětëǐ si.


\subsection{Dvôgovor 5}


— Did you see the ad in today’s paper? Russian company LLC «Trust» is looking for a sales Manager. Their requirements to the candidate: higher education, ability to work both independently and in a team. Solid experience in management and impeccable appearance will be undoubted advantages. Why are you wrinkling your forehead? What are your dislikes?

— You’ve already offered me ten different ads in the last 24 hours, and it feels like you can’t stand me leaving my job.

— What a joke! I don’t care, just thinking about you. A few months ago, someone complained a lot about our boss.…

— When was that? Now everything has changed, we’ve almost become friends. In the new year’s address to the employees, he even congratulated me on the results and gave me a puppy!

— Oh, sweet! He’d rather think about raising your salary. No, seriously, as a colleague to a colleague, you can do better. With your work book, your ability to work and your experience… just imagine: a new job, new adventures, a new life!

— Why do I deserve such care? But maybe you’re right, we should think about it.

— I can’t believe my ears! If you think long you’ll miss a great position. And then the scenario would be: you will understand everything, but it’s too late, you’re whining, start remorse, and then — depression.
— Okay, okay, sure, don’t say such terrible things. After the break I’ll call, it’s noon, there’s probably already all gone for lunch.

— Proper solution. In the meantime, I’ll go talk to our boss. And we need to help him: your place will soon be free, and therefore we need a candidate for this job. I repeat, everything is for you, everything is for you to help: and so, I will be the first candidate…

\begin{center}
	\textbf{***}
\end{center}

— Vidila li jesi objavu v dnešnomu denniku? Rusiǐska firma SOO «Trest» šukaje upravitelä prodađ. Ihnï požadavky do kandidata: vysše obrazovane, srųčnostj råbotati sámostoǐno i v družinji. Golěm opyt upravenä i vzornyǐ vid bųdųt okovidnymi plüsami. Začto mòrskaš si lòb? Čto ti ne sę podoba?

— V proběgu poslědnoga dnënočija ty jesi predložil mi desęt råzličnyh objav. Ïmám čuǐstvo, že ty ne možeš da dočękaš, dy ja opustim råbotu si.

— Směšno mi je! Bezråzlično to je, samo o tobě i mysläm. Několïko měsęcov nazad někto ožalil našoga načelnika…

— Pa, koĝda to byše? Nyně vsë je změnilo, my staha praktično kakto prijatelï. V novoročnomu poslanü k slugaram on až pozdråvih mę s rezultatami i podarih mę štenka.

— Ah, kako je mïlo! Pa po-lěpo haǐ pomysli o povyšeně tvojoji poplaty. Ne, seriozno, kako kolega do kolegy — jesi sposobna na po-vëlïke. S trudovoǐ knigoǐ si, s tvojoǐ trųdosposobnostïü i opytom ti… Samo predstavi: nova råbota, nova prigoda, nov život!

— Da načto sòm ziskala taku pěku? Ale, može byti jesi prav, trěba pomyslïti dobro.

— Ne věram na sluh-òt svôǐ! Bųdeš dôlgo dumati, hte propustiš prijimnu dòlžnostj. A dalěǐ bųde tako: vsë hte zråzumiš, ale bųde pozdno, bųdeš kanükati, hte pojavi kajane, a poslě — depresija.

— Dobro, dobro, presvědil si, ne govori vęčë takyh užastëǐ! Poslě prervy hte zatelëfonuju [zaŝvonim], nyně je pôldenj, tamo vsï obědajut, věrojętno.

— Pravidlóvo! A ja dopoĝdy hte idam da govorim s našym načelnikom. Trěba i jemu dopomognuti: město ti skoro hte bųde sŭobodne, a znači trěbame kandidata na vakansiju-ta. Povtoram, vsë dlä vas, vsë aby pomognuti vam vsï: ǐ tako, bųdu pòrvym kandidatom…


\subsection{Dvôgovor 6}

— The country has a crisis, inflation, but the number of billionaires has doubled, as well as horror-and salaries and pensions are ridiculous. And you decided to marry…

— Times change, but there’s still no perfect moment. When you and your mother got married 30 years ago, the climate in the country was even more alarming, the economy was shaken, corruption was terrible. And I got a good job, I have a reliable job, and a lot of good amassed: and I have a house, and a wheelbarrow, and a dacha. Let’s live happily ever after!

— You’re right. And will you go only to the registry office or to the Church too?

— We are sure to go to Church! Not what you were expecting? Despite the fact that she is Orthodox and I am Catholic, we have already agreed on everything. 

Last Sunday we went to the morning service, talked to the priest, received a blessing. We were allowed to get married in the temple, all as it should be.

— Well and good, it’s already time for you to marry, so to speak, to start a new life. And where will the festive feast be organized?

— That’s sick. I want to go to some awesome restaurant, service and live music, and she wants the holiday to be full of old farm, in the village, with traditional music, accordion, balalaika, etc, as well as environmentally friendly products, a real Russian feast, she says. I mean, honestly, it’s kind of freaking me out, but I can’t lose face in front of my future wife. On Saturday we go to watch some old farm with future mother-in-law and father-in-law, whether they were born there, or grew up, but in General it is very dear to them. It is true there are repairs to do, Yes, we still have time. So in the coming months I will work hard at work, and in the evenings on the farm, so that everything is ready for the wedding, as soon as possible. If only forces sufficed, because there is so much work, and I am one…

— Well, I see what you’re getting at.…

— Father, what is it worth, you’re a foreman at the biggest construction site in the city!

— Okay, got it, I’ll send you their builders, one you clearly not cope!

\begin{center}
	\textbf{***}
\end{center}

— Ïmáme krizu v dòržava-ta, ïmáme inflaciju, pa samo broǐ milïardarov je sę povysil, a tako hudo — i råbotna plata, i penza sų směšno malě. A vy nahtěla sų da zaženate.

— Doba sę měni, ale idejalnoga momenta ne hte naměrima. Ĝda va s mamoǐ sę ženiha 30 ročin nazad, položenije v dòržavji byše po-nepokôǐnym, ékonomika byše råzložena, korupcija byše strašna. A ja sę ziskah dobro, råbotu mám tvòrdu, ta ǐ vękostj blåg nažil sòm: dom ïmám, vozidlu, vilu [hatku]. Bųdeme žiti zadovôlno!

— Da, jesi pravym. Vy samo do SRCS [Služba registracijy civïlnyh sprav] hte idate ili takođe do còrkvy?

— Obovęzno hte idame do còrkvy! Čto, ne čakal jesi? Navzdor če ona je pravoslávnoju, a jesòm kaŧolikom, my vsë sma dogovorila. V mïnulu nedělü hodihma do jutróvoǐ služby, govorihma s svęštenikom [duhóvnikom], dostiĝala sma blågoslovnostj. Dostala sma zvolene za da sę věnčati [korônovati] v hramu kakto trěba.

— Ta ǐ dobro, vremę juž je da sę ženiš, kako kažut, započïnati nov život. A de bųde svatëbne veselije?

— To je vëlïkyǐ zapyt. Htem v nějakomu dužo dobromu restoranu dabi posluga byla dobra i živa gudba, a ona hte dabi prázdnik byl na nějakomu staromu hutoru, v selištu, s tradiciǐnoju gudboǐ, bajanom, balalaǐkoǐ i td, a takođe ékologično čïstymi produktami pa da s russkym stolovanëm, kako ona kaže.To mi pravdě izvodi iz sebę, ale ne mogam da sę zastydim pred bųdučëǐ ženoǐ. V sųbotu idujema glědati nějak star hutor s bųdučïma tëstëǐ i tëstëm. Da li rodila sų tamo, da li izrostnula, ale ona je mnogo dråga dlä nih. Ïstinskě, tamo trěba dělati remont, ama ïmáme vremę. Tomu, v predny měsęcy hte bųdu råbotati, a věčorom pråcovati na hutoru dabi vsë bylo gotovym k svatjbě. Samo dostati sïl, bo tako mnogo råboty tamo je.

— Pa, viđu, dlä čoga ty govoriš…

— Otče, pa čto ti to stoji, jesi nadzornikom na naǐ-vëlïkoji búdovji vo grådu!

— Dobro, ugovoril jesi mę, hte poslam ti búdóvnikov si, sám ty věrno ne sę spraviš!


\subsection{Dvôgovor 7}


— I’m going on vacation, which is very happy to: so great it is in late February to be in Bali!

— Wow! Where did you buy your ticket, in some Agency?

— No, accidentally on the Internet I found a great offer among the burning tours, paid by credit card and that’s all.

— And we are not going anywhere this winter: there is no money at all. We just finished repairing, and he poured us a pretty penny.

— Why? You had only a kitchen the neighbors flooded to fix as I remember.

— We decided to do everything at once, and to entrust the repair to one familiar architect — Yes, more expensive, but we have not lost. He made us such a bathroom: with a chrome shower column, with hydromassage, made a unique mosaic on the wall.

— Well, then you will spend your vacation in Moscow, walking among the two-meter snowdrifts!

— Laugh… It’s a real nightmare in the streets: neither car can pass nor you. And there are rumors that in one neighboring region there is even more snow than we have. Local authorities are there in a complete panic, trying to avoid the collapse of roofs, but can’t do anything, so people themselves daily clean them from the snow, using saws and shovels.

— This region has always had a fairly low socio-economic level, so I am not surprised that it is there now such a situation.

— Okay, enough about the sad part. Tell me more about what’s on the tour.

— All inclusive: visa, all excursions and meals, and there is an additional 10\% discount, and the hotel — 4 stars! All for nothing!

— How nice! When are you coming?

— We leave on February 29.

— No way.…

— How could that? Look, it’s written here.

— This year February has only 28 days… So here’s the thing: you looked through this tiny line… this tour is for the next year!

\begin{center}
	\textbf{***}
\end{center}

— Idu v otpusk, jesòm vëlïmi radym tomu: tako dobro je v koncu sěčnä sę postati na Bali!

— Voŭ! De ty si kupil pųtovane, v nějakoji aĝencijy?

— Ne, slučaǐno najil sòm v Mïromrežji odličnu predlogu među goręčïh pųtovaniǐ, zaplatil sòm kreditnoju kartoǐ i vse.

— A my v zimu-ta nida ne idame: upòlno nemáme pěnęŝëǐ. Samo opravku zakončihu, i byše toǐ vëlïmi drågym.
— Začto? Aǐ vy trěbahu opraviti samo kuhnü, vas, ako pamętam, sôsědy zatopihu.

— My rěšihu razom vsë dělati, a opravläti dati jednomu znajomomu arhitektaru — da, po-drågo, ale my ne sme sę pohybali. On nam stvoriše taku kupalnü: s hromóvoju pròšticoǐ, s vodnym masažom, nepovtorimu ukladnicu [mozaǐku] na stěnji.

— Pa znači, če hte provëdate odpusk v Moskvji, hođačï među dvôhmetróvyh sněgohòlmov.

— Směǐ sę, směǐ sę… Napravdu, užas je na vulicah: ni sámovozom projěhati, ni pehom proǐdati. I kažut, že v sôsidnoji obvlåstji má boljš sněgu, něž u nas. Tųdešna vlåda je v poplohu [panicji], dějut za da izběgat provala strěh, ale ničto mogut zpraviti, tomu lüdi sámy kòždodenno čïstut jih od sněgu, korystačï pïly i lopaty.

— V kraju-òt vsëĝda byše dostj nïzek socijalno-ékonomičnyǐ urovenj, tomu ne sòm zadiven, če tamo je ïsto taka situacija.

— Dobro, dostj o hudom. Lěpš kaži, čto máš objimano v pųtovanü.

— Vsë je objimano: i viza, i vsï proglědky, i jědlo, dodatkóvo tamo je 10\% znïžka i 4-ŝvězdóva gostinica. I vsë to blïzkodarmóvo.

— Kako dobro je! A koĝda vy idate?

— Vylětame 29 sěčnä.

— Ne može byti…

— Kako ne može? Glědaǐ, zdě je napisano.

— V ovoji ročinji je samo 28 dnëǐ… Ta ôt čto to je: ty ne poglědil si ôt na ov mal rędek...To je pųtovane na slědnu ročinu!


\subsection{Dvôgovor 8}


— I couldn’t make up with my husband all week.

— What’s the matter with you?

— I wanted to go to the Tretyakov gallery or to the Garage with a colleague because he is interested in art and especially in painting. Oddly enough that my husband said that he would go with us, but I know that he was neither large nor small museums have never been interesting…

— In my opinion, you exaggerate: remember when we all went to Peterhof together, he liked it very much.

— Of course! He was especially happy when I did not notice one fountain and remained wet to the skin.

— Well, my dear, I remember at the beginning of the week warned you that you should take care of any walks and trips…

— I rather fear your stupid horoscopes, and even your persistence will not bring any results — I will not read them.

— Okay, okay, tell me more about your cultural program at the Museum.

— We went to the «Garage», there was an exhibition of contemporary art and the exhibition of some informal author, who has long been world famous. He impressed me very much: in his paintings one can feel the inner conflict that his predecessors expressed through their works in the 60s. And my husband is pretty loud, at all began to say that it would be a picture written in one day, that even a child could draw… To my horror, the hall was the artist himself, he heard what my husband said and here is started! I’m not one of those who easily panic, but the artist began to shout that my husband does not know what art is, that will ask the organizers of the event to kick him out of the exhibition. Well, all I had to do was throw a tantrum at my husband at home.

— Yeah, you shouldn’t have gone there with your husband.…

— No, I shouldn’t have invited my colleague. If my husband hadn’t been jealous, he would never have gone to the Museum!

\begin{center}
	\textbf{***}
\end{center}

— Cělu sedmicu na možěh popraviti odnošene s mųžom.

— Čto sę staše u vas?

— Htěh idati v Tretjakovsku galeriju či v «Ĝaraž» s koleĝom, bo on sę zajima umetstvom, osoblivo malarstvom. Stranno je, ale mųž mi kazal je, če hte ida s nami, ale ja ž znam, če ni maly, ni golěmy muzejy mu niĝda sę podobahu.

— Z mojoga poglědu, ty premnožiš: pamętaš li, koĝda my vsï zajedno hodihme v Petergof, toǐ mu sę vëlïmi byše podobal.

— Razbira sę! Osoblivo mu byše směšno, koĝda ja ne primětih jednoga vodotryska i stah upòlno mokroju.

— Pa, dråga mi, pamętam, že ješto v počętku sedmicy upredih tę, če trěbaš da sę hråniš od vsękyh prohodek i pojězdek.

— Ja boljš bám sę tvojyh glupyh goroskopov, i až upor ti ne donësi nijakyh vyslědkov — ne bųdu čitati jih.

— Dobro, dobro, råzkaži dalëǐ o vašoǐ kuljturnoǐ proĝramě v muzeju.

— Poǐdahme v «Ĝaraž», tamo byše vystavka dnešnoga umetstva i ékspozicija nějakoga neformalnoga tvorca, ktor davno je dostal mïróvu znalostj. On vëlïmi zahvatiše [ïmpresioniraše] mę: v plåtnah ĝo sę čujaš tyǐ vųtrešnyǐ konflikt, čto izražahu prez råboty si poprednicy ĝo. A mųž mi dostj glåsno započïnaše govoriti blïzko vsěh, če biše napisal taku råbotu za jeden denj, če až dětę mogalo biše narisovati ju… K užasu mi v salji-ta byše malar sám, on uslyšaše, če govori mųž mi i… čto sę započïnaše! Ne sòm lěgko panikujema, ale malar staše kričati, če mųž môǐ ne zna, čto je malarstvo i če hte poprosi råzporęditelëǐ vystavky izgnati ĝo s ji. Nu, máh samo histeriku mųžu da započïnam vdomu.

— Da, ne trěbaša idati s mųžom tųda…

— Ne, ne trěbah da zovam tųda koleĝu mi. Ako mųž ne biše žarlëval mę, on niĝda ne poǐdal do muzeja!


\subsection{Dvôgovor 9}


— Oh my God, my vacation turned out to be a quiet horror!

— Why? You were at a friend’s cottage, weren’t you?

— Yeah, but they said they have everything great, but when I arrived, I saw that everything is broken. The tap in the kitchen does not work, the flush is broken, you can not touch the sink, because the pipe is clogged. Said the plumber, but we waited for him for almost a week… They’re so tired of constantly filling your head with nonsense.

— Well, as they say, lovely curse — just gratifying.

— She whines that she needs to go to the gym, that she has cellulite and excess weight, and in the evening eats for five! And he says he wants to leave to the Canary Islands or with friends at the beer festival in Bavaria… No, really, I’m almost not crazy: all sedatives that I was with him, drank. And then there’s air conditioning broke down, breathe — though be hung up.

— You weren’t forced to go to the store?

— No, he went there himself, but every time he forgot the list of products he had to buy. That was wound back on the little things five times a day, bought all sorts of things, and his wife scolded at home, not something bought…

— Well, what have you been doing for so long? You could have gone!

— You know, I am a polite and considerate person: I could not do this to these hospitable and infinitely generous people. Here and suffered their miraculous antics for nearly two months. And when I remember that at home is the wife, smoking is prohibited and the store should be gone to, so immediately the life in the country appears not so bad as it seems…

\begin{center}
	\textbf{***}
\end{center}

— Mama mi, odpusk staše mi kakto tïh užas!

— Začto? Byša ž u prijatelëǐ na hatkji, ně li?

— Da, ale oni kazahu, če tamo vsë je super, ale koĝda priǐdah, zabačil, če vsë je polomano. Vodovod na kuhnji ne robi, vodozlïv je zloman, umyvadlo ne može da dotykaš, bo truba je sę zatkala. Kazahu, če byhu zovali vodovodara, ale čękahu je blïzko sedmicy… Tako dokučili su mi: vsëčasno bzduřut.

— Kako sę kaže, mïlency sę psovat samo v slådostj.

— Ona nudi, že trěba idati v sportsalu, če má celülit i prebytečnu tęžinu, a večorom jěda zaměsto pęteryh! A on kaže, če hte v odpusk do Kanar či s prijatelämi na prázdnik pjiva v Bavariju… Ne, pravda, blïzko ošalih s nima: izpih vesj sedativ, če máh s sobom. A tamo ješto klïmatizatar sę sgųbiše, dyšati byše nemožno — hotj sę věšaǐ.

— Ne li tę prinuđahu hoditi do maĝazina [sklada]?

— Ne, on sám hođaše tųda, ale vsęk kråt zabezpamętaše spisek produktov, ktory trěbal da kúpi. Ôt i hođaše dotųde malečno pętïkråtno prez denj, kúpaše něčto bzduru, a žena ĝo vdomu kariše, če byše kúpil čto ne trěba…

— Pa začto tako dôlgo stojaše u nih? Mogaša ujěhati!

— Znaš, jesòm učtivym i taktičnym čelověkom: ne možeh az tako napraviti s tymi gostoprijimnymi i bezgrånično štedrymi lüdima. Ôt i tërpah onyšny podivny vybrycy okolo dvôh měsęcov. Ta ǐ kako vozpamętaš, če vdomu je žena si, dymiti je zabråneno i trěbaš da hodiš do sklada, tak razom život na hatkji postoji ne tako hudym…


