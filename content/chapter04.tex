\chapter{Phrase}

Phrase, or collocation, is a semantic and grammar union of some words. Usually we speak about 2-5 words in a collocation. The phrase always has the main word and dependent words. They are connected with each other by some kinds of links. They are: Coherence, Management and Adjunction.

The word with a coherence link is of the same grammar form as the main word. That means, if we change the main word (i.e. the case), we will have to change also the form of the word with a coherence link.

The word with a management link has a determined grammar form. So, if we have the main word and the word we want to connect to it, we should once change grammar form of a dependent word with the rule-case of the management link and then, when we change form of the main word, we will keep the form of the dependent one.

Adjunction link means that we simply add a dependent word to our collocation. We usually speak about this type of linking when we have immutable words.

Here you should remember what we have spoken about cases in the previous chapter. Using the proper case in a collocation is a great mean of constructing a euphonious and understandable sentence. However, we cannot define all the cases of such links, so you can feel-in the language to achieve this art. If you are a Slav, simply try to use those cases that you use in your native language.

\section{Emphasis}

Emphasis is a logical accent in a phrase. Often we allocate the accented words with the voice while speaking, but sometimes we need to show the accent in writing or strongify it. So here you can take a look at tools of showing the accent in your phrase.

The most suitable way of it is to use particles “že”, “ž”, “pretož”, “vědj”. The accent depends on place where the particle stands in your sentence. 

\begin{itemize}
	\item In the beginning - this makes the accent on the first word in a phrase (sentence)
	\item After the first word - this makes the accent on the whole phrase (sentence)
	\item After the second … last word - this makes the accent on the second … last word in your phrase (sentence)
\end{itemize}

Examples:

\section{Phraseology}

Ja mám něčto - Něčto je u menę

There are two constructions that we can use to describe an English variant “I have something”.
As you see, the exact translation of the English phrase is the first variant. Here we use the verb “máti” (to have). However, there is the second variant in Novoslovnica too. In this variant we can watch the interchange between the actor and the object of an action by syntax roles. In “Ja mám něčto” we see that the actor is the subject of this short sentence, and the action object is has a syntax role as a direct object. In “Něčto je u menę” we see, that the actor has a role of indirect object and the object has a subject syntax role.    

There are some connections between the second variant of the phrase and the English “There is/are” construction. In English we use this when the object is situated in some place, so as in the second variant (if we change the actor (indirect object) with the adverbial (or indirect object of location). In Novoslovnica we still can use this construction with the soulful objects such as personal pronouns or animated nouns. 

Jesòm někto - Jesòm někym

You can notice that there are different construction of translation a phrase “to be somebody” in Novoslovnica. These two variants, with normal form and Instrumentative, are equal. 
The accent of using these two forms lays in the linguistic formalism that you use in your speech. Using a normal form instance of cases (Remember, that you cannot use Nominative itself in the non-Subject roles in your sentence) is more analytic while using Instrumentative is more synthetic.
Nevertheless, they both can be used in speech.
