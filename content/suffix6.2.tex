\section{Suffix}

Suffix - is a \gls{pow}, that is placed after the word stem. It is a postfix that is not an ending of the word. That means that added a suffix to the word makes a new one with its own semantic value. Here is the list of main suffixes in Novoslovnica with their semantic value and examples.

\textbf{Ak}

This suffix has a semantic value of “person of some quality”.

\underline{Examples:}

\textit{Bědnäk - Běden (adj)} - Poor man - Poor (adj)

\textit{Glupak - Glup (adj)} - Fool - fool (adj)

\textbf{An}

This suffix has a semantic value of “having some property”.

\underline{Examples:}

Bědnäk - Běden (adj) - Poor man - Poor (adj)

Glupak - Glup (adj) - Fool - fool (adj)

\textbf{Ar}

This suffix has a semantic value of “person of some profession”. The suffix has a similar meaning with English "-er".

\underline{Examples:}

\textit{Rybar} - Fisher

\textit{Ovčar} - Shepherd

\textit{Ač}

This suffix has a value of "person that likes to perform an action"

\underline{Examples:}

\textit{Grač - Grati} - Player - To play

\textit{Běgač - Běgati} - Runner - To run

\textbf{B}

The suffix of a noun describing a process of continuously performing an action

\underline{Examples:}

\textit{Borba - Bořati} - Struggle - To struggle

\textit{Sųdba - Sųđati} - Fate - To judge

\textbf{Dl}

This suffix describes an object (tool) for performing an action.

\underline{Examples:}

\textit{Mydlo} - Myti

\textit{Vozidlo} - Voziti

\textbf{Ot}

The suffix represents a state of some attribute.

\underline{Examples:}

\textit{Lěpota} - Beauty

\textit{Hlådnota} - Cold

\textbf{Ostj}

This suffix is an equivalent of English "ness" suffix. In make a noun from some attribute value expressed by an adjective.

\underline{Examples:}

\textit{Vëlïkostj} - Greatness

\textit{Samostj} - Loneliness

\textbf{Ec}

The suffix is for a person with some attribute.

\underline{Examples:}

\textit{Hlåpec} - Boy

\textit{Běglec} - Escaper

\textbf{Izn}

The suffix represent a state of doing some action.

\underline{Examples:}

\textit{Žiznj - Žiti} - Life - To live

\textit{Bělizna - Běliti} - White - To white

Ik

In

Ih

Ic

\textbf{Ïc}

This suffix represent a female animal.

\underline{Examples:}

\textit{Lisïca} - Fox (female)

\textit{Kotïca} - Cat (female)

\textbf{K}

This suffix has two meanings. The first one describes a process of the verb. 

\underline{Examples:}

\textit{Myǐka - Myti} - Washing - To wash

\textit{Budka - Buđati} - Waking - To wake

The second one is deminutive.

\underline{Examples:}

\textit{Kot - Kotek} - (small) cat

\textit{Rot - Rotek} - (small) mouth

\textbf{N}

The attributive suffix that describes an attribute of being something defined by a noun.

\underline{Examples:}

\textit{Běda - Běden} - Poor - Poor

\textbf{Nik}

The suffix is for a person characterized by a NP.

\underline{Examples:}

Bezdělnik - Bez děla

Učobnik - Učoba

\textbf{Sl}

This suffix describes an object (tool) for performing an action.

\textbf{Examples:}

\textit{Vëdslo - Vëdati} - Paddle (the verb "to lead")

\textit{Ĝadslo - Ĝadati} - Password (the verb "to guess)

\textbf{Stv}

It is a suffix for collective noun of a subject.

\textbf{Examples:}

\textit{Lïstva - Lïst} - Leafage - Leaf

\textit{Čelověkstvo - Čelověk} - Mankind - Man

\textbf{Telj}

This suffix represents a person perfoming the action of the main word.

\textbf{Examples:}

\textit{Daritelj - Dariti} - Presenter - To present

\textit{Učitelj - Učiti} - Teacher - To teach

This is not the whole list. Though, you can understand a principle of common semantic shift while adding the same suffix to the stem.
