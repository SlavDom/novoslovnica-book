\section{Runaway vowels}

Looking back in the Slavic language history we can find out that there were roots with two strange for an ordinary person vowels - “Ò” and “J”. First one was named “Yer” and denotes a hard mid central vowel (Shwa). The second one was named “Yerj” (with soft R) and denotes a soft mid central vowel. Now the second one is lost and we use only shwa sound in the letter “Yer”. However, the words are still and we need to pronounce them in some way. Novoslovnica uses the soft “E” sound to represent roots with old soft shwa sound.

Main feature of these sounds was to fall out from the root, when a vowel appears afterwards. That’s why there are many words with two consonants consecutively - there is an imaginary shwa sound between them that has been fallen out from the root.

Nevertheless, despite falling out of “Yer”, soft “E” in this places does not fall out. So, in the previous paragraph you could see that there are two alternations O//- and O//Ë, that are handled in the similar positions. So the answer on the question, why in the first case there is no sound and in the second there is a soft E is the fact, that words satisfying the first case comprise old hard shwa sound and remain comprise old soft shwa sound, that has transformed into soft E.

I should mention also the fact, that nowadays the letter Ò exists only in roots of the words. In suffixes the letter E is used for this sound and in the prefixes the letter O is used. Look at the examples:
pod
ek
pòk  

You should remember that speaking the words with these letters we should pronounce them just as they are written - Ò as shwa, E as E, O as O. You should not reduce all the sounds to shwa. 
