\section{Preposition managing}

Many phrases have not one determined preposition to be managed. It is a part of phraseology to detect which one suits better in which case. Here we list some examples that show those differences.

\textbf{Jeden z - Jeden od}

Both these constructions are used to describe an element of some sort. However, two different prepositions make different semantic sense here.

\begin{itemize}
	\item Preposition "z" describes that the following word is a category or an infinite set
	\item Preposition "od" describes that the following word is a finite set.
\end{itemize}

Look at the examples:

\textit{Jeden od prijatelëǐ mi kazaše mi užasnu novinu} - One of (I have just several friends) told me a terrifying news.

\textit{Jeden z vozidlov sę zlomaše včera na stanicji} - One of autobuses (We do not know how many there are autobuses) cracked yesterday.

\textbf{Blïzko do - Blïzko k}

Both these constructions describe something that is close to the following word. However, there are also semantic differences between each other.

\begin{itemize}
	\item Preposition "do" is used to show the approaching destination (direction)
	\item Preposition "k" is used to show the closity of an object to another one (placement)
\end{itemize}

Examples:

\textit{My jěhajeme juž dvě godiny, ale sme juž blïzko do doma.} - We have been riding for two hours, though we are close to home. (Our house is close to our due to our moving)

\textit{Naš dom stoji blizko kò grådu.} - Our house situates near the city. (The placement of the house is close to city)

