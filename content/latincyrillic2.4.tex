\section{Latin and Cyrillic}

You can see in table 1.8 that Novoslovnica utilizes two alphabets- the Latin one and the Cyrillic one. They both are practically equal, but is there is a preferred alphabet for the language? The answer is “yes”, Cyrillic is preferable.

The reasons of choosing such a script goes back in history. There were two scripts in the beginning of Slavic writing: the Glagolitic and Cyrillic scripts. They were created to cover all the sounds that existed in that era of Slavic languages. Glagolitic script practically has no borrowed letters from other writing systems- all letters are unique.

By this case in Cyrillic and Glagolitic scripts we can find the bijective mapping between sounds (phonemes) and letters. Latin script does not provide such orthography in any Slavic language which uses the Latin script. For example:
% “ch” for [x], while “c” is for [t͡s] and h is for [ɦ]
% “sz” for [ʂ], while “s” is for [s] and “z” is for [z]

Novoslovnica provides an artificial Latin script system, where the bijective mapping has almost been achieved. The Latin alphabet can seem strange or uncomfortable to native Slavs (though it can be used rather conveniently by non-Slavs). That’s why the Glagolitic or Cyrillic script should be used primarily.

Why hasn’t the Glagolitic script been mentioned yet? The same reason that the Latin script should not be used primarily: to prevent misunderstanding. Nowadays only one in a hundred Slavs can understand the Glagolitic script because all of its letters are original. That’s why this language, which has the goal of being used on the international level, cannot use Glagolitic script as its primary script.

The only script, that satisfies all the requirements to be the primary script of this Slavic constructed language, is the Cyrillic alphabet. In this book you will find many examples in different paragraphs. First you will see a primary (Cyrillic) variant of the example in normal font and then a Latin one in grey italic font. This will help you to learn primary script of Novoslovnica quickly. Nevertheless, if we speak about exact letters or letter combinations, I will write them only in Latin for not to mess the text of the book.

Now you know the sounds and the letters which are used in Novoslovnica and you are ready to go deeper!
