\section{Numeral}

\begin{table}[h]
	\caption{Adverb characteristics}
	\begin{tabular}{lllll}
		\textbf{Title}              & \textbf{Value}      \\
		Semantic value              & Attribute           \\
		Category                    & Independent         \\
		Subcategory                 & Nominal             \\
		Alteration                  & Declension          \\
		Alteration parameters       &               \\
		Differentiation parameters  & Type
	\end{tabular}
\end{table}

Numerals can be attributive (with a nount) or pronominal (without a noun).
	
% 	Им. п.	Р. п. 	К. п.	В. п.	Д. п.	Тв. п.	П. п.	М. п.
% М. р.	Једен	Једнога	Једногу	Једного	Једному	Једным	Једном	Једному
% Ж. р.	Једна	Једної	Једної	Једну	Једной	Једноју	Једной	Једної
% Ср. р.	Једно	Једнога	Једногу	Једного	Једному	Једным	Једном	Једному


%	Им. п.	Р. п. 	К. п.	В. п.	Д. п.	Тв. п.	П. п.	М. п.
% М. р.	Два	Двôх	Двôх	Два	Двôм	Двомя	Двôх	Двому
% Ж. р.	Двѣ	Двôх	Двôх	Двѣ	Двôм	Двомя	Двôх	Двому

% 	Им. п.	Р. п. 	К. п.	В. п.	Д. п.	Тв. п.	П. п.	М. п.
% М. р.	Три	Трёх	Трёх	Три	Трём	Трємя	Трёх	Трєму
% Ж. р.	Трѣ	Трёх	Трёх	Трѣ	Трём	Трємя	Трёх	Трєму

% Дальше наблюдается система – числительные от 5 до 20 имеют следующую парадигму склонения (о которой будет изложено чуть ниже), далее 21,31,41 и т.п. имеют первую парадигму, 22,32,42 и т.п. – вторую, 23,24,33,34 и т.п. – третью, и 25-30, 35-40, 45-50 и т.п. – четвёртую.
% Приводим четвёртую парадигму склонения на примере числительного 5 (Пѣт)


% 	Им. п.	Р. п. 	К. п.	В. п.	Д. п.	Тв. п.	П. п.	М. п.
% М. р.	Пѧт	Пѧті	Пѧті	Пѧть	Пѧті	Пѧтію	Пѧті	Пѧтї
% Ж. р.	Пѧць	Пѧті	Пѧті	Пѧць	Пѧті	Пѧтію	Пѧті	Пѧтї

\subsection{Cardinal numbers}

\begin{itemize}
	\item One (1) - Jeden, Jedna, Jedno
	\item Two (2) - Dva, Dvě
	\item Three (3) - Tri, Trě
	\item Four (4) - Četyri, Četyrě
	\item Five (5) - Pęt
	\item Six (6) - Šest
	\item Seven (7) - Sedem
	\item Eight (8) - Osem
	\item Nine (9) - Devęt
	\item Ten (10) - Desęt
	\item Eleven (11) - Jedennadesęt
	\item Twelve (12) - Dvanadesęt
	\item Thirteen (13) - Trinadesęt
	\item Fourteen (14) - Četyrinadesęt
	\item Fifteen (15) - Pętnadesęt
	\item Sixteen (16) - Šestnadesęt
	\item Seventeen (17) - Sedemnadesęt
	\item Eighteen (18) - Osemnadesęt
	\item Nineteen (19) - Devętnadesęt
	\item Twenty (20) - Dvadesęt
	\item Twenty-one (21) - Dvadesęt jeden
	\item Thirty (30) - Tridesęt
	\item Fourty (40) - Četyridesęt
	\item Fifty (50) - Pętdesęt
	\item Sixty (60) - Šestdesęt
	\item Seventy (70) - Sedemdesęt
	\item Eighty (80) Osemdesęt
	\item Ninety (90) - Devętdesęt
	\item Hundred (100) - Sto
	\item One hundred one (101) - Sto jeden
	\item One hundred ten (110) - Sto desęt
	\item Two hundred (200) - Dvěstě (Dvasta)
	\item Three hundred (300) - Trista
	\item Four hundred (400) - Četyrista
	\item Five hundred (500) - Pętsòt
	\item Six hundred (600) - Šestsòt
	\item Seven hundred (700) - Sedemsòt
	\item Eight hundred (800) - Osemsòt
	\item Nine hundred (900) - Devętsòt
	\item Thousand (1000) - Tysęča
	\item One thousand one (1001) - Tysęča jeden
	\item One thousand ten (1010) - Tysęča desęt
	\item One thousand one hundred (1100) - Tysęča sto
	\item Two thousand (2000)- Dvě tysęčy
	\item Five thousand (5000) - Pęt tysęč
	\item Million (1000000) - Milïon
	\item One million one thousand one (1001001) - Milïon sto jeden
	\item Billion (10\^9) - Bilïon (Milïard)
	\item Trillion (10\^12) - Trilïon
	\item Quadrillion (10\^15) - Kŭadrilïon
	\item Quintillion (10\^18) - Kŭintilïon
	\item Sextillion (10\^21) - Sekstilïon
	\item Septillion (10\^24) - Septilïon
	\item Octillion (10\^27) - Oktilïon
	\item Nonillion (10\^30) - Nontilïon
	\item Decillion (10\^33) - Decilïon
	\item Googol (10\^100) - Ĝuĝòl
	\item Googolplex (10\^(10\^100)) - Ĝuĝlopleks
\end{itemize}

Declension



\subsection{Ordinal numerals}

\begin{itemize}
	\item First (1) - Pòrvyǐ
	\item Second (2) - Vtoryǐ
	\item Third (3) - Tretjyǐ
	\item Fourth (4) - Četvòrtyǐ
	\item Fifth (5) - Pętyǐ
	\item Sixth (6) - Šestyǐ
	\item Seventh (7) - Sedmyǐ
	\item Eighth (8) - Osmyǐ
	\item Ninth (9) - Devętyǐ
	\item Tenth (10) - Desętyǐ
	\item Eleventh (11) - Jedennadesętyǐ
	\item Twelfth (12) - Dvanadesętyǐ
	\item Thirteenth (13) - Trinadesętyǐ
	\item Fourteenth (14) - Četyrinadesętyǐ
	\item Fifteenth (15) - Pętnadesętyǐ
	\item Sixteenth (16) - Šestnadesętyǐ
	\item Seventeenth (17) - Sedemnadesętyǐ
	\item Eighteenth (18) - Osemnadesętyǐ
	\item Nineteenth (19) - Devętnadesętyǐ
	\item Twentieth (20) - Dvadesętyǐ
	\item Twenty first (21) - Dvadesęt pòrvyǐ
	\item Thirtieth (30) - Tridesętyǐ
	\item Fourtieth (40) - Četyridesętyǐ
	\item Fiftieth (50) - Pętdesętyǐ
	\item Sixtieth (60) - Šestdesętyǐ
	\item Seventieth (70) - Sedemdesętyǐ
	\item Eightieth (80) Osemdesętyǐ
	\item Ninetieth (90) - Devętdesętyǐ
	\item Hundredth (100) - Sòtyǐ
	\item One hundred and first (101) - Sto pòrvyǐ
	\item One hundred and tenth (110) - Sto desętyǐ
	\item Two hundredth (200) - Dvôhsòtyǐ
	\item Three hundredth (300) - Tröhsòtyǐ
	\item Four hundredth (400) - Četyröhsòtyǐ
	\item Five hundredth (500) - Pętsòtyǐ
	\item Six hundredth (600) - Šestsòtyǐ
	\item Seven hundredth (700) - Sedemsòtyǐ
	\item Eight hundredth (800) - Osemsòtyǐ
	\item Nine hundredth (900) - Devętsòtyǐ
	\item Thousandth (1000) - Tysęčnyǐ
	\item One thousand and first (1001) - Tysęča pòrvyǐ
	\item One thousand and tenth (1010) - Tysęča desętyǐ
	\item One thousand and one hundredth (1100) - Tysęča sòtyǐ
	\item Two thousandth (2000)- Dvě tysęčnyǐ
	\item Five thousandth (5000) - Pęt tysęčnyǐ
	\item Millionth (1000000) - Milïonnyǐ
	\item One million one thousand and first (1001001) - Milïon sto pòrvyǐ
	\item Billionth (10\^9) - Bilïonnyǐ (Milïardnyǐ)
	\item Trillionth (10\^12) - Trilïonnyǐ
	\item Quadrillionth (10\^15) - Kŭadrilïonnyǐ
	\item Quintillionth (10\^18) - Kŭintilïonnyǐ
	\item Sextillionth (10\^21) - Sekstilïonnyǐ
	\item Septillionth (10\^24) - Septilïonnyǐ
	\item Octillionth (10\^27) - Oktilïonnyǐ
	\item Nonillionth (10\^30) - Nontilïonnyǐ
	\item Decillionth (10\^33) - Decilïonnyǐ
	\item Googolth (10\^100) - Ĝuĝòlnyǐ
	\item Googolplexth (10\^(10\^100)) - Ĝuĝlopleksnyǐ
\end{itemize}

\subsection{Pronominal numerals}

% СОБИРАТЕЛЬНОЕ ЧИСЛИТЕЛЬНОЕ
% Собирательное числительное обозначает совокупность объектов названного количество. Как и существительные данного типа, они склоняются только в единственном числе. Характерной особенностью является наличие суффикса «–ЕР–» или «–ОЈ–». 


% Числительное	Им. п.	Р. п. 	К. п.	В. п.	Д. п.	Тв. п.	П. п.	М. п.
% Два	Двојє	Двојіх		=Р.п.	Двојім	Двојіми	Двојіх	
% Три	Тројє	Тројіх		=Р.п.	Тројім	Тројіми	Тројіх	
% Четыри	Четверо	Четверых		=Р.п.	Четверым	Четверыми	Четверых	
% Пѧт	Пѧтеро	Пѧтерых		=Р.п.	Пѧтерым	Пѧтерыми	Пѧтерых	
% Шест	Шестеро	Шестерых		=Р.п.	Шестерым	Шестерыми	Шестерых	
% Седем	Седмеро	Седмерых		=Р.п.	Седмерым	Седмерыми	Седмерых	
% Осем	Осмеро	Осмерых		=Р.п.	Осмерым	Осмерыми	Осмерых	
% Девѧт	Девѧтеро	Девѧтерых		=Р.п.	Девѧтерым	Девѧтерыми	Девѧтерых	
% Десѧт	Десѧтеро	Десѧтерых		=Р.п.	Десѧтерым	Десѧтерыми	Десѧтерых	
% Једеннадесѧт	-десѧтеро	=||=	=||=	=||=	=||=	=||=	=||=	=||=
% Двадесѧт	-десѧтеро	=||=	=||=	=||=	=||=	=||=	=||=	=||=
% ИТД								

