\section{Numeral}

\begin{table}[h]
	\caption{Numeral characteristics}
	\begin{tabular}{ll}
		\textbf{Title}              & \textbf{Value}      \\
		Semantic value              & Attribute           \\
		Category                    & Independent         \\
		Subcategory                 & Nominal             \\
		Alteration                  & Declension          \\
		Alteration parameters       & Case, Number        \\
		Differentiation parameters  & Type
	\end{tabular}
\end{table}

Numerals\index{numeral} can be attributive (with a noun) or pronominal (without a noun).
	
\subsection{Cardinal numbers}

\begin{itemize}
	\item One (1) - Jeden, Jedna, Jedno
	\item Two (2) - Dva, Dvě
	\item Three (3) - Tri, Trě
	\item Four (4) - Četyri, Četyrě
	\item Five (5) - Pęt
	\item Six (6) - Šest
	\item Seven (7) - Sedem
	\item Eight (8) - Osem
	\item Nine (9) - Devęt
	\item Ten (10) - Desęt
	\item Eleven (11) - Jedennadesęt
	\item Twelve (12) - Dvanadesęt
	\item Thirteen (13) - Trinadesęt
	\item Fourteen (14) - Četyrinadesęt
	\item Fifteen (15) - Pętnadesęt
	\item Sixteen (16) - Šestnadesęt
	\item Seventeen (17) - Sedemnadesęt
	\item Eighteen (18) - Osemnadesęt
	\item Nineteen (19) - Devętnadesęt
	\item Twenty (20) - Dvadesęt
	\item Twenty-one (21) - Dvadesęt jeden
	\item Thirty (30) - Tridesęt
	\item Fourty (40) - Četyridesęt
	\item Fifty (50) - Pętdesęt
	\item Sixty (60) - Šestdesęt
	\item Seventy (70) - Sedemdesęt
	\item Eighty (80) Osemdesęt
	\item Ninety (90) - Devętdesęt
	\item Hundred (100) - Sto
	\item One hundred one (101) - Sto jeden
	\item One hundred ten (110) - Sto desęt
	\item Two hundred (200) - Dvěstě (Dvasta)
	\item Three hundred (300) - Trista
	\item Four hundred (400) - Četyrista
	\item Five hundred (500) - Pętsòt
	\item Six hundred (600) - Šestsòt
	\item Seven hundred (700) - Sedemsòt
	\item Eight hundred (800) - Osemsòt
	\item Nine hundred (900) - Devętsòt
	\item Thousand (1000) - Tysęča
	\item One thousand one (1001) - Tysęča jeden
	\item One thousand ten (1010) - Tysęča desęt
	\item One thousand one hundred (1100) - Tysęča sto
	\item Two thousand (2000)- Dvě tysęčy
	\item Five thousand (5000) - Pęt tysęč
	\item Million (1000000) - Milïon
	\item One million one thousand one (1001001) - Milïon tysęča jeden
	\item Billion ($10^9$) - Bilïon (Milïard)
	\item Trillion ($10^{12}$) - Trilïon
	\item Quadrillion ($10^{15}$) - Kŭadrilïon
	\item Quintillion ($10^{18}$) - Kŭintilïon
	\item Sextillion ($10^{21}$) - Sekstilïon
	\item Septillion ($10^{24}$) - Septilïon
	\item Octillion ($10^{27}$) - Oktilïon
	\item Nonillion ($10^{30}$) - Nontilïon
	\item Decillion ($10^{33}$) - Decilïon
	\item Googol ($10^{100}$) - Ĝuĝòl
	\item Googolplex ($10^{10^{100}}$) - Ĝuĝlopleks
\end{itemize}

\begin{table}[!htb]
	\caption{Declension of JEDEN} 
	\begin{tabular}{llll}
		& Masculine & Feminine & Neutral \\
		Nominative / Vocative & Jeden & Jedna & Jedno \\
		Genitive & Jednoga & Jednoji & Jednoga \\
		Partitive & Jednogu & Jednoji & Jednogu \\
		Accusative & Jednogo & Jednu & Jednogo \\
		Dative & Jednomu & Jednoǐ & Jednomu \\
		Instrumental & Jednym & Jednoju & Jednym \\
		Prepositional & Jednom & Jednoǐ & Jednom \\
		Locative & Jednomu & Jednoji & Jednomu \\
	\end{tabular}
\end{table}

\textbf{Declension}

The system is - Three and Four have a common declension paradigm. Numerals 5 - 20 are common with the declension of Five. Further, 21, 31, 41 etc. have the paradigm of One, 22, 32, 42 etc. - the paradigm of Two, 33, 34, 43, 44 etc. - the paradigm of Three, 25 - 30, 35 - 40 etc. - the paradigm of Five.

\begin{table}[!htb]
	\caption{Declension of DVA}
	\begin{tabular}{lll}
		& Masculine | Neutral & Feminine \\
		Nominative / Vocative & Dva & Dvě \\
		Genitive & Dvôh & Dvôh \\
		Partitive & Dvôh & Dvôh \\
		Accusative & Dva & Dvě \\
		Dative & Dvôm & Dvôm \\
		Instrumental & Dvoma & Dvoma \\
		Prepositional & Dvôh & Dvôh \\
		Locative & Dvomu & Dvomu \\
	\end{tabular}
\end{table}

\begin{table}[!htb]
	\caption{Declension of TRI}
	\begin{tabular}{lll}
		& Masculine & Feminine \\
		Nominative / Vocative & Tri & Trě \\
		Genitive & Tröh & Tröh \\
		Partitive & Tröh & Tröh \\
		Accusative & Tri & Trě \\
		Dative & Tröm & Tröm \\
		Instrumental & Trëma & Trëma \\
		Prepositional & Tröh & Tröh \\
		Locative & Trëmu & Trëmu \\
	\end{tabular}
\end{table}

\begin{table}[!htb]
	\caption{Declension of PĘT}
	\begin{tabular}{lll}
		& Masculine & Feminine \\
		Nominative / Vocative & Pęt & Pętj \\
		Genitive & Pętï & Pętï \\
		Partitive & Pętï & Pętï \\
		Accusative & Pęt & Pętj \\
		Dative & Pętï & Pętï \\
		Instrumental & Pętïü & Pętïü \\
		Prepositional & Pętï & Pętï \\
		Locative & Pętji & Pętji \\
	\end{tabular}
\end{table}



\subsection{Ordinal numerals}

\begin{itemize}
	\item First (1) - Pòrvyǐ
	\item Second (2) - Vtoryǐ
	\item Third (3) - Tretjyǐ
	\item Fourth (4) - Četvòrtyǐ
	\item Fifth (5) - Pętyǐ
	\item Sixth (6) - Šestyǐ
	\item Seventh (7) - Sedmyǐ
	\item Eighth (8) - Osmyǐ
	\item Ninth (9) - Devętyǐ
	\item Tenth (10) - Desętyǐ
	\item Eleventh (11) - Jedennadesętyǐ
	\item Twelfth (12) - Dvanadesętyǐ
	\item Thirteenth (13) - Trinadesętyǐ
	\item Fourteenth (14) - Četyrinadesętyǐ
	\item Fifteenth (15) - Pętnadesętyǐ
	\item Sixteenth (16) - Šestnadesętyǐ
	\item Seventeenth (17) - Sedemnadesętyǐ
	\item Eighteenth (18) - Osemnadesętyǐ
	\item Nineteenth (19) - Devętnadesętyǐ
	\item Twentieth (20) - Dvadesętyǐ
	\item Twenty first (21) - Dvadesęt pòrvyǐ
	\item Thirtieth (30) - Tridesętyǐ
	\item Fourtieth (40) - Četyridesętyǐ
	\item Fiftieth (50) - Pętdesętyǐ
	\item Sixtieth (60) - Šestdesętyǐ
	\item Seventieth (70) - Sedemdesętyǐ
	\item Eightieth (80) Osemdesętyǐ
	\item Ninetieth (90) - Devętdesętyǐ
	\item Hundredth (100) - Sòtyǐ
	\item One hundred and first (101) - Sto pòrvyǐ
	\item One hundred and tenth (110) - Sto desętyǐ
	\item Two hundredth (200) - Dvôhsòtyǐ
	\item Three hundredth (300) - Tröhsòtyǐ
	\item Four hundredth (400) - Četyröhsòtyǐ
	\item Five hundredth (500) - Pętsòtyǐ
	\item Six hundredth (600) - Šestsòtyǐ
	\item Seven hundredth (700) - Sedemsòtyǐ
	\item Eight hundredth (800) - Osemsòtyǐ
	\item Nine hundredth (900) - Devętsòtyǐ
	\item Thousandth (1000) - Tysęčnyǐ
	\item One thousand and first (1001) - Tysęča pòrvyǐ
	\item One thousand and tenth (1010) - Tysęča desętyǐ
	\item One thousand and one hundredth (1100) - Tysęča sòtyǐ
	\item Two thousandth (2000)- Dvě tysęčnyǐ
	\item Five thousandth (5000) - Pęt tysęčnyǐ
	\item Millionth (1000000) - Milïonnyǐ
	\item One million one thousand and first (1001001) - Milïon tysęča pòrvyǐ
	\item Billionth ($10^9$) - Bilïonnyǐ (Milïardnyǐ)
	\item Trillionth ($10^{12}$) - Trilïonnyǐ
	\item Quadrillionth ($10^{15}$) - Kŭadrilïonnyǐ
	\item Quintillionth ($10^{18}$) - Kŭintilïonnyǐ
	\item Sextillionth ($10^{21}$) - Sekstilïonnyǐ
	\item Septillionth ($10^{24}$) - Septilïonnyǐ
	\item Octillionth ($10^{27}$) - Oktilïonnyǐ
	\item Nonillionth ($10^{30}$) - Nontilïonnyǐ
	\item Decillionth ($10^{33}$) - Decilïonnyǐ
	\item Googolth ($10^{100}$) - Ĝuĝòlnyǐ
	\item Googolplexth ($10^{10^{100}}$) - Ĝuĝlopleksnyǐ
\end{itemize}

\subsection{Collective numerals}

Collective numerals define a group of objects with some exact amount. They can be used either in an attributive role with a noun or on its own in a subjective role. Collective numerals can decline only in singular so as collective nouns do. 

Peculiar trait of collective numerals is in having "-ER-" and "-OJ-" suffixes.

\begin{table}[!htb]
	\begin{tabular}{lllll}
		Case & Two & Three & Four & Five \\
		Nominative & Dvoje & Troje & Čòtvero & Pętero \\
		Genitive & Dvojix & Trojix & Čòtveryh & Pęteryh \\
		Partitive & Dvojix & Trojix & Čòtveryh & Pęteryh \\
		Accusative* & Dvojix & Trojix & Čòtveryh & Pęteryh \\
		Dative & Dvojim & Trojim & Čòtverym & Pęterym \\
		Instrumentative & Dojimu & Trojimi & Čòtverymi & Pęterymi \\
		Prepositional & Dvojix & Dvojix & Čòtveryh & Pęteryh \\
		Locative & Dvojôx & Trojôx & Čòtverôh & Pęterôh \\
		Vocative & Dvoje & Troje & Čòtvero & Pętero \\
	\end{tabular}
\end{table}

\begin{table}[!htb]
	\begin{tabular}{lllll}
		Case & Six & Seven & Eight & Nine \\
		Nominative & Šestero & Sedmeryh & Osmero & Devętero \\
		Genitive & Šesteryh & Sedmeryh & Osmeryh & Devęteryh \\
		Partitive & Šesteryh & Sedmeryh & Osmeryh & Devęteryh \\
		Accusative* & Šesteryh & Sedmeryh & Osmeryh & Devęteryh \\
		Dative & Šesterym & Sedmerym & Osmerym & Devęterym \\
		Instrumentative & Šesterymi & Sedmerymi & Osmerymi & Devęterymi  \\
		Prepositional & Šesteryh & Sedmeryh & Osmeryh & Devęteryh \\
		Locative & Šesterôh & Sedmerôh & Osmerôh & Devęterôh \\
		Vocative & Šestero & Sedmero & Osmero & Devętero \\
	\end{tabular}
\end{table}

P.S. - In Accusative -ih/-yh form is used for animate nouns. For inanimate ones the -o form is used.

For the numeral "ten" the form "desętero" is used. It is declined so as "four" - "nine" do. The following numerals have the declination correlating with the cardinal form.
