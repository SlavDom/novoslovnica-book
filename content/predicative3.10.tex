\section{Predicative}

\begin{table}[!htb]
	\caption{Predicative characteristics}
	\begin{tabular}{ll}
		\textbf{Title}              & \textbf{Value}                            \\
		Semantic value              & Predicate                                 \\
		Category                    & Independent                               \\
		Subcategory                 & Nominal                                   \\
		Alteration                  & None                                      \\
		Alteration parameters       & None                                      \\
		Differentiation parameters  & Tense                                  
	\end{tabular}
\end{table}

Predicative\index{predicative} is a POS that is closely deriving to the predicate. Formally, it is an adverb that plays the role of the predicate (concluded in the predicate as a part of it). 

You should divide a syntax and morphological definitions of predicative. Now we are talking about the second one. In russian philology you can find the term of “condition category” - the equivalent for predicative (morphological definition). In this paragraph we will speak just about this term.

Predicative determines the class of words indicating an attribute or a condition of a person, environment etc., especially mental. Predicatives cannot be alternated, though they can be differentiated by tenses using an auxiliary verb “to be”. 

Predicatives have homonyms with adverbs, and nouns.  

\textbf{Examples:}

\textit{Ona hlådno mi kazala, če ne hoče da vidi mę} - She coldly said to me she does not want to see me (Adverb)

\textit{Dnësj je hlådno.} - It is cold today. (Predicative)

The most popular case of using predicative are modal constructions like "it is necessary", "it is needed" in passive form. Look at the following examples to compare active and passive forms.

\textit{Ja trěbam idati v còrkvu} - I should go to the church

\textit{Je nuđno mi da idam v còrkvu.} -  It is necessary for me to go to the church