\section{Predicative}

\begin{table}[h]
	\caption{Noun characteristics}
	\begin{tabular}{lllll}
		\textbf{Title}              & \textbf{Value}                            \\
		Semantic value              & Predicate                                 \\
		Category                    & Independent                               \\
		Subcategory                 & Nominal                                   \\
		Alteration                  & None                                      \\
		Alteration parameters       & None                                      \\
		Differentiation parameters  & Tense                                  
	\end{tabular}
\end{table}

Predicative is a POS that is closely deriving to the predicate. Formally, it is an adverb that plays the role of the predicate (concluded in the predicate as a part of it). 

You should divide a syntax and morphological definitions of predicative. Now we are talking about the second one. In russian philology you can find the term of “condition category” - the equivalent for predicative (morphological definition). In this paragraph we will speak just about this term.

Predicative determines the class of words indicating an attribute or a condition of a person, environment etc., especially mental. Predicatives cannot be alternated, though they can be differentiated by tenses using an auxiliary verb “to be”. 

Predicatives have homonyms with adverbs, and nouns.  
