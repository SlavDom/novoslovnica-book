\section{Accent}

Accent\index{accent} is a very difficult topic in most languages, because it is not permanent. There are some exceptions i.e. Czech and French, but in most cases we cannot say where accent will be put in the word without studying definite language. This causes problems for beginners.

Novoslovnica has a dynamic accent, but it has been formalized. There is a rule, that determines the place you should put the accent. 

\textbf{Rule n. 2}: The accent should be put on the first syllable of the word root.

This rule covers about 80\% of the words in the lexicon. You know the well-known 80/20 rule, or Pareto principle. It is something alike with the accent. Remained 20 per cent of the words we should cover by introducing extra-rule cases. Accepting these cases, you will be able to cover more than 99 per cent of Novoslovnica word amount.

These cases were created in attempt to unify the accents in different Slavic languages. Surely the Slavic languages have greatly changed since then, as they were one language. Therefore, accents in different Slavic languages often differ. Nevertheless, Novoslovnica tries to obliterate differences between them, producing accent patterns that could be comfortable to pronounce and to hear for all Slavs.

Below you can see the list of all these cases, that you should remember while speaking Novoslovnica.

Accent shifting cases:

\begin{itemize}
	\item{Accented endings (Nouns)}
	\subitem{-a (Dual, Nominative)}
	\subitem{-y (Singular, Genitive/Partitive)}
	\subitem{-ami (Plural, Instrumentative)}
	\subitem{-ama (Dual, Dative)}
	\subitem{-am (Plural, Dative)}
	\item{Accented endings (Verbs)}
	\subitem{-i Imperative (see paragraph about verb moods)}
	\item{Accented suffixes}
	\subitem{-ova- (Verb)}
	\subitem{-ôva- (Verb)}
	\subitem{-ava- (Verb)}
	\subitem{-óv- (Adjective)}
	\subitem{-ak- (Noun)}
	\subitem{-ok- (Noun)}
	\item{verb suffixes in Present Concrete Tense (see paragraph about verb tenses)}
	\item{Accent shift in the root}
	\subitem{If the word is a borrowed one, then the accent is put on the place it is in the original word.}
	\subitem{If the root loses its vowel, the accents moves one vowel to the left (if it is possible)}
	\subitem{Words that have more than one root (complex words) have their accent on the first syllable of the main word’s root. (see paragraph about complex words for what root is main)}
	\subitem{Adverbs or other parts of speech, formed with the prepositional construction, have their accents on the first syllable of the main word (see paragraph about collocations)}
\end{itemize}

These rules are enough for you to speak Novoslovnica properly with a few efforts for it. 
