\section{Participle}

\begin{table}[h]
	\caption{Adjective characteristics}
	\begin{tabular}{lllll}
		\textbf{Title}              & \textbf{Value}               \\
		Semantic value              & Attribute                    \\
		Category                    & Independent                  \\
		Subcategory                 & Verbal                       \\
		Alteration                  & Declension                   \\
		Alteration parameters       & Case, Number                 \\
		Differentiation parameters  & Type, Voice, Tense, Gender
	\end{tabular}
\end{table}

Participle is an independent verbal POS, that defines the action of another actor as its attribute. Participle is a POS between the nominal and verbal type, so as a pronoun, because it has properties of tense, voice, gender, case and number.

There are essential participles and auxiliary participles. The first ones can be used as attributes of nouns, they are far distanciated from the verb. They are Active and Passive Participles.

Auxiliary participles are quite closer to verbs, they participate in grammar constructions. They are Aorist and Imperfect participles.

Let us look at the tables to define how to use different participles.
