\section{Noun}

\begin{table}[h]
	\caption{Noun characteristics}
	\begin{tabular}{lp{12em}}
		\textbf{Title}              & \textbf{Value}                            \\
		Semantic value              & Concept                                   \\
		Category                    & Independent                               \\
		Subcategory                 & Nominal                                   \\
		Alteration                  & Declension                                \\
		Alteration parameters       & Case, Numbers, Gender, Type of Declension \\
		Differentiation parameters  & Gender, Animacy, Types of  Declension                                  
	\end{tabular}
\end{table}

\newglossaryentry{noun}{name=N, description={Noun}}


Nouns\index{noun} (\gls{noun}) can be differentiated by three parameters: gender, animacy and the type of declension.

\underline{Animacy\index{animacy}} determines whether the object is animate (we are able to ask “Who is it?” to the object) and inanimate (we are able to ask “What is it?” to the object).

\underline{Gender\index{gender}} determines whether the object is masculine, feminine or undefined (we cannot say it is one of the previous genders). Hence, there are three genders: masculine (with masculine properties), feminine (with feminine properties) and neutral (with undefined properties).

Despite English, Novoslovnica make us always show word gender explicitly of both animacies. We can say “it” to the object if we aren’t coupled with it in English. In Novoslovnica (as in every Slavic language) we should use the predefined gender when we speak about some concept (noun). Using wrong genders shows your ignorance and language nescience.

\underline{Type of declension} is a parameter of declension function. Declension\index{declension} is a function of word alteration. It has two input parameters - the word itself and the type of declension that includes the terms of animacy, gender and some morphological features (such as word endings) in it. The output is a list of forms that the noun can be changed into. Novoslovnica supports 27-cell output list with (3 numbers) * (9 cases) elements in it.

Further you can see tables of different declension types. The tables cover all use cases of declension function. You can find all of them in the eighth chapter.

\newglossaryentry{animate}{name=An, description={Animate}}
\newglossaryentry{inanimate}{name=Inan, description={Inanimate}}

P.S. In all tables abbreviations “\gls{animate}” and “\gls{inanimate}” are for “animate” and “inanimate” respectively.


\subsection{Collective nouns}

Collective nouns denote a set of homogeneous objects or living beings as an indivisible whole.

Collective nouns have special grammatical features:

\begin{itemize}
	\item Number immutability, only a singular form exists.
	\item Incongruity with quantitative numerals and with words denoting units of measure.
	\item Can be combined with the words \textit{many/few} or \textit{how many} in the singular form.
\end{itemize}

Collective nouns are indicated with the following suffixes:

\begin{itemize}
	\item ij
	\item stv
\end{itemize}

\textbf{Examples:}

- \textit{Bratija} - Brethren

- \textit{Pijonerija} - Pioneers

- \textit{Bratstvo} - Brotherhood

- \textit{Čelověkstvo} - Mankind


\subsection{Verbal Noun}

Verbal Noun\index{noun!verbal} is a group of nouns that were created from the verbs or their forms. Verbal forms can be formed twice: directly from the verb (from infinitive) and indirectly from gerund.

Direct verbal nouns are formed by reducing verb with its ending “-ti” and then adding some needed suffixes and endings.

There are two different means of forming a verbal noun such a way.

\begin{itemize}
	\item Adding suffix “-k-” or "-b-" and “-a” ending. The latter fact means that all these nouns are of the first type of declension. 
	\item Reducing verbal suffix (“-a-”,  “-e-” etc.) and adding null-ending. These nouns will be of the second declension.
\end{itemize}

Usually, the second mean corelates with English short nouns derived from verbs (to run - run, to breathe - breath etc.)

\textbf{Examples:}

\textit{Běg} - Run

\textit{Dyh} - Breath

\textit{Stroǐka} - Building

\textit{Myǐka} - Washing

\textit{Borba} - Struggle 

\textit{Sųdba} - Fate

However, not all verbs allow to create verbal nouns in such ways. So there is a better way of creating verbal nouns indirectly.

Indirect verbal nouns are formed by adding a suffix “-iǐ-” to the gerund. These nouns are of the second declension with “-je” ending. Remember the fact that before the vowel “-ǐ-” transforms into “-j-”. 

You can use the second type everywhere in your sentences. Nevertheless, if there is a verbal noun of the first type within the word lexeme, the verbal noun of the first type is preferred to be used.

Let us see the \textbf{examples}:

\textit{Pisati} (verb) - pisane (gerund) - pisanije (verbal noun)

\textit{Strojati} (verb) - \textit{strojka} (verbal noun, I type (preferred)) - \textit{strojane} (gerund) - \textit{strojanije} (verbal noun, II type (is not preferred))

\textit{Běgati} (verb) - \textit{běg} (verbal noun, I type (preferred)) - \textit{běgane} (gerund) - \textit{běganije} (verbal noun, II type (is not preferred))  


