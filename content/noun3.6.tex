\section{Noun}

%Table

Nouns can be differentiated by three parameters: gender, animacy and the type of declension.

Animacy determines whether the object is animate (we are able to ask “Who is it?” to the object) and inanimate (we are able to ask “What is it?” to the object).

Gender determines whether the object is masculine, feminine or undefined (we cannot say it is one of the previous genders). Hence, there are three genders: masculine (with masculine properties), feminine (with feminine properties) and neutral (with undefined properties).

Despite English, Novoslovnica make us always show word gender explicitly of both animacies. We can say “it” to the object if we aren’t coupled with it in English. In Novoslovnica (as in every Slavic language) we should use the predefined gender when we speak about some concept (noun). Using wrong genders shows your ignorance and language nescience.

Type of declension is a parameter of declension function. Declension is a function of word alteration. It has two input parameters - the word itself and the type of declension that includes the terms of animacy, gender and some morphological features (such as word endings) in it. The output is a list of forms that the noun can be changed into. Novoslovnica supports 27-cell output list with (3 numbers) * (9 cases) elements in it. Further you can see tables of different declension types. These tables cover all use cases of declension function.

P.S. In tables abbrevs “A” and “I” are for “animate” and “inanimate” respectively.

% Tables