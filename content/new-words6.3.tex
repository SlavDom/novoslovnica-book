\section{New word creating}

The process of creating the language’s vocabulary is one of the most important processes in language constructing. We need to build an understandable lexeme, moreover, it can be named as a Slavic one. 

Here I want to tell you about this process we follow every time when the new word of Novoslovnica appears in this world. By following this algorithm you will be able to create new words that are suitable for Novoslovnica by yourself. Now look here:

We are looking through the vocabularies of Slavic languages and find out the translations from our native language for the exact word.

Then we look at the frequency of roots within this list.

\begin{enumerate}
	\item If there is an absolute leader there, and it is Slavic, we take this root.
	\item Then we look at the frequency of suffixes and prefixes within this list.
	\item If there is an absolute leader, and it stays for the primary semantic value, we take this form.
	\item If not, we create a new form by following described above semantic values of different prefixes and suffixes.
	\item If there is some forms that are prevail others, we can add them all into Novoslovnica vocabulary.
	\item If not, we check whether there are some non-absolute leaders within the words. If they are, we go to 3.a to each of them.
	\item If not, and there is a non-Slavic absolute or some relative leaders, we should take into account the next points:
	\subitem There is no Slavic root among them
	\subitem We cannot form an artificial Slavic form for a word.
	\item If the previous forms are true, we add a non-Slavic word into vocabulary. However, if we can form a Slavic word - we add it into vocabulary with additional non-Slavic “crutch”. 
\end{enumerate}

Frankly, this is the whole algorithm. You can apply it in your everyday speaking. However, it takes time to build a new word for the only semantic value, so it is difficult to create all your words by these rules by yourself. It is better for you to use a dictionary, that have already been created by our team and contains more than 5 thousand forms of Novoslovnica’s vocabulary.. 