\section{Particle}

Particle\index{particle} is a dependent part of word. Particles add an auxiliary meaning to the main word. There are some groups of particles.

Particles added to words are close to some additional functions. If you delete them from your phrase, there will be no change in the whole meaning (that is why one cannot say they are of an auxiliary POS), but with presence of particles you will get more additional semantic or sometimes emotional information named \textit{color}. English language has very few particles. The most known is “to” as an infinitive indicator particle of a verb. Controversially, Slavic languages have a bit more particles that are rather popular in the spoken language. They form several groups by their semantic color.

\textbf{Examples:}

\textit{Ja sòm govoril tobě.} - I have told you.

\textit{Ja sòm govoril že tobě!} - I have told you!

Additional semantic meaning. The speaker shifts emphasis from undefined (neutral phrase) to the word “govoril” (told). So the interlocutor now has a determined emphasis of the phrase. The speaker wants to say that he has already told the same fact to the interlocutor and he was right because something happened confirming them.

Positive

* Aga - Yeap

* Ugu - Yeah

* Da - Yes

Negative

* Ne - Not
* Ni...ni - Neither...nor

Interrogative

* Či - Whether

* Li - Whether

Estimative

* Kakto - Like

* Mòl - Supposedly

Comparative

Incentive

Exclamative

Amplificative


Specifying

Restrictive

Demonstrative


