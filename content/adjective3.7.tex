\section{Adjective}

\begin{table}[h]
	\caption{Adjective characteristics}
	\begin{tabular}{lllll}
		\textbf{Title}              & \textbf{Value}               \\
		Semantic value              & Attribute                    \\
		Category                    & Independent                  \\
		Subcategory                 & Nominal                      \\
		Alteration                  & Declension                   \\
		Alteration parameters       & Case, Number, Gender, Degree \\
		Differentiation parameters  & Gender, Type, Form
	\end{tabular}
\end{table}

Adjective is one of POS that determines an attribute of the concept. There are two types of adjectives - relative and qualitative. 

Relative adjectives are called so, because they show relations between two concepts or a concept and an action.

\textbf{Examples:}

\textit{South (adj) pole} = South (noun) <= relation <= Pole

\textit{Južnyǐ pôl} = Jug <= relation <= Pôl

Qualitative adjectives are called so, because they show the quality of a concept’s property. This quality could be relative or quantitative or purely qualitative (showing concept condition, position, measure etc).

\textbf{Examples:}

\subsection{Full adjectives}

There is the only type of adjective declension. However, we will divide declension tables by gender and base softness.


\begin{table}
	\caption{Masculine}
	\begin{tabular}{lllllll}
		\textbf{Masculine}       
		& \multicolumn{2}{c}{Singular} 
		& \multicolumn{2}{c}{Dual} 
		& \multicolumn{2}{c}{Plural} \\
		& Hard   & Soft  & Hard   & Soft   & Hard  & Soft \\
		Nominative    & -yǐ & -ïǐ     
		& -aja  & -äja        
		& -yji & -ïji \\
		Genitive      & -oga & -ëga 
		& -oŭ & -oŭ
		& -yh & -ïh \\
		Partitive     & -ogu & -ëgu 
		& -oŭ & -oŭ
		& -yh & -ïh \\
		Accusative    & -ogo & -ëgo     
		& -aja & -äja
		& -yji* & -ïji*  \\
		Dative		  & -omu & -ëmu
		& -oma & -ëma 
		& -ym & -ïm \\  
		Instrumental  & -ym & -ïm     
		& -yma & -ïma   
		& -ymi & -ïmi \\
		Prepositional & -om & -ëm  
		& -yvěh & -ïvěh     
		& -ěh & -ěh \\
		Locative      & -omu & -ëmu      
		& -oŭ & -oŭ
		& -yh & -ïh \\
		Vocative       & -yǐ & -ïǐ     
		& -aja  & -äja        
		& -yji & -ïji 
	\end{tabular}
\end{table}


\begin{table}
	\caption{Feminine}
	\begin{tabular}{lllllll}
		\textbf{Feminine}       
		& \multicolumn{2}{c}{Singular} 
		& \multicolumn{2}{c}{Dual} 
		& \multicolumn{2}{c}{Plural} \\
		& Hard   & Soft  & Hard   & Soft   & Hard  & Soft \\
		Nominative    & -aja & -äja     
		& \multicolumn{2}{c}{-ěja}        
		& -yje & -ïje \\
		Genitive      & -oji & -ëji
		& -oŭ & -öŭ
		& -yh & -ïh \\
		Partitive    & -oji & -ëji
		& -oŭ & -öŭ
		& -yh & -ïh \\
		Accusative    & -uju & -üju     
		& \multicolumn{2}{c}{-ěja} 
		& -yje* & -ïje*  \\
		Dative		  & -oǐ & -ëǐ
		& -oma & -ëma 
		& -ym & -ïm \\  
		Instrumental  & -oju & -ëju
		& -yma & -ïma   
		& -ymi & -ïmi \\
		Prepositional  & -oǐ & -ëǐ
		& -yvěh & -ïvěh     
		& -ěh & -ěh \\
		Locative      & -oji & -ëji      
		& -oŭ & -oŭ
		& -yh & -ïh \\
		Vocative      & -aja & -äja     
		& \multicolumn{2}{c}{-ěja}        
		& -yje & -ïje 
	\end{tabular}
\end{table}

\begin{table}
	\caption{Neutral}
	\begin{tabular}{lllllll}
		\textbf{Neutral}       
		& \multicolumn{2}{c}{Singular} 
		& \multicolumn{2}{c}{Dual} 
		& \multicolumn{2}{c}{Plural} \\
		& Hard   & Soft  & Hard   & Soft   & Hard  & Soft \\
		Nominative    & -oje & -ëje     
		& -aja  & -äja        
		& -yje & -ïje \\
		Genitive      & -oga & -ëga 
		& -oŭ & -oŭ
		& -yh & -ïh \\
		Partitive     & -ogu & -ëgu 
		& -oŭ & -oŭ
		& -yh & -ïh \\
		Accusative    & -ogo & -ëgo     
		& -aja & -äja
		& -yji* & -ïji*  \\
		Dative		  & -omu & -ëmu
		& -oma & -ëma 
		& -ym & -ïm \\  
		Instrumental  & -ym & -ïm     
		& -yma & -ïma   
		& -ymi & -ïmi \\
		Prepositional & -om & -ëm  
		& -yvěh & -ïvěh     
		& -ěh & -ěh \\
		Locative      & -omu & -ëmu      
		& -oŭ & -oŭ
		& -yh & -ïh \\
		Vocative       & -oje & -ëje     
		& -aja  & -äja        
		& -yje & -ïje 
	\end{tabular}
\end{table}

\subsection{Short adjectives}

We can find differences in using short and full forms of adjectives only in some cases - Nominative and Accusative. In other cases short adjectives correspond with appropriate full form adjective.

To understand how to change a full adjective to receive its short form look at the next table (C is for “consonant”, V is for “vowel”).

\begin{table}
	\begin{tabular}{ll}
		Full form adjective endings & Short form adjective endings \\
		C + C + yǐ & C + ò C \\
	    V + C + yǐ & V + C \\
		aja & a \\
		oje  & o \\
		yje & y \\
	\end{tabular}
\end{table}

What about using full and short adjectives… There are some recommendations that you should follow in your speech. I tried to unite them into one table with cases when you should use either full form or short one.

\begin{table}
	\begin{tabular}{ll}
		Full form & Short form \\
		Before the modified noun & In a complex predicate \\
		In denominative sentence & - \\
	\end{tabular}
\end{table}


\subsection{Degrees of comparison}

When we speak about qualitative adjectives, we use a term of a degree of comparison. It’s the condition whether temporal adjective is greater in its measure than another one or not. There are three degrees: positive, comparative and superlative.

\underline{Positive} (or neutral, basic) degree is used when we do not mind about the comparison with other attributes. We could call it an undefined degree.

\underline{Comparative} degree is used when our attribute is greater than another one (attributes must be comparable).  

\underline{Superlative} degree is used when our concept has the attribute with most valuable degree in considered area.

There are two ways of comparison - synthetic and analytic. Synthetic comparison changes the word itself, while analytic comparison uses analytic constructions with other words to create a degree of comparison. You can you both analytic or synthetic comparison forms in your speech.

\textbf{Synthetic comparison}

Comparative degree can be formed in two ways: 

\begin{itemize}
	\item Adding between word base and word ending “-š-”, that means that the attribute has a more strongly pronounced property than another concept, expressed by a noun.
	\item Adding suffix “-š-” plus suffix “-ëǐ-” for soft word base and suffix “-aǐ-” for hard word base before it.
\end{itemize}

\textbf{Examples:}

Bolïǐ – boljšyǐ. Mnogyǐ - mnogšyǐ. Vëlïkyǐ - vëlïkšyǐ.

Bolïǐ – bolëǐšyǐ. Mnogyǐ - množaǐšyǐ. Vëlïkyǐ - vëlïčaǐšyǐ.

Superlative degree also has two variants, that are formed by adding a prefix “-naǐ-” to the comparative form. This prefix has a similar meaning with the English word “most”.

\textbf{Examples:}

Bolïǐ – boljšyǐ - naǐboljšyǐ. Mnogyǐ - mnogšyǐ - naǐmnogšyǐ. Vëlïkyǐ - vëlïkšyǐ - naǐvëlïkšyǐ.

Bolïǐ – bolëǐšyǐ - naǐbolëǐšyǐ. Mnogyǐ - množaǐšyǐ - naǐmnožaǐšyǐ. Vëlïkyǐ - vëlïčaǐšyǐ - naǐvëlïčaǐšyǐ.

Some says Novoslovnica has five degrees of comparison instead of three degrees with doubly forms. Let us know this classification.

\begin{itemize}
	\item \textit{Positive} degree equals to the one in ordinary classification. (Bolïǐ)

	\item \textit{Defined Comparative} degree matches the first variant of comparative form in ordinary classification. It is used when there are two objects and the temporal one has a prevailed property to another one. (\textit{Bolïšyǐ})

	\item \textit{Undefined Comparative} degree matches the second variant of comparative form in ordinary classification. It is used when we have some objects (more than two) and temporal object has a prevailed property to a few objects in the set (probably its power is an undefined number). (\textit{Bolëǐšyǐ})

	\item \textit{Relative superlative} degree matches the first variant of superlative form in ordinary classification. It is used when temporal object is in the set and has a superior property in it, but we cannot say that this property would have a superior value in other sets. (\textit{Naǐboljšyǐ})

	\item \textit{Absolute superlative} degree matches the second variant of superlative form in ordinary classification. It is used when there are no doubts in superiority of the temporal object’s property. (\textit{Naǐbolëǐšyǐ})
\end{itemize}

\textbf{Analytic forms}

There are two variants of how to use analytic comparison of adjectives: to use prefixes or to use an auxiliary adverb.

To create a comparative or a superlative form you should add a prefix “po-” or “naǐ-” respectively to the word though a defis.

\textbf{Examples:}

Kråtkyǐ - po-kråtkyǐ - naǐ-kråtkyǐ

Sïlnyǐ - po-sïlnyǐ - naǐ-sïlnyǐ

Analytic comparison forms have only three ones - positive, comparative and superlative. Analytic comparison with an auxiliary adverb is formed by adding to the positive form of the adjective a comparative or a superlative form of an auxiliary adverb (look at the paragraph about adverb degrees of comparison). However, you cannot use analytic comparison with adjectives “bolïǐ” and “mënïǐ”, because they are the basic forms of these auxiliary adverbs.

\textbf{Examples:}

Kråtkyǐ - bolěǐ kråtkyǐ - naǐbolěǐ kråtkyǐ

Bolïǐ - bolěǐ bolïǐ (\textbf{you cannot do that}!)
