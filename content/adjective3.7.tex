\section{Adjective}

\begin{table}[h]
	\caption{Adjective characteristics}
	\begin{tabular}{lllll}
		\textbf{Title}              & \textbf{Value}               \\
		Semantic value              & Attribute                    \\
		Category                    & Independent                  \\
		Subcategory                 & Nominal                      \\
		Alteration                  & Declension                   \\
		Alteration parameters       & Case, Number, Gender, Degree \\
		Differentiation parameters  & Gender, Type, Form
	\end{tabular}
\end{table}

Adjective is one of POS that determines an attribute of the concept. There are two types of adjectives - relative and qualitative. 

Relative adjectives are called so, because they show relations between two concepts or a concept and an action.

\textbf{Examples:}

\textit{South (adj) pole} = South (noun) <= relation <= Pole

\textit{Južnyǐ pôl} = Jug <= relation <= Pôl

Qualitative adjectives are called so, because they show the quality of a concept’s property. This quality could be relative or quantitative or purely qualitative (showing concept condition, position, measure etc).

\textbf{Examples:}

There is the only type of adjective declension. However, we will divide declension tables by gender and base softness.
