\section{Simple sentense}

Originally, a simple sentence consists of a single independent clause with a finite verb in it. Complex and compound sentences (look further) may be of multiple clauses, though a simple sentence has the only one.

\subsection{Subject}
The actor, that subject is indeed, can be described by different means. However, the most frequently used one is a noun. Every POS that is a subject in the sentence should be put into initial form. For a noun, its initial form is Nominative case. Noun in other cases cannot play the role of a subject.
Frankly speaking, a lot of POS in proper cases can play role of a subject. Now I want to list the most popular ones to let you know what is the point.

% Table

The most popular ones are first three rows.

At the end I will list the examples that could help you in understanding of how to put the word into its initial form to let it be the subject of a sentence. 

\textbf{Examples:}


\subsection{Predicate}
The action, which is provided by the actor, can be also described in different ways. The most popular is the verb, of course. The verb, determining the tense of the action, the person that does this action, the mood of the action, can be formed into every available for a verb form… Novoslovnica does not provide a noun being as a predicate (as it is available in Russian). Such sentences are transformed into the ones with the auxiliary verb “to be” as connection between the term and the determining word. 

\subsection{Auxiliary members}
The subject and the predicate are the main members of a simple sentence. However, you can notice that they do not provide  enough information to the interlocutor. That is why there are auxiliary members of a sentence, that give us necessary information.

\subsubsection{Object}
This auxiliary member gives the information about objects that are influenced by the action actor does. 
Here we should speak about one more classification of verb: by the mean of transitivity.

The transitive verb can have an object in accusative case without any prepositions.

The intransitive verb can have no objects or only with a preposition. 

\subsubsection{Adverbial}
Adverbial helps us to get some more information about predicate. Sometimes, they are involved in semantic connection with the verb and become actants. 

\subsubsection{Modifier}
Modifier brings some additional information about attributes of the subject, objects etc. They help us to build the exact picture of the situation that is described by a predicate.

\subsection{Word order}
There are languages with free word order and with strict one. Novoslovnica is a language with a declared word order, when changing the order of the words shifts a semantic value of the whole sentence. However, you can change some words in the sentences as you like. Let us see what the order is.

The common structure of simple sentence with neutral semantic value is SVO: \textit{Subject + Verb + Object}. From this thesis we can write down two rules:

Predicate is placed after the Subject

Object is placed after the Predicate it is connected to.

These rules are main if you want to create a grammatically proper sentence with neutral semantic value. 

There are also three additional rules:

\begin{itemize}
	\item Consistent grammar modifiers should be placed before the modified word.
	\item Inconsistent grammar modifiers should be placed after the modified word.
	\item Adverbials should be placed after the Predicate.
	\item Interrogatives should be placed at the beginning of the sentence.
\end{itemize}

These six rules can help you to understand what order is suitable for a Slavic sentence. There are also some constraints of what should not appear in your sentence:

\begin{itemize}
	\item Participial should be placed after or before the modified word.
	\item The particle “b/by” should not be placed in the beginning of the sentence.
\end{itemize}

Thus, now you are able to create a normal simple sentence with neutral semantic value. Nevertheless, you can gain a certain accent on the word you want by replacing words within the sentence.

Using SOV idiom you put the accent on the Object. 

Using VSO idiom you put the accent on the Verb (the Predicate)

Using OVS idiom you put the accent on the pair of Object and Predicate.

Using OSV idiom you put the accent on the pair of Object and Subject.

Using VOS idiom is equal to using a VSO idiom.

The same thing is about auxiliary members of the sentence - if you want to put an accent on the word, you should place it at the beginning of the sentence. Subject is an exception - its normal place is first, so you can put an accent on it only in your pronunciation.

\subsection{Incomplete sentence}
However, sometimes we do not need to say everything in our sentence, because it will cause tautology. Then we reduce some words calling for ellipsis. A sentence is named incomplete, when the subject or the predicate is not used in it.

\subsection{Substantive sentence}
Sometimes it is enough to say only a collocation or even the only word to show your idea to the interlocutor. This is what the substantive sentence is like.

We would not speak about them, because we suppose everything is quite clear. Just use phrases or the only word to express your idea.

\subsection{Relative clause}
Relative clause is one of grammar modifiers. It is a separate definition or adverbial, which has turned into a sentence that depends on the main one.
