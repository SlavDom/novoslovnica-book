\section{Simple sentense}

Originally, a simple sentence consists of a single independent clause with a single \gls{vp} in it. Complex and compound sentences (look further) may be of multiple clauses, though a simple sentence has the only one.

\subsection{Subject}

The actor\index{actor}, that subject\index{subject} is indeed, can be described by different means. However, the most frequently used one is a noun.  Every POS that is a subject in the sentence should be put into initial form. For a noun, its initial form is Nominative case. Noun in other cases cannot play the role of a subject.

To be frank, a lot of POS in proper cases can play role of a subject. Now I want to list the most popular ones to let you know what is the point.

\begin{table}[h]
	\begin{tabular}{ll}
		\textbf{POS} & \textbf{Initial form} \\
		Noun & Nominative \\
		Numeral & Cardinal number \\
		Pronoun & Nominative \\
		Verb & Infinitive \\
		Allocation & The main word in initial form \\
		Adjective & Nominative \\
	\end{tabular}
\end{table}

The most popular ones are first three rows. Though, you can meet the other ones in the texts.

\subsection{Predicate}

The action\index{action}, which is provided by the actor, can be described by \gls{vp}.\index{predicate} The verb, determining the tense of the action, the person that does this action, the mood of the action, can be formed into every available verb form. Novoslovnica does not allow a noun to be a predicate (as it is available in Russian). Such sentences are transformed into the ones with the auxiliary verb “to be” as connection between the term and the determining word.

\textbf{Examples:}

\textit{Kniga je napisana u mïnuloji ročinji} - The book was written last year.

\textit{Pętero prijatelëǐ sę postojahu u menę 5 dnëǐ} - Five friends have stayed at me for 5 days 

\textit{On sę zova Peter} - His name is Peter

\textit{Pjiti vodku je hudo} - Drinking vodka is bad.

\subsection{Auxiliary members}

The subject and the predicate are the main members of a simple sentence. However, you can notice that they do not provide  enough information to the interlocutor. That is why there are auxiliary members of a sentence, that give us the necessary information.

\subsubsection{Object}

This auxiliary\index{object} member gives the information about objects that are influenced by the action actor does. 
Here we should speak about one more classification of verb: by the mean of transitivity.

The transitive\index{verb!transitive} verb can have an object in accusative case without any prepositions.

The intransitive\index{verb!intransitive} verb can have no objects or only with a preposition. 

\textbf{Examples:}

\textit{Ja lübam ovo sëlišto.} - I like this settlement. (Transitive verb "lübati" (to love, to like))

\textit{Ja glědam na ovo sëlišto.} - I look at this settlement. (Intransitive verb "glědati" (to look at, to watch))

\subsubsection{Modifier}

Modifier\index{modifier} brings some additional information about attributes of the subject, objects etc. They help us to build the exact picture of the situation that is described by a predicate.

\textbf{Example:}

\textit{Lěp krasen sòn.} - Pretty beautiful dream.

\textbf{Adverbial modifier}

Adverbial modifier\index{modifier!adverbial} helps us to get some more information about predicate. Sometimes, they are involved in semantic connection with the verb and become actants. 

\textbf{Example:}

\textit{Lěpo tïho spati.} - To sleep pretty quite.

\subsection{Word order}

There are languages with free word order and with strict one\index{word order}. Novoslovnica is a language with a declared word order, when changing the order of the words shifts a semantic value of the whole sentence. However, you can change some words in the sentences as you like. Let us see what the order is.

The common structure of a simple sentence with neutral semantic value is SVO: \textit{Subject + Verb + Object}. From this thesis we can write down two rules:

\begin{itemize}
	\item Predicate is placed after the Subject
	\item Object is placed after the Predicate it is connected to.
\end{itemize}

These rules are main if you want to create a grammatically proper sentence with neutral semantic value. 

There are also three additional rules:

\begin{itemize}
	\item Consistent grammar modifiers should be placed before the modified word.
	\item Inconsistent grammar modifiers should be placed after the modified word.
	\item Adverbials should be placed after the Predicate.
	\item Interrogatives should be placed at the beginning of the sentence.
\end{itemize}

These six rules can help you to understand what order is suitable for a Slavic sentence. There are also some constraints of what should not appear in your sentence:

\begin{itemize}
	\item Participial should be placed right after or before the modified word.
	\item The particle “b/by” should not be placed in the beginning of the sentence.
\end{itemize}

Thus, now you are able to create a normal simple sentence with neutral semantic value.

Nevertheless, you can gain a certain accent on the word you want by replacing words within the sentence.

\begin{itemize}
	\item Using SOV idiom you put the accent on the Object.
	\item Using VSO idiom you put the accent on the Verb (the Predicate)
	\item Using OVS idiom you put the accent on the pair of Object and Predicate.
	\item Using OSV idiom you put the accent on the pair of Object and Subject.
	\item Using VOS idiom is similar to using a VSO idiom with a low accent on the object.
\end{itemize}

The same thing is about auxiliary members of the sentence - if you want to put an accent on the word, you should place it at the beginning of the sentence. Subject is an exception - its normal place is first, so you can put an accent on it only in your pronunciation or by using particles.

\textbf{Examples:} (\textit{My house has two beautiful windows.})

\textit{Môǐ dom ïmá dva lěpyh prozorca.} - SVO, no accent

\textit{Môǐ dom dva lěvyh prozorca ïmá.} - SOV, accent on the object

\textit{Ïmá môǐ dom dva lěpyh prozorca.} - VSO, accent on the verb

\textit{Ïmá dva lěvyh prozorca môǐ dom.} - VOS, accent on the verb and the object

\textit{Dva lěpyh prozorca ïmá môǐ dom.} - OVS, accent on the object and the verb

\textit{Dva lěpyh prozorca môǐ dom ïmá.} - OSV, accent on the object and the subject

\subsection{Incomplete sentence}

However, sometimes we do not need to say every sentence parts in our sentence, because it will cause tautology. Then we reduce some words calling for ellipsis\index{ellipsis}.

A sentence is named \textit{incomplete}\index{sentence!incomplete}, when the subject or the predicate is not used in it.

\textbf{Examples:}

- \textit{Oběd (stoji) na stolištu.} - The dinner (is) on the table. (The verb "is placed" is omitted)

- \textit{(Ja) šukaju knigu-ta.} - I am looking for the book. (The pronoun is omitted).

The second example shows the most frequently used case in Novoslovnica. If pronoun should be used as a subject in a sentence it is often omitted because the following verb can represent the information about the person and the number has been used in it. Omitting predicate is much rarer phenomenon. Mostly it is used in some kind of "frozen" allocations that are not an aphorism, but has a common meaning within language community.

\subsection{Substantive sentence}

Sometimes it is enough to say only a collocation or just a word to show your idea to the interlocutor. This is what the substantive\index{sentence!substantive} sentence is like.

\textbf{Examples:}

- \textit{Tvôǐ kot sę zova Musjka? - Môǐ.}

- \textit{Koĝda hte poǐdaš na věčôrku s prijatelämi? - Dnesj.}

When you give a short positive or negative answer it is also a kind of a substantive sentence.

\textbf{Example:}

\textit{Da li ty pušaš? - Da} (Do you smoke? - Yes)

\textit{Da li priǐdaš do nas na nastepnoji sedmicji? - Ne.} (Do you come to us next week? - No)

\subsection{Relative clause}

Relative\index{clause!relative} clause is a kind of grammar modifiers. It is a separate definition or an adverbial modifier, which has turned into a sentence that depends on the main one.

\textbf{Examples:}

\textit{Učujah ŝvųk, užasajučï ušy mi.} - I heard a sound terrifying my ears.