\section{Gerund}

Gerund is a form of a verb that determines the process of an action. That means it is not an independent POS, but has some features that do not allow us to include the description about gerund in the paragraph about verb.

Gerund is formed from the infinitive of a verb by reducing a “-ti” ending and adding a suffix “-n-” and an ending “-e-”. However, gerunds are often created only from verbs of A-type. However, you can construct gerunds from verbs of other types. Look at the examples:

Examples:

Pisati (verb) - pisane (gerund)

Nositi (verb) - nosine (gerund, bad form) - nošane (gerund, recommended form)

Pěti (verb) - Pěne (gerund, bad form) - pějane (gerund, recommended form)

Unlike infinitive or supine, gerund can be declined, but only in the singular. This fact unites it with the nominal POS. Simply speaking, gerund declines so as soft neutral nouns do despite the nominal/accusative form of “e”.

Case
Ending
Nominal
-e
Genitive
-ä
Partitive
-ü
Dative
-ü
Accusative
-e
Instrumental
-ëm
Prepositional
-ě
Locative
-ü
Vocative
no form

Let us look at some examples:
Examples:


