\section{Verbal Noun}

Verbal Noun\index{noun!verbal} is a group of nouns that were created from the verbs or their forms. Verbal forms can be formed twice: directly from the verb (from infinitive) and indirectly from gerund.

Direct verbal nouns are formed by reducing verb with its ending “-ti” and then adding some needed suffixes and an endings.

What are the suffixes suitable for forming a verbal noun? 

There are two different means of forming a verbal noun such a way.
Adding suffix “-k-” and ending “-a”. The latter fact means that all these nouns are of the first type of declension. 

Reducing verbal suffix (“-a-”,  “-e-” etc.) and adding null-ending. These nouns will be of the second declension.

However, not all verbs allow creating verbal nouns in such ways. So there is a better way of creating verbal nouns indirectly.

Indirect verbal nouns are formed by adding a suffix “-iǐ-” to the gerund. These nouns are of the second declension with “-je” ending. Remember the fact that before the vowel “-ǐ-” transforms into “-j-”. 

You can use the second type everywhere in your sentences. Nevertheless, if there is a verbal noun of the first type within the word lexeme, the verbal noun of the first type is preferred to be used.

Let us see the examples:

\textit{Pisati} (verb) - pisane (gerund) - pisanije (verbal noun)

\textit{Strojati} (verb) - \textit{strojka} (verbal noun, I type (preferred)) - \textit{strojane} (gerund) - \textit{strojanije} (verbal noun, II type (is not preferred))

\textit{Běgati} (verb) - \textit{běg} (verbal noun, I type (preferred)) - \textit{běgane} (gerund) - \textit{běganije} (verbal noun, II type (is not preferred))  

