\section{Interjection}

This interesting POS is used to describe emotions within the sentence. Interjections\index{interjection} are neither independent nor dependent POS. They even are not involved into sentence structure. They just show the color of the sentence to let the interlocutor understand your feelings. 

Interjection is a POS that has no semantic meanings. Interjections are not words in our common comprehension. They are sounds that we produce. Sometimes it is rather difficult to say what differences are between interjections and senseless sounds that we can produce (i.e. some spontaneous exclaims, murmuring etc.). Interjections are divided into intended exclaims (with bright emotional color) and sound imitations (i.e. animal voices imitation) that are united in Novoslovnica within the POS of Onomatopoetics. So, Interjections in Novoslovnica are only used to express emotional exclaims.

Interjections are divided into three groups depending on their aim: emotional, imperative and etiquette. 

\textbf{Emotional interjections}

Emotional interjections can be divided into negative and positive interjections.

\textbf{Positive:}

Ah - Ah

Ŭaŭ - Wow

Ŭah - Wow

Ura - Hurray

Ogo - Wow

Uf - Uh

\textbf{Negative}

Oh - Oh

O-o - Oh

Be - Phew

Hehe - Huh

Heh - Heh

Éh - Eh

Jo - Yo

Fu - Phew

\textbf{Ambiguous} (depend on the context):

Uh - Uh

Oǐ - Oh

Aǐ - Aw

Hm - Hmm

\textbf{Imperative interjections}

Let us look at them:

Éǐ - Hey

Na - Take it

Stop - Stop

Bre - Man

A-u - Hey

Allo - Hello

Brysj - Go out

Von - Out

\textbf{Etiquette interjections}

These interjections are often whole words, that we use without the sentence context in some situations that need our etiquette.

Hvála - Thanks

Dobrodošli - Welcome

Dękujem - Thanks

Zdråveǐ - Hi

Zdråveǐte - Hello
