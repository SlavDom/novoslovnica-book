\section{Interjection}

This interesting POS is used to describe emotions within the sentence. Interjections are neither independent nor dependent POS. They even are not involved into sentence structure. They just show the color of the sentence to let the interlocutor show your feelings. Interjections are divided into three groups depending on their aim: emotional, imperative and etiquette. 

\textbf{Emotional interjections}

Emotional interjections can be divided into negative and positive interjections.

Positive:

Ah - Ah

Ŭaŭ - Wow

Ŭah - Wow

Ura - Hurray

Ogo - 

Uf - 

\textbf{Negative}

Oh - 

O-o - Oh

Be - 

Hehe -

Heh -  

Éh - 

Jo - 

Fu - 

\textbf{Ambiguous} (depend on the context):

Uh - 

Oǐ - 

Aǐ - 

Išty - 

Hm - Hmm

\textbf{Imperative interjections}

Let us look at them:

Éǐ - Hey

Na - Take it

Stop - Stop

Bre - Man

A-u - 

Allo - Hello

Brysj - 

Von -

\textbf{Etiquette interjections}

These interjections are often whole words, that we use without the sentence context in some situations that need our etiquette.

Hvála - Thanks

Dobrodošli - Welcome

Dękujem - Thanks

Zdråveǐ - Hi

Zdråveǐte - Hello
