\section{Verb}

Verb\index{verb} is a POS that describes the predicate and its properties. 


\subsection{Verbal types}

Learning Slavic languages you could mention that there is a set of verb suffixes that is very eliminated. Novoslovnica provides a theory that allows you to construct a right verb form.
Verbs with different verb suffixes represent different verbal types. There are four types of verbs in Novoslovnica.

\begin{itemize}
	\item A-type\index{verb!a-type}: verbs of this type define that the action in common sense.  
	\item E-type\index{verb!e-type}: verbs of this type define that the action is long-termed.
	\item I-type\index{verb!i-type}: verbs of this type define that the action is short-termed. This type comprises verbs with suffixes I and O. The suffix O is used when the consonant before the constructed suffix is involved in alterations depending on soft vowels that the vowel I is.
	\item U-type\index{verb!u-type}: verbs of this type define that the action is dotty.
	\item Extra type\index{verb!extra-type}. Is formed with the suffix “-OVA-” and defines the repeated action.  
\end{itemize}

When you speak about the action, you find what characteristic is suitable for the action and then use one of the predefined verbal types.

Somebody can ask what are the differences in tenses and types, because it might be confusing. However, verbal type determines the durability of the action (or its repeating property) while tense determines tense characteristic of the action such as completeness, stability in time, result, order etc.

All tenses provide difference between conjugation of different verbal types except imperfect. This tense has the common conjugation table for all verbal types. 

Further we will speak about conjugation itself. We will look at verb conjugation in indicative mood first of all. Then we will speak about other moods.

\subsection{Active voice}

Active\index{voice!active} voice shows that the person makes the action by himself. So, the subject of the sentence and the actor are the same.

\subsubsection{Indicative mood}

Verbs in indicative\index{mood!indicative} mood can be found in every tense that Novoslovnica possesses. Let us look at some tables with verb of different verbal types conjugation.

\subsubsection{Present Tenses}

\begin{table}[!htb]
	\caption{A-type conjugation in Present Indefinite}
	\begin{tabular}{llll}
		Present Indefinite & Singular & Dual & Plural \\
		1 person & -am & -ama & -ame \\
		2 person & -aš & -ata & -ate \\
		3 person & -a & -at & -ut
	\end{tabular}
\end{table}


\begin{table}[!htb]
	\caption{A-type conjugation in Present Definite}
	\begin{tabular}{llll}
		Present Definite & Singular & Dual & Plural \\
		1 person & -aju & -ajema & -ajeme \\
		2 person & -aješ & -ajeta & -ajete \\
		3 person & -aje & -ajat & -ajut
	\end{tabular}
\end{table}

\begin{table}[!htb]
	\caption{E-type conjugation in Present Indefinite}
	\begin{tabular}{llll}
		Present Indefinite & Singular & Dual & Plural \\
		1 person & -em & -ema & -eme \\
		2 person & -eš & -eta & -ete \\
		3 person & -e & -at & -ut
	\end{tabular}
\end{table}


\begin{table}[!htb]
	\caption{E-type conjugation in Present Definite}
	\begin{tabular}{llll}
		Present Definite & Singular & Dual & Plural \\
		1 person & -u & -ujema & -ujeme \\
		2 person & -uješ & -ujeta & -ujete \\
		3 person & -uje & -ujat & -ujut
	\end{tabular}
\end{table}


\begin{table}[!htb]
	\caption{I-type conjugation in Present Indefinite}
	\begin{tabular}{llll}
		Present Indefinite & Singular & Dual & Plural \\
		1 person & -im & -ima & -ime \\
		2 person & -iš & -ita & -ite \\
		3 person & -i & -at & -ut
	\end{tabular}
\end{table}


\begin{table}[!htb]
	\caption{I-type conjugation in Present Definite}
	\begin{tabular}{llll}
		Present Definite & Singular & Dual & Plural \\
		1 person & -ujim & -ujima & -ujime \\
		2 person & -ujiš & -ujita & -ujite \\
		3 person & -uji & -ujat & -ujut
	\end{tabular}
\end{table}


\begin{table}[!htb]
	\caption{U-type conjugation in Present Indefinite}
	\begin{tabular}{llll}
		Present Indefinite & Singular & Dual & Plural \\
		1 person & -nam & -nama & -name \\
		2 person & -naš & -nata & -nate \\
		3 person & -na & -nat & -nut
	\end{tabular}
\end{table}


\begin{table}[!htb]
	\caption{U-type conjugation in Present Definite}
	\begin{tabular}{llll}
		Present Definite & Singular & Dual & Plural \\
		1 person & - & - & - \\
		2 person & - & - & - \\
		3 person & - & - & -
	\end{tabular}
\end{table}

\begin{table}[!htb]
	\caption{Extra-type conjugation in Present Indefinite}
	\begin{tabular}{llll}
		Present Indefinite & Singular & Dual & Plural \\
		1 person & -ovam & -ovama & -ovame \\
		2 person & -ovaš & -ovata & -ovate \\
		3 person & -ova & -ovat & -ovut
	\end{tabular}
\end{table}


\begin{table}[!htb]
	\caption{Extra-type conjugation in Present Definite}
	\begin{tabular}{llll}
		Present Definite & Singular & Dual & Plural \\
		1 person & -uju & -ujema & -ujeme \\
		2 person & -uješ & -ujeta & -ujete \\
		3 person & -uje & -ujat & -ujut
	\end{tabular}
\end{table}

\begin{table}[!htb]
	\caption{The verb "Byti" (Exception)}
	\begin{tabular}{lllllll}
		Pr. Indef.
			& \multicolumn{2}{c}{Singular}
			 & \multicolumn{2}{c}{Dual}
			 & \multicolumn{2}{c}{Plural} \\
		1 person & Jesòm & Sòm & Jesma & Sma & Jesme & Sme \\
		2 person & Jesi & Si & Jesta & Sta & Jeste & Ste \\
		3 person & Jestj & Je & Jesų & Sų & Jesu & Su
	\end{tabular}
\end{table}

\subsubsection{Future Tenses}

\textbf{Future Definite Tense (Bųdešt čas)}

The Future Definite Tense is formed with the following:

\begin{itemize}
	\item Using HTE (3p. sg. of the verb "hteti" (to will)) + the verb in Present Definite or Indefinite Tense
	\item Using the future form of the verb "byti" + infinitive
	\item Using single-form conjugation 
\end{itemize}

The second variant semantically is close to English Future Continuous Tense. However, they both can be used to indicate the future action (see more details in the Tense section).

\begin{table}[!htb]
	\caption{Future single-form conjugation}
	\begin{tabular}{llll}
		Future
		& Singular
		& Dual
		& Plural \\
		1 person & -ahtem & -ahtema & ahteme \\
		2 person & -ahteš & -ahteta & -ahtete \\
		3 person & -ahte & -ahtat & -ahtut
	\end{tabular}
\end{table}

\begin{table}[!htb]
	\caption{The verb "Byti" future conjugation (Exception)}
	\begin{tabular}{llll}
		Future
		& Singular
		& Dual
		& Plural \\
		1 person & Bųdu & Bųdema & Bųdeme \\
		2 person & Bųdeš & Bųdeta & Bųdete \\
		3 person & Bųde & Bųdat & Bųdut
	\end{tabular}
\end{table}

\textbf{Pre-Future Tense}

\begin{table}[!htb]
	\caption{The verb "Byti" pre-future conjugation (Exception)}
	\begin{tabular}{llll}
		Future
		& Singular
		& Dual
		& Plural \\
		1 person & Bųdeh & Bųdehma & Bųdehme \\
		2 person & Bųdeša & Bųdehta & Bųdehte \\
		3 person & Bųdeše & Bųdeha & Bųdehu
	\end{tabular}
\end{table}

\textbf{Future Indefinite Tense}

\subsubsection{Past Tenses}

\begin{table}[!htb]
	\caption{A-type conjugation in Aorist}
	\begin{tabular}{llll}
		Present Definite & Singular & Dual & Plural \\
		1 person & -ah & -ahma & -ahme \\
		2 person & -aša & -ahta & -ahte \\
		3 person & -aše & -aha & -ahu
	\end{tabular}
\end{table}

\begin{table}[!htb]
	\caption{E-type conjugation in Aorist}
	\begin{tabular}{llll}
		Present Definite & Singular & Dual & Plural \\
		1 person & -eh & -ehma & -ehme \\
		2 person & -eša & -ehta & -ehte \\
		3 person & -eše & -eha & -ehu
	\end{tabular}
\end{table}

\begin{table}[!htb]
	\caption{I-type conjugation in Aorist}
	\begin{tabular}{llll}
		Present Definite & Singular & Dual & Plural \\
		1 person & -ih & -ihma & -ihme \\
		2 person & -iša & -ihta & -ihte \\
		3 person & -iše & -iha & -ihu
	\end{tabular}
\end{table}

\begin{table}[!htb]
	\caption{U-type conjugation in Aorist}
	\begin{tabular}{llll}
		Present Definite & Singular & Dual & Plural \\
		1 person & -uh & -uhma & -uhme \\
		2 person & -uša & -uhta & -uhte \\
		3 person & -uše & -uha & -uhu
	\end{tabular}
\end{table}

\begin{table}[!htb]
	\caption{Extra-type conjugation in Aorist}
	\begin{tabular}{llll}
		Present Definite & Singular & Dual & Plural \\
		1 person & -ovah & -ovahma & -ovahme \\
		2 person & -ovaša & -ovahta & -ovahte \\
		3 person & -ovaše & -ovaha & -ovahu
	\end{tabular}
\end{table}

\begin{table}[!htb]
	\caption{The verb "byti" conjugation in Aorist}
	\begin{tabular}{llll}
		Present Definite & Singular & Dual & Plural \\
		1 person & byh & byhma & byhme \\
		2 person & byša & byhta & byhte \\
		3 person & byše & byha & byhu 
	\end{tabular}
\end{table}

\begin{table}[!htb]
	\caption{Сonjugation in Imperfect}
	\begin{tabular}{llll}
		Present Definite & Singular & Dual & Plural \\
		1 person & -ěh & -ěhma & -ěhme \\
		2 person & -ěša & -ěhta & -ěhte \\
		3 person & -ěše & -ěha & -ěhu
	\end{tabular}
\end{table}

\begin{table}[!htb]
	\caption{Сonjugation in Perfect}
	\begin{tabular}{llll}
		Present Definite & Singular & Dual & Plural \\
		1 person & sòm + PPP & jesma + PPP & jesme + PPP \\
		2 person & si + PPP & jesta + PPP & jeste  + PPP \\
		3 person & je + PPP & jesų + PPP & jesu + PPP 
	\end{tabular}
\end{table}

\begin{table}[!htb]
	\caption{Conjugation in Plusquamperfect}
	\begin{tabular}{llll}
		Present Definite & Singular & Dual & Plural \\
		1 person & byh + PPP & byhma + PPP & byhme + PPP \\
		2 person & byša + PPP & byhta + PPP & byhte  + PPP \\
		3 person & byše + PPP & byha + PPP & byhu + PPP 
	\end{tabular}
\end{table}

\subsubsection{Future-in-the-Past Tenses}

\textbf{Future-in-the-Past}

\textbf{Pre-Future-in-the-Past}

\subsubsection{Subjunctive mood}

Subjunctive\index{mood!subjunctive} mood shows two states of an actions. In the first state it can show a desirable action. In the second one it can show an action that has not become a real one. 

Subjunctive mood has only one-tense form. It is constructed with the verb “byti” in a subjunctive form with an aorist or imperfect participle. Special forms of the verb “byti” you can see in the next table.

\begin{table}[!htb]
	\begin{tabular}{llll}
		Subjunctive mood & Singular & Dual & Plural \\
		1 person & Bih & Bihma & Bihme \\
		2 person & Biša & Bihta & Bihte \\
		3 person & Biše & Biha & Bihu
	\end{tabular}
\end{table}

\subsubsection{Imperative mood}

Imperative\index{mood!imperative} mood shows that the actor make somebody do some action (imperate another object). Imperative mood also has only one tense to be used in. 

\begin{table}[!htb]
	\caption{A-type}
	\begin{tabular}{llll}
		Imperative mood & Singular & Dual & Plural \\
		1 person &  & -aǐma & -aǐmo \\
		2 person & -aǐ & -aǐta & -aǐte \\
		3 person &  &  & 
	\end{tabular}
\end{table}



\begin{table}[!htb]
	\caption{E-type}
	\begin{tabular}{llll}
		Imperative mood & Singular & Dual & Plural \\
		1 person &  & -eǐma & -eǐmo \\
		2 person & -i & -eǐta & -eǐte \\
		3 person &  &  & 
	\end{tabular}
\end{table}



\begin{table}[!htb]
	\caption{I-Type}
	\begin{tabular}{llll}
		Imperative mood & Singular & Dual & Plural \\
		1 person &  & -iǐma & -iǐmo \\
		2 person & -i & -iǐta & -iǐte \\
		3 person &  &  & 
	\end{tabular}
\end{table}

\begin{table}[!htb]
	\caption{U-Type}
	\begin{tabular}{llll}
		Imperative mood & Singular & Dual & Plural \\
		1 person &  & -niǐma & -niǐmo \\
		2 person & -ni & -niǐta & -niǐte \\
		3 person &  &  & 
	\end{tabular}
\end{table}


\begin{table}[!htb]
	\caption{Extra-Type}
	\begin{tabular}{llll}
		Imperative mood & Singular & Dual & Plural \\
		1 person &  & -uǐǐma & -uǐǐmo \\
		2 person & -uǐ & -uǐǐta & -uǐte \\
		3 person &  &  & 
	\end{tabular}
\end{table}

\subsubsection{Inferential mood}

Inferential\index{mood!inferential} mood is used when we speak about actions that we did not observe by ourselves. This mood mostly is used in tales, histories two show that we were not witnesses of what we are speaking about.

Inferential mood is created by using the verb “ïmáti’ with the past passive participle (PPP). This mood can be used in all tenses of the indicative mood (we should only place the verb “ïmáti” into this tense and add a PPP of the main verb to it). 

\textbf{Examples:}

\textit{Ja sòm rođen svojoǐ mamoǐ} - I am born by my own mother. (This is a real fact, but the actor (Mother) is not at the subject place in the sentence - Passive Voice).

\textit{Ona má rođeno ove rebętko.} - Some say she bore this child (This is a rumor, and the actor is in the place of a subject - Inferential mood of AV)

\subsection{Passive voice}

Passive\index{voice!passive} voice shows that the person is an object for the action. That means, that the subject of the sentence semantically is not an actor, but the object. 

Passive voice has only two moods - Indicative and Subjunctive. Passive voice is created by using the verb “byti” with the PPP. Here is the narrow border between Inferential mood of Active Voice and Indicative mood of Passive Voice.

We can describe the difference between these two terms in such a way. If we speak about real actions that we have imagined or we heard about them (however, the actor of these sentences is placed in the subject) - we use Inferential mood. And if we speak about any action that has no actor in the subject syntax role - we use Passive Voice. Let’s look at the examples.


\subsection{Infinitive and Supine}

Infinitive\index{infinitive} is a normal main form of the verb. In fact all languages that have an infinitive form of the verb use it in that way. So when you look through the dictionary looking for a verb, you should remember that it will stand in infinitive. Infinitive form is created by adding the ending “-ti” to the end of the verb. (Compare with English, you put the particle “to” before the verb to create the same form - there is some similarity in both cases). Infinitive has no parameters for declination.

Infinitive is a form to describe a verb as a process or an action itself.

The second unchangeable form of the verb is supine. It has no equivalent in English. Semantically, it is similar to the construction “to be going to do something”, determining the aims of the subject. Supine in Novoslovnica is built by adding the “-tj” ending to the end of the verb.

Supine\index{supine} had a great usage area in the past, now it is still used in Uppersorbian, Lowersorbian and Slovenian languages. It is mostly used with the verbs of motion, such as “to go”, “to swim”, “to move”, “to fly” etc.

\textbf{Examples:}

\textit{Ty hteš kazati mi něčto, da li?} - You want to say me something, don’t you? (Infinitive)

\textit{Mamo, ja idaju spatj dnesj po-rano.} - Mom, I’m going to sleep now earlier. (Supine)



