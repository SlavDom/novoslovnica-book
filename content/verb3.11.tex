\section{Verb}

Verb\index{verb} is a POS that describes the predicate and its properties. 


\subsection{Verbal types}

Learning Slavic languages you could mention that there is a set of verb suffixes that is very eliminated. Novoslovnica provides a theory that allows you to construct a right verb form.
Verbs with different verb suffixes represent different verbal types. There are four types of verbs in Novoslovnica.

\begin{itemize}
	\item A-type\index{verb!a-type}: verbs of this type define that the action in common sense.  
	\item E-type\index{verb!e-type}: verbs of this type define that the action is long-termed.
	\item I-type\index{verb!i-type}: verbs of this type define that the action is short-termed. This type comprises verbs with suffixes I and O. The suffix O is used when the consonant before the constructed suffix is involved in alterations depending on soft vowels that the vowel I is.
	\item U-type\index{verb!u-type}: verbs of this type define that the action is dotty.
	\item Extra type\index{verb!extra-type}. Is formed with the suffix “-OVA-” and defines the repeated action.  
\end{itemize}

When you speak about the action, you find what characteristic is suitable for the action and then use one of the predefined verbal types.

Somebody can ask what are the differences in tenses and types, because it might be confusing. However, verbal type determines the durability of the action (or its repeating property) while tense determines tense characteristic of the action such as completeness, stability in time, result, order etc.

All tenses provide difference between conjugation of different verbal types except imperfect. This tense has the common conjugation table for all verbal types. 

Further we will speak about conjugation itself. We will look at verb conjugation in indicative mood first of all. Then we will speak about other moods.

\subsection{Active voice}

Active\index{voice!active} voice shows that the person makes the action by himself. So, the subject of the sentence and the actor are the same.

\subsubsection{Indicative mood}

Verbs in indicative\index{mood!indicative} mood can be found in every tense that Novoslovnica possesses. Let us look at some tables with verb of different verbal types conjugation.

\subsubsection{Present Tenses}

There are two present tenses in Novoslovnica - Present Indefinite and Present Definite (see the chapter about tenses). In the appendix you can see the conjugations of five verbal types in Novoslovnica. The only exception is the verb "byti" that has a unique type of conjugation.

As you can see in the table 8.26 the verb "byti" has two forms - a full one and a short one. Usually, the full form is used separately without a NP with it. Using a subject with a verb leads to short form using. Look at the examples:

\textbf{Examples:}

\textit{Jesòm iz Moskvy} - I am from Moscow (no pronoun)

\textit{Ja sòm iz Moskvy} - I am from Moscow (with a pronoun) 

\subsubsection{Future Tenses}

\textbf{Future Definite Tense (Bųdešt čas)}

The Future Definite Tense is formed with the following:

\begin{itemize}
	\item Using HTE (3p. sg. of the verb "hteti" (to will)) + the verb in Present Definite or Indefinite Tense
	\item Using the future form of the verb "byti" + infinitive
	\item Using single-form conjugation 
\end{itemize}

The second variant semantically is close to English Future Continuous Tense. However, they both can be used to indicate the future action (see more details in the Tense section).

The verb "byti" also has a unique type of conjugation in Future Definite Tense.

\textbf{Pre-Future Tense}

This tense is form doubly: 

\begin{itemize}
	\item Using \textit{hte} with Perfect tense form of the main verb.
	\item Using the verb "byti" in pre-future form with the infinitive of the main verb.
\end{itemize}

The conjugation of the verb "byti" in pre-future tense you can find in the table in the appendix.

\textbf{Future Indefinite Tense}

Future Indefinite Tense is formed by \textit{hte} along with the Plusquamperfect main verb form.

\subsubsection{Past Tenses}

There are four past tenses in Novoslovnica (see the "tense" chapter). Though, only one of them distinguishes by verbal types (Aorist). Imperfect has a common conjugation for all verbal types while Perfect and Plusquamperfect have complex verb phrases.

The verb "byti" has also a separate declension in Aorist.

\subsubsection{Future-in-the-Past Tenses}

\textbf{Future-in-the-Past}

This tense is formed by the verb "hteti" in Imperfect with DA-form of the main verb.

\textbf{Pre-Future-in-the-Past}

This tense is formed by the verb "hteti" in Imperfect with perfect DA-form of the main verb.

\subsubsection{Declarative mood}

Declarative mood of the verb has all tenses existing in Indicative. It is formed using the verb "imati" (to have) with the corresponding passive participle (\gls{ppp}) of the main verb.

Look for the corresponding tables in the eighth chapter to see the conjugation in Declarative mood in Present Indefinite, Present Definite and Aorist.

All other tenses are formed equally to Indicative for the verb "ïmáti" (to have) with addition of \gls{ppp} of the main verb.

\subsubsection{Subjunctive mood}

Subjunctive\index{mood!subjunctive} mood shows two states of an actions. In the first state it can show a desirable action. In the second one it can show an action that has not become a real one. 

Subjunctive mood has only one-tense form. It is constructed with the verb “byti” in a subjunctive form with an aorist or imperfect participle. Special forms of the verb “byti” you can see in the next table.

\begin{table}[!htb]
	\begin{tabular}{llll}
		Subjunctive mood & Singular & Dual & Plural \\
		1 person & Bih & Bihma & Bihme \\
		2 person & Biša & Bihta & Bihte \\
		3 person & Biše & Biha & Bihu
	\end{tabular}
\end{table}

\subsubsection{Imperative mood}

Imperative\index{mood!imperative} mood shows that the actor make somebody do some action (imperate another object). Imperative mood also has only one tense to be used in. 

\subsubsection{Inferential mood}

Inferential\index{mood!inferential} mood is used when we speak about actions that we did not observe by ourselves. This mood mostly is used in tales, histories two show that we were not witnesses of what we are speaking about.

\newglossaryentry{ppp}{name=PPP, description={Past Passive Particle}}

Inferential mood is created by using the verb “ïmáti’ with the past passive participle (\gls{ppp}). This mood can be used in all tenses of the indicative mood (we should only place the verb “ïmáti” into this tense and add a PPP of the main verb to it). 

\textbf{Examples:}

\textit{Ja sòm rođen svojoǐ mamoǐ} - I am born by my own mother. (This is a real fact, but the actor (Mother) is not at the subject place in the sentence - Passive Voice).

\textit{Ona má rođeno ove rebętko.} - Some say she bore this child (This is a rumor, and the actor is in the place of a subject - Inferential mood of AV)

\subsection{Reflexive voice}

Reflexive voice has an equivalent amount of moods and tenses to be expressed with. The difference is in using the reflexive pronoun next to the word. Usually, the reflexive mark is placed right before the main verb in the verbal phrase.

Look at the example to see it:

\textit{Ja hoču \textbf{sę učiti} anĝliǐskomu jazyku} - I want to study English.

\textit{On hte \textbf{sę dojedna} do toga ruha} - He will join that movement.

\subsection{Passive voice}

\newglossaryentry{pv}{name=PV, description={Passive Voice}}

Passive\index{voice!passive} (\gls{pv}) voice shows that the person is an object for the action. That means, that the subject of the sentence semantically is not an actor, but the object. 

Passive voice has only two moods - Indicative and Subjunctive. Passive voice is created by using the verb “byti” with the \gls{ppp}. Here is the narrow border between Inferential mood of Active Voice and Indicative mood of Passive Voice.

We can describe the difference between these two terms in such a way. If we speak about real actions that we have imagined or we heard about them (however, the actor of these sentences is placed in the subject) - we use Inferential mood. And if we speak about any action that has no actor in the subject syntax role - we use Passive Voice. Let’s look at the examples.

\textbf{Examples:}

- \textit{Moja råbota je byla ocěnena dobro.} - My work was truly appreciated.

- \textit{Kniga je napisana vëlïkym tvorcom.} - The book is written by a great author.

\subsection{Infinitive and Supine}

Infinitive\index{infinitive} is a normal main form of the verb. In fact all languages that have an infinitive form of the verb use it in that way. So when you look through the dictionary looking for a verb, you should remember that it will stand in infinitive. Infinitive form is created by adding the ending “-ti” to the end of the verb. (Compare with English, you put the particle “to” before the verb to create the same form - there is some similarity in both cases). Infinitive has no parameters for declination.

Infinitive is a form to describe a verb as a process or an action itself.

The second unchangeable form of the verb is supine. It has no equivalent in English. Semantically, it is similar to the construction “to be going to do something”, determining the aims of the subject. Supine in Novoslovnica is built by adding the “-tj” ending to the end of the verb.

Supine\index{supine} had a great usage area in the past, now it is still used in Uppersorbian, Lowersorbian and Slovenian languages. It is mostly used with the verbs of motion, such as “to go”, “to swim”, “to move”, “to fly” etc.

\textbf{Examples:}

\textit{Ty hteš kazati mi něčto, da li?} - You want to say me something, don’t you? (Infinitive)

\textit{Mamo, ja idaju spatj dnesj po-rano.} - Mom, I’m going to sleep now earlier. (Supine)

\section{Transgressive}

Transgressive is a verbal independent POS, however some scientist suppose that it is only a form of a verb. so as infinitive and supine.
Transgressive cannot be declined, it has verbal characteristics (i.e. forms) but it is derived from participle with its suffixes. Let us look at the table with transgressive forms.

Imperfect form
Perfect form
Suffix
%-ја-(-чі-) || -јучі
%-в-(-шы)

We see that there are no endings here, that means transgressive cannot be declined. Perfect form determines the action that is completed (in past, present or future). Imperfect form determines the action is being done (in past, present or future). This fact differ transgressives from participles, because, as you can see, participles are divided by tenses, not by forms. However, transgressive is “over” the tenses. 

I should say that a consonant “j” in imperfect form transforms into a soft symbol after consonants. 

There are recommendation about using different forms of transgressive. Perfect forms with “-v-” are used when we speak about actions in the past (Aorist), while forms with “-všy-” are used when we speak about resultative action by the present moment (Perfect). Imperfect full forms with “-jačï-” are less preferred than with “-jučï-”. Moreover, these full forms are recommended to be used as a reflection of actions in the past (Imperfect), while short form of “-ja-” should be used as a reflection of actions in Present Concrete tense.

Transgressive can be replaced with a relative clause in the sentence with the relation “when”. Look at the examples:
% Казавшы му нѣчто, той изидаше из дома.
% Коґда той казал быше му нѣчто, он изидаше из дома.


