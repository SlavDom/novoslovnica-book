\section{Verb}

Verb is a POS that describes the predicate and its properties. 


Verbal types
Learning Slavic languages you could mention that there is a set of verb suffixes that is very eliminated. Novoslovnica provides a theory that allows you to construct a right verb form.
Verbs with different verb suffixes represent different verbal types. There are four types of verbs in Novoslovnica.

A-type: verbs of this type define that the action in common sense.  
E-type: verbs of this type define that the action is long-termed.
I-type: verbs of this type define that the action is short-termed. This type comprises verbs with suffixes I and O. The suffix O is used when the consonant before the constructed suffix is involved in alterations depending on soft vowels that the vowel I is.
U-type: verbs of this type define that the action is dotty.
Extra type. Is formed with the suffix “-OVA-” and defines the repeated action.  

When you speak about the action, you find what characteristic is suitable for the action and then use one of the predefined verbal types.

Some can ask what are the differences in tenses and types, because it might confuse you. However, verbal type determines the durability of the action (or its repeating property) while tense determine tense characteristic of the action such as completeness, stability in time, result, order etc.

All tenses provides difference between conjugation of different verbal types except imperfect. This tense have the common conjugation table for all verbal types. 

Further we will speak about conjugation itself. Be ready for a lot of tables! We will look at verb conjugation in indicative mood first of all. Then we will speak about other moods.

Active voice
Active voice shows that the person makes the action by himself. So, the subject of the sentence and the actor are the same.
Indicative mood
Verbs in indicative mood can be found in every tense that novoslovnica possesses. Let us look at some tables with verb of different verbal types conjugation.
Present Tenses
A-type conjugation
