\section{Vowels}

In the beginning of this paragraph the description of a vowel should be given.

A \textit{vowel} - a sound produced with no constriction in the vocal tract.

With this information we can distinguish different types of vowels. The classification of vowels is based on two main factors:

\begin{itemize}
	\item{What is the row of the sound}
	\item{What is the height of the sound}
	\item{Whether the vowel is rounded or not}
\end{itemize}

The \textbf{row} is the position of the tongue when you pronounce a vowel. There are three main rows: front, central and back. When you pronounce a vowel at the front row, you move your tongue toward the teeth. The descriptions of central and back rows are similar - you move your tongue to the center or toward the larynx to pronounce them.

The \textbf{height} of the sound is a characteristic of tongue convexity and tension in your mouth. If it is positive, it means your tongue does not touches the palate nor bottom of the mouth, and the tip of your tongue is tense- or closed. If your tongue is flat and parallel to the bottom of your oral cavity, moreover, it lies on it - it is an open one. Between these two positions a middle position can be found.

The \textbf{roundedness} of the vowel is the amount of rounding in the lips during the articulation of a vowel. Vowels can be categorized as rounded and unrounded. Thus, to pronounce a rounded vowel you should round your lips. To pronounce an unrounded vowel, you should relax your lips during the articulation of a vowel.

Finally, we can talk about Novoslovnica phonology. Novoslovnica consists of 20 ordinary vowel phonemes. Among them there are seven closed vowels, three open vowels, ten middle vowels as well as seven front vowels, five center vowels and eight back vowels. On the table 1.1 you can see a chart position of the vowels in Novoslovnica.

\begin{table}[h]
\caption{Vowels in Novoslovnica}
	\begin{tabular}{lllll}
		& Front & Center & Near-back & Back \\
		Close      &  \textipa{i}          & \textipa{1|\textbf{0}} &                      &  \textipa{W|\textbf{u}} \\
		Near-close &  \textipa{\textbf{I}} &                        & \textipa{\textbf{U}} &                         \\
		Close-mid  &  \textipa{e}          & \textipa{\textbf{8}}   &                      &   \textipa{\textbf{o}}         \\
		Mid        &  \textipa{\|`e}       & \textipa{@}            &                      &  \textipa{\textbf{\|`o}} \\
		Open-mid   &  \textipa{E}          & \textipa{3}            &                      &   \textipa{2|\textbf{O}}    \\
		Near-open  &  \textipa{\ae}        &                        &                      &                 \\
		Open       &  \textipa{a}          &                        &                      &   \textipa{A} 
	\end{tabular}
\end{table}

In this table you can also see different vowels are font-styled differently. The bold ones are for rounded vowels. The normal ones are for unrounded vowels.

If you know Czech or Finnish, you might be concerned by the absence long vowels in this chart. It’s time to speak about allophones in Novoslovnica.

Novoslovnica has allophones of open and long vowels. This means that it does not matter how you pronounce a “modified” vowel - as a long one or as an open one - the meaning of the word will not change. To make this more clear, look at table 1.2.

\begin{table}[h]
	\caption{Long vowel allophones in Novoslovnica}
	\begin{tabular}{lll}
  \multirow{2}{*}{Main vowel} & \multicolumn{2}{c}{Modified vowel} \\
  	&	First allophone & Second allophone \\
  \textipa{E} & \textipa{E:} & \textipa{3} \\
  \textipa{a} & \textipa{a:} & \textipa{A} \\
  \textipa{u} & \textipa{u:} & \textipa{U} \\
  \textipa{o} & \textipa{o:} & \textipa{O} \\
  \textipa{\|`o} & - & \textipa{W}
	\end{tabular}
\end{table}

\textbf{Exception:}

\textipa{\v*{o}} (only one modified vowel - \textipa{W})

\textbf{Examples:}

Buda \textipa{[‘buda]} (Buddha) - búda \textipa{[‘bu:da]} (building) - \textipa{[‘bUda]}

If you have some knowledge about Slavic languages and their origins, you should know that the Proto-Slavic language had nasal vowels, which we can nowadays be found in Polish and Lithuanian. Do they exist in Novoslovnica? Of course they do! There are two allophones for pronouncing nasal vowels. The first one is actually a nasal vowel, when you pronounce an ordinary vowel through your nose. The second is more common, such as in French, when you add a nasal consonant \textipa{[N]} to an ordinary vowel. Look at the sounds on table 1.3

\begin{table}[h]
	\caption{Nasal vowel allophones in Novoslovnica}
	\begin{tabular}{lll}
  \multirow{2}{*}{Ordinary vowel} & \multicolumn{2}{c}{Nasal vowel} \\
  &	First allophone & Second allophone \\
  \textipa{E} & \textipa{\~E} & \textipa{EN} \\
  \textipa{i} & \textipa{\~E} & \textipa{iN} \\
  \textipa{a} & \textipa{\~a} & \textipa{aN} \\
  \textipa{u} & \textipa{\~u} & \textipa{uN} \\
  \textipa{o} & \textipa{\~o} & \textipa{oN}
	\end{tabular}
\end{table}

\textbf{Examples:}

D\textipa{\k{u}b} (oak) \textipa{[d\~{O}b]} - \textipa{[duNb]}

M\textipa{\k{e}so} (meet)\textipa{[‘m\~{E}so]} - \textipa{[`meNso]}

As you can see, Novoslovnica distinguishes between nasal vowels in two categories - O-nasality (hard) and E-nasality (soft). 

The last topic that we will speak of pertaining to vowels, is the firmness and the softness of the vowels. Scientists argue about what is primary in making sound soft or hard - consonants or vowels. Novoslovnica claim vowels are softer or hard primarily, although consonants itself also can be either soft or hard.

There are vowels that tend to make their surrounding soft and there are vowels that tend to make their surrounding hard. As you already know, there are two nasal vowels - one hard (O-nasality) and one soft (E-nasality). However, there are other pairs of hard-soft vowels among ordinary vowels. Look at table 1.4 for more information.


\begin{table}
	\caption{Hard and soft vowels}
	\begin{tabular}{ll}
		Hard vowel & Soft vowel \\
		\textipa{\~o} & \textipa{\~E} \\
		\textipa{u} & \textipa{0} \\
		\textipa{1} & \textipa{i} \\
		\textipa{E} & \textipa{\|`e} \\
		\textipa{o} & \textipa{8} \\
		\textipa{a} & \textipa{\ae} \\
		\textipa{I} & \textipa{e}
	\end{tabular}
\end{table}

\textbf{Examples:}

Dodatek (addition) [do’datek] - Sm"ertj (death) \textipa{[sm\t{e}rt’]}

It should be mentioned that the sounds \textsc and e that are shown in italic are the allophones for one phoneme. However, both these sounds exist in Novoslovnica and you can use whatever you like.

There is only one vowel that has no pair in softness/hardness. It is \textipa{@}. This vowel is named the “Schwa” sound and it can be described as the most “middle” sound among vowels. To pronounce it you should relax your oral cavity and pronounce a sound with weakened muscles. This is the “Schwa” sound.



