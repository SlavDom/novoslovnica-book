\section{Case}

Case\index{case} is a grammatical property of a nominal POS (Part of speech) that shows what references this nominal POS has with other words in a sentence (phrase). This property is widely known in fusional languages, while analytical languages do not often possess this property. Thus, English has only two active cases - Nominative and Oblique one. Moreover, oblique case is used practically only within pronouns while nouns have no such a case. That means case is not the only way to show references between nominal POS and other words in a sentence. Case is one of the ways to show it and Slavic languages as being fusional widely use this grammar category.

Different Slavic languages have different number of cases. For example, Russian language has six cases while Serbian language has seven. We can find exceptions in Bulgarian and Macedonian languages, which are analytical that is why they have only one case for a noun and adjective and three cases for a pronoun.

Different cases are referred to different semantic links between words. It is the cause why we see ambiguity of cases in different languages (that have different amount of cases and different usage rules of cases). Novoslovnica provides most common and wide means to use cases with almost full determination. When you speak Novoslovnica you have to use the case of exact semantic value and not of the longstanding phraseology of your own language.

With this principle Novoslovnica establishes nine cases. Nine changing patterns that determine alterations of all words of nominal POS. This is the unification of Slavic languages in the sphrere of fusial word linking. Here they are:

\begin{itemize}
	\item Nominative
	\item Genitive
	\item Partitive
	\item Dative
	\item Accusative
	\item Instrumental
	\item Prepositional
	\item Locative
	\item Vocative
\end{itemize}

In this chapter we will speak about cases in general. All examples will disclose case features through nouns as examples.

Nominative\index{case!nominative} case (\textit{Imenóvnik}) is used when we are talking about a concept as an actor. If the sentence is full, the subject is in Nominative. You can ask questions like «Who? What?» to it.

This case is basic in most languages, so POSes in this case are supposed to be in the normal form (that we can find in a dictionary). In Novoslovnica nominative also determines a normal from of the word. In the examples you can see full sentenses, where subject is used in nominative.

\textbf{Examples:}

\textit{Dom-òt je vëlïkym.} - The house is big.

\textit{Izučilišto, de ja sę učam, je starym}. - The school I attend is old.

\textit{Klüč-òt je od ovoŭ vråtoŭ.} - The key is to these doors.

Genitive\index{case!genitive} case (\textit{Čyǐnik}) is used when we are talking to an object being related to another one. Thus this case show what generation the object is of and what is it made from or whom does it belong to. The questions that determine the case are «Whose? Which? What?».

In Novoslovnica possessive case equals to genitive one, so English «'s» constructions should be translated in genitive (example 1). Further, genitive in Novoslovnica could be related to usage of nouns with «of» preposition (example 2, 3).

\textbf{Examples:}

\textit{Kniga brata je vëlïmi zajimliva.} — My brother's book is very fascinating.

\textit{Cěna uspěha je mnogo vëlïka.} — The price of success is very high.

\textit{Sklad je na koncu ulicy-ta.} — The shop is at the end of the street.

Partitive\index{case!partitive} case (\textit{Ličóvnik}) is used when we are talking about some amount of object having or being supposed to have uncountable properties. The questions that determine the case are «Of what? With what? How much of?».

This case has many coinciding forms with genitive, though in masculine gender it has another endings (examples 1, 3). In English it should be translated with «of» construction (example 1). If uncountable nouns are used with the predicate directly or with adverbs of measure, they should be translated into Oblique case in English (example 2, 3).

\textbf{Examples:}

\textit{Daǐ mi čašku čaju.} — Give me a cup of tea.

\textit{Dodaǐ do pïroga němnogo vody.} — Add some water to a cake.

\textit{Predaǐ mi cukru.} — Pass me sugar.

Dative case (\textit{Datelnik}) is used when we are talking about a noun to which something is given. We can ask a question for a word in this case as «Whom? For whom?».

It's simple with pronouns, because there is a dative case within pronouns in English (example 1). In some use cases we can find a noun with «for» preposition to be translated into Novoslovnica's dative (example 2). However, in these cases form with «dlä» preposition with genitive can be used instead (example 3).

However, there are cases when some direct objects in English will be translated with dative in Novoslovnica (example 4), so you need to consider the semantic value of dative — give somebody something.

\textbf{Examples:}

\textit{Kaži mi, čto ty hteš da dostęžiš ovym.} — Tell me what do you want to achieve with this.

\textit{Jesòm stvoril podarek tatě.} — I've made a present for daddy.

\textit{Jesóm stvoril podarek dlä taty.} — I've made a present for daddy.

\textit{Jesòm podal bratu moju pomočj, dy on zapytaše mę o tom.} — I helped my brother, when he asked about it.

Accusative\index{case!accusative} case (\textit{Vinitelnik}) is used to describe a direct object of the action. The questions determining the case are «What? Whom?».

If a noun has a role of a direct object in English sentense (Oblique case), you should translate it with accusative in Novoslovnica (example 1-3).

\textbf{Examples:}

\textit{Ja viđu lěpyǐ lěs predò mnom.} — I see a beautiful forest ahead of me.

\textit{On pokazaše mi kota, ke glåsisto kričěše.} — He showed me a cat, crying loudly.

\textit{Znaš li ty rěku, če tëče z severu na jug?} — Do you know a river that flows from north to south?

Instrumental\index{case!instrumental} case (\textit{Tvornik}) is used to describe an instrument of an action that affects the object of the action. The questions related with this case are: «With whom? With what?».

As you can see from auxiliary questions, English phrases with «with» expressions should be translated to instrumental case in Novoslovnica. Moreover, «by» expressions also are translated to instrumental case. The difference is in preposition:

«with»-expressions are translated in

«s»+instrumental case (keeping preposition) with animate nouns or pronouns (examples 1, 2)
instrumental case (loosing preposition) with inanimate nouns (example 4)
«by»-expressions are translated in instrumental case, loosing preposition (example 3).

\textbf{Examples:}

\textit{Idaǐ na věčôrku sò mnom.} — Let's go to the party with me.

\textit{Kaži mi, s kym ty hteš poǐdati na věčôrku?} — Tell me, with whom do you want to go to the party?

\textit{Kniga-ta je napisana vëlïkym tvorcom.} — The book is written by a great author.

\textit{Ta búda je vybúdana kamenom.} — That building is built with stone.

Prepositional\index{case!prepositional} case (\textit{Predložnik}) is used when something is an object of speaking. It can be related to auxiliary questions «About what? About whom?».

In English there are two prepositions that show that it should be translated to prepositional case in Novoslovnica — «about», «of». Note, that you should divide phrases with the object of speaking from genitive forms.

This case is called prepositional in Novoslovnica, because it is used only with preposition «o» (about).

\textbf{Examples:}

\textit{Råzkaži mi o tom slučajě.} — Tell me about that case.

\textit{On ne zna ničto o tom městě.} — He knows nothing about that place.

\textit{Ja ne kazal sòm mu ničto o sobě.} — I have told nothing to him about myself.

Locative\index{case!locative} case (\textit{Městnik}) is used when we speak about something as a place where the action related with the predicate takes place. It can be related to an auxiliary question: «where?».

In English phrases with both «at» and «in» prepositions should be translated to locative in Novoslovnica (examples 1-3). Note, that «in» preposition is usually translated with locative, while «into» preposition should be translated with accusative (example 4).

\textbf{Examples:}

\textit{Hođu v lěsu.} — I am walking in the forest.

\textit{Ja běše v domu, koĝda ty pozovaše mę izvòn.} — I was at home, when you called me out.

\textit{On živa v golěmomu grådu.} — He lives in a big city.

\textit{Hođu v lěs.} — I am walking into the forest.

Vocative\index{case!vocative} case (\textit{Zvatelnik}) is a special case in Novoslovnica that binds a new object to the action by direct mentioning it.

Vocative is usually used with human names (example 1) or animate nouns (example 2), but can also be used with every noun that is supposed to be a receiver of an action result (example 3).

Vocative is usually divided from the main sentense by a comma.

\textbf{Examples:}

\textit{Ivane, začto ne odgovořaš mi?} — Ivan, why don't you answer me?

\textit{Otče, koĝda hte mi kupiš dvojokol?} — Father, when will you buy me a bike?

\textit{Větre, začto ne sę zaspokojiš?} — The Wind, why don't you calm down?

These cases cover 99,99\% of possible nominal POS declension. Some extra ordinary cases exist, but they should not be mentioned here.
