\section{Case}

Case is a grammatical property of a nominal POS, that shows what references this nominal POS has with other words in a sentence (phrase). This property is widely known in fusional languages, while analytical languages do not often possess this property. Thus English has only two active cases - Nominative and Oblique case. Moreover, oblique case is used practically only within pronouns while nouns have no such case. That means case is not the only way to show references between nominal POS and other words in a sentence. Case is one of the ways to show and Slavic languages as being fusional widely use this grammar category.

Different Slavic languages have different amounts of cases. For example, Russian language has six cases when Serbian language has seven. We can find exceptions in Bulgarian and Macedonian languages, which are analytical, that’s why they have only one case for a noun and adjective and three cases for a pronoun.

Different cases are referred to different semantic links between words. That’s why we see ambiguity of cases in different languages (that have different amount of cases and different usages of cases). Novoslovnica provides most common and wide means to use cases with most determination. When you speak Novoslovnica, you have to use the case of exact semantic value and not of the longstanding phraseology of your own language.

With this principle Novoslovnica establishes nine cases. Nine changing patterns that determine alterations of all words of nominal POS. Here I want to introduce them to you:

\begin{itemize}
	\item Nominative
	\item Genitive
	\item Partitive
	\item Dative
	\item Accusative
	\item Instrumental
	\item Prepositional
	\item Locative
	\item Vocative
\end{itemize}

\textbf{Nominative} case is used when we are talking to a concept as an actor. If the sentence is full, the subject is in Nominative. You can ask questions like “Who? What?” to it.

\textbf{Genitive} case is used when we are talking to an object being related to another one. Thus this case show what generation the object is of and what is it made from or to whom does it belong. The questions that determine the case are “Whose? Which? What?”

\textbf{Partitive} case is used when we are talking about some amount of an object having or being supposed to have uncountable properties. The questions that determine the case are “Of what? With what?”. 

\textbf{Dative} case is used when we are talking about a subject of perception. We can ask a question for a word in this case as “Whom? For whom?”  …. 

\textbf{Accusative} case is used to describe a direct object of the action. The questions determining the case are “What? Whom?”

\textbf{Instrumental} case is used to describe an instrument of an action that affect the object of the action. The questions related with this case are: “With whom? With what?”.

\textbf{Prepositional} case 

\textbf{Locative} case

\textbf{Vocative} case

These cases cover 99,99\% of possible nominal POS declension.  
