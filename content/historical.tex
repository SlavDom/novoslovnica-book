\section{Historical background}

One of the Central ideas of pan-Slavism is the creation of the Slavic language. In one form or another, this problem was solved many times and in different ways, starting with the old Church Slavonic language Svv. brothers Constantine-Cyril and Methodius. With the advent of the first common Slavic language is associated with the appearance of the first writing, which was to become common Slavic — Glagolitic.

But over time, the part of the Slavs adopted the Cyrillic alphabet, the piece took Latin, and some in course of some time was the Glagolitic alphabet, the Latin alphabet and a modified Cyrillic alphabet, known as bosancica or horvatich (there are other names). Glagolitic also changed, becoming from rounded "Bulgarian" angular "Croatian". But the main thing — changed the language itself. Once a single drevnearmyansky began to share on Harry, and Harry began to converge with the living Slavic languages. Of the harass is possible, for example, to call: slavyanorusskogo, Slav, slavyanoserbsky, slavyanobolgarskaya and other.

Attempts to solve the problem of separation of Slavic languages and scripts gave several options:

\begin{itemize}
	\item To revive the Church in one way or another, but adding to it elements of living languages or without adding.
	\item Take one of the living languages, but with certain modifications — the addition of letters, words from other Slavic languages, grammatical structures and so on.
	\item As in the previous paragraph, but do not change anything, and try to spread the living language through educational courses and other forms of education.
	\item As in the previous point, but forcibly.
\end{itemize}

Create a new artificial / semi-artificial language based on those elements of the Slavic languages that remain common or can be relatively easily reduced to common + a number of different elements, to facilitate the "entry" into the language of the Slavs who speak different Slavic languages, or on the basis of only common elements.

The second part of the problem is writing. Due to the fact that the Glagolitic alphabet and bosancica virtually disappeared and remained only in limited use (bosancica disappeared almost completely, and supported the Glagolitic alphabet in Croatia), the remaining candidates for the Slavic alphabet is Cyrillic and Latin.
In General it can be said that novoalekseevsky projects often use the Latin alphabet that cuts obsessiveness projects such as Cyrillic and Latin alphabets still are native to Slavic writing systems: in one case, a literature began almost immediately with the Latin alphabet, the other with the Cyrillic alphabet, somewhere in the course of the Cyrillic and the Latin alphabet from the Cyrillic alphabet there was a transition to the Latin alphabet.

In Yugoslavia, Cyrillic and Latin tried to lead to a common denominator through the creation of the so-called Slavica. The essence of Slavica was to leave the backbone of the Latin alphabet (all the basic letters + letters, the same for the Latin and Cyrillic), replace all the digraphs, letters with diacritics and digraphs with diacritics corresponding Cyrillic letters. In Yugoslavia, this was all the easier to do because the Yugoslav Cyrillic and Latin alphabet are completely identical in composition, being mutual transliteration of each other. However, because of the resistance of nationalists in Croatia, this project failed.