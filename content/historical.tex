\section{Script background}

One of the Central ideas of pan-Slavism is the creation of a common Slavic language. This problem was solved many times in one form or another and in different ways starting with the old Church Slavonic language created by St. brothers Constantine-Cyril and Methodius. The invention of the first common Slavic language is associated with the appearance of the first writing, which was to become common Slavic — Glagolitic.

However, over time, the part of the Slavs adopted the Cyrillic alphabet, the part took Latin, and some for a while had the Glagolitic alphabet, the Latin alphabet and a modified Cyrillic alphabet, known as bosancica or horvatica (there are other names). Glagolitic also changed, becoming from rounded "Bulgarian" an angular "Croatian". But the main thing was — the language itself changed. Once a common old Church Slavonic began to divide into version that started to converge with the live Slavic languages. We can name some versions of Church Slavonic: Russian Slavonic, Bulgarian Slavonic, Serbian Slavonic, Croatian Slavonic etc.

Attempts to solve the problem of separation of Slavic languages and scripts gave several options:

\begin{itemize}
	\item To revive the Church Slavonic language anyhow with adding elements of live languages to it or without adding.
	\item Take one of the live languages, but with certain modifications — adding new letters or words from other Slavic languages, new grammatical structures and so on.
	\item As in the previous item, but do not change anything and try to spread the live language through educational courses and other forms of education.
	\item As in the previous point, but forcibly.
	\item Create a new artificial / semi-artificial language based on those elements of the Slavic languages that remain common or can be relatively easily reduced to common + a number of different elements, to facilitate the "entry" into the language of the Slavs who speak different Slavic languages, or on the basis of only common elements.
\end{itemize}

The second part of the problem is writing. Due to the fact that the Glagolitic alphabet and bosancica virtually disappeared and remained only in limited use (bosancica disappeared almost completely, and the Glagolitic alphabet is supported in Croatia), the remaining candidates for the Slavic alphabet stays Cyrillic and Latin.

Generally, it can be said that new interslavic projects often use the Latin alphabet that cuts obsessiveness of these projects, because Cyrillic and Latin alphabets still are native to Slavic writing systems:

\begin{itemize}
	\item in one case, a literature began almost immediately with the Latin alphabet
	\item in the other a literature started with the Cyrillic alphabet
	\item somewhere both Latin and Cyrillic scripts are used
	\item somewhere the transition from Cyrillic to Latin script took place
\end{itemize}

In Yugoslavia, Cyrillic and Latin tried to lead to a common denominator through the creation of the so-called Slavica. The essence of Slavica was to leave the backbone of the Latin alphabet (all the basic letters + letters, the same for the Latin and Cyrillic), replace all the digraphs, letters with diacritics and digraphs with diacritics corresponding Cyrillic letters. In Yugoslavia, this was all the easier to do because the Yugoslav Cyrillic and Latin alphabet are completely identical in composition having a mutual transliteration of each other. However, this project failed because of the resistance of nationalists in Croatia.