\section{Transgressive}

Transgressive is a verbal independent POS, however some scientist suppose that it is only a form of a verb. so as infinitive and supine.

Transgressive cannot be declined, it has verbal characteristics (i.e. forms) but it is derived from participle with its suffixes. Let us look at the table with transgressive forms.

\begin{table}
	\begin{tabular}{lll}
		& Imperfect form & Perfect form \\
		Suffix & -ja (-čï) || -jučï & -v (-šy)
	\end{tabular}
\end{table}

We see that there are no endings here, that means transgressive cannot be declined. Perfect form determines the action that is completed (in past, present or future). Imperfect form determines the action is being done (in past, present or future). This fact differ transgressives from participles, because, as you can see, participles are divided by tenses, not by forms. However, transgressive is “over” the tenses. 

I should say that a consonant “j” in imperfect form transforms into a soft symbol after consonants. 

There are recommendation about using different forms of transgressive. Perfect forms with “-v-” are used when we speak about actions in the past (Aorist), while forms with “-všy-” are used when we speak about resultative action by the present moment (Perfect). Imperfect full forms with “-jačï-” are less preferred than with “-jučï-”. Moreover, these full forms are recommended to be used as a reflection of actions in the past (Imperfect), while short form of “-ja-” should be used as a reflection of actions in Present Concrete tense.

Transgressive can be replaced with a relative clause in the sentence with the relation “when”.

Look at the examples:

\textit{Kazavšy mu něčto, toǐ izidaše iz doma.}

\textit{Koĝda toǐ kazal byše mu něčto, on izidaše iz doma.}
