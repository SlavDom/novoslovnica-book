\section{Reported speech}

There are some ways of describing person’s speech in the sentence. The first way is to use direct\index{speech!direct} speech. Look at the examples:

\textit{I say: “We need to go to the theatre”.} - (in quotes is situated my own phrase about going to the theatre)

\textit{- Come on, we need to go!, - exclaimed his brother.} - (after the hyphen we see a phrase of somebody’s brother).

Another way is to use a reported speech\index{speech!reported} - a way of describing what somebody has said without using of quotes. Look at the examples of reported speech in English.

\textit{He says (that) we’re going to stay here today.} (We just mention what HE says to us).

\textit{I said that we had finished this work the day before.} (I mention my words that I said previously without exact reproducing of them).

Reported speech is used when we refer to somebody’s words in our sentence implicitly. It is constructed with the ordinary sentence describing real situation and then a relative cause of a reported speech with reference to another expression, which is connected to the main clause with the relative pronoun or a subordinative conjunction.

However, like English, Novoslovnica has a tense-shift in using reported speech. You can see how the tenses shift in Indicative in the next table:

\begin{table}
	\begin{tabular}{ll}
		Tense in Direct Speech & Tense in Reported Speech \\
		Present Common & Present Common/Aorist \\
		Present Concrete & Aorist/Imperfect \\
		Future I & Future-in-the-Past I \\
		Future II & Future-in-the-Past II \\
		Future-in-the-Past I & Future-in-the-Past I \\
		Future-in-the-Past II & Future-in-the-Past II \\
		Aorist & Plusquamperfect (aorist part.) \\
		Imperfect & Plusquamperfect (imperfect part.) \\
		Perfect & Plusquamperfect \\
		Plusquamperfect & Plusquamperfect 
	\end{tabular}
\end{table}


\underline{Subjunctive}\index{mood!subjunctive} mood is reproduced in reported speech so as it is written in the direct one. 

\textbf{Examples:}

\textit{On kazal mamě, če ja bih htěl byti vozarom} - He told my mother that I would like to be a driver.

\textit{On je kazal, če htěl biše dojěhati do Gimalaja.} - He said he would like to get to Himalayas.

\textit{Môǐ prijatelj govorěh, če biše izučil vsï eŭropsky jazycy.} - My friend said he would learn all European languages.

\underline{Imperative}\index{mood!imperative} mood is reproduced in reported speech with using modal verbs and subjunctive mood of the main verb.

\textbf{Examples:}

\textit{Brat mi je kazal, če trěbam bih sgotvil oběd dnësj.} - My brother said I should prepare a dinner today.

\textit{Učitelj mi ukazaše mi že musim bih stvoril vsï děly.} - My teacher told me I must do all homework.

Declarative\index{mood!declarative} mood can be reproduced in reported speech doubly:

\begin{itemize}
	\item with using modal analogs of the English verb “might” with a tense-shifted verb in Indicative mood
	\item the verb in Inferential mood itself can be tense-shifted (we shift the tense of the auxiliary verb “ïmáti”).
\end{itemize}

\underline{Inferential} mood is already a type of reported speech, so it does not shift while using reported speech.
