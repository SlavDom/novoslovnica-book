\chapter{Grammar}

Grammar\index{grammar} is the core of every language. Grammar comprises different themes that are united in a system of changing words enough to build sensible syntax constructions. We will look at Novoslovnica grammar from the point of the parts of speech.

\newglossaryentry{pos}{name=POS, description={Part of speech}}

\textit{Part of speech\index{part of speech} (\gls{pos})} - is a category of words which have similar grammatical properties.

Thus, we unite a group of words into parts of speech when they have a lot of similar grammar properties such as “number”, “person”, “case”, “tense” etc. In this chapter we will go through different parts of speech and study what differences they have and how we should identify each of them and combine them to be able of creating phrases and sentences. 

Parts of speech comprise two main categories - independent\index{independent POS} POS and auxiliary\index{auxiliary POS} POS. The fact the POS is independent or not refers to its semantic value. An independent POS has a full semantic value that can be used separately.  That means when you say a word of an independent POS you reproduce some semantic meaning that the interlocutor can recognize. An auxiliary POS a semantic value that is partially defined and can be distinguished only in pair with a word of an independent POS.

\textbf{Example:}

\textit{River} - this word can be recognized by interlocutor as some concept of water flowing in a restricted area.

\textit{Beautiful} - this word is recognized as an attribute of any concept that is nice, pleasant to the person (i.e. speaker or interlocutor)

\textit{Came} - this word can be recognized as any concept’s action of going to the destination point and having reached it. 

And so on. Nouns, adjectives, verbs, adverbs, participles, numerals, pronouns refer to independent POS. You see that every word has its own meaning full enough to imagine yourself some information you have received. This is not true for a word of an auxiliary POS. 

\textit{On} - this word can be recognized as a placement of an object against some horizontal surface. It can refer to some time moment (be on time). It can also be used with a fully different value (come on - to increase the activity of doing something).

\textit{For} - it can be recognized as an aim for something or an object that participates in some action. Also this word can refer to a duration of a process.

Moreover, words “and” or “to” cannot give you any map in your mind into any sensible meanings. Though these words have no determined semantic value, they are extremely important in the whole phrase connecting two words of the same POS with logical value (AND, OR, NOT). English language is an analytical one, that is why words are mostly connected with each other in the phrase by an auxiliary POS. Without them you are not able to understand what the person is speaking about. Slavic languages are fusional, however, there are enough analytic features in them, hence auxiliary parts of speech are also important. Articles, prepositions and conjunctions are referred to auxiliary parts of speech.

There are also two additional categories - particles and interjections. Some allocate them into separate categories, some claim they belong to an auxiliary category. Nevertheless, these are both separated parts of speech because they have different grammar properties.

These are all categories of POS. If we speak about an independent POS, we should take into account that there are different semantic, morphological and syntax functions can be described by it. There are several types of semantic functions: the concept, the attribute, the predicate and the demonstrative.

The \textit{concept} is something that correlates with the object or subject in the real world. It could be either abstract or imaginary, but we can ask a questions “Who? What?” about it.

The \textit{predicate} is something that determines the action, corresponding to a concept. We often ask questions “What to do?” to reveal a predicate.

The \textit{attribute} is something that is correspond to a concept or an action. We ask questions “What concept is like?” or “What action is like?” to find out the determine value of the attribute.

The \textit{demonstrative} points out the concept. It has the same question with the attribute yet has no semantic value but demonstrating the concept it corresponds with. 

Parts of speech that have properties of a concept are: \textit{nouns, cardinal numerals, verbal noun and some kinds of pronouns}.

Parts of speech that have properties of an attribute are: \textit{adjectives, participles, ordinal numerals and adverbs}.

Parts of speech that have properties of a predicate are: \textit{verbs, transgressive, and gerund}.

Parts of speech that have demonstrative properties are most kinds of \textit{pronouns}.

Thus, noun, adjective, verb, adverb, numeral, pronoun, gerund and participle are independent POSes, while preposition, particle, interjection (with Onomatopoetic), article and conjunction are auxiliary ones.

Independent parts of speech are also divided into nominal and verbal ones. It is extremely important because this division shows differences in grammar forms of nominal and verbal POS. Verb, participle, transgressive, gerund are verbal parts of speech, while noun, adjective, numeral are nominal. Adverb and pronoun stay separately - the first one because of its immutability and the second one because of its heterogenity. 

In the chapter it will be spoken about the very POS exists in Novoslovnica. The table with grammar and semantic properties of a POS will be given in the corresponding section.

First of all you should know some facts about different grammatical properties in Novoslovnica.

\section{Case}

Case is a grammatical property of a nominal POS, that shows what references this nominal POS has with other words in a sentence (phrase). This property is widely known in fusional languages, while analytical languages do not often possess this property. Thus English has only two active cases - Nominative and Oblique case. Moreover, oblique case is used practically only within pronouns while nouns have no such case. That means case is not the only way to show references between nominal POS and other words in a sentence. Case is one of the ways to show and Slavic languages as being fusional widely use this grammar category.

Different Slavic languages have different amounts of cases. For example, Russian language has six cases when Serbian language has seven. We can find exceptions in Bulgarian and Macedonian languages, which are analytical, that’s why they have only one case for a noun and adjective and three cases for a pronoun.

Different cases are referred to different semantic links between words. That’s why we see ambiguity of cases in different languages (that have different amount of cases and different usages of cases). Novoslovnica provides most common and wide means to use cases with most determination. When you speak Novoslovnica, you have to use the case of exact semantic value and not of the longstanding phraseology of your own language.

With this principle Novoslovnica establishes nine cases. Nine changing patterns that determine alterations of all words of nominal POS. Here I want to introduce them to you:

Nominative
Genitive
Partitive
Dative
Accusative
Instrumental
Prepositional
Locative
Vocative

Nominative case is used when we are talking to a concept as an actor. If the sentence is full, the subject is in Nominative. You can ask questions like “Who? What?” to it.

Genitive case is used when we are talking to an object being related to another one. Thus this case show what generation the object is of and what is it made from or to whom does it belong. The questions that determine the case are “Whose? Which? What?”

Partitive case is used when we are talking about some amount of an object having or being supposed to have uncountable properties. The questions that determine the case are “Of what? With what?”. 

Dative case is used when we are talking about a subject of perception. We can ask a question for a word in this case as “Whom? For whom?”  …. 

Accusative case is used to describe a direct object of the action. The questions determining the case are “What? Whom?”

Instrumental case is used to describe an instrument of an action that affect the object of the action. The questions related with this case are: “With whom? With what?”.

Prepositional case 

Locative case

Vocative case

These cases cover 99,99% of possible nominal POS declension.  

\section{Number}

How do people understand what is the numeric characteristic of the object? Of course, the simpliest way is to use numerals. We can call to a numeral and link it with some noun, thus, people will understand that there is an amount shown by a numeral of the concept shown by a noun. But it is rather uncomfortable to use near every noun an additional numeral. Mostly due to the fact we seldom know the exact amount of something. That is why there exists a term of number.

Number\index{number} shows what is the amount of some concept without using numeral before the noun.

Number is a grammar property of the word, its alteration. That means when we change number of the word, we change the word itself and not add some additional words or particles around the word we are speaking about.

There are three numbers in Novoslovnica: \textit{singular}, \textit{dual} and \textit{plural}. Singular\index{number!singular} and plural\index{number!plural} are familiar to an English speaker. They show whether the object is single or not. Dual is rather peculiar \footnote{Using non-boolean logic in a number property is rather specific. Triple logic (Single-Dual-Plural) can be found in some derived from Indo-European languages: Sanscrit, Slavic, though other branches lost the dual number in ancient time (Greek, Latin, German, Baltic); Arabic, Hebrew, Khoe languages also have triple number logic. Quadruple logic (Single-Dual-Triple-Plural) can be found i.e. in Tok-Pisin. Sursurunga is famous for having a five-way grammatical number distinction.}, so we should take an additional account on it.

A dual\index{number!dual} number \footnote{In modern Slavic languages Slovenian, Upper- and Lower-Sorbian languages still have a dual number.} is used when we speak about a pair of something - hands, legs, eyes etc. of one person. Two-doors gates, two boolean values, two antipodes etc. In these cases we use a dual number. Having a pair is a rather frequent fact, so this number appeared in Proto-Indo-European. Dual number so as plural one depends on the word which is spoken so we cannot determine a static rule about choosing a form for a dual number. We can get a dual form of the word by using a declension function with a type of declension corresponding with the word given as a parameter.

There also exists an additional number form we should speak about that is so-called as a counting form. A counting form is used with the nouns. It only occurs when we use the \underline{noun} \textbf{with} the numerals “\underline{two}”, “\underline{three}” or “\underline{four}” (cardinal numbers). The counting form is equal to a dual number in writing so we can speak of it as an extension of a dual number usage.

\textbf{Examples:}

\textit{Dva doma} (two houses) - counting form

\textit{Doma} (two ones) - dual number

\textit{Tri doma} (three houses) - counting form

\textit{Domy} (three ones) - plural number 

\section{Person}

This grammar category determines the person who is spoken about. There are three points of view:

\begin{itemize}
	\item the point of the speaker (First person)
	\item the point of the interlocutor (Second person)
	\item the point of other persons, that are not involved into discussion (Third person)
\end{itemize}

That is how a Slavic discussion could be seen. Practically, this concept is similar to all European languages, particularly English. There is a total equivalency with English in Novoslovnica, so it is not necessary to describe the usage all of these person types. Just look at the following examples to get sure of it:

\textbf{Examples:}

\textit{Ja glědaju v prozorec cěl věčôr.} - I am looking outside the window for the whole evening. (The first person)

\textit{Vy kažete že sámo ïsto byše včera? }- Do you mean that the very same thing was yesterday? (The second person)

\textit{Ony hlåpcy niĝda ne mogut pjiti tïho.} - Those guys never can drink quitely. (The third person)

\section{Tense}

Grammatical tense is a category that expresses time reference with reference to the moment of speaking. In many languages there are three main categories of present, past and future, that refer to the moment of speaking, the period before and the period after it respectively. English is of the same group of languages with tense-rich grammar. It has 16 tenses, divided by categories of time and perfection. Novoslovnica has 12 tenses, based on the two principles like English. They are:

\begin{itemize}
	\item Present Indefinite Tense
	\item Present Definite Tense
	\item Future Indefinite Tense
	\item Future Definite Tense
	\item Pre-future Tense
	\item Future-in-the-Past Tense
	\item Pre-future-in-the-Past Tense
	\item Aorist
	\item Perfect
	\item Imperfect
	\item Plusquamperfect
	\item Past Indefinite Tense
\end{itemize}

First two tenses describe the present moment, the next three ones refers to the future period of time and the last six ones describe the period of time that has already passed.

We can divide past tenses in two groups - past tenses itself and future-in-the-past tenses, that describe actions that refer to the future moment with reference to the moment in the past.
Novoslovnica tense system can be described better with the help of the following diagram:

\begin{figure}
	\includegraphics[width=\linewidth]{./sources/tenses.jpg}
	\caption{Tenses in Novoslovnica}
	\label{fig:tenses}
\end{figure}

Present group of tenses consists of Present Indefinite Tense (\textit{Pritomen čas}) and Present Definite Tense (\textit{Sëdyšen čas}).

Present Indefinite Tense (\textit{Pritomen čas}) is used when the action does not depend on time. For example, in the first example we know that the Earth always revolves and nothing but the apocalypsis can change it. This tense is very similar to the English one and has similar use cases. We can find this tense in indicative (examples 1, 2) and declarative (example 3, 4) moods mostly (in some cases it can be found in different moods too).

Note that in declarative there are two variants of present indefinite, because of its resultive semantics. In example 3 you can see the verb in declarative mood, in present indefinite tense, and in example 4 you can see the same tense and mood but within a predicateless sentence.

Also take care of the translation in example 3. You can mess it with English Present Perfect Tense (He has bought two cars), but this sentence should be translated with the verb «to have» in Present Indefinite with the participle III of the verb «to buy» (bought).

\textbf{Examples:}

\textit{Zemlä sę vreta okolo slånca.} - The Earth revolves around the Sun.

\textit{Vsękyǐ denj ja hođam do učilišta prez lěpyǐ park.} — Every day I walk to school through the beautiful park.

\textit{On imá kupeno dva vozidla.} — He has two cars bought.

\textit{Lěs i tïhota...} — There are forest and silence around me...

Present Definite Tense (\textit{Sëdyšen čas}) is used if the action depends on time. In the first example we can see a sentense with the following semantics: now I think about the rest, though in an hour I can forget about this thought.

Generally, this tense is close to Present Continuous Tense in English, though has a different interpretation. If you want to get in differences, we can give you such an example: «The sun is revolving around the Sun at the moment». Read this hypothetic sentence that is syntaxicly valid in English.

What would be the differences of English and Novoslovnica here? It is in a term of differentiation. English operates with the time and Novoslovnica operates with the mutability. That is why here Novoslovnica will still use Present Definite Tense with the word «sëdy» (at the moment).

\textbf{Examples:}

\textit{Ja myslü o počïvkě.} — I am thinking about the rest.

\textit{On idaje do råboty, zato ne može da odgovori vam.} — He is going to the work, that is why he cannot answer you.

The group of future tenses comprises two ones: Future Tense and Pre-future Tense.

Future Definite Tense (\textit{Bųdešt čas}) defines that the action will appear in the future with reference with the moment of speaking. There are three variants of how you can use Future Tense in Novoslovnica.

Firstly, you can use the verb «hteti» (to will) in 3-person form with the main verb in Present Indefinite Tense (example 1). This varitant is most similar to English Future Simple Tense.

Secondly, you can use the verb «byti» (to be) with inifinitive form of the main verb (example 2). This variant is close to English Future Continuous Tense. It is often used with verbs of A-type (read the chapter about verbal types in Novoslovnica).

Finally, you can use a synthetic verb form with the future tense conjugation (example 3). In English, it should be translated in Future Simple so as the first one.

\textbf{Examples:}

\textit{Ja hte kazam ti něčto.} — I will tell you something.

\textit{Ja bųdu sluhati gudbu.} — I will listen to music. (I will be listening to music).

\textit{Nakonec ja tę viđahtem}. – Finally, I will meet you.

Pre-future Tense (\textit{Predbųdešt čas}) is a grammatical tense that is used when the speaker should explicitly show that the action will take place before another future action. So, this tense deals with the comparison of future actions rather than with the moment of speaking.

In Novoslovnica Pre-future Tense is formed by using the verb «hteti» in 3-person form with the main verb in perfect form (examples 1, 2). This tense should be translated in English with Future Perfect Tense.

\textbf{Examples:}

\textit{Ja hte sòm kupil květy, dy ty hte priǐdaš do města.} — I will have bought flowers, when you will come to the place.

\textit{On hte je zakončil učilišto, dy otec mu hte sę vreta dodomu}. — He will have graduated from school, when his father will come back home.

Future Indefinite Tense (\textit{Něĝdašen čas}) is a rather rare tense to be used in Novoslovnica. It has no direct equivalents in English and should be translated in Future Simple with some keywords (such as «somewhen», «ever», «once»). The sense of this tense is to show that the action takes place in a future moment or period of time, but we do not care of when it will occur or how long it will last.
This tense is formed by Pre-future form of the verb «byti» with the past passive participle of the main verb. Look at the examples to get acquainted.

\textbf{Examples:}

\textit{On hte je byl nagrådil medalom mę.} — Once he will grant me with a medal.

\textit{Ja hte sòm byl kupil vašu věčj.} — Once I will but your item.

All other tenses are related to the past period of time. Firstly, we will consider real past tenses and then future-in-the-past ones.

Aorist (\textit{Prost minul čas}) determines the fact of the committed action without semantic details. It is similar in usage with English Past Simple Tense. Using Aorist we consider only the time the action occurs, but do not think about its duration.

In Novoslovnica it is formed by verb-base vowels with past definite endings: «-h», «-ša», «-še», «-hma», «-hta», «-ha», «-hme», «-hte», «-hu».

\textbf{Examples:}

\textit{Ja dělah råbotu-ta včera.} — I did the work yesterday.

\textit{Pred dvě ročiny ja jěŝih v gråd-òt.} — Two years ago I travelled to the city.

Imperfect (\textit{Neporęden minul čas}) determines the imperfect aspect of the past action. That means it is used with habitual, durable repeated actions etc, that took place in the past. Using imperfect we add the information, that our action took some exact time in the past. This tense is similar in usage with English Past Continuous Tense or Past Perfect.

It is formed so as Aorist with the one difference. There is a «-ě-» vowel before endings, not the verb-base vowel. So, for any verb type (see a chapter about verbal types) there is only a single imperfect form.

\textbf{Examples:}

\textit{Ja pišěh nadomnu råbotu dvě godiny včera.} — I was writing my homework for two hours yesterday.

\textit{Prijatelj mi ráděše na zavodu dvě ročiny.} — My friend had worked on a plant for two years.

Perfect (\textit{Svòršen minul čas}) determines the action has been committed before the present moment. That means, that we take account on the result of the action, on the fact it is finished, while Aorist determines the fact of the action itself and the time when it occured. So, it is a full equivalent of English Present Perfect Tense.

In Novoslovnica Perfect is formed by the auxiliary verb «byti» in Present Indefinite and an L-participle following it.

\textbf{Examples:}

\textit{On je izmyslïl novu ŧeoriju o problemě-ta.} — He has devised a new theory on the problem.

\textit{Môǐ brat je zakončil vysšojučilišto.} — My brother has graduated from University.

Plusquamperfect (\textit{Predminul čas}) determines the fact our action is further from us than another action in the past. It is akin Pre-Future tense, just with past actions. Simply speaking, it is an equivalent of English Past Perfect Tense. So as Pre-Future tense, actions in Plusquamperfect are usually used in pair with another past action that stands in Aorist.

Plusquamperfect in Novoslovnica is formed with aorist form of the auxiliary verb «byti» with an L-participle of the main verb.

\textbf{Examples:}

\textit{Ja byh doǐdal do ulicy-ta, dy on mi odŝvoniše, če ne može da priǐde.} — I had arrived to the street by the time he called be to say he cannot come.

\textit{On byše zdělal model, ĝda ja kazaše mu, če to ne máše potrěbnostï.}

Now we should consider two last tenses: Future-in-the-Past tense and Pre-Future-in-the-Past tense.
Future-in-the-Past tense (Bųdešt v minulom čas) is used when we speak about past actions, that occured after some other past actions. To emphasize this fact we use a past variant of the future tense — Future-in-the-Past. English has an equivalent, so it is easy to build a parallel between these two ones. In Novoslovnica this time is build with imperfect of the auxiliary verb «hteti» and DA-construction of the main verb (example 1).

This tense has also another meaning, that can be expressed with «would like to» phrase in English. It shows a polite intention to do something (example 2).

\textbf{Examples:}

\textit{My zapytahme dali vozidlo htěše da priǐde včas.} — We wondered if the bus would arrive on time.

\textit{Ja htěh da kazam ti něčto.} — I would like to say you something.

Pre-Future-in-the-Past tense (\textit{Predbųdešt v minulom čas}) is used, when we speak about some actions, that happened earlier than some future actions in the past (analogue of pre-future tense in the past). It has a similar meaning with Future-in-the-Past Perfect tense in English and it is rather rarely used.

It is formed with imperfect of the auxiliary verb «hteti» with perfect form of the main verb via DA-construction (look at the example).

\textbf{Examples:}

\textit{Vy kazahte, če htěhte da jeste podpisali pismo pred tym, kak htěhte da započïnate råbotati.} — You said you would have signed the paper before you would start working.

Past Indefinite Tense (\textit{Davnominul čas}) is the last official tense in Novoslovnica. It describes an action that occured in some moment in the past we do not know. In English we should translate it with Past Simple or with the «used to»-construction. It can be considered as a pair to Future Indefinite Tense.

\textbf{Examples:}

\textit{On je byl pisal knigu.} — He used to write a book.

Let us summarize all the equivalents of Novoslovnica's Tenses in English in the next table.

\begin{table}
	\caption{English equivalents of tenses in Novoslovnica}
	\begin{tabular}{ p{11em} p{11em} }
		\textbf{Tense in Novoslovnica}       & \textbf{English equivalent}                  \\
		Present Indefinite Tense             & Present Simple                               \\
		Present Definite Tense               & Present Continuous                           \\
		Future Indefinite Tense              & “once” + Future Simple                       \\
		Future Definite Tense (with “hteti”) & Future Simple                                \\
		Future Definite Tense (with “byti”)  & Future Continuous                            \\
		Future Definite Tense (single form)  & Future Simple                                \\
		Pre-future Tense                     & Future Perfect (Cont.)                       \\
		Future-in-the-Past Tense             & Future-in-the-Past Simple or “would like to” \\
		Pre-future-in-the-Past Tense         & Future-in-the-Past Perfect (Cont.)           \\
		Aorist                               & Past Simple                                  \\
		Perfect                              & Present Perfect (Cont.)                      \\
		Imperfect                            & Past Continuous                              \\
		Plusquamperfect                      & Past Perfect (Cont.)                         \\
		Past Indefinite Tense                & “used to”                                    
	\end{tabular}
\end{table}

The third verbal category can be found in Novoslovnica which is shown only in differences between Aorist and Imperfect: perfection, while other tenses differ only by determinacy and time. So, you can use complex tenses as Perfect, Plusquamperfect, Past Indefinite etc. with aorist and imperfect participles and receive two shades of perfection meaning.

\section{Mood}

Mood\index{mood} is a grammatical feature of verbs, used for signaling modality \cite{mood}. Mood is distinct from grammatical tense or grammatical aspect, although the same word patterns are used for expressing more than one of these meanings at the same time. There are several moods in Novoslovnica \cite{nsl-naklony}:

\begin{itemize}
	\item Indicative
	\item Declarative
	\item Subjunctive
	\item Conjunctive I
	\item Conjunctive II
	\item Imperative
	\item Optative
	\item Jussive
	\item Hortative
	\item Supine
	\item Inferential
\end{itemize}

Moods are divided into realis and irrealis moods. Realis\index{mood!realis} mood is a grammatical mood which is used principally to indicate that something is a statement of fact. More precisely, it is used to express what the speaker considers to be a known state of affairs. Novoslovnica has two realis moods - indicative and declarative.

Indicative mood\index{mood!indicative} (\textit{Oznamitelen})  is used when we need simply to indicate that something is a statement of fact. Indicative has a variety of forms, including positive, negative and interrogative. Due to this fact it is chosen to be a basic mood, that is compared to the others. Indicative mood will be extensively considered in the next chapters. Just look at few examples.

\textbf{Examples:}

\textit{On glěděše iz prozorca doma si.} - He was looking out of the window of his house.

\textit{Ja bųdu hoditi vò vysšojučilišto črez dvě godiny.} - I will attend the university in two years.

Declarative mood\index{mood!declarative} (\textit{Objavitelen}) is used when we describe a state of something, considering the action it was caused by. Declarative mood is of the same degree of variety as indicative mood, though is not used as often as indicative.

The first difference with indicative mood is in using auxiliary verb “imáti” (to have) instead of “byti” (to be) while forming sentences. This makes some semantic shifts in Novoslovnica. Look at the second example: the auxiliary verb is used in present tense, though it has a resultive semantic value. For present semantics the verbless sentences are used (first example).

The second difference is in the particle being used with the auxiliary verb in two moods. In indicative we use L-particles, while in declarative mood we use passive ones.

\textbf{Examples:}

\textit{Rěka krasna i slånco.} - (There are) Beautiful river and the sun (around me).

\textit{Ja mám rođeno dvôh synoŭ.} - I have two sons born.

Irrealis\index{mood!irrealis} mood indicates that a certain situation or action is not known to have happened at the moment the speaker is talking. 

Subjunctive mood\index{mood!subjunctive} (\textit{Predpokladen}) is used when we want to express various states of unreality: wish, emotion, possibility, obligation etc. Subjunctives occur most often, although not exclusively, in subordinate clauses, using particles “aby”, “žeby”, “čtoby”, “išby” (example 1).

Take a note about absence of auxiliary verb while using L-particles. We just use a subjunctive particle with them. Though, we can also use infinitive instead of L-particle with the object in dative (example 2). Moreover, DA-forms can be used to express subjunctive (example 3). Nevertheless, there are cases you can use subjunctive in the main clause (example 4). 

\textbf{Examples:}

\textit{Az upëram aby ty odidal.} - I would like you to leave.

\textit{Az upëram aby ti odidati.} - It’s better for you to leave.

\textit{Ja htem da ty odidaš.} - I want you to leave.

\textit{Ty by odidal odde.} - You better leave here.

Conjunctive mood I\index{mood!conjunctive I} (\textit{Domyslen I}) is used to express real wishes relatively present moment. It means, using Conjunctive I we show that our wish is still able to be implemented. It is a classic variant of conjunctive and is formed by conjunctive form of the verb “byti” (to be) with an L-participle.

In fact, this mood is rather similar to Conditional II in English with the same tense analogues used (examples 2, 3). If we use just conjunctive clause, it is similar with “wish-construction” usage area in English (example 1). However, we can express conjunctive with the single clause using past transgressive (example 4).

Note, that having two clauses we need to use conditional conjunctions (“ako”, “jestjli”, “koli”, “dali”, “či” etc.).

\textbf{Examples:}

\textit{Az bih htěl da viđu slånco v ovyǐ obvlåčnyǐ denj.} - I wish I see the sun in this cloudy day.

\textit{On biše doǐdal do nas, ako my zazovahme ĝo.} - He would come if we called him.

\textit{Ako znaše on novoslovnicu, mogal biše da govori so vsïmi slověnami.} - If he knew Novoslovnica, he would speak with all Slavs.

\textit{Znavšy novoslovnicu, on mogal biše da govori so vsïmi slověnami}. - If he knew Novoslovnica, he would speak with all Slavs.

Conjunctive mood II\index{mood!conjunctive II} (\textit{Domyslen II}) is the second conjunctive mood and it is used to express unreal wishes that have not been realized relatively to a moment in the past. It can be formed by the verb “byti” in conjunctive form either  with supine (example 1) or with plusquamperfekt form of the main verb (example 2). It can be related to Conditional III in English. 

Note, that using supine you do not have to use conditional conjunctions between clauses. Moreover, we can use past active participle in the conditional clause (example 3). Attention! Compare this with the past transgressive in the single clause in Conjunctive I.

\textbf{Examples:}

\textit{Htětj da sę vidim s nim včera, bih byl mogal to da sdělam.} - If  I had wanted to see him yesterday, I would have meet him.

\textit{Ako byh htel da sę vidim s nim včera, bih byl mogal to da sdělam.} - If  I had wanted to see him yesterday, I would have meet him.

\textit{Ako byh htel da sę vidim s nim včera, bih byl mogavšym to sdělati.} - the same translation

Imperative mood\index{mood!imperative} (\textit{Zapověden})  is used when we want to tell somebody a command or a request. In Novoslovnica it has only I-person (in dual and plural) and II-person (in singular, dual and plural) forms. It is indicated by special endings (look forward for details). In the examples you can see how imperative is used.

Please note that imperative is indicated only by single clause. Complex or compound sentences are not of this mood.

\textbf{Examples:}

\textit{Piši ovu knigu.} - [Please] write this book. (you, sg)

\textit{Kažite mu poslanije-to.} - [Please] tell him the message. (you, pl)

\textit{Pročitaǐte ovu knigu.} - [Please] read this book. (you, pl)

\textit{Predgotuǐmo juž sëdy pokôǐ-ot.} - [Please] prepare the room now! (we, pl)

Optative mood\index{mood!optative} (\textit{Žadatelen}) is a grammatical mood that expresses wish or hope. It is used when we want something or somebody to succeed in any action. It is formed by DA-construction in the main clause (look at the examples). 

In English we can find similarity in LET-forms (examples 1, 2) or some general expressions that reveal our wish (example 3).

\textbf{Examples:}

\textit{Da daǐ mi pomognuti ti.} - Let me help you.

\textit{Da bųde tako.} - Let it be so.

\textit{Da žive Bòlĝarija.} - Long live Bulgaria.

Jussive mood\index{mood!jussive} (\textit{Umožnitelen}) is a grammatical mood of verbs for issuing orders and commanding. In Novoslovnica it is used to make orders for third-person expressions. There are no direct equivalents in English for this mood, but we can translate it with impersonal sentences (example 1) or with MUST-modal expressions. It is formed by indicative with “haǐ” modal word.

\textbf{Examples:}

\textit{Haǐ toǐ ne tòpta trevy.} - Do not walk on grass.

\textit{Haǐ on podide.} - He must come closer.

Hortative mood\index{mood!hortative} (\textit{Predložen})  is a grammatical mood that let verb express encourage or discourage of doing something. It can be translated with “let us” (encourage) or “might not” (discourage) constructions in English. In Novoslovnica it is formed with “něhaǐ” modal word with indicative and is able to have negative (example 4) and positive (examples 1-3) form. It is generally used just with I-person expressions.

\textbf{Examples:}

\textit{Něhaǐ grame v ovu gru.} - Let us play this game.

\textit{Něhaǐ pějama pěsnü.} - Let us (both) sing a song.

\textit{Něhaǐ govorim to otcu.} - I might say this to father.

\textit{Něhaǐ ne idame v dom-òt.} - We might not go into the house.


Supine\index{supine} (\textit{Dostęgatelen}) is rather a grammar form than a mood. Nevertheless, it is often used to express aiming at something and the action of approaching to the goal that has been defined. It is similar to infinitive but the ending (infinitive has “-ti” ending, while supine has “-tj” endind). It is usually used with modal verbs and verbs of moving.

Supine can be translated in English through complex predicate. That is because in Novoslovnica supine is practically never used as a single verb form. It is generally used with main verb that determine the background action of the circumstance. Look at the examples to get acquainted with it.

\textbf{Examples:}

\textit{Moǐca je priǐdala povědatj dobru novinu.} - Mojca has come to message a good news.

\textit{Běgi skoro sę preoblekatj.} - Run fast to change clothes.

\textit{Poǐdame kupovatj.} - Let’s go to buy something.

Inferential mood\index{mood!inferential} (\textit{Prekazatelen}) is used to report an unwitnessed event without confirming it. It is often used in stories or fiction books and is very similar with indicative. The only difference is in 3-person in past tenses, where the auxiliary verb “byti” disappears and we use just L-participle. In English it should be translated with ordinary indicative (examples 1-4), sometimes “there”-forms also can be used (example 5). Note that we use L-participle in Novoslovnica in the past tense (you can mess it with Perfect tense), while it should be translated in Past Simple.
Look at the examples.

\textbf{Examples:}

\textit{Jesòm mu ĝo kupil.} - I bought it for him.

\textit{Byl sòm kupil naǐ-prosty martenicy, ale toǐ izbral naǐ-råzkošny.} - I had bought simpliest martenitsy, but he bought the most luxurious.

\textit{On byl bogatym.} - He was rich.

\textit{Kųde byl master?} - Where was the master?

\textit{Žila žena i mųž v malomu domu.} - There lived a woman and a man in a small house.

\section{Noun}

%Table

Nouns can be differentiated by three parameters: gender, animacy and the type of declension.

Animacy determines whether the object is animate (we are able to ask “Who is it?” to the object) and inanimate (we are able to ask “What is it?” to the object).

Gender determines whether the object is masculine, feminine or undefined (we cannot say it is one of the previous genders). Hence, there are three genders: masculine (with masculine properties), feminine (with feminine properties) and neutral (with undefined properties).

Despite English, Novoslovnica make us always show word gender explicitly of both animacies. We can say “it” to the object if we aren’t coupled with it in English. In Novoslovnica (as in every Slavic language) we should use the predefined gender when we speak about some concept (noun). Using wrong genders shows your ignorance and language nescience.

Type of declension is a parameter of declension function. Declension is a function of word alteration. It has two input parameters - the word itself and the type of declension that includes the terms of animacy, gender and some morphological features (such as word endings) in it. The output is a list of forms that the noun can be changed into. Novoslovnica supports 27-cell output list with (3 numbers) * (9 cases) elements in it. Further you can see tables of different declension types. These tables cover all use cases of declension function.

P.S. In tables abbrevs “A” and “I” are for “animate” and “inanimate” respectively.

% Tables
\section{Adjective}

\begin{table}[h]
	\caption{Noun characteristics}
	\begin{tabular}{lllll}
		\textbf{Title}              & \textbf{Value}               \\
		Semantic value              & Attribute                    \\
		Category                    & Independent                  \\
		Subcategory                 & Nominal                      \\
		Alteration                  & Declension                   \\
		Alteration parameters       & Case, Number, Gender, Degree \\
		Differentiation parameters  & Gender, Type, Form
	\end{tabular}
\end{table}

Adjective is one of POS that determines an attribute of the concept. There are two types of adjectives - relative and qualitative. 

Relative adjectives are called so, because they show relations between two concepts or a concept and an action.

\textbf{Examples:}

\textit{South (adj) pole} = South (noun) <= relation <= Pole

\textit{Južnyǐ pôl} = Jug <= relation <= Pôl

Qualitative adjectives are called so, because they show the quality of a concept’s property. This quality could be relative or quantitative or purely qualitative (showing concept condition, position, measure etc).

\textbf{Examples:}

There is the only type of adjective declension. However, we will divide declension tables by gender and base softness.

\section{Pronoun}

\begin{table}[h]
	\caption{Pronoun characteristics}
	\begin{tabular}{lllll}
		\textbf{Title}              & \textbf{Value}               \\
		Semantic value              & Attribute, Concept           \\
		Category                    & Independent                  \\
		Subcategory                 & Nominal                      \\
		Alteration                  & Declension                   \\
		Alteration parameters       & Case, Numbers, Gender, Person\\
		Differentiation parameters  & Gender, Type, Group
	\end{tabular}
\end{table}

Pronoun\index{pronoun} is a POS that has different meanings. It can play the role of an attribute or a concept, depending on what is replaced with the pronoun. However, pronoun has a verbal property of person.

Pronouns are divided into three types: nominal\index{pronoun!nominal} (noun-like declension), substantive\index{pronoun!substantive} (substantive declension), attributive\index{pronoun!attributive} (adjective-like declension).

There are also several groups of pronouns, depending on their semantic value. They are: personal, possessive, interrogative, relative, indefinite, definitive, reflexive, demonstrative, negative, reciprocal.

\subsection{Personal pronouns}

One of the most important groups are personal pronouns\index{pronoun!personal}. They replace nouns in sentences, where we do not want to use an unreasonable repeat. Personal pronouns have different forms for every person-number cell. In the table you can see them.

\begin{table}
	\begin{tabular}{llll}
		& Singular & Dual & Plural \\
		1 person & Ja (Az) & Ma & My \\
		2 person & Ty & Va & Vy \\
		3 person, ms & On & Ona & Oni \\
		3 person, fm & Ona & One & Oně \\
		3 person, neu & Ono & Oná & Onji
	\end{tabular}
\end{table}

I should mention the form “Vy” (You). As in English or polite Russian, we can use this pronoun for a single person when we want to emphasize our respect to the interlocutor. Moreover, we also have to use a plural form of the verb while speaking “Vy” in polite form.

Also you can see two forms of the 1 person - singular pronoun (I). Etymologically the form “Az” is full while “Ja” is only a short one. However, different Slavic languages have retained different forms and now we should support both ones. In Novoslovnica the difference between using “Az” and “Ja” lies in phonetics. As you see, “Az” begins with the vowel and ends with a consonant, “Ja” - controversially. If we remember rule 1 we will get such number of rules:

\begin{itemize}
	\item If the word before the pronoun ends with a vowel and the word after begins with a consonant - you should use “Ja”
	\item If the word before the pronoun ends with a consonant and the word after begins with a vowel - you should use “Az”
	\item In other cases you can use either one or another variant (“Az” is more formal than “Ja”)
\end{itemize}

Now let us speak about declension of personal pronouns. Personal pronouns relate to nominal pronouns (with noun-like declension). This is the most difficult type to learn. But everything has its order and beauty.

\begin{table}[!htb]
	\begin{tabular}{llll}
		I & Singular & Dual & Plural \\
		Nominative & Ja (Az) & Ma & My \\
		Genitive & Menę & Naju & Nas \\
		Partitive & Menä & Naju & Nas \\
		Accusative & Mene & Naju & Nas \\
		Dative & Meni & Nama & Nam \\
		Instrumental & Mnom (Mnoǐ) & Nama & Nami \\
		Prepositional & O mně & O naju & O nas \\
		Locative & Vo mnu & V naju & V nas \\
		Vocative & - & - & -
	\end{tabular}
\end{table}

\begin{table}[!htb]
	\begin{tabular}{llll}
		You (sg) & Singular & Dual & Plural \\
		Nominative & Ty & Va & Vy \\
		Genitive & Tebę & Vaju & Vas \\
		Partitive & Tebä & Vaju & Vas \\
		Accusative & Tebe & Vaju & Vas \\
		Dative & Tebi & Vama & Vam \\
		Instrumental & Tobom (Toboǐ) & Vama & Vami \\
		Prepositional & O mně & O vaju & O vas \\
		Locative & Vo mnu & V vaju & V vas \\
		Vocative & - & - & -
	\end{tabular}
\end{table}

\begin{table}[!htb]
	\begin{tabular}{llll}
		He & Singular & Dual & Plural \\
		Nominative & On & Ona & Oni \\
		Genitive & Jeĝa & Onaju & Ih \\
		Partitive & Jeĝu & Onaju & Ih \\
		Accusative & Jeĝo & Onaju & Ih \\
		Dative & Jemu & Onama & Im \\
		Instrumental & Nim & Onama & Nimi \\
		Prepositional & O nëm & Ob onaju & O nih \\
		Locative & V nëmu & V onaju & V nih \\
		Vocative & - & - & -
	\end{tabular}
\end{table}

\begin{table}[!htb]
	\begin{tabular}{llll}
		She & Singular & Dual & Plural \\
		Nominative & Ona & Oně & Oni \\
		Genitive & Ji & Oněju & Ih \\
		Partitive & Ji & Oněju & Ih \\
		Accusative & Ju & Oněju & Ih \\
		Dative & Ji & Oněma & Im \\
		Instrumental & Neju & Oněju & Nimi \\
		Prepositional & O neǐ & Ob oněju & O nih \\
		Locative & V neji & V oněju & V nih \\
		Vocative & - & - & -
	\end{tabular}
\end{table}

\begin{table}[!htb]
	\begin{tabular}{llll}
		It & Singular & Dual & Plural \\
		Nominative & Ono & Oná & Oni \\
		Genitive & Jeĝa & Náju & Ih \\
		Partitive & Jeĝu & Oněju & Ih \\
		Accusative & Jeĝo & Oněju & Ih \\
		Dative & Jemu & Onáma & Ih \\
		Instrumental & Nim & Onáma & Nimi \\
		Prepositional & O nem & O náju & O nih \\
		Locative & V nemu & V náju & V nih \\
		Vocative & - & - & -
	\end{tabular}
\end{table}


\subsection{Reflexive pronoun}

This\index{pronoun!reflexive} is a separate group of pronouns. There is the only reflexive pronoun in Novoslovnica - “sebę”. Its feature is the absence of nominative form. It has only 7 cases to be alternated. Vocative and Nominative are absent.

\begin{table}
	\begin{tabular}{lll}
		Case & Full & Short \\
		Genitive & Sebę & Sę \\
		Partitive & Sebä & Sä \\
		Accusative & Sebe & Se \\
		Dative & Sebi & Si \\ 
		Insrumental & Sobom (Soboǐ) & - \\ 
		Prepositional & O sobě & - \\
		Locative & V sobu & -
	\end{tabular}
\end{table}


“Sę” is used very often as a reflexive suffix in verbs. It determines a way of creating medial voice sentences (look the paragraph about it).

However, there is a term of a complex reflexive form (CRF) also. It is formed by the sequence of a personal pronoun and the definitive pronoun “sám”. This form shows a partly-developed reflection of a subject.  

There is practically no difference in using a reflexive pronoun or a complex reflexive form. However, it is recommended to us a CRF for Nominative and to use a reflexive pronoun in other cases.

\subsection{Possessive pronouns}

Possessive\index{pronoun!possessive} pronouns show whom any object belongs to. For example, we can say “It is a thing of Bob (which Bob possesses)”. In Novoslovnica you can use a similar expression: “To je věčj Boba” (remember rules of case using). Also Novoslovnica has another expression with a possessive adjective: “To je Bobóva věčj”. This adjective shows that this thing belongs to Bob. However, if we have already used the name of Bob in our sentence, we should not repeat it again. English also uses possessive pronouns in such cases: “Bob is my friend and that’s his thing”. We do not repeat the word Bob, we just say - his (which he possesses). That is what the possessive pronouns look like.

These pronouns are attributive, so we have no need to rewrite their declension, just to name nominative forms. Further, you take a nominative form of a possessive pronoun, look through declension tables for adjectives and transform your pronoun so as you did it with an adjective. Now let us look at the nominative forms of possessive pronouns.

\begin{table}
	\begin{tabular}{llll}
		& Singular & Dual & Plural \\
		1 person & Môǐ & Naš & Naš \\
		2 person & Tvôǐ & Vaš & Vaš \\
		3 person, ms & Jegôǐ & Onyš & Ih \\
		3 person, fm & Jejôǐ & Oneš & Ih \\
		3 person, neu & Jegôǐ & Onyš & Ih
	\end{tabular}
\end{table}

As usual, bold letters are under an accent.

The only feature of possessive pronouns is that you should add a “virtual” ending for 1-person and 2-person pronouns to start declining of it.

\textbf{Examples:}

- \textit{Môǐ - Mojyǐ} (virtual full form) - \textit{Mojoga, mojomu, mojym} (ordinary attributive declension) 

To avoid this difficulty you may use an “adjectived” form

\begin{table}
	\begin{tabular}{llll}
		& Singular & Dual & Plural \\
		1 person & Môǐnyǐ & Našnyǐ & Našnyǐ \\
		2 person & Tvôǐnyǐ & Vašnyǐ & Vašnyǐ \\
		3 person, ms & Jegôǐnyǐ & Onyšnyǐ & Ihnyǐ \\
		3 person, fm & Jejôǐnyǐ & Onešnyǐ & Ěhnyǐ \\
		3 person, neu & Jegôǐnyǐ & Onyšnyǐ & Ihnyǐ
	\end{tabular}
\end{table}

\subsection{Interrogative and relative pronouns}

Interrogative\index{pronoun!interrogative} pronouns are used when we create a question. They are never in plural or dual. Also they have no declension. The only role of interrogative pronouns is to reveal an aim of the question (How much? Who? What?) - to guide the interlocutor to the right answer (you need).

Relative\index{pronoun!relative} pronouns aim is to introduce a relative clause.

\textit{Relative clause}\index{clause!relative} is a special kind of subordinate clause whose primary function is as modifier to a noun or nominal. We examine the case of relative clause modifiers in NPs first, and then extend the description to cover less prototypical relative constructions.\cite{english-grammar}

Interrogative and relative pronouns are practically the same, only usage differs, that is why I talk about them in the same paragraph. Let us look at them in the table.

\begin{table}
	\begin{tabular}{lll}
		English equivalent & Interrogative pronoun & Relative pronoun \\
		Who & Kto? & Ke \\
		What & Čto & Če \\
		What & Kakyǐ? & Kak \\
		Which & Ktoryǐ? & Ktor \\
		Whose & Čyǐ? & Čyǐ \\
		What & Jakyǐ? & Jak \\
		What & Kakvyǐ ? & Kakòv \\
		What & Kolïkyǐ ? & Kolïk 
	\end{tabular}
\end{table}

Pronouns “Kolïko”, “” do not decline. Pronouns “Kakyǐ”, “Ktoryǐ” decline so as adjectives do. Other pronouns have a similar declension with possessive pronouns.
  
\subsection{Indefinite and negative pronouns}

Another group of pronouns that are similar to each other are indefinite\index{pronoun!indefinite} and negative\index{pronoun!negative} pronouns. They are derived from interrogative pronouns by adding the negative “ni-” and the indefinite “ne-” prefixes. Their declension equals to their ancestor’s one.

\begin{table}
	\begin{tabular}{lll}
		Interrogative pronoun & Negative pronoun & Indefinite pronoun \\
		Kto? & Nikto & Někto \\
		Čto & Ničto & Něčto \\
		Kakyǐ? & Nikakyǐ & Někakyǐ \\
		Ktoryǐ? & Niktoryǐ & Něktoryǐ \\
		Čyǐ? & Ničyǐ & Něčyǐ 
	\end{tabular}
\end{table}

\subsection{Demonstrative pronouns}

Demonstrative\index{pronoun!demonstrative} pronouns are usually used to make the interlocutor pay attention to something.
There are only four demonstrative pronouns. And this topic is closely connected with the articles. In the table … you can see these pronouns, divided by terms of visibility and distance of the object to name with respect to the speaker.


\begin{table}
	\begin{tabular}{lll}
		& Visible & Invisible \\
		Far & Onyǐ & Tyǐ \\
		\multirow{2}{*}{Close} & Ovyǐ & - \\ & \multicolumn{2}{c}{Sïǐ}  
	\end{tabular}
\end{table}

These pronouns decline so as adjectives do, so that is very easy. However, you see that there is no pronoun for a close object that is not seen to you. Maybe it is logical, maybe not, nevertheless, Slavic languages do not support this semantic value.

Examples:

\subsection{Definitive pronouns}

Attributive\index{pronoun!definitive} pronouns indicate a generalized feature of an object. 

\begin{table}
	\begin{tabular}{ll}
		English equivalent & Definitive pronoun \\
		Whole & Vesj \\
		All & Vsë, vsä, vsï \\
		Every & Vsękyǐ, lübyǐ \\
		Each & Káždyǐ \\
		Another, other & Ïnyǐ, drugyǐ \\
		Reflexive pronouns & Personal pronouns + sám
	\end{tabular}
\end{table}

All these pronouns decline as adjectives. I should say just about the last one - the pronoun “sám”. This pronoun is used with personal pronouns to create a complex reflexive form (look paragraph about reflexive pronouns). However, it keeps adjective-like declension.

\subsection{Reciprocal pronouns}


The last type of pronouns is reciprocal\index{pronoun!reciprocal}. It is used to refer to a noun phrase mentioned earlier in a sentence. English has only two such pronouns - each other and another one.

Slavic languages have much more variants:

\textit{Drug so drugom}

\textit{Raz za razom}


\section{Numeral}

\begin{table}[h]
	\caption{Adverb characteristics}
	\begin{tabular}{lllll}
		\textbf{Title}              & \textbf{Value}      \\
		Semantic value              & Attribute           \\
		Category                    & Independent         \\
		Subcategory                 & Nominal             \\
		Alteration                  & Declension          \\
		Alteration parameters       &               \\
		Differentiation parameters  & Type
	\end{tabular}
\end{table}

Numerals can be attributive (with a nount) or pronominal (without a noun).
	
% 	Им. п.	Р. п. 	К. п.	В. п.	Д. п.	Тв. п.	П. п.	М. п.
% М. р.	Једен	Једнога	Једногу	Једного	Једному	Једным	Једном	Једному
% Ж. р.	Једна	Једної	Једної	Једну	Једной	Једноју	Једной	Једної
% Ср. р.	Једно	Једнога	Једногу	Једного	Једному	Једным	Једном	Једному


%	Им. п.	Р. п. 	К. п.	В. п.	Д. п.	Тв. п.	П. п.	М. п.
% М. р.	Два	Двôх	Двôх	Два	Двôм	Двомя	Двôх	Двому
% Ж. р.	Двѣ	Двôх	Двôх	Двѣ	Двôм	Двомя	Двôх	Двому

% 	Им. п.	Р. п. 	К. п.	В. п.	Д. п.	Тв. п.	П. п.	М. п.
% М. р.	Три	Трёх	Трёх	Три	Трём	Трємя	Трёх	Трєму
% Ж. р.	Трѣ	Трёх	Трёх	Трѣ	Трём	Трємя	Трёх	Трєму

% Дальше наблюдается система – числительные от 5 до 20 имеют следующую парадигму склонения (о которой будет изложено чуть ниже), далее 21,31,41 и т.п. имеют первую парадигму, 22,32,42 и т.п. – вторую, 23,24,33,34 и т.п. – третью, и 25-30, 35-40, 45-50 и т.п. – четвёртую.
% Приводим четвёртую парадигму склонения на примере числительного 5 (Пѣт)


% 	Им. п.	Р. п. 	К. п.	В. п.	Д. п.	Тв. п.	П. п.	М. п.
% М. р.	Пѧт	Пѧті	Пѧті	Пѧть	Пѧті	Пѧтію	Пѧті	Пѧтї
% Ж. р.	Пѧць	Пѧті	Пѧті	Пѧць	Пѧті	Пѧтію	Пѧті	Пѧтї

\subsection{Cardinal numbers}

\begin{itemize}
	\item One (1) - Jeden, Jedna, Jedno
	\item Two (2) - Dva, Dvě
	\item Three (3) - Tri, Trě
	\item Four (4) - Četyri, Četyrě
	\item Five (5) - Pęt
	\item Six (6) - Šest
	\item Seven (7) - Sedem
	\item Eight (8) - Osem
	\item Nine (9) - Devęt
	\item Ten (10) - Desęt
	\item Eleven (11) - Jedennadesęt
	\item Twelve (12) - Dvanadesęt
	\item Thirteen (13) - Trinadesęt
	\item Fourteen (14) - Četyrinadesęt
	\item Fifteen (15) - Pętnadesęt
	\item Sixteen (16) - Šestnadesęt
	\item Seventeen (17) - Sedemnadesęt
	\item Eighteen (18) - Osemnadesęt
	\item Nineteen (19) - Devętnadesęt
	\item Twenty (20) - Dvadesęt
	\item Twenty-one (21) - Dvadesęt jeden
	\item Thirty (30) - Tridesęt
	\item Fourty (40) - Četyridesęt
	\item Fifty (50) - Pętdesęt
	\item Sixty (60) - Šestdesęt
	\item Seventy (70) - Sedemdesęt
	\item Eighty (80) Osemdesęt
	\item Ninety (90) - Devętdesęt
	\item Hundred (100) - Sto
	\item One hundred one (101) - Sto jeden
	\item One hundred ten (110) - Sto desęt
	\item Two hundred (200) - Dvěstě (Dvasta)
	\item Three hundred (300) - Trista
	\item Four hundred (400) - Četyrista
	\item Five hundred (500) - Pętsòt
	\item Six hundred (600) - Šestsòt
	\item Seven hundred (700) - Sedemsòt
	\item Eight hundred (800) - Osemsòt
	\item Nine hundred (900) - Devętsòt
	\item Thousand (1000) - Tysęča
	\item One thousand one (1001) - Tysęča jeden
	\item One thousand ten (1010) - Tysęča desęt
	\item One thousand one hundred (1100) - Tysęča sto
	\item Two thousand (2000)- Dvě tysęčy
	\item Five thousand (5000) - Pęt tysęč
	\item Million (1000000) - Milïon
	\item One million one thousand one (1001001) - Milïon sto jeden
	\item Billion (10\^9) - Bilïon (Milïard)
	\item Trillion (10\^12) - Trilïon
	\item Quadrillion (10\^15) - Kŭadrilïon
	\item Quintillion (10\^18) - Kŭintilïon
	\item Sextillion (10\^21) - Sekstilïon
	\item Septillion (10\^24) - Septilïon
	\item Octillion (10\^27) - Oktilïon
	\item Nonillion (10\^30) - Nontilïon
	\item Decillion (10\^33) - Decilïon
	\item Googol (10\^100) - Ĝuĝòl
	\item Googolplex (10\^(10\^100)) - Ĝuĝlopleks
\end{itemize}

Declension



\subsection{Ordinal numerals}

\begin{itemize}
	\item First (1) - Pòrvyǐ
	\item Second (2) - Vtoryǐ
	\item Third (3) - Tretjyǐ
	\item Fourth (4) - Četvòrtyǐ
	\item Fifth (5) - Pętyǐ
	\item Sixth (6) - Šestyǐ
	\item Seventh (7) - Sedmyǐ
	\item Eighth (8) - Osmyǐ
	\item Ninth (9) - Devętyǐ
	\item Tenth (10) - Desętyǐ
	\item Eleventh (11) - Jedennadesętyǐ
	\item Twelfth (12) - Dvanadesętyǐ
	\item Thirteenth (13) - Trinadesętyǐ
	\item Fourteenth (14) - Četyrinadesętyǐ
	\item Fifteenth (15) - Pętnadesętyǐ
	\item Sixteenth (16) - Šestnadesętyǐ
	\item Seventeenth (17) - Sedemnadesętyǐ
	\item Eighteenth (18) - Osemnadesętyǐ
	\item Nineteenth (19) - Devętnadesętyǐ
	\item Twentieth (20) - Dvadesętyǐ
	\item Twenty first (21) - Dvadesęt pòrvyǐ
	\item Thirtieth (30) - Tridesętyǐ
	\item Fourtieth (40) - Četyridesętyǐ
	\item Fiftieth (50) - Pętdesętyǐ
	\item Sixtieth (60) - Šestdesętyǐ
	\item Seventieth (70) - Sedemdesętyǐ
	\item Eightieth (80) Osemdesętyǐ
	\item Ninetieth (90) - Devętdesętyǐ
	\item Hundredth (100) - Sòtyǐ
	\item One hundred and first (101) - Sto pòrvyǐ
	\item One hundred and tenth (110) - Sto desętyǐ
	\item Two hundredth (200) - Dvôhsòtyǐ
	\item Three hundredth (300) - Tröhsòtyǐ
	\item Four hundredth (400) - Četyröhsòtyǐ
	\item Five hundredth (500) - Pętsòtyǐ
	\item Six hundredth (600) - Šestsòtyǐ
	\item Seven hundredth (700) - Sedemsòtyǐ
	\item Eight hundredth (800) - Osemsòtyǐ
	\item Nine hundredth (900) - Devętsòtyǐ
	\item Thousandth (1000) - Tysęčnyǐ
	\item One thousand and first (1001) - Tysęča pòrvyǐ
	\item One thousand and tenth (1010) - Tysęča desętyǐ
	\item One thousand and one hundredth (1100) - Tysęča sòtyǐ
	\item Two thousandth (2000)- Dvě tysęčnyǐ
	\item Five thousandth (5000) - Pęt tysęčnyǐ
	\item Millionth (1000000) - Milïonnyǐ
	\item One million one thousand and first (1001001) - Milïon sto pòrvyǐ
	\item Billionth (10\^9) - Bilïonnyǐ (Milïardnyǐ)
	\item Trillionth (10\^12) - Trilïonnyǐ
	\item Quadrillionth (10\^15) - Kŭadrilïonnyǐ
	\item Quintillionth (10\^18) - Kŭintilïonnyǐ
	\item Sextillionth (10\^21) - Sekstilïonnyǐ
	\item Septillionth (10\^24) - Septilïonnyǐ
	\item Octillionth (10\^27) - Oktilïonnyǐ
	\item Nonillionth (10\^30) - Nontilïonnyǐ
	\item Decillionth (10\^33) - Decilïonnyǐ
	\item Googolth (10\^100) - Ĝuĝòlnyǐ
	\item Googolplexth (10\^(10\^100)) - Ĝuĝlopleksnyǐ
\end{itemize}

\subsection{Pronominal numerals}

% СОБИРАТЕЛЬНОЕ ЧИСЛИТЕЛЬНОЕ
% Собирательное числительное обозначает совокупность объектов названного количество. Как и существительные данного типа, они склоняются только в единственном числе. Характерной особенностью является наличие суффикса «–ЕР–» или «–ОЈ–». 


% Числительное	Им. п.	Р. п. 	К. п.	В. п.	Д. п.	Тв. п.	П. п.	М. п.
% Два	Двојє	Двојіх		=Р.п.	Двојім	Двојіми	Двојіх	
% Три	Тројє	Тројіх		=Р.п.	Тројім	Тројіми	Тројіх	
% Четыри	Четверо	Четверых		=Р.п.	Четверым	Четверыми	Четверых	
% Пѧт	Пѧтеро	Пѧтерых		=Р.п.	Пѧтерым	Пѧтерыми	Пѧтерых	
% Шест	Шестеро	Шестерых		=Р.п.	Шестерым	Шестерыми	Шестерых	
% Седем	Седмеро	Седмерых		=Р.п.	Седмерым	Седмерыми	Седмерых	
% Осем	Осмеро	Осмерых		=Р.п.	Осмерым	Осмерыми	Осмерых	
% Девѧт	Девѧтеро	Девѧтерых		=Р.п.	Девѧтерым	Девѧтерыми	Девѧтерых	
% Десѧт	Десѧтеро	Десѧтерых		=Р.п.	Десѧтерым	Десѧтерыми	Десѧтерых	
% Једеннадесѧт	-десѧтеро	=||=	=||=	=||=	=||=	=||=	=||=	=||=
% Двадесѧт	-десѧтеро	=||=	=||=	=||=	=||=	=||=	=||=	=||=
% ИТД								


\section{Adverb}

\begin{table}[h]
	\caption{Adverb characteristics}
	\begin{tabular}{lllll}
		\textbf{Title}              & \textbf{Value}      \\
		Semantic value              & Attribute           \\
		Category                    & Independent         \\
		Subcategory                 & Nominal             \\
		Alteration                  & Comparison          \\
		Alteration parameters       & Degree              \\
		Differentiation parameters  & Type, Group
	\end{tabular}
\end{table}

\section{Predicative}

\begin{table}[h]
	\caption{Noun characteristics}
	\begin{tabular}{lllll}
		\textbf{Title}              & \textbf{Value}                            \\
		Semantic value              & Predicate                                 \\
		Category                    & Independent                               \\
		Subcategory                 & Nominal                                   \\
		Alteration                  & None                                      \\
		Alteration parameters       & None                                      \\
		Differentiation parameters  & Tense                                  
	\end{tabular}
\end{table}

Predicative\index{predicative} is a POS that is closely deriving to the predicate. Formally, it is an adverb that plays the role of the predicate (concluded in the predicate as a part of it). 

You should divide a syntax and morphological definitions of predicative. Now we are talking about the second one. In russian philology you can find the term of “condition category” - the equivalent for predicative (morphological definition). In this paragraph we will speak just about this term.

Predicative determines the class of words indicating an attribute or a condition of a person, environment etc., especially mental. Predicatives cannot be alternated, though they can be differentiated by tenses using an auxiliary verb “to be”. 

Predicatives have homonyms with adverbs, and nouns.  

\section{Verb}

Verb\index{verb} is a POS that describes the predicate and its properties. 


\subsection{Verbal types}

Learning Slavic languages you could mention that there is a set of verb suffixes that is very eliminated. Novoslovnica provides a theory that allows you to construct a right verb form.
Verbs with different verb suffixes represent different verbal types. There are four types of verbs in Novoslovnica.

\begin{itemize}
	\item A-type\index{verb!a-type}: verbs of this type define that the action in common sense.  
	\item E-type\index{verb!e-type}: verbs of this type define that the action is long-termed.
	\item I-type\index{verb!i-type}: verbs of this type define that the action is short-termed. This type comprises verbs with suffixes I and O. The suffix O is used when the consonant before the constructed suffix is involved in alterations depending on soft vowels that the vowel I is.
	\item U-type\index{verb!u-type}: verbs of this type define that the action is dotty.
	\item Extra type\index{verb!extra-type}. Is formed with the suffix “-OVA-” and defines the repeated action.  
\end{itemize}

When you speak about the action, you find what characteristic is suitable for the action and then use one of the predefined verbal types.

Somebody can ask what are the differences in tenses and types, because it might be confusing. However, verbal type determines the durability of the action (or its repeating property) while tense determines tense characteristic of the action such as completeness, stability in time, result, order etc.

All tenses provide difference between conjugation of different verbal types except imperfect. This tense has the common conjugation table for all verbal types. 

Further we will speak about conjugation itself. We will look at verb conjugation in indicative mood first of all. Then we will speak about other moods.

\subsection{Active voice}

Active\index{voice!active} voice shows that the person makes the action by himself. So, the subject of the sentence and the actor are the same.

\subsubsection{Indicative mood}

Verbs in indicative\index{mood!indicative} mood can be found in every tense that Novoslovnica possesses. Let us look at some tables with verb of different verbal types conjugation.

\subsubsection{Present Tenses}

\begin{table}[!htb]
	\caption{A-type conjugation in Present Indefinite}
	\begin{tabular}{llll}
		Present Indefinite & Singular & Dual & Plural \\
		1 person & -am & -ama & -ame \\
		2 person & -aš & -ata & -ate \\
		3 person & -a & -at & -ut
	\end{tabular}
\end{table}


\begin{table}[!htb]
	\caption{A-type conjugation in Present Definite}
	\begin{tabular}{llll}
		Present Definite & Singular & Dual & Plural \\
		1 person & -aju & -ajema & -ajeme \\
		2 person & -aješ & -ajeta & -ajete \\
		3 person & -aje & -ajat & -ajut
	\end{tabular}
\end{table}

\begin{table}[!htb]
	\caption{E-type conjugation in Present Indefinite}
	\begin{tabular}{llll}
		Present Indefinite & Singular & Dual & Plural \\
		1 person & -em & -ema & -eme \\
		2 person & -eš & -eta & -ete \\
		3 person & -e & -at & -ut
	\end{tabular}
\end{table}


\begin{table}[!htb]
	\caption{E-type conjugation in Present Definite}
	\begin{tabular}{llll}
		Present Definite & Singular & Dual & Plural \\
		1 person & -u & -ujema & -ujeme \\
		2 person & -uješ & -ujeta & -ujete \\
		3 person & -uje & -ujat & -ujut
	\end{tabular}
\end{table}


\begin{table}[!htb]
	\caption{I-type conjugation in Present Indefinite}
	\begin{tabular}{llll}
		Present Indefinite & Singular & Dual & Plural \\
		1 person & -im & -ima & -ime \\
		2 person & -iš & -ita & -ite \\
		3 person & -i & -at & -ut
	\end{tabular}
\end{table}


\begin{table}[!htb]
	\caption{I-type conjugation in Present Definite}
	\begin{tabular}{llll}
		Present Definite & Singular & Dual & Plural \\
		1 person & -ujim & -ujima & -ujime \\
		2 person & -ujiš & -ujita & -ujite \\
		3 person & -uji & -ujat & -ujut
	\end{tabular}
\end{table}


\begin{table}[!htb]
	\caption{U-type conjugation in Present Indefinite}
	\begin{tabular}{llll}
		Present Indefinite & Singular & Dual & Plural \\
		1 person & -nam & -nama & -name \\
		2 person & -naš & -nata & -nate \\
		3 person & -na & -nat & -nut
	\end{tabular}
\end{table}


\begin{table}[!htb]
	\caption{U-type conjugation in Present Definite}
	\begin{tabular}{llll}
		Present Definite & Singular & Dual & Plural \\
		1 person & - & - & - \\
		2 person & - & - & - \\
		3 person & - & - & -
	\end{tabular}
\end{table}

\begin{table}[!htb]
	\caption{Extra-type conjugation in Present Indefinite}
	\begin{tabular}{llll}
		Present Indefinite & Singular & Dual & Plural \\
		1 person & -ovam & -ovama & -ovame \\
		2 person & -ovaš & -ovata & -ovate \\
		3 person & -ova & -ovat & -ovut
	\end{tabular}
\end{table}


\begin{table}[!htb]
	\caption{Extra-type conjugation in Present Definite}
	\begin{tabular}{llll}
		Present Definite & Singular & Dual & Plural \\
		1 person & -uju & -ujema & -ujeme \\
		2 person & -uješ & -ujeta & -ujete \\
		3 person & -uje & -ujat & -ujut
	\end{tabular}
\end{table}

\begin{table}[!htb]
	\caption{The verb "Byti" (Exception)}
	\begin{tabular}{lllllll}
		Pr. Indef.
			& \multicolumn{2}{c}{Singular}
			 & \multicolumn{2}{c}{Dual}
			 & \multicolumn{2}{c}{Plural} \\
		1 person & Jesòm & Sòm & Jesma & Sma & Jesme & Sme \\
		2 person & Jesi & Si & Jesta & Sta & Jeste & Ste \\
		3 person & Jestj & Je & Jesų & Sų & Jesu & Su
	\end{tabular}
\end{table}

\subsubsection{Future Tenses}

\textbf{Future Definite Tense (Bųdešt čas)}

The Future Definite Tense is formed with the following:

\begin{itemize}
	\item Using HTE (3p. sg. of the verb "hteti" (to will)) + the verb in Present Definite or Indefinite Tense
	\item Using the future form of the verb "byti" + infinitive
	\item Using single-form conjugation 
\end{itemize}

\begin{table}[!htb]
	\caption{Future single-form conjugation}
	\begin{tabular}{llll}
		Future
		& Singular
		& Dual
		& Plural \\
		1 person & -ahtem & -ahtema & ahteme \\
		2 person & -ahteš & -ahteta & -ahtete \\
		3 person & -ahte & -ahtat & -ahtut
	\end{tabular}
\end{table}

\begin{table}[!htb]
	\caption{The verb "Byti" future conjugation (Exception)}
	\begin{tabular}{llll}
		Future
		& Singular
		& Dual
		& Plural \\
		1 person & Bųdu & Bųdema & Bųdeme \\
		2 person & Bųdeš & Bųdeta & Bųdete \\
		3 person & Bųde & Bųdat & Bųdut
	\end{tabular}
\end{table}

\textbf{Pre-Future Tense}

\begin{table}[!htb]
	\caption{The verb "Byti" pre-future conjugation (Exception)}
	\begin{tabular}{llll}
		Future
		& Singular
		& Dual
		& Plural \\
		1 person & Bųdeh & Bųdehma & Bųdehme \\
		2 person & Bųdeša & Bųdehta & Bųdehte \\
		3 person & Bųdeše & Bųdeha & Bųdehu
	\end{tabular}
\end{table}

\textbf{Future Indefinite Tense}

\subsubsection{Past Tenses}

\begin{table}[!htb]
	\caption{A-type conjugation in Aorist}
	\begin{tabular}{llll}
		Present Definite & Singular & Dual & Plural \\
		1 person & -ah & -ahma & -ahme \\
		2 person & -aša & -ahta & -ahte \\
		3 person & -aše & -aha & -ahu
	\end{tabular}
\end{table}

\begin{table}[!htb]
	\caption{E-type conjugation in Aorist}
	\begin{tabular}{llll}
		Present Definite & Singular & Dual & Plural \\
		1 person & -eh & -ehma & -ehme \\
		2 person & -eša & -ehta & -ehte \\
		3 person & -eše & -eha & -ehu
	\end{tabular}
\end{table}

\begin{table}[!htb]
	\caption{I-type conjugation in Aorist}
	\begin{tabular}{llll}
		Present Definite & Singular & Dual & Plural \\
		1 person & -ih & -ihma & -ihme \\
		2 person & -iša & -ihta & -ihte \\
		3 person & -iše & -iha & -ihu
	\end{tabular}
\end{table}

\begin{table}[!htb]
	\caption{U-type conjugation in Aorist}
	\begin{tabular}{llll}
		Present Definite & Singular & Dual & Plural \\
		1 person & -uh & -uhma & -uhme \\
		2 person & -uša & -uhta & -uhte \\
		3 person & -uše & -uha & -uhu
	\end{tabular}
\end{table}

\begin{table}[!htb]
	\caption{Extra-type conjugation in Aorist}
	\begin{tabular}{llll}
		Present Definite & Singular & Dual & Plural \\
		1 person & -ovah & -ovahma & -ovahme \\
		2 person & -ovaša & -ovahta & -ovahte \\
		3 person & -ovaše & -ovaha & -ovahu
	\end{tabular}
\end{table}

\begin{table}[!htb]
	\caption{The verb "byti" conjugation in Aorist}
	\begin{tabular}{llll}
		Present Definite & Singular & Dual & Plural \\
		1 person & byh & byhma & byhme \\
		2 person & byša & byhta & byhte \\
		3 person & byše & byha & byhu 
	\end{tabular}
\end{table}

\begin{table}[!htb]
	\caption{Сonjugation in Imperfect}
	\begin{tabular}{llll}
		Present Definite & Singular & Dual & Plural \\
		1 person & -ěh & -ěhma & -ěhme \\
		2 person & -ěša & -ěhta & -ěhte \\
		3 person & -ěše & -ěha & -ěhu
	\end{tabular}
\end{table}

\begin{table}[!htb]
	\caption{Сonjugation in Perfect}
	\begin{tabular}{llll}
		Present Definite & Singular & Dual & Plural \\
		1 person & sòm + PPP & jesma + PPP & jesme + PPP \\
		2 person & si + PPP & jesta + PPP & jeste  + PPP \\
		3 person & je + PPP & jesų + PPP & jesu + PPP 
	\end{tabular}
\end{table}

\begin{table}[!htb]
	\caption{Conjugation in Plusquamperfect}
	\begin{tabular}{llll}
		Present Definite & Singular & Dual & Plural \\
		1 person & byh + PPP & byhma + PPP & byhme + PPP \\
		2 person & byša + PPP & byhta + PPP & byhte  + PPP \\
		3 person & byše + PPP & byha + PPP & byhu + PPP 
	\end{tabular}
\end{table}

\subsubsection{Future-in-the-Past Tenses}


\textbf{Future-in-the-Past}

\textbf{Pre-Future-in-the-Past}

\subsubsection{Subjunctive mood}

Subjunctive\index{mood!subjunctive} mood shows two states of an actions. In the first state it can show a desirable action. In the second one it can show an action that has not become a real one. 

Subjunctive mood has only one-tense form. It is constructed with the verb “byti” in a subjunctive form with an aorist or imperfect participle. Special forms of the verb “byti” you can see in the next table.

\begin{table}[!htb]
	\begin{tabular}{llll}
		Subjunctive mood & Singular & Dual & Plural \\
		1 person & Bih & Bihma & Bihme \\
		2 person & Biša & Bihta & Bihte \\
		3 person & Biše & Biha & Bihu
	\end{tabular}
\end{table}

\subsubsection{Imperative mood}

Imperative\index{mood!imperative} mood shows that the actor make somebody do some action (imperate another object). Imperative mood also has only one tense to be used in. 

\begin{table}[!htb]
	\caption{A-type}
	\begin{tabular}{llll}
		Imperative mood & Singular & Dual & Plural \\
		1 person &  & -aǐma & -aǐmo \\
		2 person & -aǐ & -aǐta & -aǐte \\
		3 person &  &  & 
	\end{tabular}
\end{table}



\begin{table}[!htb]
	\caption{E-type}
	\begin{tabular}{llll}
		Imperative mood & Singular & Dual & Plural \\
		1 person &  & -eǐma & -eǐmo \\
		2 person & -i & -eǐta & -eǐte \\
		3 person &  &  & 
	\end{tabular}
\end{table}



\begin{table}[!htb]
	\caption{I-Type}
	\begin{tabular}{llll}
		Imperative mood & Singular & Dual & Plural \\
		1 person &  & -iǐma & -iǐmo \\
		2 person & -i & -iǐta & -iǐte \\
		3 person &  &  & 
	\end{tabular}
\end{table}

\begin{table}[!htb]
	\caption{U-Type}
	\begin{tabular}{llll}
		Imperative mood & Singular & Dual & Plural \\
		1 person &  & -niǐma & -niǐmo \\
		2 person & -ni & -niǐta & -niǐte \\
		3 person &  &  & 
	\end{tabular}
\end{table}


\begin{table}[!htb]
	\caption{Extra-Type}
	\begin{tabular}{llll}
		Imperative mood & Singular & Dual & Plural \\
		1 person &  & -uǐǐma & -uǐǐmo \\
		2 person & -uǐ & -uǐǐta & -uǐte \\
		3 person &  &  & 
	\end{tabular}
\end{table}

\subsubsection{Inferential mood}

Inferential\index{mood!inferential} mood is used when we speak about actions that we did not observe by ourselves. This mood mostly is used in tales, histories two show that we were not witnesses of what we are speaking about.

Inferential mood is created by using the verb “ïmáti’ with the past passive participle (PPP). This mood can be used in all tenses of the indicative mood (we should only place the verb “ïmáti” into this tense and add a PPP of the main verb to it). 

\textbf{Examples:}

\textit{Ja sòm rođen svojoǐ mamoǐ} - I am born by my own mother. (This is a real fact, but the actor (Mother) is not at the subject place in the sentence - Passive Voice).

\textit{Ona má rođeno ove rebętko.} - Some say she bore this child (This is a rumor, and the actor is in the place of a subject - Inferential mood of AV)

\subsection{Passive voice}

Passive\index{voice!passive} voice shows that the person is an object for the action. That means, that the subject of the sentence semantically is not an actor, but the object. 

Passive voice has only two moods - Indicative and Subjunctive. Passive voice is created by using the verb “byti” with the PPP. Here is the narrow border between Inferential mood of Active Voice and Indicative mood of Passive Voice.

We can describe the difference between these two terms in such a way. If we speak about real actions that we have imagined or we heard about them (however, the actor of these sentences is placed in the subject) - we use Inferential mood. And if we speak about any action that has no actor in the subject syntax role - we use Passive Voice. Let’s look at the examples.


\subsection{Infinitive and Supine}

Infinitive is the main form of the verb. In fact all languages that have an infinitive form of the verb use it in that way. So when you look through the dictionary looking for a verb, you should remember that it will stand in infinitive. Infinitive form is created by adding the ending “-ti” to the end of the verb. (Compare with English, you put the particle “to” before the verb to create the same form - there is some similarity in both cases). Infinitive has no parameters for declination.

The second unchangeable form of the verb is supine. It has no equivalent in English. Semantically, it is similar to the construction “to be going to do something”, determining the aims of the subject. Supine in Novoslovnica is built by adding the “-tj” ending to the end of the verb.

Supine had a great usage area in the past, now it is still used in Uppersorbian, Lowersorbian and Slovenian languages. It is mostly used with the verbs of motion, such as “to go”, “to swim”, “to move”, “to fly” etc.

Examples:

Ty hteš kazati mi něčto, da li? - You want to say me something, don’t you? (Infinitive)

Mamo, ja idaju spatj dnesj po-pano. - Mom, I’m going to sleep now earlier. (Supine)




\section{Participle}

\begin{table}[h]
	\caption{Adjective characteristics}
	\begin{tabular}{lllll}
		\textbf{Title}              & \textbf{Value}               \\
		Semantic value              & Attribute                    \\
		Category                    & Independent                  \\
		Subcategory                 & Verbal                       \\
		Alteration                  & Declension                   \\
		Alteration parameters       & Case, Number                 \\
		Differentiation parameters  & Type, Voice, Tense, Gender
	\end{tabular}
\end{table}

Participle is an independent verbal POS, that defines the action of another actor as its attribute. Participle is a POS between the nominal and verbal type, so as a pronoun, because it has properties of tense, voice, gender, case and number.

There are essential participles and auxiliary participles. The first ones can be used as attributes of nouns, they are far distanciated from the verb. They are Active and Passive Participles.

Auxiliary participles are quite closer to verbs, they participate in grammar constructions. They are Aorist and Imperfect participles.

Let us look at the tables to define how to use different participles.

\section{Gerund}

Gerund is a form of a verb that determines the process of an action. That means it is not an independent POS, but has some features that do not allow us to include the description about gerund in the paragraph about verb.

Gerund is formed from the infinitive of a verb by reducing a “-ti” ending and adding a suffix “-n-” and an ending “-e-”. However, gerunds are often created only from verbs of A-type. However, you can construct gerunds from verbs of other types. Look at the examples:

\textbf{Examples:}

\textit{Pisati} (verb) - pisane (gerund)

\textit{Nositi} (verb) - nosine (gerund, bad form) - nošane (gerund, recommended form)

\textit{Pěti} (verb) - Pěne (gerund, bad form) - pějane (gerund, recommended form)

Unlike infinitive or supine, gerund can be declined, but only in the singular. This fact unites it with the nominal POS. Simply speaking, gerund declines so as soft neutral nouns do despite the nominal/accusative form of “e”.

\begin{table}
	\begin{tabular}{ll}
	Case & Ending \\
	Nominal & -e \\
	Genitive & -ä \\
	Partitive & -ü \\
	Dative & -ü \\
	Accusative & -e \\
	Instrumental & -ëm \\
	Prepositional & -ě \\
	Locative & -ü \\
	Vocative & no form
	\end{tabular}
\end{table}

Let us look at some examples:

Examples:



\section{Article}

If you tried to learn a Slavic language, you would notice that there are no articles\index{article} in it. The term of definiteness\index{definiteness} is ruled mostly by demonstrative pronouns. However, there are two Slavic languages that provide this possibility - Bulgarian\footnote{With Pomak dialects that are considered to be a separate language by some scientists.} and Macedonian: these are Slavic analytic marvel (a pair of languages, that has worked out a completely different model of grammar). Novoslovnica let you use this achievement. Going further, we will compare an English and a Slavic systems of articles.

Firstly, you should remember that there is no indefinite article. The word itself shows that you are speaking about a designatum and there is no concrete information about it, no details. However, if you want to accentuate the term of indefiniteness you can use an indefinite pronoun before the word (“Někyǐ”, “Něktoryǐ”).

Secondly, there are huge differences between definite articles\index{article!definite} in English and in Novoslovnica. English has only one article - “the”. It shows only the term of definiteness. Novoslovnica provides in articles additional meanings, which you have already seen in the paragraph about demonstrative pronouns - distance and visibility of the object. So, there are three definite articles in Novoslovnica: “-òt”, “-òn”, “-òv”. Also you should know that these articles have the differentiation by gender and number. 

òt – ta – to – te

òv – va – vo – ve

òn – na – no – ne

Despite of the other POS articles have only two numbers (singular and plural) and are differentiated by gender only in singular. Words in dual manage the article in plural. To remind you, we will repeat that the article “òv” is used, when the object is in the field of view and it is rather close to you. The article “òn” is used, when you still see the object, but it is rather far from you. The article “òt” is used, when you cannot see the object or it is abstract, however you are talking about the definite instance of the designatum.

English article "the" is derived from a demonstrative\index{pronoun!demonstrative} pronoun "that". Novoslovnica articles are derived as it follows: \textit{òn} from \textit{onyǐ}, \textit{òv} - from \textit{òvyǐ} and \textit{òt} - from \textit{tyǐ}. Nonetheless, you can see that the pronoun “sïǐ” has not developed into the article and there is no article equivalent.

Finally, there are differences in the way to use the article. In English you place the article just before the word, dividing the word and the article by the space (different words). In Novoslovnica you put the article just after the word, dividing the word and the article by the hyphen (one word).

\textbf{Examples:} 

\textit{Kniga-na je na stolu} - The book is on table.

\textit{Dom-òv je mnogo starym} - This house is very old.

\subsection{Adjective article}

Articles in Novoslovnica are used with nouns only. However, there are tools for expressing definiteness of adjectives.

Adjective article is a form of using short personal pronouns with adjectives.\footnote{A similar phenomenon is used nowadays in Macedonian to express the definiteness of an object of speaking. However, the "double object" (which stands for a personal pronoun form here) is placed right before the predicate.}

\textbf{Examples:}

Dobòr - Dobòr \textbf{i} 

Dobra - Dobra \textbf{ja} 

Dobro - Dobro \textbf{je} 

You can use such an old form. Though, better you a full adjective form which has been derived from adjective articles in East Slavic languages.

\textbf{Examples:}

Dobòr \textbf{i} - Dobryǐ

Dobra \textbf{ja} - Dobraja

Dobro \textbf{je} - Dobroje

Follow the next rules of using articles:

\begin{itemize}
	\item Use only one definite form in a \gls{np}.
	\item If there are several adjectives in a definite \gls{np}, use a full adjective only with the first one.
	\item If the full form of an adjective is equal to the short one (Due to cases), you should use an article on the noun in \gls{np} or an adjective article itself (in a split form).
\end{itemize}
\section{Particle}

Particle\index{particle} is a dependent part of word. Particles add an auxiliary meaning to the main word. There are some groups of particles.

Particles added to words are close to some additional functions. If you delete them from your phrase, there will be no change in the whole meaning (that is why one cannot say they are of an auxiliary POS), but with presence of particles you will get more additional semantic or sometimes emotional information named \textit{color}. English language has very few particles. The most known is “to” as an infinitive indicator particle of a verb. Controversially, Slavic languages have a bit more particles that are rather popular in the spoken language. They form several groups by their semantic color.

\textbf{Examples:}

\textit{Ja sòm govoril tobě.} - I have told you.

\textit{Ja sòm govoril že tobě!} - I have told you!

Additional semantic meaning. The speaker shifts emphasis from undefined (neutral phrase) to the word “govoril” (told). So the interlocutor now has a determined emphasis of the phrase. The speaker wants to say that he has already told the same fact to the interlocutor and he was right because something happened confirming them.

Positive

* Aga - Yeap

* Ugu - Yeah

* Da - Yes

Negative

* Ne - Not
* Ni...ni - Neither...nor

Interrogative

* Či - Whether

* Li - Whether

Estimative

* Kakto - Like

* Mòl - Supposedly

Comparative

Incentive

Exclamative

Amplificative


Specifying

Restrictive

Demonstrative



\section{Preposition}

Prepositions\index{preposition} also are not an independent POS. They are closely related to the main word (often it is a noun). Most prepositions show the direction or the location of the action. 

We can divide prepositions into two main groups, so as we did with adverbs. We can distinguish \textit{primary} and \textit{secondary} prepositions. Primary prepositions are very ancient and we cannot refer to any word form they have formed from. Secondary prepositions are longer, and they appeared by semantic shift of an adverb, transgressive or a cased-noun. 

Here I will list primary prepositions with English translations and controlled cases. Complex cases are commented in notes.

\textit{Bez} (Gen.) - Without

\textit{V} (Acc.) - In, into

\textit{Dlä} (Gen) - For

\textit{Do} (Gen.) - To

\textit{Za} (Instr.) - For

\textit{Iz} (Gen.) - From (inside the object)

\textit{K} (Dat.) - To

\textit{Krôz} (Skrôz) (Gen.) - Through

\textit{Na} (Acc.) - On

\textit{Nad} (Instr.) - Above

\textit{O} (Prep.) - About

\textit{Od} (Gen.) - From (the object)

\textit{Po} (Dat.) - Along

\textit{Pod} (Instr.) - Under

\textit{Pri} (Loc.) - At

\textit{Pro} (Acc.) - About (the difference between “O” and “Pro” is in the detail view on the object. When we say the second variant we just mention the object in our speech, while using the first one we talk about it in details).

\textit{S} (Instr.) - With

\textit{U} (Gen.) - At (the difference between “Pri” and “U” is in the object of speaking. When we use “U” we mention real object in space and place the object of speaking near it. “Pri” is used when we speak about proximity in time, i.e. some events are close to each other.)

\textit{Črez} (Acc.) - After,  in (time)

The following figure shows the semantics of most primary prepositions.

\begin{figure}
	\includegraphics[width=\linewidth]{./sources/prepositions.jpeg}
	\caption{Prepositions in Novoslovnica}
	\label{fig:prepositions}
\end{figure}

Secondary prepositions are derived from nouns or adverbs with the shift of semantic from independent to an auxiliary one. For examples, preposition "Pred" is derived from the noun "Pred" (Front). Using separately in refers to a frontal part of something. Using with an additional independent word it becomes a preposition defining the frontal part of the word that follows it.

\textbf{Examples:}

\textit{Svòrh} - Over

\textit{Među} - Between
\section{Conjunction}

You saw that there is a rather big amount of prepositions on Novoslovnica. However, the amount of conjunctions\index{conjunction} is much smaller. 

Conjunctions are divided into two classes: coordinating and subordinating conjunctions.

Coordinating\index{conjunction!coordinating} conjunctions usually connect sentence elements of the same grammatical class (N + N, V + V etc.). There are four kinds of coordinating conjunctions: copulative, adversative, disjunctive and illative.

\textbf{Conjunctive}:

I - and

Da - and

Ta - and

Ili - or

Či - or

Abo -or

\textbf{Adversative}:

Ale - but

Ama - but

No - but

Subordinating\index{conjunction!subordinating} conjunctions are used to connect clauses in subordinate sentence. They complement the functionality of corresponding adverbs.

\textbf{Subordinative}:

Dabi - for

Aby - for

Da - for

This is the whole list of existing conjunctions in Novoslovnica today.

\section{Interjection}

This interesting POS is used to describe emotions within the sentence. Interjections\index{interjection} are neither independent nor dependent POS. They even are not involved into sentence structure. They just show the color of the sentence to let the interlocutor show your feelings. Interjections are divided into three groups depending on their aim: emotional, imperative and etiquette. 

\textbf{Emotional interjections}

Emotional interjections can be divided into negative and positive interjections.

Positive:

Ah - Ah

Ŭaŭ - Wow

Ŭah - Wow

Ura - Hurray

Ogo - 

Uf - 

\textbf{Negative}

Oh - 

O-o - Oh

Be - 

Hehe -

Heh -  

Éh - 

Jo - 

Fu - 

\textbf{Ambiguous} (depend on the context):

Uh - Uh

Oǐ - Oh

Aǐ - 

Išty - 

Hm - Hmm

\textbf{Imperative interjections}

Let us look at them:

Éǐ - Hey

Na - Take it

Stop - Stop

Bre - Man

A-u - 

Allo - Hello

Brysj - Go out

Von - Out

\textbf{Etiquette interjections}

These interjections are often whole words, that we use without the sentence context in some situations that need our etiquette.

Hvála - Thanks

Dobrodošli - Welcome

Dękujem - Thanks

Zdråveǐ - Hi

Zdråveǐte - Hello

\section{Onomatopoetic words}

This group of words determines the reflection of animal, bird, babe, technic sounds that person reproduces in his speech.


