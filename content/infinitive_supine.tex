\subsection{Infinitive and Supine}

Infinitive is the main form of the verb. In fact all languages that have an infinitive form of the verb use it in that way. So when you look through the dictionary looking for a verb, you should remember that it will stand in infinitive. Infinitive form is created by adding the ending “-ti” to the end of the verb. (Compare with English, you put the particle “to” before the verb to create the same form - there is some similarity in both cases). Infinitive has no parameters for declination.

The second unchangeable form of the verb is supine. It has no equivalent in English. Semantically, it is similar to the construction “to be going to do something”, determining the aims of the subject. Supine in Novoslovnica is built by adding the “-tj” ending to the end of the verb.

Supine had a great usage area in the past, now it is still used in Uppersorbian, Lowersorbian and Slovenian languages. It is mostly used with the verbs of motion, such as “to go”, “to swim”, “to move”, “to fly” etc.

Examples:

Ty hteš kazati mi něčto, da li? - You want to say me something, don’t you? (Infinitive)

Mamo, ja idaju spatj dnesj po-pano. - Mom, I’m going to sleep now earlier. (Supine)
