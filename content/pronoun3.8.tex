\section{Pronoun}

\begin{table}[h]
	\caption{Pronoun characteristics}
	\begin{tabular}{lllll}
		\textbf{Title}              & \textbf{Value}               \\
		Semantic value              & Attribute, Concept           \\
		Category                    & Independent                  \\
		Subcategory                 & Nominal                      \\
		Alteration                  & Declension                   \\
		Alteration parameters       & Case, Numbers, Gender, Person\\
		Differentiation parameters  & Gender, Type, Group
	\end{tabular}
\end{table}

Pronoun\index{pronoun} is a POS that has different meanings. It can play the role of an attribute or a concept, depending on what is replaced with the pronoun. However, pronoun has a verbal property of person.

Pronouns are divided into three types: nominal\index{pronoun!nominal} (noun-like declension), substantive\index{pronoun!substantive} (substantive declension), attributive\index{pronoun!attributive} (adjective-like declension).

There are also several groups of pronouns, depending on their semantic value. They are: personal, possessive, interrogative, relative, indefinite, definitive, reflexive, demonstrative, negative, reciprocal.

\subsection{Personal pronouns}

One of the most important groups are personal pronouns\index{pronoun!personal}. They replace nouns in sentences, where we do not want to use an unreasonable repeat. Personal pronouns have different forms for every person-number cell. In the table you can see them.

\begin{table}
	\begin{tabular}{llll}
		& Singular & Dual & Plural \\
		1 person & Ja (Az) & Ma & My \\
		2 person & Ty & Va & Vy \\
		3 person, ms & On & Ona & Oni \\
		3 person, fm & Ona & One & Oně \\
		3 person, neu & Ono & Oná & Onji
	\end{tabular}
\end{table}

I should mention the form “Vy” (You). As in English or polite Russian, we can use this pronoun for a single person when we want to emphasize our respect to the interlocutor. Moreover, we also have to use a plural form of the verb while speaking “Vy” in polite form.

Also you can see two forms of the 1 person - singular pronoun (I). Etymologically the form “Az” is full while “Ja” is only a short one. However, different Slavic languages have retained different forms and now we should support both ones. In Novoslovnica the difference between using “Az” and “Ja” lies in phonetics. As you see, “Az” begins with the vowel and ends with a consonant, “Ja” - controversially. If we remember rule 1 we will get such number of rules:

\begin{itemize}
	\item If the word before the pronoun ends with a vowel and the word after begins with a consonant - you should use “Ja”
	\item If the word before the pronoun ends with a consonant and the word after begins with a vowel - you should use “Az”
	\item In other cases you can use either one or another variant (“Az” is more formal than “Ja”)
\end{itemize}

Now let us speak about declension of personal pronouns. Personal pronouns relate to nominal pronouns (with noun-like declension). This is the most difficult type to learn. But everything has its order and beauty.

\begin{table}[!htb]
	\begin{tabular}{llll}
		I & Singular & Dual & Plural \\
		Nominative & Ja (Az) & Ma & My \\
		Genitive & Menę & Naju & Nas \\
		Partitive & Menä & Naju & Nas \\
		Accusative & Mene & Naju & Nas \\
		Dative & Meni & Nama & Nam \\
		Instrumental & Mnom (Mnoǐ) & Nama & Nami \\
		Prepositional & O mně & O naju & O nas \\
		Locative & Vo mnu & V naju & V nas \\
		Vocative & - & - & -
	\end{tabular}
\end{table}

\begin{table}[!htb]
	\begin{tabular}{llll}
		You (sg) & Singular & Dual & Plural \\
		Nominative & Ty & Va & Vy \\
		Genitive & Tebę & Vaju & Vas \\
		Partitive & Tebä & Vaju & Vas \\
		Accusative & Tebe & Vaju & Vas \\
		Dative & Tebi & Vama & Vam \\
		Instrumental & Tobom (Toboǐ) & Vama & Vami \\
		Prepositional & O mně & O vaju & O vas \\
		Locative & Vo mnu & V vaju & V vas \\
		Vocative & - & - & -
	\end{tabular}
\end{table}

\begin{table}[!htb]
	\begin{tabular}{llll}
		He & Singular & Dual & Plural \\
		Nominative & On & Ona & Oni \\
		Genitive & Jeĝa & Onaju & Ih \\
		Partitive & Jeĝu & Onaju & Ih \\
		Accusative & Jeĝo & Onaju & Ih \\
		Dative & Jemu & Onama & Im \\
		Instrumental & Nim & Onama & Nimi \\
		Prepositional & O nëm & Ob onaju & O nih \\
		Locative & V nëmu & V onaju & V nih \\
		Vocative & - & - & -
	\end{tabular}
\end{table}

\begin{table}[!htb]
	\begin{tabular}{llll}
		She & Singular & Dual & Plural \\
		Nominative & Ona & Oně & Oni \\
		Genitive & Ji & Oněju & Ih \\
		Partitive & Ji & Oněju & Ih \\
		Accusative & Ju & Oněju & Ih \\
		Dative & Ji & Oněma & Im \\
		Instrumental & Neju & Oněju & Nimi \\
		Prepositional & O neǐ & Ob oněju & O nih \\
		Locative & V neji & V oněju & V nih \\
		Vocative & - & - & -
	\end{tabular}
\end{table}

\begin{table}[!htb]
	\begin{tabular}{llll}
		It & Singular & Dual & Plural \\
		Nominative & Ono & Oná & Oni \\
		Genitive & Jeĝa & Náju & Ih \\
		Partitive & Jeĝu & Oněju & Ih \\
		Accusative & Jeĝo & Oněju & Ih \\
		Dative & Jemu & Onáma & Ih \\
		Instrumental & Nim & Onáma & Nimi \\
		Prepositional & O nem & O náju & O nih \\
		Locative & V nemu & V náju & V nih \\
		Vocative & - & - & -
	\end{tabular}
\end{table}


\subsection{Reflexive pronoun}

This\index{pronoun!reflexive} is a separate group of pronouns. There is the only reflexive pronoun in Novoslovnica - “sebę”. Its feature is the absence of nominative form. It has only 7 cases to be alternated. Vocative and Nominative are absent.

\begin{table}
	\begin{tabular}{lll}
		Case & Full & Short \\
		Genitive & Sebę & Sę \\
		Partitive & Sebä & Sä \\
		Accusative & Sebe & Se \\
		Dative & Sebi & Si \\ 
		Insrumental & Sobom (Soboǐ) & - \\ 
		Prepositional & O sobě & - \\
		Locative & V sobu & -
	\end{tabular}
\end{table}


“Sę” is used very often as a reflexive suffix in verbs. It determines a way of creating medial voice sentences (look the paragraph about it).

However, there is a term of a complex reflexive form (CRF) also. It is formed by the sequence of a personal pronoun and the definitive pronoun “sám”. This form shows a partly-developed reflection of a subject.  

There is practically no difference in using a reflexive pronoun or a complex reflexive form. However, it is recommended to us a CRF for Nominative and to use a reflexive pronoun in other cases.

\subsection{Possessive pronouns}

Possessive\index{pronoun!possessive} pronouns show whom any object belongs to. For example, we can say “It is a thing of Bob (which Bob possesses)”. In Novoslovnica you can use a similar expression: “To je věčj Boba” (remember rules of case using). Also Novoslovnica has another expression with a possessive adjective: “To je Bobóva věčj”. This adjective shows that this thing belongs to Bob. However, if we have already used the name of Bob in our sentence, we should not repeat it again. English also uses possessive pronouns in such cases: “Bob is my friend and that’s his thing”. We do not repeat the word Bob, we just say - his (which he possesses). That is what the possessive pronouns look like.

These pronouns are attributive, so we have no need to rewrite their declension, just to name nominative forms. Further, you take a nominative form of a possessive pronoun, look through declension tables for adjectives and transform your pronoun so as you did it with an adjective. Now let us look at the nominative forms of possessive pronouns.

\begin{table}
	\begin{tabular}{llll}
		& Singular & Dual & Plural \\
		1 person & Môǐ & Naš & Naš \\
		2 person & Tvôǐ & Vaš & Vaš \\
		3 person, ms & Jegôǐ & Onyš & Ih \\
		3 person, fm & Jejôǐ & Oneš & Ih \\
		3 person, neu & Jegôǐ & Onyš & Ih
	\end{tabular}
\end{table}

As usual, bold letters are under an accent.

The only feature of possessive pronouns is that you should add a “virtual” ending for 1-person and 2-person pronouns to start declining of it.

\textbf{Examples:}

- \textit{Môǐ - Mojyǐ} (virtual full form) - \textit{Mojoga, mojomu, mojym} (ordinary attributive declension) 

To avoid this difficulty you may use an “adjectived” form

\begin{table}
	\begin{tabular}{llll}
		& Singular & Dual & Plural \\
		1 person & Môǐnyǐ & Našnyǐ & Našnyǐ \\
		2 person & Tvôǐnyǐ & Vašnyǐ & Vašnyǐ \\
		3 person, ms & Jegôǐnyǐ & Onyšnyǐ & Ihnyǐ \\
		3 person, fm & Jejôǐnyǐ & Onešnyǐ & Ěhnyǐ \\
		3 person, neu & Jegôǐnyǐ & Onyšnyǐ & Ihnyǐ
	\end{tabular}
\end{table}

\subsection{Interrogative and relative pronouns}

Interrogative\index{pronoun!interrogative} pronouns are used when we create a question. They are never in plural or dual. Also they have no declension. The only role of interrogative pronouns is to reveal an aim of the question (How much? Who? What?) - to guide the interlocutor to the right answer (you need).

Relative\index{pronoun!relative} pronouns aim is to introduce a relative clause.

\textit{Relative clause}\index{clause!relative} is a special kind of subordinate clause whose primary function is as modifier to a noun or nominal. We examine the case of relative clause modifiers in NPs first, and then extend the description to cover less prototypical relative constructions.\cite{english-grammar}

Interrogative and relative pronouns are practically the same, only usage differs, that is why I talk about them in the same paragraph. Let us look at them in the table.

\begin{table}
	\begin{tabular}{lll}
		English equivalent & Interrogative pronoun & Relative pronoun \\
		Who & Kto? & Ke \\
		What & Čto & Če \\
		What & Kakyǐ? & Kak \\
		Which & Ktoryǐ? & Ktor \\
		Whose & Čyǐ? & Čyǐ \\
		What & Jakyǐ? & Jak \\
		What & Kakvyǐ ? & Kakòv \\
		What & Kolïkyǐ ? & Kolïk 
	\end{tabular}
\end{table}

Pronouns “Kolïko”, “” do not decline. Pronouns “Kakyǐ”, “Ktoryǐ” decline so as adjectives do. Other pronouns have a similar declension with possessive pronouns.
  
\subsection{Indefinite and negative pronouns}

Another group of pronouns that are similar to each other are indefinite\index{pronoun!indefinite} and negative\index{pronoun!negative} pronouns. They are derived from interrogative pronouns by adding the negative “ni-” and the indefinite “ne-” prefixes. Their declension equals to their ancestor’s one.

\begin{table}
	\begin{tabular}{lll}
		Interrogative pronoun & Negative pronoun & Indefinite pronoun \\
		Kto? & Nikto & Někto \\
		Čto & Ničto & Něčto \\
		Kakyǐ? & Nikakyǐ & Někakyǐ \\
		Ktoryǐ? & Niktoryǐ & Něktoryǐ \\
		Čyǐ? & Ničyǐ & Něčyǐ 
	\end{tabular}
\end{table}

\subsection{Demonstrative pronouns}

Demonstrative\index{pronoun!demonstrative} pronouns are usually used to make the interlocutor pay attention to something.
There are only four demonstrative pronouns. And this topic is closely connected with the articles. In the table … you can see these pronouns, divided by terms of visibility and distance of the object to name with respect to the speaker.


\begin{table}
	\begin{tabular}{lll}
		& Visible & Invisible \\
		Far & Onyǐ & Tyǐ \\
		\multirow{2}{*}{Close} & Ovyǐ & - \\ & \multicolumn{2}{c}{Sïǐ}  
	\end{tabular}
\end{table}

These pronouns decline so as adjectives do, so that is very easy. However, you see that there is no pronoun for a close object that is not seen to you. Maybe it is logical, maybe not, nevertheless, Slavic languages do not support this semantic value.

Examples:

\subsection{Definitive pronouns}

Attributive\index{pronoun!definitive} pronouns indicate a generalized feature of an object. 

\begin{table}
	\begin{tabular}{ll}
		English equivalent & Definitive pronoun \\
		Whole & Vesj \\
		All & Vsë, vsä, vsï \\
		Every & Vsękyǐ, lübyǐ \\
		Each & Káždyǐ \\
		Another, other & Ïnyǐ, drugyǐ \\
		Reflexive pronouns & Personal pronouns + sám
	\end{tabular}
\end{table}

All these pronouns decline as adjectives. I should say just about the last one - the pronoun “sám”. This pronoun is used with personal pronouns to create a complex reflexive form (look paragraph about reflexive pronouns). However, it keeps adjective-like declension.

\subsection{Reciprocal pronouns}


The last type of pronouns is reciprocal\index{pronoun!reciprocal}. It is used to refer to a noun phrase mentioned earlier in a sentence. English has only two such pronouns - each other and another one.

Slavic languages have much more variants:

\textit{Drug so drugom}

\textit{Raz za razom}

