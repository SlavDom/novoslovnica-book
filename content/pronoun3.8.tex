\section{Pronoun}

\begin{table}[h]
	\caption{Pronoun characteristics}
	\begin{tabular}{lllll}
		\textbf{Title}              & \textbf{Value}               \\
		Semantic value              & Attribute, Concept           \\
		Category                    & Independent                  \\
		Subcategory                 & Nominal                      \\
		Alteration                  & Declension                   \\
		Alteration parameters       & Case, Numbers, Gender, Person\\
		Differentiation parameters  & Gender, Type, Group
	\end{tabular}
\end{table}

Pronoun is a POS that has different meanings. It can play the role of an attribute or a concept, depending on what is replaced with the pronoun. However, pronoun has a verbal property of person.

Pronouns are divided into three types: nominal (noun-like declension), substantive (substantive declension), attributive (adjective-like declension).

There are also several groups of pronouns, depending on their semantic value. They are: personal, possessive, interrogative, relative, indefinite, definitive, reflexive, demonstrative, negative, reciprocal.

\subsection{Personal pronouns}

One of the most important groups are personal pronouns. They replace nouns in sentences, where we do not want to use an unreasonable repeat. Personal pronouns have different forms for every person-number cell. In the table you can see them.


