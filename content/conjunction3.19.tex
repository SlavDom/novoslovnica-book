\section{Conjunction}

You see that there is a rather big amount of prepositions on Novoslovnica. However, the amount of conjunctions\index{conjunction} is much smaller. 

Conjunctions are divided into two classes: \textit{coordinating} and \textit{subordinating} conjunctions.

Coordinating\index{conjunction!coordinating} conjunctions usually connect sentence elements of the same grammatical class (N + N, V + V etc.). Syntactically, coordinating conjunctions connect either homogenuous sentense parts or independent clauses in a compound sentense. There are four kinds of coordinating conjunctions: copulative, adversative, disjunctive and illative.

\textbf{Conjunctive}:

I - and

Da - and

Ta - and

Ili - or

Či - or

Abo -or

\textbf{Adversative}:

Ale - but

Ama - but

No - but

Jednako - but

\textbf{Disjunctive}

Alïbo - either ... or

Lïbo - either ... or

A - and

\textbf{Illative}

Bo - because, that is why

Tako - thus, so

Subordinating\index{conjunction!subordinating} conjunctions are used to connect clauses in subordinate sentence. They complement the functionality of corresponding adverbs.

\textbf{Subordinative}:

Dabi - for

Aby - for

Da - for

This is the whole list of conjunctions existing in Novoslovnica today.
