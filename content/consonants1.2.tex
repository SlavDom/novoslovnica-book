\section{Consonants}

In this paragraph about consonants, I would like to begin with the definition of a consonant.

Consonant - a sound that is articulated with complete or particular closure of the vocal tract. 
Likewise vowels, consonants have three characteristics that determine their position in your articulation. These three parameters are:
Place, where the consonant is pronounced in the mouth
Manner, how the consonant is pronounced 
Sonority, whether you use your vocal cords or not

Place of the consonant can be quite different. Here are possible types: Bilabial, labiodental, dental, alveolar, postalveolar, palato-alveolar, retroflex, alveolo-palatal, palatal, labio-velar, velar. There are more types, but they do not exist in Novoslovnica.

Manner is the way how you pronounce the sound. There are also different manners, that are used in Novoslovnica. They are: nasal, stop, affricate (sibilant), sibilant fricative, non-sibilant fricative, approximant, trill and lateral approximant.

Sonority is the boolean attribute of pronunciation. You can either use your voice with the sound you pronounce or not. Notice that vowels cannot be pronounced without the use of your voice. 

Combining these three parameters, we get the unique consonant that we want to pronounce. I cannot draw  a 3-dimensional table, because there are three parameters on input, so we will combine information into 2-dimensional space as in paragraph about vowels. So, look at table 1.5 and see the different consonants that are used in Novoslovnica.



Different colors of the cells show the sonority of the consonant. Yellow color shows that the sound is voiced, while green ones are for voiceless sounds.

Blue cells in the table show that sounds in it can be used both in voiced and voiceless forms as allophones.

Novoslovnica has 51 consonants, 21 of them are voiceless and 30 are voiced.

However, not all of these consonants are language phonemes. So, let’s talk about the allophones among these sounds.